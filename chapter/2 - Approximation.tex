\newpage
%
%	Аппроксимация функция
%
\section{Аппроксимация функция}
Задача о приближении функции ставится следующим образом:
данную функцию $f(x)$ необходимо заменить 
обобщенным полиномом $Q_m(x)$ заданного порядка $m$ 
так, чтобы отклонение (в известном смысле) функции $f(x)$ 
от обобщенного полинома $Q_m(x)$ на указанном множестве 
$\vec{x}=\{x\}$ было наименьшим. 
При этом полином $Q_m(x)$ в общем случае 
называется аппроксимирующим.

Если множество $\vec{x}$ состоит из отдельных точек 
$x\in\{x_0, x_1, x_2, \dots x_n\}$ (узлов),
то приближение называется \textit{точечным}.
Если $\vec{x}$ есть отрезок $x_a<x<x_b$, 
то приближение называется \textit{интегральным}. 
Для практики важным является приближение функций 
алгебраическими и тригонометрическими полиномами.

\subsection{Точечное квадратичное аппроксимирование функций}
На практике часто бывает, что заданный порядок $m$ 
приближающего полинома $Q_m(x)$ меньше числа 
узлов аппроксимации ${m<n}$, в которых 
известно значение функции $y_i=f(x_i)$ ($i=1,2, \cdots, n$).
В этом случае обычно используют точечный 
метод наименьших квадратов и
рассматривается полином степени $m$ вида:
\begin{gather*}
Q_m(x)=a_0+a_1\cdot{x}+a_2\cdot{x^2}+\dots+a_m\cdot{x^m}=
\sum\limits_{j=0}^{m}a_j\cdot{x^j},
\end{gather*}
где $S_m$ -- мера отклонения полинома $Q_m(x)$ 
от известной функции $y(x)$ на множестве точек 
$\{x_0, x_1, x_2,\cdots,x_n\}$, которая принимается 
равная сумме квадратов отклонений полинома от этой функции 
на заданной системе точек:
\begin{gather*}
S_m = \sum_{i=0}^{n}\left(Q_m(x_i)-y_i\right)^2
\end{gather*}

Следует отметить, что $S_m=S_m(a_0, a_1, \cdots, a_n)$ 
есть функция многих переменных, т.е. коэффициентов полинома
$a_i$ ($i=0,1,\dots,m$), которые необходимо подобрать так, 
чтобы величина $S_m$ была наименьшей $S_m\to{\min}$.
Полученный полином называется аппроксимирующим 
для данной функции, а процесс построения этого полинома -- 
точечной квадратичной аппроксимацией или 
точечным квадратичным аппроксимированием функции. 

Для решения задачи точечного квадратичного аппроксимирования,
т.е. определения числовых значений всех коэффициентов 
полинома $Q_m(x)$, необходимо найти \emph{положения минимума 
функции} многих переменных $S_m$.

Определим частные производные от величины суммы квадратов отклонений и 
воспользовавшись условием экстремума функции многих переменных, 
составим систему уравнений вида:
\begin{gather*}
\pdiff{S_m}{a_0}=\pdiff{S_m}{a_1}=\pdiff{S_m}{a_2}=\cdots=\pdiff{S_m}{a_m}=0
\end{gather*}

Для определения неизвестных коэффициентов полинома
$a_0, a_1, a_2,\cdots a_m$ необходимо решить систему 
$m+1$ уравнений с $m +1$ неизвестными: 
\begin{gather*}
\left\{\begin{matrix}
\pdiff{S_m}{a_0}&=2\cdot\sum\limits_{i=0}^{n}\left(a_0+a_1\cdot{x_i}+a_2\cdot{x_i^2}+a_3\cdot{x_i^3}+\dots+a_m\cdot{x_i^m} - y_i\right)\cdot1=0\\[14pt]
\pdiff{S_m}{a_1}&=2\cdot\sum\limits_{i=0}^{n}\left(a_0+a_1\cdot{x_i}+a_2\cdot{x_i^2}+a_3\cdot{x_i^3}+\dots+a_m\cdot{x_i^m} - y_i\right)\cdot{x_i}=0\\[14pt]
\pdiff{S_m}{a_2}&=2\cdot\sum\limits_{i=0}^{n}\left(a_0+a_1\cdot{x_i}+a_2\cdot{x_i^2}+a_3\cdot{x_i^3}+\dots+a_m\cdot{x_i^m} - y_i\right)\cdot{x_i^2}=0\\[14pt]
\hdotsfor{2}\\[14pt]
\pdiff{S_m}{a_m}&=2\cdot\sum\limits_{i=0}^{n}\left(a_0+a_1\cdot{x_i}+a_2\cdot{x_i^2}+a_3\cdot{x_i^3}+\dots+a_m\cdot{x_i^m} - y_i\right)\cdot{x_i^m}=0\\[14pt]
\end{matrix}\right.
\end{gather*}

Таким образом, задача точечной квадратичной аппроксимации 
функции сводится к решению системы линейных уравнений вида:
\begin{gather*}
\begin{matrix}
\mathbf{C}\cdot\vec{a}=\vec{b}
&\text{или}&
\begin{pmatrix}
c_{00}&c_{01}&\cdots&c_{0m}\\
c_{10}&c_{11}&\cdots&c_{1m}\\
\vdots&\vdots&\ddots&\vdots\\
c_{m0}&c_{m1}&\cdots&c_{mm}\\
\end{pmatrix}
\cdot
\begin{pmatrix}a_0\\a_1\\\vdots\\a_m\end{pmatrix}
=\begin{pmatrix}b_0\\b_1\\\vdots\\b_m\end{pmatrix}
\end{matrix},\end{gather*}
относительно неизвестных коэффициентов полинома 
$\{a_0, a_1, a_2,\cdots, a_m\}$.

Если среди точек $\{x_0, x_1, x_2, \cdots, x_n\}$ 
нет совпадающих, а также степень полинома 
меньше чем число узлов аппроксимации $m<n$, 
то определитель системы не равен нулю.
Следовательно, эта система имеет единственное решение 
$\{a_0, a_1, a_2,\cdots, a_m\}$,
а полином $Q_m(x)$ с такими коэффициентами $a_i$ 
будет обладать минимальным квадратичным отклонением 
$S_m=S_{min}$. 

%
%	Аппроксимирования функций полиномом второй степени $Q_2(x)$
%
\subsection{Аппроксимирования функций полиномом
второй степени $Q_2(x)$}
Известна таблица данных некоторой функциональной зависимости 
$y(x)$.
\begin{center}
\begin{tabular}{r*{5}{r}}
\toprule
$x_i$&$-0.76$&$-0.48$&$-0.09$&$0.22$&$0.55$\\
\midrule
$y_i$&$5.15$&$4.39$&$4.10$&$5.71$&$5.30$\\
\bottomrule
\end{tabular}
\end{center}

\end{document}

Необходимо аппроксимировать экспериментальные данные 
полиномом второй степени $Q_2(x)$:
\begin{gather*}
Q_2(x)=a_0 + a_1\cdot{x} + a_2\cdot{x^2}
\end{gather*}

\begin{enumerate}
\item
Построим меру отклонения полинома $Q_2(x)$ 
от таблично заданной функции $y_i=f(x_i)$
на множестве точек $\{x_0, x_1, x_2, x_3, x_4\}$:
\begin{gather*}
S_2=\sum_{i=0}^{4}\left(a_0+a_1\cdot{x_i}+a_2\cdot{x_i^2}-y_i\right)^2
,\end{gather*}
где $y_i=f(x_i)$ -- значение функции в точке $x_i$.

\item
Запишем меру отклонения $S_2$ в явном виде 
на основе данных из условия задачи:
\begin{gather*}
\begin{split}
S_2=
&\left(a_0 + a_1\cdot(-0.76) + a_2\cdot(-0.76)^2 - 5.15 \right)^2+\\
+&\left(a_0 + a_1\cdot(-0.48) + a_2\cdot(-0.48)^2 - 4.39 \right)^2+\\
+&\left(a_0 + a_1\cdot(-0.09) + a_2\cdot(-0.09)^2 - 4.10 \right)^2+\\
+&\left(a_0 + a_1\cdot(0.22) + a_2\cdot(0.22)^2 - 5.71 \right)^2+\\
+&\left(a_0 + a_1\cdot(0.55) + a_2\cdot(0.55)^2 - 5.30 \right)^2
\end{split}
\end{gather*}

\item
Определим частную производную от меры отклонений $S_2$ 
по аргументу $a_0$ и приравняем её нулю:
\begin{gather*}
\begin{split}
\pdiff{S_2}{a_0}=
&2\cdot\left(a_0 + a_1\cdot(-0.76) + a_2\cdot(-0.76)^2 - 5.15 \right)\cdot 1+\\
+&2\cdot\left(a_0 + a_1\cdot(-0.48) + a_2\cdot(-0.48)^2 - 4.39 \right)\cdot 1+\\
+&2\cdot\left(a_0 + a_1\cdot(-0.09) + a_2\cdot(-0.09)^2 - 4.10 \right)\cdot 1+\\
+&2\cdot\left(a_0 + a_1\cdot(0.22) + a_2\cdot(0.22)^2 - 5.71 \right)\cdot 1+\\
+&2\cdot\left(a_0 + a_1\cdot(0.55) + a_2\cdot(0.55)^2 - 5.30 \right)\cdot 1=0
\end{split}
\end{gather*}

Коэффициенты первой строки матрицы $\mathbf{C}$:
\begin{gather*}
\begin{array}{rcl}
c_{00}&=&1+1+1+1+1=5\\
c_{01}&=&(-0.76) + (-0.48) + (-0.09) + (0.22) + (0.55) = -0.56\\
c_{02}&=&(-0.76)^2 + (-0.48)^2 + (-0.09)^2 + (0.22)^2 + (0.55)^2=1.18\\
\end{array}
\end{gather*}
Первый элемент вектора правой части $\vec{b}$:
\begin{gather*}
b_0=5.15 + 4.39 + 4.10 + 5.71 + 5.30=24.65
\end{gather*}
\item
Определим частную производную от меры отклонений 
$S_2$ по аргументу $a_1$ и приравняем её нулю:
\begin{gather*}
\begin{split}
\pdiff{S_2}{a_1}=
&2\cdot\left(a_0 + a_1\cdot(-0.76) + a_2\cdot(-0.76)^2 - 5.15 \right)\cdot(-0.76)+\\
+&2\cdot\left(a_0 + a_1\cdot(-0.48) + a_2\cdot(-0.48)^2 - 4.39 \right)\cdot(-0.48)+\\
+&2\cdot\left(a_0 + a_1\cdot(-0.09) + a_2\cdot(-0.09)^2 - 4.10 \right)\cdot(-0.09)+\\
+&2\cdot\left(a_0 + a_1\cdot(0.22) + a_2\cdot(0.22)^2 - 5.71 \right)\cdot(0.22)+\\
+&2\cdot\left(a_0 + a_1\cdot(0.55) + a_2\cdot(0.55)^2 - 5.30 \right)\cdot(0.55)=0
\end{split}
\end{gather*}

Коэффициенты второй строки матрицы $\mathbf{C}$:
\begin{gather*}
\begin{array}{rcl}
c_{10}&=&(-0.76) + (-0.48) + (-0.09) + (0.22) + (0.55) = -0.56\\
c_{11}&=&(-0.76)^2 + (-0.48)^2 + (-0.09)^2 + (0.22)^2 + (0.55)^2=1.18\\
c_{12}&=&(-0.76)^3 + (-0.48)^3 + (-0.09)^3 + (0.22)^3 + (0.55)^3=-0.38\\
\end{array}
\end{gather*}

Второй элемент вектора правой части $\vec b$:
\begin{gather*}
\begin{split}
b_1&=&5.15\cdot(-0.76) +4.39\cdot(-0.48) +4.10\cdot(-0.09) +\\
&+5.71\cdot(0.22) +5.30\cdot(0.55)=-2.24
\end{split}
\end{gather*}

\item
Определим частную производную от меры отклонений $S_2$ по аргументу $a_2$ и приравняем её нулю:
\begin{gather*}
\begin{split}
\pdiff{S_2}{a_2}=
&2\cdot\left(a_0 + a_1\cdot(-0.76) + a_2\cdot(-0.76)^2 - 5.15 \right)\cdot(-0.76)^2+\\
+&2\cdot\left(a_0 + a_1\cdot(-0.48) + a_2\cdot(-0.48)^2 - 4.39 \right)\cdot(-0.48)^2+\\
+&2\cdot\left(a_0 + a_1\cdot(-0.09) + a_2\cdot(-0.09)^2 - 4.10 \right)\cdot(-0.09)^2+\\
+&2\cdot\left(a_0 + a_1\cdot(0.22) + a_2\cdot(0.22)^2 - 5.71 \right)\cdot(0.22)^2+\\
+&2\cdot\left(a_0 + a_1\cdot(0.55) + a_2\cdot(0.55)^2 - 5.30 \right)\cdot(0.55)^2=0
\end{split}
\end{gather*}

Коэффициенты третьей строки матрицы $\mathbf{C}$:
\begin{gather*}
\begin{array}{rcl}
c_{20}&=&(-0.76)^2 + (-0.48)^2 + (-0.09)^2 + (0.22)^2 + (0.55)^2=1.18\\
c_{21}&=&(-0.76)^3 + (-0.48)^3 + (-0.09)^3 + (0.22)^3 + (0.55)^3=-0.38\\
c_{22}&=&(-0.76)^4 + (-0.48)^4 + (-0.09)^4 + (0.22)^4 + (0.55)^4=0.49\\
\end{array}
\end{gather*}

Третий элемент вектора правой части $\vec b$:
\begin{gather*}
\begin{split}
b_2=
&5.15\cdot(-0.76)^2 +4.39\cdot(-0.48)^2 +4.10\cdot(-0.09)^2 +\\
&+5.71\cdot(0.22)^2 +5.30\cdot(0.55)^2=5.94
\end{split}
\end{gather*}

\item
Таким образом, для определения неизвестных коэффициентов аппроксимирующего полинома 
необходимо решить систему линейных алгебраических уравнений:
\begin{gather*}
\left\{\begin{matrix}
&5\cdot a_0&-&0.56\cdot a_1&+&1.18\cdot a_2&=&24.65\\
-&0.56\cdot a_0&+&1.18\cdot a_1&-&0.38\cdot a_2&=&-2.24\\
&1.18\cdot a_0&-&0.38\cdot a_1&+&0.49\cdot a_2&=&5.94\\
\end{matrix}\right.
\end{gather*}

\item
Решение системы линейных алгебраических уравнений находится с помощью метода Гаусса.
$a_0=4.66$, $a_1=0.80$, $a_2=1.52$.
Таким образом. аппроксимирующий полином имеет вид:
\begin{gather*}
Q_2(x)=4.66 + 0.80\cdot{x} + 1.52\cdot{x^2}
\end{gather*}

\item
Построение графиков таблично заданной 
функции $y(x)$ и аппроксимирующего 
полинома второго порядка $Q_2(x)$:
% *******************************
%	График функций
%
\begin{center}
\begin{tikzpicture}
%[background rectangle/.style={fill=olive!10}, show background rectangle]
\begin{axis}
[font=\small,
%every axis/.style={color=black, solid, thick},
xlabel = {$x$},		% подпись оси x
ylabel = {$Q_2(x)$},	% подпись оси y
xmin=-1, xmax=0.7, xtick={-0.8,-0.4,0,0.4},
ymin=3.8, ymax=6,
xtick style={thick, black},
ytick style={thick, black},
grid=major,		
major grid style={color=black!20, dashed, thin},
]
\addplot[only marks,mark size=3pt,
mark=ball,mark size=4pt, 
mark options={thin,draw=darkblue,ball color=darkblue!50}]
coordinates 
{(-0.76,5.15) (-0.48,4.39) (-0.09,4.10) (0.22,5.71) (0.55,5.30)};
\addplot[thick,color=darkblue,domain=-0.9:0.6, samples=50] 
{4.66 + 0.80*x + 1.52*x^2};
%\draw[color=red] (axis cs: -0.35,4.75) node[right] {$Q_2(x)$};
\end{axis}
\end{tikzpicture}
\end{center}

\end{enumerate}