% Required Package and Librarie
%\pgfplotsset{compat=1.18}
%\usepgfplotslibrary{fillbetween}
\pgfdeclarelayer{pre main}
\pgfdeclarelayer{background}
\pgfdeclarelayer{foreground}
\pgfsetlayers{background,pre main,main,foreground} 
%\usetikzlibrary{backgrounds}
%
%	Нелинейные уравнения
%
\newpage
\section{Решение нелинейных уравнений}
Пусть задана функция $f(x)$ действительного переменного и 
необходимо найти корни уравнения или, что то же самое, нули функции $f(x)$:
\begin{gather}\label{eq:NLE:f(x)=0}
f(x)=0.
\end{gather}

На примере алгебраического многочлена известно, что нули $f(x)$ могут быть 
как действительными, так и комплексными числами.
Поэтому \textit{более точная} постановка задачи состоит в нахождении корней уравнения,
расположенных в заданной области комплексной плоскости. 
Можно рассматривать также задачу о нахождении действительных корней уравнения,
которые расположены в пределах заданного отрезка $x\in[a,b]$.
% *******************************
%	График функций
%
\begin{figure}[H]\centering
\begin{tikzpicture}
\begin{axis}
[
xlabel={\empty},		% подпись оси x
ylabel={\empty},	% подпись оси y
xmin=-1.5, xmax=2.75, xtick={-1,0,1,1.5,2}, xticklabels={$x_1$,$\mathbf{a}$,$x_2$,$\mathbf{b}$,$x_3$},
ymin=-2, ymax=3, ytick={0}, yticklabels={$f(x)=0$},
]
\addplot[color=black,domain=-1.5:2.75, samples=2] {0};
\addplot[thick,color=darkred,domain=-1.25:2.5, samples=100] {(x-1)*(x+1)*(x-2)};
\addplot[only marks,ball darkred] coordinates {(-1,0) (1,0) (2,0)};
\draw[red] (axis cs:0,2.2) node[right] {$f(x)$};
\end{axis}
\end{tikzpicture}
\caption{График функции $y=f(x)$}
\label{fig:NLE:f(x)=0}
\end{figure}
На рисунке \eqref{fig:NLE:f(x)=0} представлены 
$x_1$, $x_2$ и $x_3$ -- действительные корни уравнения
\eqref{eq:NLE:f(x)=0}, т.е. $f(x_1)=0, f(x_2)=0, f(x_3)=0$

Задача нахождения корней уравнения $f(x)=0$ обычно решается в два этапа:
\begin{enumerate}
\item
На первом этапе изучается расположение корней (в общем случае на комплексной плоскости)
и проводится их разделение, т. е. \emph{выделяются области} в комплексной плоскости, 
\emph{содержащие только один корень}.
Кроме того, изучается вопрос о кратности корней.
Тем самым находятся некоторые начальные приближения для корней уравнения. 
\item
На втором этапе, \emph{используя заданное начальное приближение},
строится итерационный процесс, позволяющий \emph{уточнить значение отыскиваемого корня}.
\end{enumerate}

\begin{tcolorbox}
Следует отметить, что не существует каких-то общих регулярных приемов решения задачи 
о расположении корней произвольной функции $f(x)$.
\end{tcolorbox} 

Численные методы решения нелинейных уравнений являются,
как правило, итерационными методами, которые предполагают 
задание достаточно близких к искомому решению начальных данных.

%
%	Итервальный метод поиска корня уравнения
%
\emptyline
\subsection{Итервальный метод}
Итервальный метод поиска корня уравнения $f(x)=0$ 
состоит следующем:
\begin{enumerate}%[label=(\arabic*)]
\item
Область поиска корня $[a,b]$ разбивается на заранее заданное количество интервалов $N$:
\begin{gather*}
[a,b]=[a,x_1] \cup [x_1,x_2] \cup \cdots \cup [x_i,x_{i+1}] \cup \cdots [x_{N},b]
\end{gather*}
\item
Вычисляется таблица значений функции $\{f(x_i)\}$ на границах этих интервалов $\{x_i\}$.
\item
Проводится последовательный перебор таблицы значений функции $\{f(x_i)\}$
$(i=1,2,\cdots,N)$.
\item
Если при некотором $i$ значения функции $f(x_i)$ и $f(x_{i+1})$ имеют разные знаки,
то это означает, что на интервале $x\in(x_i, x_{i+1})$ 
имеет по крайней мере один действительный корень уравнения $f(x)=0$.
\item
В качестве новой, более узкой $\left(|x_i, x_{i+1}|<|a,b|\right)$, 
области поиска выбирается отрезок $(x_i, x_{i+1})$, т.е. полагают
\begin{gather*}
x_{i}=a,\quad x_{i+1}=b
\end{gather*}
и с помощью аналогичной процедуры (1) процесс происка корня уравнения $f(x)=0$ 
повторяют до тех пор пока, область поиска не станет меньше заранее заданной величины
$\varepsilon$ (погрешности поиска корня уравнения):
\begin{gather*}
|a,b|<\varepsilon
\end{gather*}
\end{enumerate}

% *******************************
%	График функций
%
\begin{figure}[H]\centering
\begin{tikzpicture}
\begin{axis}
[
%axis on top,
xlabel={\empty},	% подпись оси x
ylabel={\empty},	% подпись оси y
xmin=-0.5, xmax=2.15, xtick={0,0.75,1.25,1.75}, xticklabels={a,$x_i$,$x_{i+1}$,b},
ymin=-2, ymax=3, ytick={0}, yticklabels={$f(x)=0$},
]
\addplot[gray,thin,domain=-0.5:2.15, samples=2] {0};
\addplot[name path=A,color=black,domain=-0.35:1.95,samples=100] {(x-1)*(x+1)*(x-2)};
\addplot[only marks,ball darkred] coordinates {(0,2) (0.25,1.640625) (0.5,1.125) (0.75,0.546875)};
\draw[darkred] (axis cs:0.72,0.47) node[left] {$f(x_i)>0$};
\addplot[only marks,ball darkblue] coordinates {(1.25,-0.421875) (1.5,-0.625) (1.75,-0.515625)};
\draw[darkblue] (axis cs:1.2,-0.5) node[left] {$f(x_{i+1})<0$};
%
\path [name path=B] (axis cs: 0,-2) -- (axis cs: 1.75,-2);
\addplot[orange!15] fill between [of=A and B, soft clip={domain=0:1.75}];
\addplot[blue!15] fill between [of=A and B, soft clip={domain=0.75:1.25},];
\end{axis}
\end{tikzpicture}
\caption{Иллюстрация итервального метода поиска корня уравнения}
%$[a,b]$ -- начальная область поиска корня уравнения $f(x)=0$;\\
%$[x_i,x_{i+1}]$ -- интервал, содержащий корень уравнения.
\end{figure}

\end{document}
%
%	Метод бисекции
%
\subsection{Метод бисекции}
Метод бисекции основан на теореме \textit{Больцано-Коши} (теорема о промежуточном значении):
``Если непрерывная функция $f(x)$, определённая на вещественном интервале $[a,b]$,
принимает два различных значения $f(a)\ne f(b)$, тогда существует такое $c\in[a,b]$,
что эта функция в этой точке принимает промежуточное значение $f(a)\le f(c)\le f(b)$``.

Следствие теоремы Больцано-Коши (теорема о нуле непрерывной функции).
``Если функция $f(x)$ непрерывна на некотором отрезке $[a,b]$ 
и на концах этого отрезка принимает значения $f(a)$ и $f(b)$ противоположных знаков,
то существует точка $x_0$, в которой значение функции равно нулю $f(x_0)=0$.

Если непрерывная функция строго монотонна на отрезке $[a,b]$,
т.е. для любого $x\in[a,b]$ выполняется условие 
$f^\prime(x)>0$ либо $f^\prime(x)<0$, то в соответствие со следствие теоремы
Больцано-Коши в пределах отрезка $[a,b]$ сущетсвует \textit{единственный}
корень уравнения $f(x)=0$.

\begin{tcolorbox}[width=\textwidth,colback={orange!10}, boxrule=0mm, arc=0mm]
Метод бисекции (деления пополам) является регулярным способом поиска 
действительного корня уравнения $f(x)=0$, однако для реализации этого метода
необходимо \textit{правильно выбрать область поиска}, т.е. начальный отрезок $[a,b]$ 
на концах которого функция $f(x)$ принимает значения разных знаков 
$f(a)\cdot f(b)<0$ и в пределах этого отрезка строго монотонна 
$f^{\prime}(x)<0$ либо $f^{\prime}(x)>0$.
\end{tcolorbox}

Алгоритм метода деления отрезка пополам (метод бисекции):
\begin{enumerate}[label=(\arabic*)]
\item
Область поиска корня уравнения отрезок $[a,b]$ делится пополам:
\begin{gather*}
c=\dfrac{a+b}{2}
\end{gather*}
\item
Вычисляется значение функции $f(x)$ в середине отрезка $f(c)$.
\item
Проводится сравнение знаков функции в середине отрезка $f(c)$
и, например, на левом конце отрезка $f(a)$:
\begin{enumerate}[leftmargin=0.75cm]
\item
если $f(a)\cdot f(c)<0$, функция $f(x)$ на концах отрезка $[a,c]$
принимает значения разных знаков, следовательно, искомый корень уравнения
$f(x)=0$ находится внутри отрезка $[a,c]$,
поэтому правый конец отрезка ``переносится`` в его середину.
\item
если $f(a)\cdot f(c)>0$, функция $f(x)$ на концах отрезка $[a,c]$
принимает значения одного знака, следовательно, искомый корень уравнения
$f(x)=0$ находится внутри отрезка $[c,b]$,
поэтому левый конец отрезка ``переносится`` в его середину.
\end{enumerate}
\begin{gather*}
\rm sign\left(f(a)\cdot f(c)\right)=\begin{cases}<0,&b=c\\>0,&a=c\end{cases}
\end{gather*}
Таким образом, область поиска корня уравнения $f(x)=0$ ``сужается наполовину``.
\item
Процесс вычислений (1)--(3) повторятся до тех пока, 
длина вновь полученного интервала $[a,b]$ станет меньше заранее заданного числа
$\varepsilon$ (погрешности поиска корня уравнения):
\begin{gather*}
|a,b|<\varepsilon
\end{gather*}
\end{enumerate}

В качестве корня уравнения $x_0$ приближенно принимаются середину
последнего полученного интервала $[a,b]$.

% *******************************
%	График функций
%
\begin{figure}
\begin{center}
\begin{tikzpicture}
\begin{axis}
[
	every axis/.style={color=black, solid, thick},
	xlabel = {\empty},		% подпись оси x
	ylabel = {\empty},	% подпись оси y
	xmin=-0.5, xmax=3.5, xtick={0,1.5,3}, xticklabels={a,c,b},
	ymin=-15, ymax=25, ytick={0}, yticklabels={$f(x)=0$},
	xtick style={thick, black},
	ytick style={thick, black},
%	grid=major,		
	major grid style={color=black!20, dashed, thin},
]
\path [name path=B] (axis cs: 0,-15) -- (axis cs: 3,-15);
\addplot[very thick, color=black,domain=-0.5:3.5, samples=2] {0};
\addplot[name path=A,very thick, color=black,domain=0:3, samples=100] {(x+2)*(x-2)*(x-5)};
\addplot[orange!20] fill between [of=A and B, soft clip={domain=1.5:3}];
% f(a)
\addplot[only marks, mark=*, mark size=4pt, mark options={fill=red!75, draw=black, solid}]
coordinates {(0,20)};
\draw[red] (axis cs:0,22) node[right] {$f(a)>0$};
\addplot[thin,dashed,color=black] coordinates {(0,-15) (0,20)};
% f(c)
\addplot[only marks, mark=*, mark size=4pt, mark options={fill=red, draw=black, solid}]
coordinates {(1.5,6.125)};
\draw[red] (axis cs:1.5,8) node[right] {$f(c)>0$};
\addplot[thin,dashed,color=black] coordinates {(1.5,-15) (1.5,6.125)};
% f(b)
\addplot[only marks, mark=*, mark size=4pt, mark options={fill=blue!75, draw=black, solid}]
coordinates {(3,-10)};
\draw[blue] (axis cs:3,-12) node[left] {$f(b)<0$};
\addplot[thin,dashed,color=black] coordinates {(3,-15) (3,-10)};
%\addplot [olive!10] fill between [of=A and B, soft clip={domain=0:1.5},];
\end{axis}
\end{tikzpicture}
\caption{Иллюстрация метода бисекции (деления отрезка пополам)}

$f(a)\cdot f(c)>0$, поэтому новая область поиска корня отрезок $[c,b]$
\end{center}
\end{figure}

\begin{tcolorbox}[width=\textwidth,colback={orange!10}, boxrule=0mm, arc=0mm]
Следует отметить, что если условие строгой монотонности для функции $f(x)$
на отрезке $[a,b]$ не выполняется и на отрезке имеется несколько корней,
то указанный процесс сойдется к одному из корней, 
но \textit{заранее неизвестно, к какому именно}.
\end{tcolorbox}

%
%	Метод выделения корней
%
\subsection{Метод выделения корней}
Один из недостатков интервального метода и метода бисекции 
является сходимость итерационного процесса к заранее неизвестному
корню уравнения $f(x)=0$. 
Этот недостаток можно устранить удалением уже найденного корня. 

Если $x_1$ простой корень уравнения $f(x)=0$ и функция $f(x)$ 
непрерывна по Липшицу, то вспомогательная функция
\begin{gather*}
g(x)=\dfrac{f(x)}{(x-x_1)}
\end{gather*}
непрерывна, причем все нули функций $f(x)$ и $g(x)$ совпадают, 
за исключением $x_1$, так как $g(x_1)\ne0$.

Поэтому найденный корень $x_1$ можно удалить, т.е. перейти 
в процессе поиска корня уравнения $f(x)=0$ от функции $f(x)$ к функции $g(x)$.
Тогда процесс нахождения остальных корней уравнения 
сведется к нахождению корней $g(x)=0$.

Когда найден какой-нибудь новый корень $x_2$ уравнения $g(x)=0$, 
то этот корень тоже можно удалить, вводя новую вспомогательную функцию:
\begin{gather*}
\varphi(x)=\dfrac{g(x)}{(x-x_2)}=\dfrac{f(x)}{(x-x_1)(x-x_2)}
\end{gather*}

Таким образом, можно последовательно найти все корни исходного уравнения $f(x)=0$.

В любом методе поиска корней уравнения $f(x)$ окончательные итерации вблизи 
определяемого корня рекомендуется делать не по функциям типа 
$g(x)$, а по исходной функции $f(х)$. 
Последние итерации, вычисленные по функции $g(x)$, используются при этом в качестве 
нулевого приближения.

%
%	Численное решение нелинейного уравнения методом бисекции
%
\subsection{Численное решение нелинейного уравнения методом бисекции}
На отрезке $x\in[-3,5]$ задана непрерывная функция:
\begin{gather*}
f(x)=\tanh(x)\cdot(1+\cos(x))-\dfrac{1}{2}
\end{gather*}

С помощью метода бисекции найдем первый положительный корень\\
$x_1>0$ нелинейного уравнения $f(x)=0$.
% *******************************
%	График функций
%
\begin{center}
\begin{tikzpicture}
\begin{axis}
[
	every axis/.style={color=black, solid, thick},
	xlabel = {$x$},		% подпись оси x
	ylabel = {$f(x)$},	% подпись оси y
	xmin=-4, xmax=6, xtick={-3,-1,0,1,3,5}, %xticklabels={$a$,$c$,$b$},
	ymin=-2, ymax=1, ytick={-2,-1.5,-1,-0.5,0,0.5,1}, yticklabels={-2,-1.5,-1,-0.5,0,0.5,1},	
	xtick style={thick, black},
	ytick style={thick, black},
	grid=major,		
	major grid style={color=black!20, dashed, thin},
]
\addplot[very thick, color=black,domain=-4:6, samples=2] {0};
\addplot[name path=A,very thick, color=red,domain=-3:5, samples=100] {tanh(x)*(1+cos(deg(x)))-0.5};
\path [name path=B] (axis cs: -4,-2) -- (axis cs: 6,-2);
\addplot [orange!20] fill between [of=A and B, soft clip={domain=1:3}];
%
%\addplot[only marks, mark=*, mark size=4pt, mark options={fill=red!75, draw=black, solid}]
%coordinates {(0,20)};
%\addplot[thin,dashed,color=black] coordinates {(0,-15) (0,20)};
%
%\addplot[only marks, mark=*, mark size=4pt, mark options={fill=blue!75, draw=black, solid}]
%coordinates {(3,-10)};
%\draw[blue] (axis cs:3,-12) node[left] {$f(b)<0$};
%\addplot[thin,dashed,color=black] coordinates {(3,-15) (3,-10)};
%
%\addplot[only marks, mark=*, mark size=4pt, mark options={fill=green, draw=black, solid}]
%coordinates {(1.5,6.125)};
%\draw[black] (axis cs:1.5,8) node[right] {$f(c)>0$};
%\addplot[thin,dashed,color=black] coordinates {(1.5,-15) (1.5,6.125)};
%
%\addplot [olive!10] fill between [of=A and B, soft clip={domain=0:1.5},];
\end{axis}
\end{tikzpicture}
\end{center}

На основе анализа графика функции $f(x)$ выбираем область поиска первого положительного корня
уравнения $(x>0)$, на границах которой функция $f(x)$ принимает значения разных знаков 
$f(a)\cdot f(b)<0$ и в пределах этого области строго монотонна 
$f^{\prime}(x)<0$ либо $f^{\prime}(x)>0$.

Таким требованиям удовлетворяет отрезок $x\in[1,3]$ (выдененная область на графике),
так как функция на отрезке монотонна $f^{\prime}(x)<0$ и на концах отрезка 
принимает значения разных знаков:
\begin{gather*}
\begin{array}{rcl}
f(1)&=&\tanh(1)\cdot(1+\cos(1))-\dfrac{1}{2}\approx0.67>0\\[1em]
f(3)&=&\tanh(3)\cdot(1+\cos(3))-\dfrac{1}{2}\approx-0.49<0
\end{array}
\end{gather*}

Зададим погрешность поиска корня уравнения $\varepsilon=0.1$ и
используем метод бисекции для поиска первого положительного корня уравнения:
\begin{enumerate}
\item
Область поиска корня уравнения отрезок $[1,3]$ делится пополам
и вычисляется значение функции в середине отрезка $f(c)$:
\begin{gather*}
c=\dfrac{1+3}{2}=2\\
f(c)=f(2)=\tanh(2)\cdot(1+\cos(2))-\dfrac{1}{2}\approx0.06285>0
\end{gather*}
Проводится сравнение знаков функции в середине отрезка $f(c)$
и на левом конце отрезка $f(a)$:
\begin{gather*}
f(a)\cdot f(c)=f(1)\cdot f(2)=0.67\cdot0.06285=0.0423>0
\end{gather*}
Следовательно, искомый корень уравнения находится внутри отрезка $[2,3]$.
Проведем сравнение длины отрезка и погрешности поиска корня:
\begin{gather*}
|a,b|=b-a=3-2=1>\varepsilon=0.1
\end{gather*}
Так как длина отрезка больше погрешности поиска, то итерационный процесс продолжаем.
% *******************************
%	График функций
%
\begin{center}
\begin{tikzpicture}
\begin{axis}
[
	every axis/.style={color=black, solid, thick},
	xlabel = {$x$},		% подпись оси x
	ylabel = {$f(x)$},	% подпись оси y
	xmin=0.5, xmax=3.5, xtick={1,2,3}, %xticklabels={$a$,$c$,$b$},
	ymin=-1, ymax=1, %ytick={0}, %yticklabels={$f(x)=0$},
	xtick style={thick, black},
	ytick style={thick, black},
	grid=major,		
	major grid style={color=black!20, dashed, thin},
]
\addplot[very thick, color=black,domain=0.5:3.5, samples=2] {0};
\addplot[name path=A,very thick, color=red,domain=1:3, samples=100] {tanh(x)*(1+cos(deg(x)))-0.5};
\path [name path=B] (axis cs: 1,-1) -- (axis cs: 3,-1);
\addplot [orange!20] fill between [of=A and B, soft clip={domain=2:3}];
% f(a)
\addplot[only marks, mark=*, mark size=4pt, mark options={fill=red!75, draw=black, solid}]
coordinates {(1,.6731)};
\draw[red] (axis cs:1,0.8) node[right] {$f(a)$};
% f(c)
\addplot[only marks, mark=*, mark size=4pt, mark options={fill=red!75, draw=black, solid}]
coordinates {(2,.628505523e-1)};
\draw[red] (axis cs:2.2,0.2) node {$f(c)$};
% f(b)
\addplot[only marks, mark=*, mark size=4pt, mark options={fill=blue!75, draw=black, solid}]
coordinates {(3,-.4900419862)};
\draw[blue] (axis cs:3,-0.3) node {$f(b)$};
\end{axis}
\end{tikzpicture}
\end{center}
% *****************************************
\item
Область поиска корня уравнения отрезок $[2,3]$ делится пополам
и вычисляется значение функции в середине отрезка $f(c)$:
\begin{gather*}
c=\dfrac{2+3}{2}=2.5\\
f(c)=f(2.5)=\tanh(2.5)\cdot(1+\cos(2.5))-\dfrac{1}{2}\approx-0.3038<0
\end{gather*}
Проводится сравнение знаков функции в середине отрезка $f(c)$
и на левом конце отрезка $f(a)$:
\begin{gather*}
f(a)\cdot f(c)=f(2)\cdot f(2.5)=0.6285\cdot(-0.3038)=-0.019094<0
\end{gather*}
Следовательно, искомый корень уравнения находится внутри отрезка $[2,2.5]$.
Проведем сравнение длины отрезка и погрешности поиска корня:
\begin{gather*}
|a,b|=b-a=2.5-2=0.5>\varepsilon=0.1
\end{gather*}
Так как длина отрезка больше погрешности поиска, то итерационный процесс продолжаем.
% *******************************
%	График функций
%
\begin{center}
\begin{tikzpicture}
\begin{axis}
[
	every axis/.style={color=black, solid, thick},
	xlabel = {$x$},		% подпись оси x
	ylabel = {$f(x)$},	% подпись оси y
	xmin=1.75, xmax=3.25, xtick={2,2.5,3}, %xticklabels={$a$,$c$,$b$},
	ymin=-0.6, ymax=0.3, %ytick={0}, %yticklabels={$f(x)=0$},
	xtick style={thick, black},
	ytick style={thick, black},
	grid=major,		
	major grid style={color=black!20, dashed, thin},
]
\addplot[very thick, color=black,domain=1.75:3.25, samples=2] {0};
\addplot[name path=A,very thick, color=red,domain=2:3, samples=100] {tanh(x)*(1+cos(deg(x)))-0.5};
\path [name path=B] (axis cs: 2,-0.6) -- (axis cs: 3,-0.6);
\addplot [orange!20] fill between [of=A and B, soft clip={domain=2:2.5}];
% f(a)
\addplot[only marks, mark=*, mark size=4pt, mark options={fill=red!75, draw=black, solid}]
coordinates {(2,.628505523e-1)};
\draw[red] (axis cs:2,0.15) node {$f(a)$};
% f(c)
\addplot[only marks, mark=*, mark size=4pt, mark options={fill=blue!75, draw=black, solid}]
coordinates {(2.5,-.3038054478)};
\draw[blue] (axis cs:2.6,-0.2) node {$f(c)$};
% f(b)
\addplot[only marks, mark=*, mark size=4pt, mark options={fill=blue!75, draw=black, solid}]
coordinates {(3,-.4900419862)};
\draw[blue] (axis cs:3,-0.4) node {$f(b)$};
\end{axis}
\end{tikzpicture}
\end{center}
% *****************************************
\item
Область поиска корня уравнения отрезок $[2,2.5]$ делится пополам
и вычисляется значение функции в середине отрезка $f(c)$:
\begin{gather*}
c=\dfrac{2+2.5}{2}=2.25\\
f(c)=f(2.25)=\tanh(2.25)\cdot(1+\cos(2.25))-\dfrac{1}{2}\approx-0.13634<0
\end{gather*}
Проводится сравнение знаков функции в середине отрезка $f(c)$
и на левом конце отрезка $f(a)$:
\begin{gather*}
f(a)\cdot f(c)=f(2)\cdot f(2.25)=0.6285\cdot(-0.13634)=-0.00857<0
\end{gather*}
Следовательно, искомый корень уравнения находится внутри отрезка $[2,2.25]$.
Проведем сравнение длины отрезка и погрешности поиска корня:
\begin{gather*}
|a,b|=b-a=2.25-2=0.25>\varepsilon=0.1
\end{gather*}
Длина отрезка меньше погрешности поиска, поэтому итерационный процесс продолжаем.

% *******************************
%	График функций
%
\begin{center}
\begin{tikzpicture}
\begin{axis}
[
	every axis/.style={color=black, solid, thick},
	xlabel = {$x$},		% подпись оси x
	ylabel = {$f(x)$},	% подпись оси y
	xmin=1.9, xmax=2.6, xtick={2,2.25,2.5}, %xticklabels={$a$,$c$,$b$},
	ymin=-0.4, ymax=0.2, %ytick={0}, %yticklabels={$f(x)=0$},
	xtick style={thick, black},
	ytick style={thick, black},
	grid=major,		
	major grid style={color=black!20, dashed, thin},
]
\addplot[very thick, color=black,domain=1.9:2.6, samples=2] {0};
\addplot[name path=A,very thick, color=red,domain=2:2.5, samples=100] {tanh(x)*(1+cos(deg(x)))-0.5};
\path [name path=B] (axis cs: 2,-0.4) -- (axis cs: 3,-0.4);
\addplot [orange!20] fill between [of=A and B, soft clip={domain=2:2.25}];
% f(a)
\addplot[only marks, mark=*, mark size=4pt, mark options={fill=red!75, draw=black, solid}]
coordinates {(2,.628505523e-1)};
\draw[red] (axis cs:2,0.1) node {$f(a)$};
% f(c)
\addplot[only marks, mark=*, mark size=4pt, mark options={fill=blue!75, draw=black, solid}]
coordinates {(2.25,-.1363440929)};
\draw[blue] (axis cs:2.3,-0.1) node {$f(c)$};
% f(b)
\addplot[only marks, mark=*, mark size=4pt, mark options={fill=blue!75, draw=black, solid}]
coordinates {(2.5,-.3038054478)};
\draw[blue] (axis cs:2.5,-0.25) node {$f(b)$};
\end{axis}
\end{tikzpicture}
\end{center}
% *****************************************
\item
Область поиска корня уравнения отрезок $[2,2.25]$ делится пополам
и вычисляется значение функции в середине отрезка $f(c)$:
\begin{gather*}
c=\dfrac{2+2.25}{2}=2.125\\
f(c)=f(2.125)=\tanh(2.125)\cdot(1+\cos(2.125))-\dfrac{1}{2}\approx-0.03959<0
\end{gather*}
Проводится сравнение знаков функции в середине отрезка $f(c)$
и на левом конце отрезка $f(a)$:
\begin{gather*}
f(a)\cdot f(c)=f(2)\cdot f(2.125)=0.6285\cdot(-0.03959)=-0.002488<0
\end{gather*}
Следовательно, искомый корень уравнения находится внутри отрезка $[2,2.125]$.
Проведем сравнение длины отрезка и погрешности поиска корня:
\begin{gather*}
|a,b|=b-a=2.125-2=0.125>\varepsilon=0.1
\end{gather*}
Длина отрезка меньше погрешности поиска, поэтому итерационный процесс продолжаем.

% *******************************
%	График функций
%
\begin{center}
\begin{tikzpicture}
\begin{axis}
[
	every axis/.style={color=black, solid, thick},
	xlabel = {$x$},		% подпись оси x
	ylabel = {$f(x)$},	% подпись оси y
	xmin=1.95, xmax=2.3, xtick={2,2.125,2.25}, %xticklabels={$a$,$c$,$b$},
	ymin=-0.2, ymax=0.2, %ytick={0}, %yticklabels={$f(x)=0$},
	xtick style={thick, black},
	ytick style={thick, black},
	grid=major,		
	major grid style={color=black!20, dashed, thin},
]
\addplot[very thick, color=black,domain=1.95:2.3, samples=2] {0};
\addplot[name path=A,very thick, color=red,domain=2:2.25, samples=100] {tanh(x)*(1+cos(deg(x)))-0.5};
\path [name path=B] (axis cs: 2,-0.2) -- (axis cs: 2.25,-0.2);
\addplot [orange!20] fill between [of=A and B, soft clip={domain=2:2.125}];
% f(a)
\addplot[only marks, mark=*, mark size=4pt, mark options={fill=red!75, draw=black, solid}]
coordinates {(2,.628505523e-1)};
\draw[red] (axis cs:2,0.1) node {$f(a)$};
% f(c)
\addplot[only marks, mark=*, mark size=4pt, mark options={fill=blue!75, draw=black, solid}]
coordinates {(2.125,-.395911619e-1)};
\draw[blue] (axis cs:2.155,-0.025) node {$f(c)$};
% f(b)
\addplot[only marks, mark=*, mark size=4pt, mark options={fill=blue!75, draw=black, solid}]
coordinates {(2.25,-.1363440929)};
\draw[blue] (axis cs:2.25,-0.1) node {$f(b)$};
\end{axis}
\end{tikzpicture}
\end{center}
% *****************************************
\item
Область поиска корня уравнения отрезок $[2,2.125]$ делится пополам
и вычисляется значение функции в середине отрезка $f(c)$:
\begin{gather*}
c=\dfrac{2+2.125}{2}=2.0625\\
f(c)=f(2.0625)=\tanh(2.0625)\cdot(1+\cos(2.0625))-\dfrac{1}{2}\approx0.01108>0
\end{gather*}
Проводится сравнение знаков функции в середине отрезка $f(c)$
и на левом конце отрезка $f(a)$:
\begin{gather*}
f(a)\cdot f(c)=f(2)\cdot f(2.0625)=0.6285\cdot0.01108=0.0007>0
\end{gather*}
Следовательно, искомый корень уравнения находится внутри отрезка $[2.0625,2.125]$.
Проведем сравнение длины отрезка и погрешности поиска корня:
\begin{gather*}
|a,b|=b-a=2.125-2.0625=0.0625<\varepsilon=0.1
\end{gather*}
Длина отрезка меньше погрешности поиска, поэтому итерационный процесс завершаем.

% *******************************
%	График функций
%
\begin{center}
\begin{tikzpicture}
\begin{axis}
[
	every axis/.style={color=black, solid, thick},
	xlabel = {$x$},		% подпись оси x
	ylabel = {$f(x)$},	% подпись оси y
	xmin=1.985, xmax=2.14, xtick={2,2.0625,2.125}, %xticklabels={$a$,$c$,$b$},
	ymin=-0.1, ymax=0.1, ytick={-0.1,-0.05,0,0.05,0.1}, yticklabels={-0.1,,0,,0.1},
	xtick style={thick, black},
	ytick style={thick, black},
	grid=major,		
	major grid style={color=black!20, dashed, thin},
]
\addplot[very thick, color=black,domain=1.95:2.14, samples=2] {0};
\addplot[name path=A,very thick, color=red,domain=2:2.125, samples=100] {tanh(x)*(1+cos(deg(x)))-0.5};
\path [name path=B] (axis cs: 2,-0.1) -- (axis cs: 2.125,-0.1);
\addplot [orange!20] fill between [of=A and B, soft clip={domain=2.0625:2.125}];
% f(a)
\addplot[only marks, mark=*, mark size=4pt, mark options={fill=red!75, draw=black, solid}]
coordinates {(2,.628505523e-1)};
\draw[red] (axis cs:2,0.08) node {$f(a)$};
% f(c)
\addplot[only marks, mark=*, mark size=4pt, mark options={fill=red!75, draw=black, solid}]
coordinates {(2.0625,.110785242e-1)};
\draw[red] (axis cs:2.07,0.025) node {$f(c)$};
% f(b)
\addplot[only marks, mark=*, mark size=4pt, mark options={fill=blue!75, draw=black, solid}]
coordinates {(2.125,-.395911619e-1)};
\draw[blue] (axis cs:2.125,-0.02) node {$f(b)$};
\end{axis}
\end{tikzpicture}
\end{center}
% *****************************************
\end{enumerate}

В качестве первого положительного корня $x_1$ уравнения $f(x)=0$ 
приближенно выберем середину последнего полученного интервала 
$[a,b]$ и для контроля определим значение функции 
в точке $f(x_1)$ приближенного корня уравнения:
\begin{gather*}
x_1=\dfrac{a+b}{2}=\dfrac{2.0625+2.125}{2}=2.09375\\
f(x_1)=f(2.09375)=\tanh(2.09375)\cdot(1+\cos(2.09375))-\dfrac{1}{2}\approx-0.01442
\end{gather*}

% *******************************
%	График функций
%
\begin{center}
\begin{tikzpicture}[background rectangle/.style={fill=olive!10}, show background rectangle]
\begin{axis}
[
	every axis/.style={color=black, solid, thick},
	xlabel = {$x$},		% подпись оси x
	ylabel = {$f(x)$},	% подпись оси y
	xmin=-4, xmax=6, xtick={-3,-1,0,1,2.09375,3,5}, xticklabels={-3,-1,0,1,$x_1$,3,5},
	ymin=-2, ymax=1, ytick={-2,-1.5,-1,-0.5,0,0.5,1}, yticklabels={-2,-1.5,-1,-0.5,0,0.5,1},
	xtick style={thick, black},
	ytick style={thick, black},
	grid=major,		
	major grid style={color=black!20, dashed, thin},
]
%\addplot[very thick, color=black,domain=-4:6, samples=2] {0};
\addplot[name path=A,very thick, color=red,domain=-3:5, samples=100] {tanh(x)*(1+cos(deg(x)))-0.5};
\addplot[only marks, mark=*, mark size=5pt, mark options={fill=lime, draw=black, solid, very thick}]
coordinates {(2.09375,-.144150229e-1)};
\draw[red] (axis cs:1.8,0.25) node[right] {$f(x_1)=0$};
\addplot[red, dashed, very thick] coordinates {(2.09375,-2) (2.09375,-.144150229e-1)};
\end{axis}
\end{tikzpicture}

График функции $f(x)$ (\textcolor{red}{сплошная линия}) и первый положительный 
корень $x_1=2.09375$ (маркер) уравнения $f(x)=0$
\end{center}

%\end{document}
