\newpage
\section{Интерполирование функций}
Задача интерполирования состоит в том, чтобы по 
известным значениям функции $f(x)$ в отдельных
точках отрезка восстановить её значения 
в остальных точках этого отрезка. Такая постановка 
задачи допускает множество решений.

Например, задача интерполирования возникает, 
в том случае, когда известны результаты измерения 
$y_i=f(x_i)$ некоторой физической величины $f$ 
в ограниченном количестве точек $x_i$ ($i=0,1,\dots,n$), 
а требуется оценить значения этой величины в других точках.

Интерполирование используется также, когда вычисление 
значений $f(x)$ трудоемко, например, значение искомой функции
может быть определено как решение сложной задачи, 
в которой $x$ играет роль параметра. При этом 
можно вычислить небольшую таблицу значений функции, 
но прямое нахождение функции при большом числе значений 
аргумента практически затруднительно или нецелесообразно. 

\emptyline
\subsection{Линейная интерполяция функции}
При \emph{линейной интерполяции} функция $f(x)$
на отрезке $x\in[a, b]$ заменяется обобщенным 
интерполяционным полиномом 
$p_n(x)$, который построен в виде линейной комбинации 
$(n+1)$ аналитических функций $\{\phi_i(x)\}$
\begin{gather}\label{eq:Interpolation_Polynom}
p_n(x)=c_0\cdot\phi_0(x)+c_1\cdot\phi_1(x)+\ldots+c_n\cdot\phi_n(x)
=\sum\limits_{i=0}^n c_i\cdot\phi_i(x),
\end{gather}
таким образом, чтобы значения полинома $p_n(x)$ 
в определённых точках отрезка $\{x_0,x_1,\dots,x_n\}$ 
(узлах сетки) совпадают со значениями функции 
в этих точках $\{y_0,y_1,\dots,y_n\}$ (условия сопряжения):
\begin{gather}\label{eq:Interpolation Conjugation}
\left\{\begin{matrix}
p_n(x_0)&=&y_0\\
p_n(x_1)&=&y_1\\
\hdotsfor{3}\\
p_n(x_n)&=&y_n\\
\end{matrix}\right.,\quad\iff\quad
% подробно
\left\{\begin{matrix}
\sum\limits_{i=0}^n c_i\cdot\phi_i(x_0)&=&y_0\\[1em]
\sum\limits_{i=0}^n c_i\cdot\phi_i(x_1)&=&y_1\\
\hdotsfor{3}\\
\sum\limits_{i=0}^n c_i\cdot\phi_i(x_n)&=&y_n\\
\end{matrix}\right..
\end{gather}

Из условий \eqref{eq:Interpolation Conjugation}, 
накладываемых на интерполяционный полином,
формулируется система линейных уравнений относительно 
неизвестных коэффициентов полинома ${c_0,c_1,\dots,c_n}$:
\begin{gather}\label{eq:Interpolation_LS}
\mathbf{A}\cdot\vect{c}=\vect{y},
\end{gather}
где $\mathbf{A}$ -- квадратная матрица $(n+1)\times(n+1)$,
$\vect{c}$ и $\vect{y}$ -- 
вектор неизвестных коэффициентов полинома $p_n(x)$
и вектор значений функции $f(x)$ в заданных точках $\{x_i\}$:
\begin{gather*}
\mathbf{A}=
\begin{pmatrix}
\phi_0(x_0)&\phi_1(x_0)&\cdots&\phi_n(x_0)\\
\phi_0(x_1)&\phi_1(x_1)&\cdots&\phi_n(x_1)\\
\vdots&\vdots&\ddots&\vdots\\
\phi_0(x_n)&\phi_1(x_n)&\cdots&\phi_n(x_n)\\
\end{pmatrix},
\quad
\vect{c}=\begin{pmatrix}c_0\\c_1\\\vdots\\c_n\end{pmatrix},
\quad
\vect{y}=\begin{pmatrix}y_0\\y_1\\\vdots\\y_n\end{pmatrix}.
\end{gather*}

Если среди узлов интерполяции $\{x_i\}$ нет совпадающих
($x_i\ne x_j$ для всех $i,j=0,1,\dots,n$) и 
определитель системы отличен от нуля $\det\mathrm{A}\ne0$,
то задача интерполяции имеет единственное решение, а
система функций $\{\phi_i(x)\}$ называется чебышевской. 
Поэтому при линейной интерполяции необходимо 
строить обобщенный полином $p_n(x)$ на основе 
\emph{чебышевской системы функций}.

Таким образом, для определения коэффициентов 
интерполяционного полинома 
\eqref{eq:Interpolation_Polynom} необходимо 
найти решение системы линейных уравнений
\eqref{eq:Interpolation_LS}, 
любыми аналитическими, приближенными
или численными методами, например, методом Гаусса.

Интерполирование не всегда дает удовлетворительное решение 
задачи о приближении функции \emph{с заданной точностью}
на данном промежутке, так как совпадение функции $f(x)$
с полиномом $p(x)$ в точках $x_i$ и $x_{i+1}$
не гарантирует малость величины
$\abs{f(x)-p(x)}$ на отрезке $[x_i,x_{i+1}]$.

%
% Интерполирование алгебраическими полиномами
%
\emptyline
\subsection{Интерполяция алгебраическими полиномами}
Задача интерполяции алгебраическими полиномами 
сводится к построению полинома степени $n$ по
чебышевской системе алгебраических функций
$\{1,x,x^2,\dots,x^n\}$:
\begin{equation}\label{eq:Interpolation_AP}
p_{n}(x)
=c_{0}+c_{1}\cdot{x}+c_{2}\cdot{x^2}+\ldots+c_{n}\cdot{x^n}
=\sum\limits_{i=0}^{n}c_i\cdot x^i.
\end{equation}

Определитель системы \eqref{eq:Interpolation_LS}
представляет собой определителем Вандермонда, 
который  отличен от нуля $\det\mathbf{A}\ne0$ , 
если среди точек $\{x_i\}$ нет совпадающих,
т.е. $x_i\ne x_j$ для всех $i,j=0,1,\dots,n$:
\begin{gather*}
\det\mathbf{A}=
\begin{vmatrix}
1&x_0&\cdots&x_0^n\\
1&x_1&\cdots&x_1^n\\
\vdots&\vdots&\ddots&\vdots\\
1&x_n&\cdots&x_n^n\\
\end{vmatrix}.
\end{gather*}

Выражение для коэффициентов алгебраического полинома
и вид самого полинома \eqref{eq:Interpolation_AP} 
можно записать различными способами.
Наиболее распространенная запись интерполяционного 
многочлена в форме Лагранжа и в форме Ньютона. 

%
%	Интерполяция функций полиномами Лагранжа
%
\emptyline
\subsection{Интерполяционный полином в форме Лагранжа}
Интерполяционная формула Лагранжа позволяет 
представить многочлен $L_{n}(x)$ в виде 
линейной комбинации значений функции $y(x)$ 
в узлах интерполирования $\{x_i\}$:
\begin{equation}\label{eq:Lagrange}
L_{n}(x)
=\lambda_{0}(x)\cdot{y_0}+\lambda_{1}(x)\cdot{y_1}+\dots+\lambda_{n}(x)\cdot{y_n}
=\sum\limits_{i=0}^n \lambda_i(x)\cdot y_i
\end{equation}
где $\lambda_0(x),\lambda_1(x),\dots,\lambda_n(x)$ -- произвольные неизвестные функции.

Для определения неизвестных функций $\lambda_i(x)$ 
из условий интерполирования следует:
\begin{equation*}\label{inter2}
\left\{\begin{matrix}
\lambda_0(x_0)\cdot{y_0}+\lambda_1(x_0)\cdot{y_1}+\dots+\lambda_n(x_0)\cdot{y_n}&=&y_0\\
\lambda_0(x_1)\cdot{y_0}+\lambda_1(x_1)\cdot{y_1}+\dots+\lambda_n(x_1)\cdot{y_n}&=&y_1\\
\hdotsfor{3}\\
\lambda_0(x_n)\cdot{y_0}+\lambda_1(x_n)\cdot{y_1}+\dots+\lambda_n(x_n)\cdot{y_n}&=&y_n\\
\end{matrix}\right.
\end{equation*}

Эта система уравнений имеет решение если выполняются условия:
\begin{equation*}
\label{uslovia_c}
\lambda_{i}(x_j)=\left\{\begin{matrix}
1, &x_j=x_{i}\\
0, &x_j\ne{x_{i}}
\end{matrix}\right.
\end{equation*}

Коэффициенты $\lambda_{i}(x)$ можно искать в виде 
многочленов степени $n$:
\begin{equation*}
\label{eq_c}
\left\{\begin{matrix}
\lambda_0(x)&=&\alpha_0\cdot(x-x_1)\cdot(x-x_2)
\cdot(x-x_2)\cdot&\ldots&\cdot(x-x_n)\\
\lambda_1(x)&=&\alpha_1\cdot(x-x_0)\cdot(x-x_2)
\cdot(x-x_3)\cdot&\ldots&\cdot(x-x_n)\\
\hdotsfor{5}\\
\lambda_n(x)&=&\alpha_n\cdot(x-x_0)\cdot(x-x_1)
\cdot(x-x_2)\cdot&\ldots&\cdot(x-x_{n-1})
\end{matrix}\right.
\end{equation*}

Определим неизвестные 
$\alpha_0, \alpha_1, \ldots, \alpha_n$ 
из условия для коэффициентов $\lambda_i(x)$:
\begin{equation*}
\left\{
\begin{matrix}
1&=&\alpha_0\cdot(x_0-x_1)\cdot(x_0-x_2)\cdot(x_0-x_2)\cdot&
\dots&\cdot(x_0-x_n)\\
1&=&\alpha_1\cdot(x_1-x_0)\cdot(x_1-x_2)\cdot(x_1-x_3)\cdot&
\dots&\cdot(x_1-x_n)\\
\hdotsfor{5}\\
1&=&\alpha_n\cdot(x_n-x_0)\cdot(x_n-x_1)\cdot(x_n-x_2)\cdot&
\dots&\cdot(x_n-x_{n-1})
\end{matrix}
\right.
\end{equation*}

Таким образом, коэффициенты $\lambda_{i}(x)$ 
интерполяционного многочлена\linebreak
Лагранжа находятся из соотношений:
\begin{equation*}
\left\{\begin{matrix}
\lambda_0(x)&=&\dfrac
{(x-x_1)\cdot(x-x_2)\cdot\dots\cdot(x-x_n)}
{(x_0-x_1)\cdot(x_0-x_2)\cdot\dots\cdot(x_0-x_n)}\\[1em]
\lambda_1(x)&=&\dfrac
{(x-x_0)\cdot(x-x_2)\cdot\dots\cdot(x-x_n)}
{(x_1-x_0)\cdot(x_1-x_2)\cdot\dots\cdot(x_1-x_n)}\\[1em]
\hdotsfor{3}\\[1em]
\lambda_n(x)&=&\dfrac
{(x-x_0)\cdot(x-x_1)\cdot\dots\cdot(x-x_{n-1})}
{(x_n-x_0)\cdot(x_n-x_1)\cdot\dots\cdot(x_n-x_{n-1})}
\end{matrix}\right.,
\end{equation*}
или в более компактной форме:
\begin{equation*}
\lambda_i(x)=\dfrac
{\prod\limits_{j \ne i}^n (x - x_j)}
{\prod\limits_{j \ne i}^n (x_i - x_j)}
\end{equation*}

Итак, интерполяционный многочлен Лагранжа 
\eqref{eq:Lagrange} имеет вид:
\begin{equation}\label{eq:Lagrange_Polynom}
L_{n}(x)=\sum\limits_{i=0}^n\dfrac
{\prod\limits_{j \ne i}^n(x-x_j) }
{\prod\limits_{j \ne i}^n(x_i-x_j)}\cdot{y_i}
\end{equation}

%
% Интерполирование таблично заданной функции
%
\emptyline
\subsection{Интерполяция функции заданной таблично}
Известно множество данных (узлов интерполяции)
$\{x_i\}$, в которых определены 
значения функции $y_i=f(x_i)$:
\begin{table}[h]
\vspace{-0.5\baselineskip}
\caption{Таблично заданная функциональная зависимость}
\label{tab:Interpolation_Data}
\begin{tabular*}{\textwidth}{%
l@{\extracolsep{\fill}}*{4}{r}p{0.25cm}}
\toprule
$i$&$0$&$1$&$2$&$3$\\
\midmidrule
$x_i$&$-0.76$&$-0.09$&$0.22$&$0.55$\\
\addlinespace% дополнительный пробел
$y_i$&$0.08$&$1.84$&$0.40$&$0.96$\\
\bottomrule
\end{tabular*}
\end{table}

Построим обобщенный интерполяционный полином $p_3(x)$
для таблично заданной функции исходя из чебышевской 
системы функций $\{1,x,e^{-x},e^x\}$:
\begin{gather*}
p_4(x) = c_0+c_1\cdot{x}+c_2\cdot e^{-x}+c_3\cdot e^x
\end{gather*}

\begin{enumerate}
\item
Составим расширенную матрица системы уравнений 
\eqref{eq:Interpolation_LS} для определения 
коэффициентов полинома $(c_0,c_1,c_2,c_3)^\mathrm{T}$:
\begin{gather*}
\begin{pmatrix}[cccc|c]
\phi_0(x_0)&\phi_1(x_0)&\phi_2(x_0)&\phi_3(x_0)&y_0\\
\phi_0(x_1)&\phi_1(x_1)&\phi_2(x_1)&\phi_3(x_1)&y_1\\
\phi_0(x_2)&\phi_1(x_2)&\phi_2(x_2)&\phi_3(x_2)&y_2\\
\phi_0(x_3)&\phi_1(x_3)&\phi_2(x_3)&\phi_3(x_3)&y_3\\
\end{pmatrix},\quad\text{здесь}\quad
\left\{\begin{matrix}
\phi_0(x)&=&1\\
\phi_1(x)&=&x\\
\phi_2(x)&=&e^{-x}\\
\phi_3(x)&=&e^x\\
\end{matrix}\right..
\end{gather*}
\item
Воспользуемся данными таблицы \ref{tab:Interpolation_Data}
и заполним числовыми значения элементы расширенной матрицы:
\begin{gather*}
\begin{pmatrix}[cccc|c]
1&-0.76&e^{0.76}&e^{-0.76}&0.08\\
1&-0.09&e^{0.09}&e^{-0.09}&1.84\\
1&0.22&e^{-0.22}&e^{0.22}&0.40\\
1&0.55&e^{-0.55}&e^{0.55}&0.96\\
\end{pmatrix}
\;\iff\;
\begin{pmatrix}[cccc|c]
1&-0.76&\fcolorbox{gray!50}{yellow}{2.138}&0.468&0.08\\
1&-0.09&1.094&0.914&1.84\\
1&0.22&0.803&1.246&0.40\\
1&0.55&0.577&1.733&0.96\\
\end{pmatrix}
\end{gather*}

\item
Решение системы линейных уравнений
найдем методом Гаусса с выбором главного элемента
в расширенной матрице (выделен цветом):
\begin{gather*}
\vect{c}=(-0.393,-81.472,-37.288,39.053)^\mathrm{T}
\end{gather*}

Следовательно, обобщенный интерполяционный полином
для функции заданной таблично можно записать в виде:
\begin{gather*}
\textcolor{darkblue}{
p_3(x) = -0.393-81.472\cdot{x}-37.288\cdot e^{-x}+39.053\cdot e^x
}
\end{gather*}

\item
В таблице \ref{tab:Interpolation_GPD} представлены 
данные расчета коэффициентов обобщенного интерполяционного 
полинома $c_i$, значений этого полинома в узлах сетки $p_3(x_i)$
и абсолютная погрешность интерполяции 
$\varepsilon_i=y_i-p_3(x_i)$.
%
% Таблица результатов
%
\vspace{-0.5\baselineskip}
\begin{table}[H]
\caption{Коэффициенты обобщенного интерполяционного 
полинома $c_i$, значения этого полинома в узлах сетки $p_3(x_i)$
и абсолютная погрешность интерполяции $\varepsilon_i$}
\label{tab:Interpolation_GPD}
\begin{tabular*}{\textwidth}{%
@{\extracolsep{\fill}}*{5}{r}p{2cm}}
\toprule
$i$&$0$&$1$&$2$&$3$\\
\midmidrule% @x,@y
$x_i$&$-0.76$&$-0.09$&$0.22$&$0.55$\\
$y_i$&$0.08$&$1.84$&$0.4$&$0.96$\\
\midrule% коэффициенты полинома
$c_i$&$-0.393$&$-81.472$&$-37.288$&$39.053$\\
\midrule% значение полинома и погрешность
$p_3(x_i)$&$0.057$&$1.832$&$0.422$&$0.973$\\
$\varepsilon_i$&$0.023$&$0.008$&$-0.022$&$-0.013$\\
\bottomrule
\end{tabular*}
\end{table}

\item
На рисунке \ref{fig:Interpolation_GP} представлена 
диаграмма рассеяния (разброса) данных 
функции заданной таблично $y_i=f(x_i)$ (маркеры) и 
результаты вычислений обобщенного интерполяционного 
полинома $p_3(x)$ (сплошная линия).
\begin{figure}[H]\centering
\begin{tikzpicture}
\begin{axis}[ylabel=$p_3(x)$,
xmin=-0.9,xmax=0.7,xtick={-0.8,-0.4,0,0.4},
ymin=-1,ymax=3]
% табличные данные
\addplot[ball darkblue,only marks] coordinates {(-0.76,0.08)(-0.09,1.84)(0.22,0.4)(0.55,0.96)};
% интерполяционный полином
\addplot[darkblue,domain=-0.8:0.6,samples=50]
{-0.393*1-81.472*x-37.288*exp(-x)+39.053*exp(x)}
node[pos=0.7,right] {$p_3(x)$};
\end{axis}
\end{tikzpicture}
\caption{График таблично заданной функции $y_i=f(x_i)$ (маркеры) 
и обобщенного интерполяционного полинома $p_3(x)$
(сплошная линия)}
\label{fig:Interpolation_GP}
\end{figure}
\end{enumerate}

%
% Таблица графика
%
% xmin = -0.8
% xmax = 0.6
% dx   = 0.028
%
\begin{table}[H]
\caption{Рассчётные значения обобщенного интерполяционного 
полинома в узлах сетки $p_3(x_i)$}
\label{tab:Interpolation Plot}
\small
\begin{tabular*}{\textwidth}{%
p{1cm}@{\extracolsep{\fill}}*{9}{r}}
\toprule
$i$&$0$&$1$&$2$&$\dots$&$\dots$&$\dots$&$47$&$48$&$49$\\
\midmidrule
$x_i$&$-0.800$&$-0.771$&$-0.743$&$\dots$&$\dots$&$\dots$&$0.543$&$0.571$&$0.600$\\
$y_i$&$-0.654$&$-0.128$&$0.330$&$\dots$&$\dots$&$\dots$&$0.920$&$1.145$&$1.419$\\
\bottomrule
\end{tabular*}
\end{table}


Построим интерполяционный полином 
в форме Лагранжа $L_3(x)$ на основе данных 
об узлах интерполяции $\{x_i\}$ 
и значений функции в этих узлах $\{y_i\}$:
\begin{gather*}
L_{3}(x)=\sum \limits_{i=0}^3\dfrac
{\prod\limits_{j \ne i}^3(x-x_j)}
{\prod\limits_{j \ne i}^3(x_i-x_j)}\cdot{y_i}
\end{gather*}

\begin{enumerate}
\item
Представим полином Лагранжа в развернутом виде:
\begin{gather*}
\begin{split}
L_{3}(x)=
&\;\dfrac{(x-x_1)\cdot(x-x_2)\cdot(x-x_3)}{(x_0-x_1)\cdot(x_0-x_2)\cdot(x_0-x_3)}\cdot y_0+\\[1ex]
&\;\dfrac{(x-x_0)\cdot(x-x_2)\cdot(x-x_3)}{(x_1-x_0)\cdot(x_1-x_2)\cdot(x_1-x_3)}\cdot y_1+\\[1ex]
&\;\dfrac{(x-x_0)\cdot(x-x_1)\cdot(x-x_3)}{(x_2-x_0)\cdot(x_2-x_1)\cdot(x_2-x_3)}\cdot y_2+\\[1ex]
&\;\dfrac{(x-x_0)\cdot(x-x_1)\cdot(x-x_2)}{(x_3-x_0)\cdot(x_3-x_1)\cdot(x_3-x_2)}\cdot y_3
\end{split}
\end{gather*}

\item
Воспользуемся численными данными об узлах интерполяции 
$\{x_i\}$ и известными значениями интерпретируемой функции 
в этих узлах $\{y_i\}$:
\begin{gather*}
\begin{split}
L_{3}(x)=
&\;\dfrac{(x-(-0.09))\cdot(x-0.22)\cdot(x-0.55)}{(-0.76-(-0.09))\cdot(-0.76-0.22)\cdot(-0.76-0.55)}\cdot0.08+\\[1ex]
&\;\dfrac{(x-(-0.76))\cdot(x-0.22)\cdot(x-0.55)}{(-0.09-(-0.76))\cdot(-0.09-0.22)\cdot(-0.09-0.55)}\cdot1.84+\\[1ex]
&\;\dfrac{(x-(-0.76))\cdot(x-(-0.09))\cdot(x-0.55)}{(0.22-(-0.76))\cdot(0.22-(-0.09))\cdot(0.22-0.55)}\cdot0.40+\\[1ex]
&\;\dfrac{(x-(-0.76))\cdot(x-(-0.09))\cdot(x-0.22)}{(0.55-(-0.76))\cdot(0.55-(-0.09))\cdot(0.55-0.22)}\cdot0.96
\end{split}
\end{gather*}

\item
Проведем необходимые арифметические действия:
\begin{gather*}
\begin{split}
L_{3}(x)=
&\;\dfrac{(x+0.09)\cdot(x-0.22)\cdot(x-0.55)}{(-0.67)\cdot(-0.98)\cdot(-1.31)}\cdot0.08+\\[1ex]
&\;\dfrac{(x+0.76)\cdot(x-0.22)\cdot(x-0.55)}{(0.67)\cdot(-0.31)\cdot(-0.64)}\cdot1.84+\\[1ex]
&\;\dfrac{(x+0.76)\cdot(x+0.09)\cdot(x-0.55)}{(0.98)\cdot(0.31)\cdot(-0.33)}\cdot0.40+\\[1ex]
&\;\dfrac{(x+0.76)\cdot(x+0.09)\cdot(x-0.22)}{(1.31)\cdot(0.64)\cdot(0.33)}\cdot0.96
\end{split}
\end{gather*}

или
\begin{gather*}
\begin{matrix}
L_{3}(x)=
&\;\dfrac{(x+0.09)\cdot(x-0.22)\cdot(x-0.55)}{-0.86}\cdot0.08+\\[1ex]
&\;\dfrac{(x+0.76)\cdot(x-0.22)\cdot(x-0.55)}{0.13}\cdot1.84+\\[1ex]
&\;\dfrac{(x+0.76)\cdot(x+0.09)\cdot(x-0.55)}{-0.10}\cdot0.40+\\[1ex]
&\;\dfrac{(x+0.76)\cdot(x+0.09)\cdot(x-0.22)}{0.28}\cdot0.96&
\end{matrix}
\end{gather*}

Продолжая делать упрощения окончательно получим:
\begin{gather*}
\begin{split}
L_{3}(x)=
&\;(x+0.09)\cdot(x-0.22)\cdot(x-0.55)\cdot(-0.09)+\\
&\;(x+0.76)\cdot(x-0.22)\cdot(x-0.55)\cdot13.84+\\
&\;(x+0.76)\cdot(x+0.09)\cdot(x-0.55)\cdot(-3.99)+\\
&\;(x+0.76)\cdot(x+0.09)\cdot(x-0.22)\cdot3.47
\end{split}
\end{gather*}

\item
Запишем выражение для интерполяционный полином Лагранжа
в каноническом виде:
\begin{gather*}\textcolor{darkred}{
L_{3}(x)=1.37 - 5.248\cdot x + 0.912\cdot x^2 + 13.23\cdot x^3
}
\end{gather*}

\item
В таблице \ref{tab:Interpolation_LPD} представлены 
данные расчета коэффициентов интерполяционного 
полинома Лагранжа $c_i$, значений этого полинома 
в узлах сетки $L_3(x_i)$ и абсолютная погрешность интерполяции 
$\varepsilon_i=y_i-L_3(x_i)$.
%
% Таблица
%
\begin{table}[H]
\caption{Коэффициенты интерполяционного 
полинома Лагранжа $c_i$, значения этого полинома 
в узлах сетки $L_3(x_i)$ и абсолютная погрешность 
интерполяции $\varepsilon_i$}
\label{tab:Interpolation_LPD}
\begin{tabular*}{\textwidth}{%
@{\extracolsep{\fill}}*{6}{r}}
\toprule
$i$&$x_i$&$y_i$&$c_i$&$L_3(x_i)$&$\varepsilon_i$\\
\midmidrule
0&$-0.76$&$0.08$&$1.37$&$0.078$&$0.002$\\
1&$-0.09$&$1.84$&$-5.248$&$1.840$&$0.000$\\
2&$0.22$&$0.40$&$0.912$&$0.400$&$0.000$\\
3&$0.55$&$0.96$&$13.23$&$0.961$&$-0.001$\\
\bottomrule
\end{tabular*}
\end{table}

\item
На рисунке \eqref{fig:Interpolation_LP} представлена 
диаграмму рассеяния (разброса) данных функции 
заданной таблично $y_i=f(x_i)$ (маркеры) и 
результаты вычислений интерполяционного 
полинома Лагранжа $L_3(x)$ (сплошная линия).
% *******************************
%	График функций
%
\begin{figure}[H]\centering
\begin{tikzpicture}
\begin{axis}[% оси координат
ylabel=$L_3(x)$,
xmin=-0.9,xmax=0.7,xtick={-0.8,-0.4,0,0.4},
ymin=-1, ymax=3,
]
% табличные данные
\addplot[ball darkred,only marks]
coordinates {(-0.76,0.08) (-0.09,1.84) (0.22,0.40) (0.55,0.96)};
% полином Лагранжа
\addplot[darkred,mark=none,domain=-0.8:0.6,samples=100] 
{1.37 - 5.248*x + 0.912*x^2 + 13.23*x^3}
node[pos=0.7,right] {$L_3(x)$};
\end{axis}
\end{tikzpicture}
\caption{График таблично заданной функции $y_i=f(x_i)$ (маркеры) 
и интерполяционного полинома Лагранжа $L_3(x)$
(сплошная линия)}
\label{fig:Interpolation_LP}
\end{figure}
% *******************************
\end{enumerate}
