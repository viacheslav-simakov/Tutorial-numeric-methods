\newpage
\section{Интерполирование функций}
Задача интерполирования состоит в том, чтобы по 
известным значениям функции $f(x)$ в отдельных
точках отрезка восстановить её значения 
в остальных точках этого отрезка. Такая постановка 
задачи допускает множество решений.

Например, задача интерполирования возникает, 
в том случае, когда известны результаты измерения 
$y_i=f(x_i)$ некоторой физической величины $f$ 
в ограниченном количестве точек $x_i$ ($i=0,1,\dots,n$), 
а требуется оценить значения этой величины в других точках.

Интерполирование используется также, когда вычисление 
значений $f(x)$ трудоемко, например, значение искомой функции
может быть определено как решение сложной задачи, 
в которой $x$ играет роль параметра. При этом 
можно вычислить небольшую таблицу значений функции, 
но прямое нахождение функции при большом числе значений 
аргумента практически затруднительно или нецелесообразно. 

Пусть на отрезке $x\in[a, b]$ выбрана сетка
$\{x_i\}$ (здесь $i=0,1,\dots,n$), в узлах которой 
известны значения функции $y_{i}=f(x_{i})$.
Произвольно выберем множество известных 
линейно независимых функций $\phi_i(x)$, 
вычисление которых не встречает серьезных трудностей.


Задача интерполирования алгебраическими полиномами 
состоит в том, чтобы построить полином степени $n$
\begin{equation}\label{eq:Interpolation_Polynom}
p_{n}(x)
=c_{0}+c_{1}\cdot{x}+c_{2}\cdot{x^2}+\ldots+c_{n}\cdot{x^n}
=\sum\limits_{i=0}^{n}c_i\cdot x^i
\end{equation}
значения которого в заданных точках $\{x_i\}$, 
совпадают со значениями функции $\{y_i\}$ в этих точках.

Т.е. полином $p_{n}(x)$ должен удовлетворять условиям сопряжения:
\begin{equation}\label{eq:Interpolation}
\left\{\begin{matrix}
p_{n}(x_0)&=&y_0\\
p_{n}(x_1)&=&y_1\\
\hdotsfor{3}\\
p_{n}(x_n)&=&y_n\\
\end{matrix}\right.
\quad\to\quad
\left\{\begin{matrix}
\sum\limits_{i=0}^{n}c_i\cdot x_0^i&=&y_0\\[1em]
\sum\limits_{i=0}^{n}c_i\cdot x_1^i&=&y_1\\
\hdotsfor{3}\\
\sum\limits_{i=0}^{n}c_i\cdot x_n^i&=&y_n\\
\end{matrix}\right.
\end{equation}

На основе условий \eqref{eq:Interpolation} формулируется
система линейных уравнений относительно неизвестных
коэффициентов полинома ${c_0,c_1,\dots,c_n}$:
\begin{gather}\label{eq:Interpolation_LS}
\mathbf{A}\cdot\vect{c}=\vect{y},
\end{gather}
где $\mathbf{A}$ -- квадратная матрица $(n+1)\times(n+1)$,
$\vect{c}$ и $\vect{y}$ -- 
вектор неизвестных коэффициентов полинома $p_n(x)$
и вектор значений функции $f(x)$ в заданных точках $\{x_i\}$:

\begin{gather*}
\mathbf{A}=
\begin{pmatrix}
1&x_0&\cdots&x_0^n\\
1&x_1&\cdots&x_1^n\\
\vdots&\vdots&\ddots&\vdots\\
1&x_n&\cdots&x_n^n\\
\end{pmatrix},
\quad
\vect{c}=\begin{pmatrix}c_0\\c_1\\\vdots\\c_n\end{pmatrix},
\quad
\vect{y}=\begin{pmatrix}y_0\\y_1\\\vdots\\y_n\end{pmatrix}.
\end{gather*}

Определитель системы \eqref{eq:Interpolation_LS}
представляет собой определителем Вандермонда, 
который  отличен от нуля $\det\mathbf{A}\ne0$ , 
если среди точек $\{x_i\}$ нет совпадающих,
т.е. $x_i\ne x_j$ для всех $i,j=0,1,\dots,n$.

Таким образом, для определения коэффициентов 
интерполяционного полинома 
\eqref{eq:Interpolation_Polynom} необходимо 
найти решение системы линейных уравнений
\eqref{eq:Interpolation_LS}, 
любыми аналитическими, приближенными
или численными методами, например, методом Гаусса.

Интерполирование не всегда дает удовлетворительное решение 
задачи о приближении функции \emph{с заданной точностью}
на данном промежутке, так как совпадение функции $f(x)$
с полиномом $p(x)$ в точках $x_i$ и $x_{i+1}$
не гарантирует малость величины
$\abs{f(x)-p(x)}$ на отрезке $[x_i,x_{i+1}]$.
%
%	Интерполяция функций полиномами Лагранжа
%
\emptyline
\subsection{Интерполяция функций полиномами Лагранжа}
Решение системы \eqref{eq:Interpolation_LS} 
можно записать различным образом. 
Наиболее распространенная запись интерполяционного 
многочлена в форме Лагранжа и в форме Ньютона. 
Интерполяционная формула Лагранжа позволяет 
представить многочлен $L_{n}(x)$ в виде 
линейной комбинации значений функции $y(x)$ 
в узлах интерполирования $\{x_i\}$:
\begin{equation}\label{eq:Lagrange}
L_{n}(x)
=\lambda_{0}(x)\cdot{y_0}+\lambda_{1}(x)\cdot{y_1}+\dots+\lambda_{n}(x)\cdot{y_n}
=\sum\limits_{i=0}^n \lambda_i(x)\cdot y_i
\end{equation}
где $\lambda_0(x),\lambda_1(x),\dots,\lambda_n(x)$ -- произвольные неизвестные функции.

Для определения неизвестных функций $\lambda_i(x)$ 
из условий интерполирования следует:
\begin{equation*}\label{inter2}
\left\{\begin{matrix}
\lambda_0(x_0)\cdot{y_0}+\lambda_1(x_0)\cdot{y_1}+\dots+\lambda_n(x_0)\cdot{y_n}&=&y_0\\
\lambda_0(x_1)\cdot{y_0}+\lambda_1(x_1)\cdot{y_1}+\dots+\lambda_n(x_1)\cdot{y_n}&=&y_1\\
\hdotsfor{3}\\
\lambda_0(x_n)\cdot{y_0}+\lambda_1(x_n)\cdot{y_1}+\dots+\lambda_n(x_n)\cdot{y_n}&=&y_n\\
\end{matrix}\right.
\end{equation*}

Эта система уравнений имеет решение если выполняются условия:
\begin{equation*}
\label{uslovia_c}
\lambda_{i}(x_j)=\left\{\begin{matrix}
1, &x_j=x_{i}\\
0, &x_j\ne{x_{i}}
\end{matrix}\right.
\end{equation*}

Коэффициенты $\lambda_{i}(x)$ можно искать в виде 
многочленов степени $n$:
\begin{equation*}
\label{eq_c}
\left\{\begin{matrix}
\lambda_0(x)&=&\alpha_0\cdot(x-x_1)\cdot(x-x_2)
\cdot(x-x_2)\cdot&\ldots&\cdot(x-x_n)\\
\lambda_1(x)&=&\alpha_1\cdot(x-x_0)\cdot(x-x_2)
\cdot(x-x_3)\cdot&\ldots&\cdot(x-x_n)\\
\hdotsfor{5}\\
\lambda_n(x)&=&\alpha_n\cdot(x-x_0)\cdot(x-x_1)
\cdot(x-x_2)\cdot&\ldots&\cdot(x-x_{n-1})
\end{matrix}\right.
\end{equation*}

Определим неизвестные 
$\alpha_0, \alpha_1, \ldots, \alpha_n$ 
из условия для коэффициентов $\lambda_i(x)$:
\begin{equation*}
\left\{
\begin{matrix}
1&=&\alpha_0\cdot(x_0-x_1)\cdot(x_0-x_2)\cdot(x_0-x_2)\cdot&
\dots&\cdot(x_0-x_n)\\
1&=&\alpha_1\cdot(x_1-x_0)\cdot(x_1-x_2)\cdot(x_1-x_3)\cdot&
\dots&\cdot(x_1-x_n)\\
\hdotsfor{5}\\
1&=&\alpha_n\cdot(x_n-x_0)\cdot(x_n-x_1)\cdot(x_n-x_2)\cdot&
\dots&\cdot(x_n-x_{n-1})
\end{matrix}
\right.
\end{equation*}

Таким образом, коэффициенты $\lambda_{i}(x)$ 
интерполяционного многочлена\linebreak
Лагранжа находятся из соотношений:
\begin{equation*}
\left\{\begin{matrix}
\lambda_0(x)&=&\dfrac
{(x-x_1)\cdot(x-x_2)\cdot\dots\cdot(x-x_n)}
{(x_0-x_1)\cdot(x_0-x_2)\cdot\dots\cdot(x_0-x_n)}\\[1em]
\lambda_1(x)&=&\dfrac
{(x-x_0)\cdot(x-x_2)\cdot\dots\cdot(x-x_n)}
{(x_1-x_0)\cdot(x_1-x_2)\cdot\dots\cdot(x_1-x_n)}\\[1em]
\hdotsfor{3}\\[1em]
\lambda_n(x)&=&\dfrac
{(x-x_0)\cdot(x-x_1)\cdot\dots\cdot(x-x_{n-1})}
{(x_n-x_0)\cdot(x_n-x_1)\cdot\dots\cdot(x_n-x_{n-1})}
\end{matrix}\right.,
\end{equation*}
или в более компактной форме:
\begin{equation*}
\lambda_i(x)=\dfrac
{\prod\limits_{j \ne i}^n (x - x_j)}
{\prod\limits_{j \ne i}^n (x_i - x_j)}
\end{equation*}

Итак, интерполяционный многочлен Лагранжа 
\eqref{eq:Lagrange} имеет вид:
\begin{equation}\label{eq:Lagrange_Polynom}
L_{n}(x)=\sum\limits_{i=0}^n\dfrac
{\prod\limits_{j \ne i}^n(x-x_j) }
{\prod\limits_{j \ne i}^n(x_i-x_j)}\cdot{y_i}
\end{equation}

%
%	Интерполирование таблично заданной функции полином Лагранжа $L_3(x)$
%
\emptyline
\subsection{Интерполяция таблично заданной функции полином Лагранжа}
Известно множество данных (узлов интерполяции)
$\{x_i\}$ ($i=0,1,2,3$), в которых определены 
значения функции $y_i=f(x_i)$:
\begin{table}[h]
\vspace{-0.5\baselineskip}
\caption{Таблично заданная функциональная зависимость}
\begin{tabular*}{\textwidth}{%
l@{\extracolsep{\fill}}*{4}{r}p{0.25cm}}
\toprule
$i$&$0$&$1$&$2$&$3$\\
\midmidrule
$x_i$&$-0.76$&$-0.09$&$0.22$&$0.55$\\
\addlinespace% дополнительный пробел
$y_i$&$0.08$&$1.84$&$0.40$&$0.96$\\
\bottomrule
\end{tabular*}
\end{table}

Построим интерполяционный полином Лагранжа $L_3(x)$
на основе данных об узлах интерполяции $\{x_i\}$ 
и значений функции в этих точках $\{y_i\}$:
\begin{gather*}
L_{3}(x)=\sum \limits_{i=0}^3\dfrac
{\prod\limits_{j \ne i}^3(x-x_j)}
{\prod\limits_{j \ne i}^3(x_i-x_j)}\cdot{y_i}
\end{gather*}

\begin{enumerate}
\item
Представим полином Лагранжа в развернутом виде:
\begin{gather*}
\begin{matrix}
L_{3}(x)=
&&\dfrac{(x-x_1)(x-x_2)(x-x_3)}{(x_0-x_1)(x_0-x_2)(x_0-x_3)}\cdot y_0&+\\[1em]
&+&\dfrac{(x-x_0)(x-x_2)(x-x_3)}{(x_1-x_0)(x_1-x_2)(x_1-x_3)}\cdot y_1&+\\[1em]
&+&\dfrac{(x-x_0)(x-x_1)(x-x_3)}{(x_2-x_0)(x_2-x_1)(x_2-x_3)}\cdot y_2&+\\[1em]
&+&\dfrac{(x-x_0)(x-x_1)(x-x_2)}{(x_3-x_0)(x_3-x_1)(x_3-x_2)}\cdot y_3&
\end{matrix}
\end{gather*}

\item
Воспользуемся численными данными об узлах интерполяции 
$\{x_i\}$ и известными значениями интерпретируемой функции 
в этих узлах $\{y_i\}$:
\begin{gather*}
\begin{matrix}
L_{3}(x)=
&&\dfrac{(x-(-0.09))(x-0.22)(x-0.55)}{(-0.76-(-0.09))(-0.76-0.22)(-0.76-0.55)}\cdot0.08&+\\[1em]
&+&\dfrac{(x-(-0.76))(x-0.22)(x-0.55)}{(-0.09-(-0.76))(-0.09-0.22)(-0.09-0.55)}\cdot1.84&+\\[1em]
&+&\dfrac{(x-(-0.76))(x-(-0.09))(x-0.55)}{(0.22-(-0.76))(0.22-(-0.09))(0.22-0.55)}\cdot0.40&+\\[1em]
&+&\dfrac{(x-(-0.76))(x-(-0.09))(x-0.22)}{(0.55-(-0.76))(0.55-(-0.09))(0.55-0.22)}\cdot0.96&
\end{matrix}
\end{gather*}

\item
Проведем необходимые арифмитические действия:
\begin{gather*}
\begin{matrix}
L_{3}(x)=
&&\dfrac{(x+0.09)(x-0.22)(x-0.55)}{(-0.67)(-0.98)(-1.31)}\cdot0.08&+\\[1em]
&+&\dfrac{(x+0.76)(x-0.22)(x-0.55)}{(0.67)(-0.31)(-0.64)}\cdot1.84&+\\[1em]
&+&\dfrac{(x+0.76)(x+0.09)(x-0.55)}{(0.98)(0.31)(-0.33)}\cdot0.40&+\\[1em]
&+&\dfrac{(x+0.76)(x+0.09)(x-0.22)}{(1.31)(0.64)(0.33)}\cdot0.96&
\end{matrix}
\end{gather*}

или
\begin{gather*}
\begin{matrix}
L_{3}(x)=
&&\dfrac{(x+0.09)(x-0.22)(x-0.55)}{-0.86}\cdot0.08&+\\[1em]
&+&\dfrac{(x+0.76)(x-0.22)(x-0.55)}{0.13}\cdot1.84&+\\[1em]
&+&\dfrac{(x+0.76)(x+0.09)(x-0.55)}{-0.10}\cdot0.40&+\\[1em]
&+&\dfrac{(x+0.76)(x+0.09)(x-0.22)}{0.28}\cdot0.96&
\end{matrix}
\end{gather*}

Продолжая делать упрощения окончательно получим:
\begin{gather*}
\begin{matrix}
L_{3}(x)=
&&(x+0.09)(x-0.22)(x-0.55)\cdot(-0.09)&+\\
&+&(x+0.76)(x-0.22)(x-0.55)\cdot13.84&+\\
&+&(x+0.76)(x+0.09)(x-0.55)\cdot(-3.99)&+\\
&+&(x+0.76)(x+0.09)(x-0.22)\cdot3.47&
\end{matrix}
\end{gather*}

\item
Запишем выражение для интерполяционный полином Лагранжа
в каноническом виде:
\begin{gather*}
L_{3}(x)=1.36963 - 5.24831\cdot x + 0.9119\cdot x^2 + 13.23\cdot x^3
\end{gather*}

\item
На одном графике представим диаграмму рассеяния 
(разброса) данных функции заданной таблично $y_i=f(x_i)$
(маркеры) и результаты вычислений 
интерполяционного полинома Лагранжа $L_3(x)$
(сплошная линия).
% *******************************
%	График функций
%
\begin{figure}[H]\centering
\begin{tikzpicture}
\begin{axis}[% оси координат
ylabel=$L_3(x)$,
xmin=-0.9,xmax=0.7,xtick={-0.8,-0.4,0,0.4},
ymin=-1, ymax=3,
]
% табличные данные
\addplot[PlotDarkRed,only marks]
coordinates {(-0.76,0.08) (-0.09,1.84) (0.22,0.40) (0.55,0.96)};
% полином Лагранжа
\addplot[PlotDarkRed,mark=none,domain=-0.8:0.6,samples=100] 
{1.36963 - 5.24831*x + 0.9119*x^2 + 13.23*x^3}
node[pos=0.7,right] {$L_3(x)$};
\end{axis}
\end{tikzpicture}
\caption{График таблично заданной функции $y_i=f(x_i)$ (маркеры) 
и интерполяционного полинома Лагранжа $L_3(x)$
(сплошная линия)}
\end{figure}
% *******************************
\end{enumerate}

%\end{document}