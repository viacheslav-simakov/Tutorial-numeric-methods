\newpage
%
%	Аппроксимация функций
%
\section{Аппроксимация функций}
Задача о приближении функции ставится следующим образом:
данную функцию $f(x)$ необходимо заменить 
обобщенным полиномом $p_m(x)$ заданного порядка $m$ 
так, чтобы отклонение (в известном смысле) функции $f(x)$ 
от обобщенного полинома $p_m(x)$ на указанном множестве 
$\vect{x}=\{x\}$ было наименьшим. 
При этом полином $p_m(x)$ в общем случае 
называется аппроксимирующим.

Если множество $\vect{x}$ состоит из отдельных точек 
$x\in\{x_0, x_1, x_2, \dots x_n\}$ (узлов),
то приближение называется \emph{точечным}.
Если $\vect{x}$ есть отрезок $x\in[a,b]$, 
то приближение называется \emph{интегральным}. 
Для практики важным является приближение функций 
алгебраическими и тригонометрическими полиномами.

В случае постановки задачи поиска аппроксимирующей 
функции, которая обеспечивает погрешность не хуже заданной, 
необходимо подбирать и структуру этой функции. 
Эта задача значительно сложнее предыдущей и подходы в её
решении основываются на переборе различных функций $p_m(x)$
и сравнении мер близости результатов расчета с исходными данными. 

%
% Точечное квадратичное аппроксимирование функций
%
\emptyline
\subsection{Точечное квадратичное аппроксимирование функций}
На практике часто бывает, что заданный порядок $m$ 
приближающего полинома $p_m(x)$ меньше числа 
узлов аппроксимации ${m<n}$, в которых 
известно значение функции $y_i=f(x_i)$ ($i=0,1,2, \cdots, n$).
В этом случае используют точечный метод наименьших квадратов
и рассматривается обобщенный линейный полином порядка $m$ вида:
\begin{equation}\label{eq:general polynom}
p_m(x)=c_0\cdot\phi_0(x)+c_1\cdot\phi_1(x)+\ldots+c_m\cdot\phi_m(x)=
\sum\limits_{j=0}^{m}c_j\cdot\phi_j(x),
\end{equation}
где $\{c_j\}$ -- неизвестные коэффициенты полинома;
$\{\phi_j(x)\}$ -- система базисная функций, которая выбирается произвольно.

Выражение для обобщенного полинома \eqref{eq:general polynom}
удобно переписать в виде скалярного произведения:
\begin{equation}\label{eq:vector polynom}
p_m(x)=\sum\limits_{j=0}^{m}c_j\cdot\phi_j(x)=
\left(\vect{c},\vect{\phi}(x)\right),
\end{equation}
где $\vect{c}=\{c_j\}$ -- вектор коэффициентов полинома;
$\vect{\phi}(x)=\{\phi_j(x)\}$ -- вектор базисных функций.

В качестве меры отклонения $\norma{r}$ полинома $p_m(x)$ 
от известной функции $y(x)$ на множестве точек 
$\{x_0, x_1, x_2,\cdots,x_n\}$, как правило, принимается 
сумма квадратов отклонений полинома от этой функции 
на заданной системе точек:
\begin{equation}\label{eq:approx:MSE}
\norma{r}=\sum_{i=0}^{n}\left(p_m(x_i)-y_i\right)^2
\end{equation}

Следует отметить, что мера отклонения полинома 
от известной функции есть функция многих переменных
$\norma{r}=\rho(c_0, c_1, \dots, c_m)$, т.е. коэффициентов полинома
$c_i$ ($i=0,1,\dots,m$), которые необходимо подобрать так, 
чтобы величина меры отклонения была наименьшей 
$\norma{r}\to{\min}$.
Полученный полином называется аппроксимирующим 
для данной функции, а процесс построения этого полинома -- 
точечной квадратичной аппроксимацией или 
точечным квадратичным аппроксимированием функции. 

Для решения задачи точечного квадратичного аппроксимирования,
т.е. определения числовых значений всех коэффициентов 
полинома $p_m(x)$, необходимо найти \emph{положения минимума 
функции} многих переменных $\rho(c_0,c_1,\ldots,c_m)$.

Определим частные производные от величины суммы квадратов отклонений и 
воспользовавшись условием экстремума функции многих переменных, 
составим систему уравнений вида:
\begin{gather*}
\pdiff{\rho}{c_0}=
\pdiff{\rho}{c_1}=
\pdiff{\rho}{c_2}=\ldots=
\pdiff{\rho}{c_m}=0
\end{gather*}

Для определения неизвестных коэффициентов полинома
$c_0, c_1, c_2,\dots, c_m$ необходимо решить систему 
$m+1$ уравнений с $m+1$ неизвестными: 
\begin{equation*}
\pdiff{\rho}{c_j}=
2\cdot\sum\limits_{i=0}^{n}\left(p_m(x_i) - y_i\right)
\cdot\phi_j(x_i)=0,\quad j=0,1,\ldots,m
\end{equation*}

Вводя сокращенные обозначения:
\begin{equation}
(\phi,\psi)=\sum\limits_{i=0}^n 
\end{equation} 


%
% Аппроксимирование функций алгебраическими полиномами
%
\emptyline
\subsubsection{Аппроксимирование функций
алгебраическими полиномами}
Для аппроксимации функции алгебраическим полиномом 
используется точечный метод наименьших квадратов и
рассматривается алгебраический полином степени $m$:
\begin{gather}\label{eq:approx:polynom}
p_m(x)=c_0+c_1\cdot{x}+c_2\cdot{x^2}+\dots+c_m\cdot{x^m}=
\sum\limits_{j=0}^{m}c_j\cdot{x^j}.
\end{gather}


Для определения неизвестных коэффициентов полинома
$c_0, c_1, c_2,\dots, c_m$ необходимо решить систему 
$m+1$ уравнений с $m+1$ неизвестными: 
\begin{gather*}
\renewcommand*{\arraystretch}{1.5}
\left\{\begin{array}{lclcl}
\pdiff{\rho}{c_0}&=&2\cdot\sum\limits_{i=0}^{n}\left(c_0+c_1\cdot{x_i}+c_2\cdot{x_i^2}+\ldots+c_m\cdot{x_i^m} - y_i\right)\cdot1&=&0\\
\pdiff{\rho}{c_1}&=&2\cdot\sum\limits_{i=0}^{n}\left(c_0+c_1\cdot{x_i}+c_2\cdot{x_i^2}+\ldots+c_m\cdot{x_i^m} - y_i\right)\cdot{x_i}&=&0\\
\pdiff{\rho}{c_2}&=&2\cdot\sum\limits_{i=0}^{n}\left(c_0+c_1\cdot{x_i}+c_2\cdot{x_i^2}+\ldots+c_m\cdot{x_i^m} - y_i\right)\cdot{x_i^2}&=&0\\
\hdotsfor{1}&=&\hdotsfor{1}&=&0\\
\pdiff{\rho}{c_m}&=&2\cdot\sum\limits_{i=0}^{n}\left(c_0+c_1\cdot{x_i}+c_2\cdot{x_i^2}+\ldots+c_m\cdot{x_i^m} - y_i\right)\cdot{x_i^m}&=&0\\
\end{array}\right.
\end{gather*}

Таким образом, задача точечной квадратичной аппроксимации 
функции сводится к решению системы линейных уравнений 
относительно неизвестных -- коэффициентов полинома 
$\{c_0, c_1, c_2,\dots, c_m\}$:
\begin{equation}\label{eq:approx:LSE}
\begin{matrix}
\mathbf{A}\cdot\vect{c}=\vect{b}
&\text{или}&
\begin{pmatrix}
a_{00}&a_{01}&\cdots&a_{0m}\\
a_{10}&a_{11}&\cdots&a_{1m}\\
\vdots&\vdots&\ddots&\vdots\\
a_{m0}&a_{m1}&\cdots&a_{mm}\\
\end{pmatrix}
\cdot
\begin{pmatrix}c_0\\c_1\\\vdots\\c_m\end{pmatrix}
=\begin{pmatrix}b_0\\b_1\\\vdots\\b_m\end{pmatrix}
\end{matrix},
\end{equation}
где $\mathbf{A}=\{a_{k\ell}\}$ и $\vect{b}=\{b_k\}$ 
-- квадратная матрица и вектор правых частей 
системы линейных уравнений, соответственно:
\begin{gather*}
a_{k\ell}=\sum\limits_{i=0}^n x_i^k\cdot x_i^\ell,
\quad
b_{k}=\sum\limits_{i=0}^n x_i^k\cdot y_i,
\quad k,\ell=0,1,2,\dots,m
\end{gather*}

Если среди узлов сетки $\{x_i\}$ 
нет совпадающих, а также степень полинома 
меньше чем число узлов аппроксимации $m<n$, 
то определитель системы не равен нулю $\det\mathbf{A}\ne0$.
Следовательно, эта система имеет единственное решение 
$\vect{c}=\{c_0, c_1, c_2,\ldots, c_m\}$,
а полином $p_m(x)$ с такими коэффициентами $\{c_i\}$ 
будет обладать минимальным квадратичным отклонением 
$\rho_{\min}$. 

Зная коэффициенты аппроксимирующего полинома, 
можно вычислить величину $\norma{r}$, например, для 
сравнения различных аппроксимирующих функций. 

Следует отметить, что все коэффициенты $\{c_i\}$ полинома 
\eqref{eq:approx:polynom} находятся из решения системы уравнений
\eqref{eq:approx:LSE}, т.е. они связаны между собой. 
Поэтому если какой-либо коэффициент $c_i$ вследствие его малости 
($c_i\approx0$) отбросить, то необходимо \emph{пересчитывать 
все оставшиеся коэффициенты} полинома $p_m(x)$.

Кроме того, при изменении даже одного значения 
исходных данных $\{x_i,y_i\}$ все коэффициенты полинома
\eqref{eq:approx:polynom} изменят свои значения, 
так как они полностью определяются исходными данными. 
Поэтому при повторении аппроксимации с несколько 
изменившимися данными (например, вследствие погрешностей 
измерения, помех, влияния неучтенных факторов и т.п.) 
получится другая аппроксимирующая функция, 
отличающаяся коэффициентами. 

%
%	Аппроксимирования функций полиномом второй степени $p_2(x)$
%
\emptyline
\subsection{Аппроксимирование функций полиномом
второй степени $p_2(x)$}
Известна таблица данных некоторой функциональной зависимости 
$y(x)$:
\begin{table}[H]
\vspace{-0.5\baselineskip}
\caption{Таблично заданная функциональная зависимость
$y_i=f(x_i)$}
\small
\begin{tabular*}{\textwidth}{%
l@{\extracolsep{\fill}}*{5}{r}p{0.25cm}}
\toprule
$i$&$0$&$1$&$2$&$3$&$4$\\
\midmidrule
$x_i$&$-0.76$&$-0.48$&$-0.09$&$0.22$&$0.55$\\
$y_i$&$5.15$&$4.39$&$4.10$&$5.71$&$5.30$\\
\bottomrule
\end{tabular*}
\end{table}

Необходимо аппроксимировать функцию $\{y_i\}$,
заданную таблично, алгебраическим полиномом 
второй степени $p_2(x)$:
\begin{gather*}
p_2(x)=c_0 + c_1\cdot x + c_2\cdot x^2
\end{gather*}

\begin{enumerate}
\item
Построим меру отклонения полинома $p_2(x)$ 
от таблично заданной функции $y_i=f(x_i)$
на множестве точек $\{x_0, x_1, x_2, x_3, x_4\}$:
\begin{gather*}
\norma{r}=\rho(c_0,c_1,c_2)=
\sum_{i=0}^{4}\left(c_0+c_1\cdot{x_i}+c_2\cdot{x_i^2}-y_i\right)^2,
\end{gather*}
где $y_i=f(x_i)$ -- значение функции в точке $x_i$.

\item
Запишем меру отклонения $\rho(c_0,c_1,c_2)$ в явном виде 
на основе данных из условия задачи:
\begin{gather*}
\begin{split}
\rho(c_0,c_1,c_2)=
&\left(c_0 + c_1\cdot(-0.76) + c_2\cdot(-0.76)^2 - 5.15 \right)^2+\\
&\left(c_0 + c_1\cdot(-0.48) + c_2\cdot(-0.48)^2 - 4.39 \right)^2+\\
&\left(c_0 + c_1\cdot(-0.09) + c_2\cdot(-0.09)^2 - 4.10 \right)^2+\\
&\left(c_0 + c_1\cdot(0.22) + c_2\cdot(0.22)^2 - 5.71 \right)^2+\\
&\left(c_0 + c_1\cdot(0.55) + c_2\cdot(0.55)^2 - 5.30 \right)^2
\end{split}
\end{gather*}

\item
Определим частную производную от меры отклонений $\rho(c_0,c_1,c_2)$ 
по аргументу $c_0$ и приравняем её нулю:
\begin{gather*}
\begin{split}
\pdiff{\rho}{c_0}=
&2\cdot\left(c_0 + c_1\cdot(-0.76) + c_2\cdot(-0.76)^2 - 5.15 \right)\cdot 1+\\
&2\cdot\left(c_0 + c_1\cdot(-0.48) + c_2\cdot(-0.48)^2 - 4.39 \right)\cdot 1+\\
&2\cdot\left(c_0 + c_1\cdot(-0.09) + c_2\cdot(-0.09)^2 - 4.10 \right)\cdot 1+\\
&2\cdot\left(c_0 + c_1\cdot(0.22) + c_2\cdot(0.22)^2 - 5.71 \right)\cdot 1+\\
&2\cdot\left(c_0 + c_1\cdot(0.55) + c_2\cdot(0.55)^2 - 5.30 \right)\cdot 1=0
\end{split}
\end{gather*}

Коэффициенты первой строки матрицы $\mathbf{A}$
и первый элемент вектора $\vect{b}$:
\begin{gather*}
\begin{array}{lcl}
a_{00}&=&1+1+1+1+1=5\\
a_{01}&=&(-0.76) + (-0.48) + (-0.09) + (0.22) + (0.55) = -0.56\\
a_{02}&=&(-0.76)^2 + (-0.48)^2 + (-0.09)^2 + (0.22)^2 + (0.55)^2=1.18\\
%
b_0&=&5.15 + 4.39 + 4.10 + 5.71 + 5.30=24.65
\end{array}
\end{gather*}

\item
Определим частную производную от меры отклонений 
$\rho(c_0,c_1,c_2)$ по аргументу $c_1$ и приравняем её нулю:
\begin{gather*}
\begin{split}
\pdiff{\rho}{c_1}=
&2\cdot\left(c_0 + c_1\cdot(-0.76) + c_2\cdot(-0.76)^2 - 5.15 \right)\cdot(-0.76)+\\
&2\cdot\left(c_0 + c_1\cdot(-0.48) + c_2\cdot(-0.48)^2 - 4.39 \right)\cdot(-0.48)+\\
&2\cdot\left(c_0 + c_1\cdot(-0.09) + c_2\cdot(-0.09)^2 - 4.10 \right)\cdot(-0.09)+\\
&2\cdot\left(c_0 + c_1\cdot(0.22) + c_2\cdot(0.22)^2 - 5.71 \right)\cdot(0.22)+\\
&2\cdot\left(c_0 + c_1\cdot(0.55) + c_2\cdot(0.55)^2 - 5.30 \right)\cdot(0.55)=0
\end{split}
\end{gather*}

Коэффициенты второй строки матрицы $\mathbf{A}$
и второй элемент вектора $\vect{b}$:
\begin{gather*}
\begin{array}{lcl}
c_{10}&=&(-0.76) + (-0.48) + (-0.09) + (0.22) + (0.55) = -0.56\\
c_{11}&=&(-0.76)^2 + (-0.48)^2 + (-0.09)^2 + (0.22)^2 + (0.55)^2=1.18\\
c_{12}&=&(-0.76)^3 + (-0.48)^3 + (-0.09)^3 + (0.22)^3 + (0.55)^3=-0.38\\
%
b_1&=&5.15\cdot(-0.76)+4.39\cdot(-0.48)+4.10\cdot(-0.09)+\\
&&5.71\cdot(0.22)+5.30\cdot(0.55)=-2.24
\end{array}
\end{gather*}

\item
Определим частную производную от меры отклонений 
$\rho(c_0,c_1,c_2)$ по аргументу $c_2$ и приравняем её нулю:
\begin{gather*}
\begin{split}
\pdiff{\rho}{c_2}=
&2\cdot\left(c_0 + c_1\cdot(-0.76) + c_2\cdot(-0.76)^2 - 5.15 \right)\cdot(-0.76)^2+\\
+&2\cdot\left(c_0 + c_1\cdot(-0.48) + c_2\cdot(-0.48)^2 - 4.39 \right)\cdot(-0.48)^2+\\
+&2\cdot\left(c_0 + c_1\cdot(-0.09) + c_2\cdot(-0.09)^2 - 4.10 \right)\cdot(-0.09)^2+\\
+&2\cdot\left(c_0 + c_1\cdot(0.22) + c_2\cdot(0.22)^2 - 5.71 \right)\cdot(0.22)^2+\\
+&2\cdot\left(c_0 + c_1\cdot(0.55) + c_2\cdot(0.55)^2 - 5.30 \right)\cdot(0.55)^2=0
\end{split}
\end{gather*}

Коэффициенты третьей строки матрицы $\mathbf{A}$
и третий элемент вектора $\vect{b}$:
\begin{gather*}
\begin{array}{lcl}
c_{20}&=&(-0.76)^2 + (-0.48)^2 + (-0.09)^2 + (0.22)^2 + (0.55)^2=1.18\\
c_{21}&=&(-0.76)^3 + (-0.48)^3 + (-0.09)^3 + (0.22)^3 + (0.55)^3=-0.38\\
c_{22}&=&(-0.76)^4 + (-0.48)^4 + (-0.09)^4 + (0.22)^4 + (0.55)^4=0.49\\
%
b_2&=&5.15\cdot(-0.76)^2 +4.39\cdot(-0.48)^2 +4.10\cdot(-0.09)^2+\\
&&5.71\cdot(0.22)^2 +5.30\cdot(0.55)^2=5.94
\end{array}
\end{gather*}

\item
Таким образом, для определения неизвестных коэффициентов 
$c_0,c_1,c_2$ аппроксимирующего полинома $p_2(x)$ 
необходимо решить систему линейных алгебраических уравнений:
\begin{equation*}
\begin{pmatrix}[rrr]
5&-0.56&1.18\\
-0.56&1.18&-0.38\\
1.18&-0.38&0.49\\
\end{pmatrix}\cdot
\begin{pmatrix}c_0\\c_1\\c_2\end{pmatrix}
=\begin{pmatrix}24.65\\-2.24\\5.94\\\end{pmatrix}
\end{equation*}

\item
Решение этой системы линейных уравнений можно найти методом Гаусса:
\begin{gather*}
\left\{\begin{array}{lcl}
c_0&=&4.66\\
c_1&=&0.80\\
c_2&=&1.52
\end{array}\right.
\end{gather*}

Таким образом, аппроксимирующий полином имеет вид:
\begin{equation}\label{eq:my approx polynom}
p_2(x)=4.66 + 0.80\cdot x + 1.52\cdot x^2
\end{equation}

\item
На одном графике представим диаграмму рассеяния 
(разброса) данных функции заданной таблично $y_i=f(x_i)$
(маркеры) и результаты вычислений
аппроксимирующего алгебраического полинома 
второго порядка $p_2(x)$ (сплошная линия).
% 
%	График функций
%
\begin{figure}[H]\centering
\begin{tikzpicture}
\begin{axis}[% оси координат
ylabel={$p_2(x)$},
xmin=-1, xmax=0.7, xtick={-0.8,-0.4,0,0.4},
ymin=3.8, ymax=6,
]
\addplot[ball darkblue,only marks]
coordinates{(-0.76,5.15) (-0.48,4.39) (-0.09,4.10) (0.22,5.71) (0.55,5.30)};
\addplot[darkblue,mark=none,domain=-0.9:0.6, samples=50] 
{4.66 + 0.80*x + 1.52*x^2} node[pos=0.5,below right] {$p_2(x)$};
\end{axis}
\end{tikzpicture}
\caption{График таблично заданной функции $y_i=f(x_i)$ (маркеры) 
и аппроксимирующего алгебраического полинома $p_2(x)$
(сплошная линия)}
\end{figure}

\item
Рассчитаем значения аппроксимирующего полинома 
\eqref{eq:my approx polynom} и ошибку аппроксимации 
таблично заданной функции в узлах сетки $\{x_i\}$
(таблица \ref{tab:approx calculate data}):
\begin{equation}
\epsilon_i=y_i-p_2(x_i)
\end{equation}
%
% Таблица: погрешность аппроксимации
%
\begin{table}[H]
\caption{Рассчитанные значения аппроксимирующего полинома $p_2(x)$}
\label{tab:approx calculate data}
\small
\begin{tabular*}{\textwidth}{%
l@{\extracolsep{\fill}}*{5}{r}p{0.25cm}}
\toprule
$i$&$0$&$1$&$2$&$3$&$4$\\
\midmidrule
$x_i$&$-0.76$&$-0.48$&$-0.09$&$0.22$&$0.55$\\
$y_i$&$5.15$&$4.39$&$4.10$&$5.71$&$5.30$\\
$p_2(x_i)$	&	$4.93$	&	$4.63$	&	$4.60$	&	$4.91$	&	$5.56$\\
$\epsilon_i$	&	$0.22$	&	$-0.24$	&	$-0.50$	&	$0.80$	&	$-0.26$\\
\bottomrule
\end{tabular*}
\end{table}

%
% График
%
\begin{figure}[H]\centering
\begin{tikzpicture}
\begin{axis}[xmin=-1, xmax=0.7, xtick={-0.8,-0.4,0,0.4},
ymin=-1, ymax=1,
xlabel=$x_i$,ylabel={$\epsilon_i=y_i-p(x_i)$}]
\addplot[ycomb,draw=darkred,mark=*,mark size=3pt,mark options={fill=white}]
coordinates{(-0.76,0.22)(-0.48,-0.24)(-0.09,-0.50)(0.22,0.80)(0.55,-0.26)};
\end{axis}
\end{tikzpicture}
\caption{Ошибка аппроксимации полином $p(x)$
функции заданной таблично $y_i=f(x_i)$}
\label{fig:error}
\end{figure}

\item
Рассчитаем среднее квадратичное отклонение 
$\delta$ полинома \eqref{eq:my approx polynom} от значений 
функции $\{y_i\}$ в узлах сетки $\{x_i\}$:
\begin{gather}
\delta = \dfrac{1}{n}\cdot\sum_{i=0}^{n-1}\epsilon_i^2=
\dfrac{0.05+0.06+0.25+0.64+0.07}{5}=0.213,
\end{gather}
где $n=5$ -- количество узлов сетки.

\end{enumerate}

%\end{document}
%Можно рассчитать количественные оценки тесноты связи 
%коэффициентов. Существует специальная теория планирования 
%экспериментов, которая позволяет обосновать и рассчитать 
%значения хi, используемые для аппроксимации, 
%чтобы получить заданные свойства коэффициентов 
%(несвязанность, минимальная дисперсия коэффициентов и т.д.) 
%или аппроксимирующей функции (равная точность описания 
%реальной зависимости в различных направлениях, 
%минимальная дисперсия предсказания значения функции и т.д.).
%
%Не следует забывать, что с повышением точности аппроксимации 
%растет и сложность функции (при полиномиальных 
%аппроксимирующих функциях), что делает ее менее удобной 
%при использовании.
%
%Пример 3.1. В ходе проведения эксперимента были получены данные, 
%представленные в таблице 3.1. Необходимо способом наименьших 
%квадратов подобрать для заданных значений x и y квадратичную 
%функцию . 
%
%Исходными данными для решения задачи является таблица 
%наблюдений -- набор значений независимых переменных и 
%соответствующие им значения функции отклика. 
%Число строк (узлов) таблично заданной функции называют 
%объемом выборки.
%
%Форма уравнения выбирается исследователем в соответствии с 
%поведением аппроксимируемой функции в области изменения 
%независимых переменных. Результатом же решения задачи 
%аппроксимации являются оценки коэффициентов этого уравнения. 
%
%Очевидно, что коэффициенты уравнения следует подбирать так, 
%чтобы рассчитываемые по уравнению значения функции отклика 
%максимально близко совпадали с заданными в исходной таблице 
%наблюдений.
%
%http://ru.bmstu.wiki/Аппроксимация_функций,_моделирующих_сигналы
%Математические модели сигналов, детально и точно описывающие 
%определенные физические объекты и процессы, могут быть очень 
%сложными и мало пригодными для практического использования, 
%как при математическом анализе физических данных, 
%так и в прикладных задачах, основанных на математическом 
%моделировании КПС. 
%
%Кроме того, практическая регистрация 
%сигналов выполняется, как правило, с определенной погрешностью 
%или с определенным уровнем шумов, которые по своим значениям 
%могут быть выше теоретической погрешности прогнозирования 
%сигналов при расчетах по сложным, хотя и очень точным формулам. 
%Не имеет большого смысла и проектирование систем обработки 
%и анализа сигналов по высокоточным формулам, 
%если повышение точности расчетов не дает ощутимого эффекта 
%в повышении точности обработки данных. 
%
%Во всех этих условиях возникает задача аппроксимации --
%представления произвольных сложных функций 
%простыми и удобными для практического использования функциями 
%таким образом, чтобы отклонение 
%в области ее задания было наименьшим по определенному критерию 
%приближения. 
%
%Математика очень часто оперирует со специальными математическими 
%функциями решения дифференциальных уравнений и интегралов, 
%которые не имеют аналитических выражений и представляются 
%табличными числовыми значениями 
%для дискретных значений независимых переменных 
%
%Аналогичными таблицами могут представляться и 
%экспериментальные данные. Точки, в которых определены 
%дискретные значения функций или данных, называются узловыми. 
%Однако на практике могут понадобиться значения данных величин 
%совсем в других точках, отличных от узловых, или с другим шагом 
%дискретизации аргументов. 
%
%Возникающая при этом задача вычисления значений функции 
%в промежутках между узами называется задачей интерполяции, 
%за пределами семейства узловых точек вперед или назад по 
%переменным -- задачей экстраполяции или прогнозирования. 
%Решение этих задач также обычно выполняется с использованием 
%аппроксимирующих функций.
%
%Сглаживание статистических данных или аппроксимация данных 
%с учетом их статистических параметров относится к задачам регрессии,
%и рассматриваются в следующей теме. 
%
%Как правило, при регрессионном анализе усреднение данных 
%производится методом наименьших квадратов (МНК).
%
