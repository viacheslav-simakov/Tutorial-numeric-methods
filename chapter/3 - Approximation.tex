\newpage
%
%	Аппроксимация функция
%
\section{Аппроксимация функция}
Задача о приближении функции ставится следующим образом:
данную функцию $f(x)$ необходимо заменить 
обобщенным полиномом $p_m(x)$ заданного порядка $m$ 
так, чтобы отклонение (в известном смысле) функции $f(x)$ 
от обобщенного полинома $p_m(x)$ на указанном множестве 
$\vec{x}=\{x\}$ было наименьшим. 
При этом полином $p_m(x)$ в общем случае 
называется аппроксимирующим.

Если множество $\vec{x}$ состоит из отдельных точек 
$x\in\{x_0, x_1, x_2, \dots x_n\}$ (узлов),
то приближение называется \textit{точечным}.
Если $\vec{x}$ есть отрезок $x_a<x<x_b$, 
то приближение называется \textit{интегральным}. 
Для практики важным является приближение функций 
алгебраическими и тригонометрическими полиномами.

\subsection{Точечное квадратичное аппроксимирование функций}
На практике часто бывает, что заданный порядок $m$ 
приближающего полинома $p_m(x)$ меньше числа 
узлов аппроксимации ${m<n}$, в которых 
известно значение функции $y_i=f(x_i)$ ($i=0,1,2, \cdots, n$).
В этом случае обычно используют точечный 
метод наименьших квадратов и
рассматривается полином степени $m$ вида:
\begin{gather*}
p_m(x)=c_0+c_1\cdot{x}+c_2\cdot{x^2}+\dots+c_m\cdot{x^m}=
\sum\limits_{j=0}^{m}c_j\cdot{x^j}.
\end{gather*}

В качестве меры отклонения $\norma{r}$ полинома $p_m(x)$ 
от известной функции $y(x)$ на множестве точек 
$\{x_0, x_1, x_2,\cdots,x_n\}$, как правило, принимается 
сумма квадратов отклонений полинома от этой функции 
на заданной системе точек:
\begin{gather*}
\norma{r}=\sum_{i=0}^{n}\left(p_m(x_i)-y_i\right)^2
\end{gather*}

Следует отметить, что мера отклонения полинома 
от известной функции есть функция многих переменных
$\norma{r}=g(c_0, c_1, \dots, c_m)$, т.е. коэффициентов полинома
$c_i$ ($i=0,1,\dots,m$), которые необходимо подобрать так, 
чтобы величина меры отклонения была наименьшей 
$\norma{r}\to{\min}$.
Полученный полином называется аппроксимирующим 
для данной функции, а процесс построения этого полинома -- 
точечной квадратичной аппроксимацией или 
точечным квадратичным аппроксимированием функции. 

Для решения задачи точечного квадратичного аппроксимирования,
т.е. определения числовых значений всех коэффициентов 
полинома $p_m(x)$, необходимо найти \emph{положения минимума 
функции} многих переменных $\norma{r}$.

Определим частные производные от величины суммы квадратов отклонений и 
воспользовавшись условием экстремума функции многих переменных, 
составим систему уравнений вида:
\begin{gather*}
\pdiff{\norma{r}}{c_0}=
\pdiff{\norma{r}}{c_1}=
\pdiff{\norma{r}}{c_2}=\cdots=
\pdiff{\norma{r}}{c_m}=0
\end{gather*}

Для определения неизвестных коэффициентов полинома
$c_0, c_1, c_2,\dots, c_m$ необходимо решить систему 
$m+1$ уравнений с $m+1$ неизвестными: 
\begin{gather*}
\renewcommand*{\arraystretch}{1.5}
\left\{\begin{array}{lclcl}
\pdiff{\norma{r}}{a_0}&=&2\cdot\sum\limits_{i=0}^{n}\left(c_0+c_1\cdot{x_i}+c_2\cdot{x_i^2}+\ldots+c_m\cdot{x_i^m} - y_i\right)\cdot1&=&0\\
\pdiff{\norma{r}}{a_1}&=&2\cdot\sum\limits_{i=0}^{n}\left(c_0+c_1\cdot{x_i}+c_2\cdot{x_i^2}+\ldots+c_m\cdot{x_i^m} - y_i\right)\cdot{x_i}&=&0\\
\pdiff{\norma{r}}{a_2}&=&2\cdot\sum\limits_{i=0}^{n}\left(c_0+c_1\cdot{x_i}+c_2\cdot{x_i^2}+\ldots+c_m\cdot{x_i^m} - y_i\right)\cdot{x_i^2}&=&0\\
\hdotsfor{1}&=&\hdotsfor{1}&=&0\\
\pdiff{\norma{r}}{a_m}&=&2\cdot\sum\limits_{i=0}^{n}\left(c_0+c_1\cdot{x_i}+c_2\cdot{x_i^2}+\ldots+c_m\cdot{x_i^m} - y_i\right)\cdot{x_i^m}&=&0\\
\end{array}\right.
\end{gather*}

Таким образом, задача точечной квадратичной аппроксимации 
функции сводится к решению системы линейных уравнений 
относительно неизвестных -- коэффициентов полинома 
$\{c_0, c_1, c_2,\dots, c_m\}$:
\begin{gather*}
\begin{matrix}
\mathbf{A}\cdot\vec{c}=\vec{b}
&\text{или}&
\begin{pmatrix}
a_{00}&a_{01}&\cdots&a_{0m}\\
a_{10}&a_{11}&\cdots&a_{1m}\\
\vdots&\vdots&\ddots&\vdots\\
a_{m0}&a_{m1}&\cdots&a_{mm}\\
\end{pmatrix}
\cdot
\begin{pmatrix}c_0\\c_1\\\vdots\\c_m\end{pmatrix}
=\begin{pmatrix}b_0\\b_1\\\vdots\\b_m\end{pmatrix}
\end{matrix},\end{gather*}
где $\mathbf{A}=\{a_{k\ell}\}$ и $\vec{b}=\{b_k\}$ 
-- квадратная матрица и вектор правых частей 
системы линейных уравнений, соответственно:
\begin{gather*}
a_{k\ell}=\sum\limits_{i=0}^n x_i^k\cdot x_i^\ell,
\quad
b_{k}=\sum\limits_{i=0}^n x_i^k\cdot y_i,
\quad k,\ell=0,1,2,\dots,m
\end{gather*}

Если среди узлов сетки $\{x_i\}$ 
нет совпадающих, а также степень полинома 
меньше чем число узлов аппроксимации $m<n$, 
то определитель системы не равен нулю $\det\mathbf{A}\ne0$.
Следовательно, эта система имеет единственное решение 
$\{\mathring{c}_0, \mathring{c}_1, \mathring{c}_2,\dots, \mathring{c}_m\}$,
а полином $p_m(x)$ с такими коэффициентами $\mathring{c}_i$ 
будет обладать минимальным квадратичным отклонением 
$\norma{r}_{\min}$. 

%
%	Аппроксимирования функций полиномом второй степени $p_2(x)$
%
\subsection{Аппроксимирования функций полиномом
второй степени $p_2(x)$}
Известна таблица данных некоторой функциональной зависимости 
$y(x)$:
\begin{table}[H]
\vspace{-0.5\baselineskip}
\caption{Таблично заданная функциональная зависимость
$y_i=f(x_i)$}
\begin{tabular*}{\textwidth}{%
l@{\extracolsep{\fill}}*{5}{r}p{0.25cm}}
\toprule
$i$&$0$&$1$&$2$&$3$&$4$\\
\midmidrule
$x_i$&$-0.76$&$-0.48$&$-0.09$&$0.22$&$0.55$\\
\addlinespace% дополнительный пробел
$y_i$&$5.15$&$4.39$&$4.10$&$5.71$&$5.30$\\
\bottomrule
\end{tabular*}
\end{table}

Необходимо аппроксимировать функцию $\{y_i\}$,
заданную таблично, алгебраическим полиномом 
второй степени $p_2(x)$:
\begin{gather*}
p_2(x)=c_0 + c_1\cdot x + c_2\cdot x^2
\end{gather*}

\begin{enumerate}
\item
Построим меру отклонения полинома $p_2(x)$ 
от таблично заданной функции $y_i=f(x_i)$
на множестве точек $\{x_0, x_1, x_2, x_3, x_4\}$:
\begin{gather*}
\norma{r}=\sum_{i=0}^{4}\left(c_0+c_1\cdot{x_i}+c_2\cdot{x_i^2}-y_i\right)^2,
\end{gather*}
где $y_i=f(x_i)$ -- значение функции в точке $x_i$.

\item
Запишем меру отклонения $\norma{r}$ в явном виде 
на основе данных из условия задачи:
\begin{gather*}
\begin{split}
\norma{r}=
&\left(c_0 + c_1\cdot(-0.76) + c_2\cdot(-0.76)^2 - 5.15 \right)^2+\\
+&\left(c_0 + c_1\cdot(-0.48) + c_2\cdot(-0.48)^2 - 4.39 \right)^2+\\
+&\left(c_0 + c_1\cdot(-0.09) + c_2\cdot(-0.09)^2 - 4.10 \right)^2+\\
+&\left(c_0 + c_1\cdot(0.22) + c_2\cdot(0.22)^2 - 5.71 \right)^2+\\
+&\left(c_0 + c_1\cdot(0.55) + c_2\cdot(0.55)^2 - 5.30 \right)^2
\end{split}
\end{gather*}

\item
Определим частную производную от меры отклонений $\norma{r}$ 
по аргументу $c_0$ и приравняем её нулю:
\begin{gather*}
\begin{split}
\pdiff{\norma{r}}{c_0}=
&2\cdot\left(a_0 + a_1\cdot(-0.76) + a_2\cdot(-0.76)^2 - 5.15 \right)\cdot 1+\\
&2\cdot\left(a_0 + a_1\cdot(-0.48) + a_2\cdot(-0.48)^2 - 4.39 \right)\cdot 1+\\
&2\cdot\left(a_0 + a_1\cdot(-0.09) + a_2\cdot(-0.09)^2 - 4.10 \right)\cdot 1+\\
&2\cdot\left(a_0 + a_1\cdot(0.22) + a_2\cdot(0.22)^2 - 5.71 \right)\cdot 1+\\
&2\cdot\left(a_0 + a_1\cdot(0.55) + a_2\cdot(0.55)^2 - 5.30 \right)\cdot 1=0
\end{split}
\end{gather*}

Коэффициенты первой строки матрицы $\mathbf{A}$
и первый элемент вектора $\vec{b}$:
\begin{gather*}
\begin{array}{lcl}
a_{00}&=&1+1+1+1+1=5\\
a_{01}&=&(-0.76) + (-0.48) + (-0.09) + (0.22) + (0.55) = -0.56\\
a_{02}&=&(-0.76)^2 + (-0.48)^2 + (-0.09)^2 + (0.22)^2 + (0.55)^2=1.18\\
%
b_0&=&5.15 + 4.39 + 4.10 + 5.71 + 5.30=24.65
\end{array}
\end{gather*}

\item
Определим частную производную от меры отклонений 
$\norma{r}$ по аргументу $c_1$ и приравняем её нулю:
\begin{gather*}
\begin{split}
\pdiff{S}{c_1}=
&2\cdot\left(c_0 + c_1\cdot(-0.76) + c_2\cdot(-0.76)^2 - 5.15 \right)\cdot(-0.76)+\\
&2\cdot\left(c_0 + c_1\cdot(-0.48) + c_2\cdot(-0.48)^2 - 4.39 \right)\cdot(-0.48)+\\
&2\cdot\left(c_0 + c_1\cdot(-0.09) + c_2\cdot(-0.09)^2 - 4.10 \right)\cdot(-0.09)+\\
&2\cdot\left(c_0 + c_1\cdot(0.22) + c_2\cdot(0.22)^2 - 5.71 \right)\cdot(0.22)+\\
&2\cdot\left(c_0 + c_1\cdot(0.55) + c_2\cdot(0.55)^2 - 5.30 \right)\cdot(0.55)=0
\end{split}
\end{gather*}

Коэффициенты второй строки матрицы $\mathbf{A}$
и второй элемент вектора $\vec{b}$:
\begin{gather*}
\begin{array}{lcl}
c_{10}&=&(-0.76) + (-0.48) + (-0.09) + (0.22) + (0.55) = -0.56\\
c_{11}&=&(-0.76)^2 + (-0.48)^2 + (-0.09)^2 + (0.22)^2 + (0.55)^2=1.18\\
c_{12}&=&(-0.76)^3 + (-0.48)^3 + (-0.09)^3 + (0.22)^3 + (0.55)^3=-0.38\\
%
b_1&=&5.15\cdot(-0.76)+4.39\cdot(-0.48)+4.10\cdot(-0.09)+\\
&&5.71\cdot(0.22)+5.30\cdot(0.55)=-2.24
\end{array}
\end{gather*}

\item
Определим частную производную от меры отклонений 
$\norma{r}$ по аргументу $c_2$ и приравняем её нулю:
\begin{gather*}
\begin{split}
\pdiff{\norma{r}}{c_2}=
&2\cdot\left(c_0 + c_1\cdot(-0.76) + c_2\cdot(-0.76)^2 - 5.15 \right)\cdot(-0.76)^2+\\
+&2\cdot\left(c_0 + c_1\cdot(-0.48) + c_2\cdot(-0.48)^2 - 4.39 \right)\cdot(-0.48)^2+\\
+&2\cdot\left(c_0 + c_1\cdot(-0.09) + c_2\cdot(-0.09)^2 - 4.10 \right)\cdot(-0.09)^2+\\
+&2\cdot\left(c_0 + c_1\cdot(0.22) + c_2\cdot(0.22)^2 - 5.71 \right)\cdot(0.22)^2+\\
+&2\cdot\left(c_0 + c_1\cdot(0.55) + c_2\cdot(0.55)^2 - 5.30 \right)\cdot(0.55)^2=0
\end{split}
\end{gather*}

Коэффициенты третьей строки матрицы $\mathbf{A}$
и третий элемент вектора $\vec{b}$:
\begin{gather*}
\begin{array}{lcl}
c_{20}&=&(-0.76)^2 + (-0.48)^2 + (-0.09)^2 + (0.22)^2 + (0.55)^2=1.18\\
c_{21}&=&(-0.76)^3 + (-0.48)^3 + (-0.09)^3 + (0.22)^3 + (0.55)^3=-0.38\\
c_{22}&=&(-0.76)^4 + (-0.48)^4 + (-0.09)^4 + (0.22)^4 + (0.55)^4=0.49\\
%
b_2&=&5.15\cdot(-0.76)^2 +4.39\cdot(-0.48)^2 +4.10\cdot(-0.09)^2+\\
&&5.71\cdot(0.22)^2 +5.30\cdot(0.55)^2=5.94
\end{array}
\end{gather*}

\item
Таким образом, для определения неизвестных коэффициентов $c_0,c_1,c_2$
аппроксимирующего полинома $p_2(x)$ необходимо решить 
систему линейных алгебраических уравнений:
\begin{gather*}
\left\{\begin{matrix}
&5\cdot c_0&-&0.56\cdot c_1&+&1.18\cdot c_2&=&24.65\\
-&0.56\cdot c_0&+&1.18\cdot c_1&-&0.38\cdot c_2&=&-2.24\\
&1.18\cdot c_0&-&0.38\cdot c_1&+&0.49\cdot c_2&=&5.94\\
\end{matrix}\right.
\end{gather*}

\item
Решение этой системы линейных уравнений можно найти методом Гаусса:
\begin{gather*}
\left\{\begin{array}{lcl}
c_0&=&4.66\\
c_1&=&0.80\\
c_2&=&1.52
\end{array}\right.
\end{gather*}

Таким образом, аппроксимирующий полином имеет вид:
\begin{gather*}
p_2(x)=4.66 + 0.80\cdot x + 1.52\cdot x^2
\end{gather*}

\item
На одном графике представим диаграмму рассеяния 
(разброса) данных функции заданной таблично $y_i=f(x_i)$
(маркеры) и результаты вычислений
аппроксимирующего алгебраического полинома 
второго порядка $p_2(x)$ (сплошная линия).
% *******************************
%	График функций
%
\begin{figure}[H]\centering
\begin{tikzpicture}
\begin{axis}[
xlabel = {$x$},		% подпись оси x
ylabel = {$p_2(x)$},	% подпись оси y
xmin=-1, xmax=0.7, xtick={-0.8,-0.4,0,0.4},
ymin=3.8, ymax=6,
xtick style={thick, black},
ytick style={thick, black},
grid=major,		
major grid style={color=black!20, dashed, thin},
]
\addplot[only marks,mark size=3pt,
mark=ball,mark size=4pt, 
mark options={thin,draw=darkblue,ball color=darkblue!50}]
coordinates 
{(-0.76,5.15) (-0.48,4.39) (-0.09,4.10) (0.22,5.71) (0.55,5.30)};
\addplot[thick,color=darkblue,domain=-0.9:0.6, samples=50] 
{4.66 + 0.80*x + 1.52*x^2};
\draw[color=darkblue] (axis cs: 0.1,4.7) node[right] {$p_2(x)$};
\end{axis}
\end{tikzpicture}
\caption{График таблично заданной функции $y_i=f(x_i)$ (маркеры) 
и аппроксимирующего алгебраического полинома $p_2(x)$
(сплошная линия)}
\end{figure}

\end{enumerate}