% Выделение цветом
\pgfkeys{/main/.code=\fcolorbox{gray}{yellow}{$#1$}}
\pgfkeys{/one/.code=\textcolor{darkred}{1}}
\pgfkeys{/zero/.code=\textcolor{gray!50}{0}}
%\pgfkeys{/main=hi!}% пример использования
%
%	Метод Гаусса решения систем линейных уравнений
%
\newpage
\section{Метод Гаусса решения систем линейных уравнений}
К решению систем линейных алгебраических уравнений сводится подавляющее 
большинство задач вычислительной математики и многие численные методы основаны 
на решении систем линейных уравнений вида:
\begin{gather}\label{eq:LS}
\left\{
\begin{matrix}
a_{11}\cdot x_1&+&a_{12}\cdot x_2&+&\dots&+&a_{1m}\cdot x_m&=&b_1\\
a_{21}\cdot x_1&+&a_{22}\cdot x_2&+&\dots&+&a_{2m}\cdot x_m&=&b_2\\
a_{31}\cdot x_1&+&a_{32}\cdot x_2&+&\dots&+&a_{3m}\cdot x_m&=&b_3\\
\hdotsfor{9}\\
a_{m1}\cdot x_1&+&a_{m2}\cdot x_2&+&\dots&+&a_{mm}\cdot x_m&=&b_m
\end{matrix}
\right.,
\end{gather}
где $x_1,x_2,\dots,x_m$ --
неизвестные, которые необходимо определить;
$\{a_{ij}\}$ и \linebreak$b_1,b_2,\dots,b_m$ -- 
известные числовые коэффициенты левой и правой 
частей системы уравнений, соответственно.

Решение системы линейных алгебраических уравнений 
\eqref{eq:LS} представляет совокупность $m$ 
действительных или мнимых чисел $\{s_1,s_2,\dots,s_m\}$, 
таких что их соответствующая подстановка вместо 
$\{x_1,x_2,\dots,x_m\}$ в систему обращает все её 
уравнения в тождества:
\begin{gather*}
\left\{
\begin{matrix}
a_{11}\cdot s_1&+&a_{12}\cdot s_2&+&\dots&+&a_{1m}\cdot s_m&\equiv&b_1\\
a_{21}\cdot s_1&+&a_{22}\cdot s_2&+&\dots&+&a_{2m}\cdot s_m&\equiv&b_2\\
a_{31}\cdot s_1&+&a_{32}\cdot s_2&+&\dots&+&a_{3m}\cdot s_m&\equiv&b_3\\
\hdotsfor{9}\\
a_{m1}\cdot s_1&+&a_{m2}\cdot s_2&+&\dots&+&a_{mm}\cdot s_m&\equiv&b_m
\end{matrix}
\right.,
\end{gather*}

Система линейных алгебраических уравнений \eqref{eq:LS}
может быть представлена в более компактной 
матричной форме:
\begin{gather}\label{eq:LSM}
\mathbf{A}\cdot\vect{x}=\vect{b},
\end{gather}
где $\mathbf{A}$ -- квадратная матрица $m\times m$,
$\vect{x}$ и $\vect{b}$ -- искомый вектор неизвестных и
заданный вектор размерности $1\times m$, правых частей
системы уравнений:
\begin{gather*}
\mathbf{A}=
\begin{pmatrix}
a_{11}&a_{12}&\cdots&a_{1m}\\
a_{21}&a_{22}&\cdots&a_{2m}\\
\vdots&\vdots&\ddots&\vdots\\
a_{m1}&a_{m2}&\cdots&a_{mm}\\
\end{pmatrix},
\quad
\vect{x}=\begin{pmatrix}x_1\\x_2\\\vdots\\x_m\end{pmatrix},
\quad
\vect{b}=\begin{pmatrix}b_1\\b_2\\\vdots\\b_m\end{pmatrix}.
\end{gather*}

Теорема Кронекера--Капелли устанавливает необходимое и 
достаточное условие совместности системы 
линейных алгебраических уравнений посредством 
свойств матричных представлений: 
система совместна тогда и только тогда, когда ранг её матрицы 
совпадает с рангом расширенной матрицы,
полученной путем добавления столбца правых частей $\vect{b}$
матрице системы $\mathbf{A}$: 
\begin{gather}\label{eq:LSAM}
(\mathbf{A}\,\vert\,\vect{b})=
\begin{pmatrix}[cccc|c]
a_{11}&a_{12}&\cdots&a_{1m}&b_1\\
a_{21}&a_{22}&\cdots&a_{2m}&b_2\\
\vdots&\vdots&\ddots&\vdots&\vdots\\
a_{m1}&a_{m2}&\cdots&a_{mm}&b_m\\
\end{pmatrix}
\end{gather}

Преимущество расширенной матрицы заключается в 
возможности выполнения тех же операций с вектором 
правых частей системы уравнений, что и со строками
матрицы.

Предполагается, что определитель матрицы $\mathbf{A}$ 
отличен от нуля $\det\mathbf{A}\ne0$,
так что решение $\vect{x}=(x_1,x_2,\dots,x_m)^\mathrm{T}$
существует и единственно.
Систему линейных уравнений можно решить 
по крайней мере двумя способами:
либо воспользовавшись \emph{формулами Крамера}, 
либо методом последовательного исключения неизвестных 
(\emph{методом Гаусса}).
При больших порядка матрицы $m$ способ Крамера, 
основанный на вычислении определителей,
требует порядка $m!$ арифметических действий, 
в то время как метод Гаусса -- $O(m^3)$ действий.

Для большинства вычислительных задач характерным 
является большой порядок матрицы 
$\mathbf{A}$ ($m\approx10^2\dots10^5$),
поэтому метод Гаусса в различных вариантах широко 
используется при решении задач линейной алгебры на ЭВМ.

%
%	Прямой ход метода Гаусса
%
\emptyline
\subsection{Прямой ход метода Гаусса}
Метод Гаусса состоит в последовательном исключении 
неизвестных $x_i$ из системы линейных уравнений
\eqref{eq:LSAM}.
Например, предположим, что $a_{11}\ne0$, 
тогда разделим первое уравнение системы на $a_{11}$, 
и в результате получим:
\begin{gather*}
\begin{pmatrix}[cccc|c]
1&c_{12}&\cdots&c_{1m}&y_1\\
a_{21}&a_{22}&\cdots&a_{2m}&b_2\\
\vdots&\vdots&\ddots&\vdots&\vdots\\
a_{m1}&a_{m2}&\cdots&a_{mm}&b_m\\
\end{pmatrix},
\end{gather*}
где $c_{1j}$ и $y_1$ -- нормированные коэффициенты 
1-ой строки и правой части 1-го уравнения, соответственно:
\begin{gather*}
c_{1j}=\dfrac{a_{1j}}{a_{11}}\quad (j=2,3,\ldots,m),\quad
y_1=\dfrac{b_1}{a_{11}}.
\end{gather*}

Последовательно умножим первое уравнение системы на $a_{i1}$ и 
вычтем полученное уравнение из каждого $i$-го уравнения системы 
$i=2,3,\dots,m$. В результате получим следующую матрицу:
\begin{gather*}
\begin{pmatrix}[crcr|r]
1&c_{12}&\cdots&c_{1m}&y_1\\
0&a_{22}-a_{21}\cdot c_{12}&\cdots&a_{2m}-a_{21}\cdot c_{1m}&b_2-a_{21}\cdot y_1\\
\vdots&\vdots&\ddots&\vdots&\vdots\\
0&a_{m2}-a_{m1}\cdot c_{12}&\cdots&a_{mm}-a_{m1}\cdot c_{1m}&b_m-a_{m1}\cdot y_1\\
\end{pmatrix}.
\end{gather*}

Запишем полученную матрицу в более компактном виде:
\begin{gather*}
\begin{pmatrix}[cccc|r]
1&c_{12}&\cdots&c_{1m}&y_1\\
0&a_{22}^{(1)}&\cdots&a_{2m}^{(1)}&b_2^{(1)}\\
\vdots&\vdots&\ddots&\vdots&\vdots\\
0&a_{m2}^{(1)}&\cdots&a_{mm}^{(1)}&b_m^{(1)}\\
\end{pmatrix},
\end{gather*}
где $a_{ij}^{(1)}$ и $b_i^{(1)}$
-- модифицированные коэффициенты при неизвестных и правой части,
соответственно.
\begin{gather*}
a_{ij}^{(1)}=(a_{ij}-a_{i1}\cdot c_{1j}), \quad 
b_i^{(1)}=(b_i-a_{i1}\cdot y_{1}), \quad
(i,j=2,3,\dots,m)
\end{gather*}

Если $a_{22}^{(1)}\ne0$, то в модифицированной системе 
аналогично можно исключить неизвестное $x_2$. 
Для этого можно разделить второе уравнение системы 
на коэффициент при второй неизвестной $a_{22}^{(1)}$,
и в результате получим:
\begin{gather*}
\begin{pmatrix}[cccc|r]
1&c_{12}&\cdots&c_{1m}&y_1\\
0&1&\cdots&c_{2m}&y_2\\
\vdots&\vdots&\ddots&\vdots&\vdots\\
0&a_{m2}^{(1)}&\cdots&a_{mm}^{(1)}&b_m^{(1)}\\
\end{pmatrix},
\end{gather*}
где $c_{2j}$ и $y_2$ -- нормированные коэффициенты 
2-ой строки и правой части 2-го уравнения, соответственно.
\begin{gather*}
c_{2j}=\dfrac{a_{2j}^{(1)}}{a_{22}^{(1)}}
\quad(j=3,4,\ldots,m),\quad
y_2=\dfrac{b_1^{(1)}}{a_{22}^{(1)}}.
\end{gather*}

Последовательно умножим второе уравнение системы 
на $a_{i2}^{(1)}$ и вычтем полученное уравнение 
из каждого $i$-го уравнения системы $i=3,4,\ldots,m$.
В результате расширенная матрица примет вид:
\begin{gather*}
\begin{pmatrix}[crcr|r]
1&c_{12}&\cdots&c_{1m}&y_1\\
0&1&\cdots&c_{2m}&y_2\\
\vdots&\vdots&\ddots&\vdots&\vdots\\
0&0&\cdots&a_{mm}^{(1)}-a_{m2}^{(1)}\cdot c_{2m}&b_m^{(1)}-a_{m2}^{(1)}\cdot y_2\\
\end{pmatrix}
\end{gather*}

или в более компактном виде:
\begin{gather*}
\begin{pmatrix}[cccc|r]
1&c_{12}&\cdots&c_{1m}&y_1\\
0&1&\cdots&c_{2m}&y_2\\
\vdots&\vdots&\ddots&\vdots&\vdots\\
0&0&\cdots&a_{mm}^{(2)}&b_m^{(2)}\\
\end{pmatrix}
\end{gather*}
где $a_{ij}^{(2)}$ и $b_i^{(2)}$
-- повторно модифицированные коэффициенты при неизвестных 
и правой части, соответственно:
\begin{gather*}
a_{ij}^{(2)}=(a_{ij}^{(1)}-a_{i2}^{(1)}\cdot c_{2j}),\quad
b_i^{(2)}=(b_i^{(1)}-a_{i2}^{(1)}\cdot y_{2}),\quad
(i,j=3,4,\dots,m)
\end{gather*}

Исключая таким же образом неизвестные $x_3,x_4,\dots,x_m$, 
исходная система линейных уравнений приводится 
к эквивалентному виду:
\begin{gather}\label{eq:UMatrix}
\begin{pmatrix}[cccc|r]
1&c_{12}&\cdots&c_{1m}&y_1\\
0&1&\cdots&c_{2m}&y_2\\
\vdots&\vdots&\ddots&\vdots&\vdots\\
0&0&\cdots&1&y_m\\
\end{pmatrix}
\end{gather}

%
%	Обратный ход метода Гаусса
%
\emptyline
\subsection{Обратный ход метода Гаусса}
Обратный ход заключается в нахождении неизвестных 
$x_1,x_2,\dots,x_m$ полученной эквивалентной системы 
в прямом ходе метода Гаусса.
Поскольку расширенная матрица системы \eqref{eq:UMatrix}
имеет треугольный вид, 
то можно последовательно найти все неизвестные 
$x_m,x_{m-1}, \dots,x_1$:
\begin{gather*}
\left\{\begin{array}{lclclcl}
x_m&=&y_m\\
x_{m-1}&=&y_{m-1}&-&c_{m-1,m}\cdot x_m\\
x_{m-2}&=&y_{m-2}&-&c_{m-2,m-1}\cdot x_{m-1}&-&c_{m-2,m}\cdot x_m\\
\hdotsfor{1}&=&\hdotsfor{5}\\
x_1&=&y_1&-&\sum\limits_{j=2}^{m} c_{1j}\cdot x_j\\
\end{array}\right.
\end{gather*}

Общие формулы обратного хода имеют вид:
\begin{gather*}
x_m=y_m,\quad 
x_i=y_i - \sum\limits_{j=i+1}^{m} c_{ij}\cdot x_j,
\quad i=(m-1,m-2,\dots,1)
\end{gather*}

%
%	Пример решения методом Гаусса
%
\emptyline
\subsection{Численное решение системы линейных 
алгебраических уравнений методом Гаусса}
Представим систему линейных алгебраических уравнений
в матричном виде и запишем расширенную матрицу этой системы,
полученную путем добавления к матрице системы
столбца правой части уравнений:
\begin{gather*}
\left\{\begin{matrix}
2x_1&+&3x_2&+&x_3&=&10\\
4x_1&+&5x_2&+&6x_3&=&31\\
3x_1&+&x_2&+&5x_3&=&22\\
\end{matrix}\right.
\quad\iff\quad
\begin{pmatrix}[ccc|c]
\pgfkeys{/main=2}&3&1&10\\
4&5&6&31\\
3&1&5&22\\
\end{pmatrix}
\end{gather*}

% Прямой ход метода Гаусса
\emph{Прямой ход метода Гаусса.}
\begin{enumerate}
\item
Разделим каждую строку матрицы на значение её
элемента в первом столбце, т.е. первую строку делим на $2$,
вторую на $4$, третью на 3:
\begin{gather*}
\begin{pmatrix}[ccc|c]
\pgfkeys{/main=2}&3&1&10\\
4&5&6&31\\
3&1&5&22\\
\end{pmatrix}\quad\to\quad
% результат
\begin{pmatrix}[ccc|c]
\pgfkeys{/main=1}&\frac{3}{2}&\frac{1}{2}&\frac{10}{2}\\
\pgfkeys{/one}&\frac{5}{4}&\frac{6}{4}&\frac{31}{4}\\
\pgfkeys{/one}&\frac{1}{3}&\frac{5}{3}&\frac{22}{3}\\
\end{pmatrix}
\end{gather*}

Вычитаем из второй и третьей строк матрицы её первую строку:
\begin{gather*}
\begin{pmatrix}[ccc|c]
\pgfkeys{/main=1}&\frac{3}{2}&\frac{1}{2}&\frac{10}{2}\\
1&\frac{5}{4}&\frac{6}{4}&\frac{31}{4}\\
1&\frac{1}{3}&\frac{5}{3}&\frac{22}{3}\\
\end{pmatrix}\quad\to\quad
% результат
\begin{pmatrix}[ccc|c]
\pgfkeys{/main=1}&\frac{3}{2}&\frac{1}{2}&\frac{10}{2}\\
\pgfkeys{zero}&-\frac{1}{4}&1&\frac{11}{4}\\
\pgfkeys{zero}&-\frac{7}{6}&\frac{7}{6}&\frac{7}{3}\\
\end{pmatrix}
\end{gather*}

\item
Разделим вторую строку матрицы на $-\frac{1}{4}$, 
третью строку на $-\frac{7}{6}$:
\begin{gather*}
\begin{pmatrix}[ccc|c]
\pgfkeys{/main=1}&\frac{3}{2}&\frac{1}{2}&\frac{10}{2}\\
\pgfkeys{zero}&\pgfkeys{/main={-\frac{1}{4}}}&1&\frac{11}{4}\\
\pgfkeys{zero}&-\frac{7}{6}&\frac{7}{6}&\frac{7}{3}\\
\end{pmatrix}\quad\to\quad
% результат
\begin{pmatrix}[ccc|c]
\pgfkeys{/main=1}&\frac{3}{2}&\frac{1}{2}&\frac{10}{2}\\
\pgfkeys{zero}&\pgfkeys{/main=1}&-4&-11\\
\pgfkeys{zero}&\pgfkeys{/one}&-1&-2\\
\end{pmatrix}
\end{gather*}

Вычитаем из третьей строки матрицы её вторую строку:
\begin{gather*}
\begin{pmatrix}[ccc|c]
\pgfkeys{/main=1}&\frac{3}{2}&\frac{1}{2}&\frac{10}{2}\\
\pgfkeys{zero}&\pgfkeys{/main=1}&-4&-11\\
\pgfkeys{zero}&1&-1&-2\\
\end{pmatrix}\quad\to\quad
% результат
\begin{pmatrix}[ccc|c]
\pgfkeys{/main=1}&\frac{3}{2}&\frac{1}{2}&\frac{10}{2}\\
\pgfkeys{zero}&\pgfkeys{/main=1}&-4&-11\\
\pgfkeys{zero}&\pgfkeys{zero}&3&9\\
\end{pmatrix}
\end{gather*}

\item
Разделим третью строку матрицы на $3$:
\begin{gather*}
\begin{pmatrix}[ccc|c]
1&\frac{3}{2}&\frac{1}{2}&\frac{10}{2}\\
\pgfkeys{zero}&1&-4&-11\\
\pgfkeys{zero}&\pgfkeys{zero}&3&9\\
\end{pmatrix}\quad\to\quad
% результат
\begin{pmatrix}[ccc|c]
\pgfkeys{/main=1}&\frac{3}{2}&\frac{1}{2}&\frac{10}{2}\\
\pgfkeys{zero}&\pgfkeys{/main=1}&-4&-11\\
\pgfkeys{zero}&\pgfkeys{zero}&\pgfkeys{/main=1}&3\\
\end{pmatrix}
\end{gather*}
\end{enumerate}

\emph{Обратный ход метода Гаусса}.\par
Последовательно определяем неизвестные
в обратном порядке их следования
$x_3\to x_2\to x_1$:
\begin{enumerate}
\item
Из третьего уравнения системы (третья строка матрицы)
определяем неизвестное $x_3$:
\begin{gather*}
x_3=3
\end{gather*}
\item
Из второго уравнения системы (вторая строка матрицы)
определяем неизвестное $x_2$:
\begin{gather*}
x_2-4\,x_3=-11\quad\to\quad
x_2=4\,x_3-11\\
x_2=4\cdot3-11=1
\end{gather*}
\item
Из первого уравнения системы (первая строка матрицы)
определяем неизвестное $x_1$:
\begin{gather*}
x_1+\frac{3}{2}\,x_2+\frac{1}{2}\,x_3=\frac{10}{2}\quad\to\quad
x_1=-\frac{3}{2}\,x_2-\frac{1}{2}\,x_3+5\\[1ex]
x_1=-\frac{3}{2}\cdot{1}-\frac{1}{2}\cdot{3}+5=2
\end{gather*}

Таким образом, найдено решение системы:
\begin{gather*}
\vect{s}=\begin{pmatrix}2\\1\\3\end{pmatrix}
\qquad\iff\qquad
\vect{s}=(2,1,3)^\mathrm{T}
\end{gather*}
\end{enumerate}

Проведём \emph{проверку решения} системы уравнений
методом прямой подстановки найденного 
вектора неизвестных $\vect{s}=(2,1,3)^\mathrm{T}$ 
в исходную систему уравнений:
\begin{gather*}
\mathbf{A}\cdot\vect{s}=\vect{b}
\quad\iff\quad
\begin{pmatrix}[ccc|c]
2&3&1&10\\
4&5&6&31\\
3&1&5&22\\
\end{pmatrix}
\cdot\begin{pmatrix}2\\1\\3\\\end{pmatrix}
\end{gather*}

После умножения расширенной матрицы
системы уравнений на вектор найденного решения
получим тождество:
\begin{gather*}
\begin{pmatrix}[c|c]
2\cdot2+3\cdot1+1\cdot3&10\\
4\cdot2+5\cdot1+6\cdot3&31\\
3\cdot2+1\cdot1+5\cdot3&22\\
\end{pmatrix}
=
\begin{pmatrix}[c|c]
10&10\\
31&31\\
22&22\\
\end{pmatrix}
\end{gather*}

%
%	Метод Гаусса с выбором главного элемента
%
\emptyline
\subsection{Метод Гаусса с выбором главного элемента}
На практике, часто может оказаться, что система \eqref{eq:LSM}
имеет единственное решение, хотя какой-либо 
из угловых миноров матрицы $\mathbf{A}$ равен нулю. 
Кроме того, заранее обычно неизвестно, 
все ли угловые миноры матрицы $\mathbf{A}$ отличны от нуля.
В этих случаях обычный метод Гаусса может оказаться 
\emph{непригодным}.
Избежать указанных трудностей позволяет метод Гаусса 
с выбором главного элемента.

\emph{Основная идея метода} состоит в том, чтобы 
на очередном шаге исключать не следующее по номеру неизвестное,
а то неизвестное, коэффициент при котором 
является \emph{наибольшим по модулю}.
Таким образом, в качестве ведущего элемента здесь выбирается 
\alert{главный}, т.е. наибольший по модулю элемент.
Поэтому, если $\det\mathbf{A}\ne0$, то в процессе вычислений 
не будет происходить деление на нуль.

На практике чаще всего применяется и метод Гаусса 
с выбором главного элемента по всей матрице, 
когда в качестве ведущего выбирается максимальный 
по модулю элемент \emph{среди всех элементов} матрицы системы.

\emptyline
\subsection{Численное решение системы линейных 
алгебраических уравнений методом Гаусса
с выбором главного элемента по всей матрице}
Рассмотрим на примере решение системы линейных уравнений
методом Гаусса с выбором главного элемента по всей
матрице системы:
\begin{gather*}
\left\{\begin{matrix}
2x_1&+&3x_2&+&x_3&=&10\\
4x_1&+&5x_2&+&6x_3&=&31\\
3x_1&+&x_2&+&5x_3&=&22\\
\end{matrix}\right.
\quad\iff\quad
\begin{pmatrix}[ccc|c]
2&3&1&10\\
4&5&6&31\\
3&1&5&22\\
\end{pmatrix}
\end{gather*}

\emph{Прямой ход метода Гаусса с выбором главного элемента}.
\begin{enumerate}
\item
Выбираем максимальный по модулю элемент в матрице 
(главный элемент) во второй строке, третьем столбце
(выделен цветом):
\begin{gather*}
\begin{pmatrix}[ccc|c]
2&3&1&10\\
4&5&\pgfkeys{/main=6}&31\\
3&1&5&22\\
\end{pmatrix}
\end{gather*}
Разделим каждую строку матрицы на значение элемента
матрицы в столбце главного элемента, т.е.
первую строку делим на $1$, вторую строку на $6$,
а третью строку на $5$:
\begin{gather*}
\begin{pmatrix}[ccc|c]
2&3&1&10\\
4&5&\pgfkeys{/main=6}&31\\
3&1&5&22\\
\end{pmatrix}\quad\to\quad
% результат
\begin{pmatrix}[ccc|c]
2&3&\pgfkeys{/one}&10\\
\frac{4}{6}&\frac{5}{6}&\pgfkeys{/main=1}&\frac{31}{6}\\
\frac{3}{5}&\frac{1}{5}&\pgfkeys{/one}&\frac{22}{5}\\
\end{pmatrix}
\end{gather*}

Вычитаем из первой и третьей строки вторую строку:
\begin{gather*}
\begin{pmatrix}[ccc|c]
2&3&1&10\\
\frac{4}{6}&\frac{5}{6}&\pgfkeys{/main=1}&\frac{31}{6}\\
\frac{3}{5}&\frac{1}{5}&1&\frac{22}{5}\\
\end{pmatrix}\quad\to\quad
% результат
\begin{pmatrix}[ccc|c]
\frac{4}{3}&\frac{13}{6}&\pgfkeys{/zero}&\frac{29}{6}\\
\frac{4}{6}&\frac{5}{6}&\pgfkeys{/main=1}&\frac{31}{6}\\
-\frac{1}{15}&-\frac{19}{30}&\pgfkeys{/zero}&-\frac{23}{30}\\
\end{pmatrix}
\end{gather*}

\item
Исключаем из рассмотрения строку с текущим главным 
элементом (вторую строку) и выбираем новый 
главный элемент матрицы (первая строка, второй столбец): 
\begin{gather*}
\begin{pmatrix}[ccc|c]
\frac{4}{3}&\pgfkeys{/main=\frac{13}{6}}&\pgfkeys{/zero}&\frac{29}{6}\\
\frac{4}{6}&\frac{5}{6}&\pgfkeys{/main=1}&\frac{31}{6}\\
-\frac{1}{15}&-\frac{19}{30}&\pgfkeys{/zero}&-\frac{23}{30}\\
\end{pmatrix}
\end{gather*}

Делим каждую строку матрицы на значение элемента
матрицы в столбце главного элемента, т.е.
первую строку делим на $\frac{13}{6}$, а третью строку на $-\frac{19}{30}$:
\begin{gather*}
\begin{pmatrix}[ccc|c]
\frac{4}{3}&\pgfkeys{/main=\frac{13}{6}}&\pgfkeys{/zero}&\frac{29}{6}\\
\frac{4}{6}&\frac{5}{6}&\pgfkeys{/main=1}&\frac{31}{6}\\
-\frac{1}{15}&-\frac{19}{30}&\pgfkeys{/zero}&-\frac{23}{30}\\
\end{pmatrix}\quad\to\quad
% результат
\begin{pmatrix}[ccc|c]
\frac{8}{13}&\pgfkeys{/main=1}&\pgfkeys{/zero}&\frac{29}{13}\\
\frac{4}{6}&\frac{5}{6}&\pgfkeys{/main=1}&\frac{31}{6}\\
\frac{2}{19}&\pgfkeys{/one}&\pgfkeys{/zero}&\frac{23}{19}\\
\end{pmatrix}
\end{gather*}

Вычитаем из третьей строки матрицы её первую строку:
\begin{gather*}
\begin{pmatrix}[ccc|c]
\frac{8}{13}&\pgfkeys{/main=1}&\pgfkeys{/zero}&\frac{29}{13}\\
\frac{4}{6}&\frac{5}{6}&\pgfkeys{/main=1}&\frac{31}{6}\\
\frac{2}{19}&1&\pgfkeys{/zero}&\frac{23}{19}\\
\end{pmatrix}\quad\to\quad
% результат
\begin{pmatrix}[ccc|c]
\frac{8}{13}&\pgfkeys{/main=1}&\pgfkeys{/zero}&\frac{29}{13}\\
\frac{4}{6}&\frac{5}{6}&\pgfkeys{/main=1}&\frac{31}{6}\\
-\frac{126}{247}&\pgfkeys{/zero}&\pgfkeys{/zero}&-\frac{252}{247}\\
\end{pmatrix}
\end{gather*}

\item
Исключаем из рассмотрения строку с текущим главным 
элементом (первую) и выбираем новый главный 
элемент матрицы (третья строка, первый столбец): 
\begin{gather*}
\begin{pmatrix}[ccc|c]
\frac{8}{13}&\pgfkeys{/main=1}&\pgfkeys{/zero}&\frac{29}{13}\\
\frac{4}{6}&\frac{5}{6}&\pgfkeys{/main=1}&\frac{31}{6}\\
\pgfkeys{/main=-\frac{126}{247}}&\pgfkeys{/zero}&\pgfkeys{/zero}&-\frac{252}{247}\\
\end{pmatrix}
\end{gather*}

Делим каждую строку матрицы на значение элемента
матрицы в столбце главного элемента, т.е.
третью строку делим на $-\frac{126}{247}$:
\begin{gather*}
\begin{pmatrix}[ccc|c]
\frac{8}{13}&\pgfkeys{/main=1}&\pgfkeys{/zero}&\frac{29}{13}\\
\frac{4}{6}&\frac{5}{6}&\pgfkeys{/main=1}&\frac{31}{6}\\
\pgfkeys{/main=-\frac{126}{247}}&\pgfkeys{/zero}&\pgfkeys{/zero}&-\frac{252}{247}\\
\end{pmatrix}\quad\to\quad
% результат
\begin{pmatrix}[ccc|c]
\frac{8}{13}&\pgfkeys{/main=1}&\pgfkeys{/zero}&\frac{29}{13}\\
\frac{4}{6}&\frac{5}{6}&\pgfkeys{/main=1}&\frac{31}{6}\\
\pgfkeys{/main=1}&\pgfkeys{/zero}&\pgfkeys{/zero}&2\\
\end{pmatrix}
\end{gather*}
\end{enumerate}

\emph{Обратный ход метода Гаусса 
с выбором главного элемента}.\par
Определим неизвестные из уравнений системы 
в обратном порядке следования номеров столбцов главных элементов,
т.е. $x_1\to x_2\to x_3$:
\begin{enumerate}
\item
Из третьего уравнения системы определим неизвестное $x_1$:
\begin{gather*}
x_1=2
\end{gather*}
\item
Из первого уравнения системы определим неизвестное $x_2$:
\begin{gather*}
\frac{8}{13}\,x_1+x_2=\frac{29}{13}\quad\to\quad
x_2=-\frac{8}{13}\,x_1+\frac{29}{13}\\[1ex]
x_2=-\frac{8}{13}\cdot2+\frac{29}{13}=1
\end{gather*}
\item
Из второго уравнения системы определим неизвестное $x_3$:
\begin{gather*}
\frac{4}{6}\,x_1+\frac{5}{6}\,x_2+x_3=\frac{31}{6}\quad\to\quad
x_3=-\frac{4}{6}\,x_1-\frac{5}{6}\,x_2+\frac{31}{6}\\[1ex]
x_3=-\frac{4}{6}\cdot2-\frac{5}{6}\cdot1+\frac{31}{6}=3
\end{gather*}
Таким образом, найдено решение системы линейных уравнений:
\begin{gather*}
\vect{s}=\begin{pmatrix}2\\1\\3\end{pmatrix}
\qquad\iff\qquad
\vect{s}=(2,1,3)^\mathrm{T}
\end{gather*}
\end{enumerate}

%\end{document}
