%\input{../../../preamble}
%\begin{document}
%
%	Метод Гаусса решения систем линейных уравнений
%
\newpage
\section{Метод Гаусса решения систем линейных уравнений}
К решению систем линейных алгебраических уравнений сводится подавляющее 
большинство задач вычислительной математики и многие численные методы основаны 
на решении систем линейных уравнений вида:
\begin{gather*}
\mathbf{A}\cdot\vec{x}=\vec{b},
\end{gather*}
где $\mathbf{A}$ -- квадратная матрица $m\times m$,
$\vec{x}$ и $\vec{b}$ -- искомый вектор неизвестных и
заданный вектор размерности $1\times m$:
\begin{gather*}
\begin{matrix}
\vec{A}=
\begin{pmatrix}
a_{11}&a_{12}&\cdots&a_{1m}\\
a_{21}&a_{22}&\cdots&a_{2m}\\
\vdots&\vdots&\ddots&\vdots\\
a_{m1}&a_{m2}&\cdots&a_{mm}\\
\end{pmatrix},
\hspace{1em}&
\vec{x}=\begin{pmatrix}x_1\\x_2\\\vdots\\x_m\end{pmatrix},
\hspace{1em}&
\vec{b}=\begin{pmatrix}b_1\\b_2\\\vdots\\b_m\end{pmatrix}
\end{matrix}.
\end{gather*}

Предполагается, что определитель матрицы $\mathbf{A}$ 
отличен от нуля $\det\mathbf{A}\ne0$,
так что решение $\vec{x}$ существует и единственно.
Систему линейных уравнений можно решить 
по крайней мере двумя способами:
либо воспользовавшись \emph{формулами Крамера}, 
либо методом последовательного исключения неизвестных 
(\emph{методом Гаусса}).
При больших порядка матрицы $m$ способ Крамера, 
основанный на вычислении определителей,
требует порядка $m!$ арифметических действий, 
в то время как метод Гаусса -- $O(m^3)$ действий.

Для большинства вычислительных задач характерным 
является большой порядок матрицы 
$\mathbf{A}$ ($m\approx10^2\dots10^5$),
поэтому метод Гаусса в различных вариантах широко 
используется при решении задач линейной алгебры на ЭВМ.

Систему линейных алгебраических уравнений 
можно записать в развернутом виде:
\begin{gather*}
\left\{
\begin{matrix}
a_{11}\cdot x_1&+&a_{12}\cdot x_2&+&\cdots&a_{1m}\cdot x_m&=&b_1\\
a_{21}\cdot x_1&+&a_{22}\cdot x_2&+&\cdots&a_{2m}\cdot x_m&=&b_2\\
a_{31}\cdot x_1&+&a_{32}\cdot x_2&+&\cdots&a_{3m}\cdot x_m&=&b_3\\
\vdots&&\vdots&&\ddots&\vdots&&\vdots\\
a_{m1}\cdot x_1&+&a_{m2}\cdot x_2&+&\cdots&a_{mm}\cdot x_m&=&b_m
\end{matrix}
\right.
\end{gather*}

%
%	Прямой ход метода Гаусса
%
\subsection{Прямой ход метода Гаусса}

Метод Гаусса состоит в последовательном исключении 
неизвестных $x_i$ из системы линейных уравнений.
Например, предположим, что $a_{11}\ne0$, 
тогда разделим первое уравнение системы на $a_{11}$, 
и в результате получим:
\begin{gather*}
\left\{
\begin{matrix}
x_1&+&c_{12}\cdot x_2&+&\cdots&c_{1m}\cdot x_m&=&y_1\\
a_{21}\cdot x_1&+&a_{22}\cdot x_2&+&\cdots&a_{2m}\cdot x_m&=&b_2\\
a_{31}\cdot x_1&+&a_{32}\cdot x_2&+&\cdots&a_{3m}\cdot x_m&=&b_3\\
\vdots&&\vdots&&\ddots&\vdots&&\vdots\\
a_{m1}\cdot x_1&+&a_{m2}\cdot x_2&+&\cdots&a_{mm}\cdot x_m&=&b_m\\
\end{matrix}\right.,
\end{gather*}
где $c_{1j}$ и $y_1$ -- нормированные коэффициенты 
1-ой строки и правой части 1-го уравнения, соответственно:
\begin{gather*}
c_{1j}=\dfrac{a_{1j}}{a_{11}}\quad (j=2,3,\ldots,m),\quad
y_1=\dfrac{b_1}{a_{11}}.
\end{gather*}

Последовательно умножим первое уравнение системы на $a_{i1}$ и 
вычтем полученное уравнение из каждого $i$-го уравнения системы 
$i=2,3,\ldots,m$. В результате получим следующую систему уравнений:
\begin{gather*}
\left\{
\begin{matrix}
x_1&+&c_{12}\cdot x_2&+&\cdots&c_{1m}\cdot x_m&=&y_1\\
&&(a_{22}-c_{12}\cdot a_{21})\cdot x_2&+&\cdots&(a_{2m}-c_{1m}\cdot a_{21})\cdot x_m&=&b_2-y_{1}\cdot a_{21}\\
&&(a_{32}-c_{12}\cdot a_{31})\cdot x_2&+&\cdots&(a_{3m}-c_{1m}\cdot a_{31})\cdot x_m&=&b_3-y_{1}\cdot a_{31}\\
&&\vdots&&\ddots&\vdots&&\vdots\\
&&(a_{m2}-c_{12}\cdot a_{m1})\cdot x_2&+&\cdots&(a_{mm}-c_{1m}\cdot a_{m1})\cdot x_m&=&b_m-y_{1}\cdot a_{m1}\\
\end{matrix}\right.
\end{gather*}

Запишем полученную систему уравнений в более компактном виде:
\begin{gather*}
\left\{\begin{matrix}
x_1&+&c_{12}\cdot x_2&+&\cdots&c_{1m}\cdot x_m&=&y_1\\[1ex]
&&a_{22}^{(1)}\cdot x_2&+&\cdots&a_{2m}^{(1)}\cdot x_m&=&b_2^{(1)}\\[1ex]
&&a_{32}^{(1)}\cdot x_2&+&\cdots&a_{3m}^{(1)}\cdot x_m&=&b_3^{(1)}\\[1ex]
&&\vdots&&\ddots&\vdots&&\vdots\\
&&a_{m2}^{(1)}\cdot x_2&+&\cdots&a_{mm}^{(1)}\cdot x_m&=&b_m^{(1)}\\
\end{matrix}\right.,
\end{gather*}
где $a_{ij}^{(1)}$ и $b_i^{(1)}$
-- модифицированные коэффициенты при неизвестных и правой части,
соответственно.
\begin{gather*}
a_{ij}^{(1)}=(a_{ij}-c_{1j}\cdot a_{i1}), \quad 
b_i^{(1)}=(b_i-y_{1}\cdot a_{i1}), \quad
(i,j=2,3,\cdots,m)
\end{gather*}

Если $a_{22}^{(1)}\ne0$, то из модифицированной системы 
аналогично можно исключить неизвестное $x_2$. 
Для этого разделим второе уравнение системы на $a_{22}^{(1)}$,
и в результате получим:
\begin{gather*}
\left\{\begin{matrix}
x_1&+&c_{12}\cdot x_2&+&\cdots&c_{1m}\cdot x_m&=&y_1\\
&&x_2&+&\cdots&c_{2m}\cdot x_m&=&y_2\\[1ex]
&&a_{32}^{(1)}\cdot x_2&+&\cdots&a_{3m}^{(1)}\cdot x_m&=&b_3^{(1)}\\
&&\vdots&&\ddots&\vdots&&\vdots\\
&&a_{m2}^{(1)}\cdot x_2&+&\cdots&a_{mm}^{(1)}\cdot x_m&=&b_m^{(1)}\\
\end{matrix}\right.,
\end{gather*}
где $c_{2j}$ и $y_2$ -- нормированные коэффициенты 
2-ой строки и правой части 2-го уравнения, соответственно.
\begin{gather*}
c_{2j}=\dfrac{a_{2j}^{(1)}}{a_{22}^{(1)}}
\quad(j=3,4,\ldots,m),\quad
y_2=\dfrac{b_1^{(1)}}{a_{22}^{(1)}}.
\end{gather*}

Последовательно умножим второе уравнение системы 
на $a_{i2}^{(1)}$ и вычтем полученное уравнение 
из каждого $i$-го уравнения системы $i=3,4,\ldots,m$.
В результате получим следующую систему уравнений:
\begin{gather*}
\left\{\begin{matrix}
x_1&+&c_{12}\cdot x_2&+&\cdots&c_{1m}\cdot x_m&=&y_1\\
&&x_2&+&\cdots&c_{2m}\cdot x_m&=&y_2\\[1ex]
&&&&\cdots&\left(a_{3m}^{(1)}-c_{2m}\cdot a_{32}^{(1)}\right)\cdot x_m&=&b_3^{(1)}-y_2\cdot a_{32}^{(1)}\\
&&&&\ddots&\vdots&&\vdots\\
&&&&\cdots&\left(a_{mm}^{(1)}-c_{2m}\cdot a_{m2}^{(1)}\right)\cdot x_m&=&b_m^{(1)}-y_2\cdot a_{m2}^{(1)}\\
\end{matrix}\right.,
\end{gather*}
или в более компактном виде:
\begin{gather*}
\left\{\begin{matrix}
x_1&+&c_{12}\cdot x_2&+&\cdots&c_{1m}\cdot x_m&=&y_1\\
&&x_2&+&\cdots&c_{2m}\cdot x_m&=&y_2\\[1ex]
&&&&\cdots&a_{3m}^{(2)}\cdot x_m&=&b_3^{(2)}\\
&&&&\ddots&\vdots&&\vdots\\
&&&&\cdots&a_{mm}^{(2)}\cdot x_m&=&b_m^{(2)}\\
\end{matrix}\right.,
\end{gather*}
где $a_{ij}^{(2)}$ и $b_i^{(2)}$
-- повторно модифицированные коэффициенты при неизвестных 
и правой части, соответственно:
\begin{gather*}
a_{ij}^{(2)}=\left(a_{ij}^{(1)}-c_{2j}\cdot a_{i2}^{(1)}\right),
\quad b_i^{(2)}=\left(b_i^{(1)}-y_{2}\cdot a_{i2}^{(1)}\right),
\quad (i,j=3,4,\cdots,m)
\end{gather*}

Исключая таким же образом неизвестные $x_3,x_4,\cdots,x_m$, 
исходная система линейных уравнений приводится 
к эквивалентному виду:
\begin{gather*}
\left\{\begin{matrix}
x_1&+&c_{12}\cdot x_2&+&c_{13}\cdot x_3&+&\cdots&c_{1m}\cdot x_m&=&y_1\\
&&x_2&+&c_{23}\cdot x_3&+&\cdots&c_{2m}\cdot x_m&=&y_2\\
&&&&x_3&+&\cdots&c_{3m}\cdot x_m&=&y_3\\
&&&&&&\ddots&\vdots&=&\vdots\\
&&&&&&&x_m&=&y_m\\
\end{matrix}\right.
\end{gather*}

%
%	Обратный ход метода Гаусса
%
\subsection{Обратный ход метода Гаусса}
Обратный ход заключается в нахождении неизвестных 
$x_1,x_2,\dots,x_m$ полученной эквивалентной системы 
в прямом ходе метода Гаусса.
Поскольку матрица системы имеет треугольный вид, 
то можно последовательно, начиная с $x_m$, найти 
все неизвестные $x_{m-1}, x_{m-2},\dots,x_1$:
\begin{gather*}
%\begin{array}{rcl}
%x_m&=&y_m\\
%x_{m-1}&=&y_{m-1}-c_{m-1,m}\cdot x_m\\
%x_{m-2}&=&y_{m-2}-c_{m-2,m-1}\cdot x_{m-1}-c_{m-2,m}\cdot x_m\\
%%x_{m-3}&=y_{m-3}-c_{m-3,m-2}\cdot x_{m-2}-c_{m-3,m-1}\cdot x_{m-1}-c_{m-3,m}\cdot x_m\\
%%\vdots&=&\cdots\\
%\cdots&=&\cdots\\
%x_1&=&y_1 - \sum\limits_{j=2}^{m} c_{1j}\cdot x_j
%\end{array}
\left\{\begin{array}{lclclcl}
x_m&=&y_m\\
x_{m-1}&=&y_{m-1}&-&c_{m-1,m}\cdot x_m\\
x_{m-2}&=&y_{m-2}&-&c_{m-2,m-1}\cdot x_{m-1}&-&c_{m-2,m}\cdot x_m\\
\cdots&=&\hdotsfor{5}\\
x_1&=&y_1&-&\sum\limits_{j=2}^{m} c_{1j}\cdot x_j\\
\end{array}\right.
\end{gather*}

Общие формулы обратного хода имеют вид:
\begin{gather*}
x_i=y_i - \sum\limits_{j=i+1}^{m} c_{ij}\cdot x_j,
\quad i=(m-1,m-2,\cdots,1), \quad x_m=y_m 
\end{gather*}

%
%	Метод Гаусса с выбором главного элемента
%
\subsection{Метод Гаусса с выбором главного элемента}

На практике, часто может оказаться, что система 
имеет единственное решение, хотя какой-либо 
из угловых миноров матрицы $\mathbf{A}$ равен нулю. 
\begin{gather*}
\mathbf{A}\cdot\vec{x}=\vec{b}.
\end{gather*}

Кроме того, заранее обычно неизвестно, 
все ли угловые миноры матрицы $\mathbf{A}$ отличны от нуля.
В этих случаях обычный метод Гаусса может оказаться 
\emph{непригодным}.
Избежать указанных трудностей позволяет метод Гаусса 
с выбором главного элемента.

\textbf{Основная идея метода} состоит в том, чтобы 
на очередном шаге исключать не следующее по номеру неизвестное,
а то неизвестное, коэффициент при котором 
является \emph{наибольшим по модулю}.
Таким образом, в качестве ведущего элемента здесь выбирается 
\alert{главный}, т.е. наибольший по модулю элемент.
Поэтому, если $\det(\mathbf{A})\ne0$, то в процессе вычислений 
не будет происходить деление на нуль.

На практике чаще всего применяется и метод Гаусса 
с выбором главного элемента по всей матрице, 
когда в качестве ведущего выбирается максимальный 
по модулю элемент \emph{среди всех элементов} матрицы системы.

%
%	Пример решения методом Гаусса
%
\subsection{Численное решение системы линейных 
алгебраических уравнений методом Гаусса}

Пример численного решения системы линейных 
алгебраических уравнений методом Гаусса:
\begin{gather*}\left\{\begin{matrix}
2x_1&+&3x_2&+&x_3&=&10\\
4x_1&+&5x_2&+&6x_3&=&31\\
3x_1&+&x_2&+&5x_3&=&22\\
\end{matrix}\right.\end{gather*}

\emph{Прямой ход метода Гаусса.}
\begin{enumerate}
\item
Разделим 1-ю строку на коэффициент при $x_1$ (2)
\begin{gather*}\left\{\begin{matrix}
x_1&+&\dfrac{3}{2}x_2&+&\dfrac{1}{2}x_3&=&5\\
4x_1&+&5x_2&+&6x_3&=&31\\
3x_1&+&x_2&+&5x_3&=&22\\
\end{matrix}\right.
\end{gather*}

\item
Умножим 1-ю строку на коэффициент при $x_1$ (4) 
во 2-ом уравнении
\begin{gather*}\begin{matrix}
4x_1&+&6x_2&+&2x_3&=&20\\
\end{matrix}\end{gather*}
и вычтем результат умножения из 2-й строки
\begin{gather*}\left\{\begin{matrix}
x_1&+&\dfrac{3}{2}x_2&+&\dfrac{1}{2}x_3&=&5\\
&-&x_2&+&4x_3&=&11\\
3x_1&+&x_2&+&5x_3&=&22\\
\end{matrix}\right.\end{gather*}

\item
Умножим 1-ю строку на коэффициент при $x_1$ (3) 
в 3-ем уравнении
\begin{gather*}\begin{matrix}
3x_1&+&\dfrac{9}{2}x_2&+&\dfrac{3}{2}x_3&=&15\\
\end{matrix}\end{gather*}
и вычтем результат умножения из 3-й строки
\begin{gather*}\left\{\begin{matrix}
x_1&+&\dfrac{3}{2}x_2&+&\dfrac{1}{2}x_3&=&5\\[1ex]
&-&x_2&+&4x_3&=&11\\[1ex]
&-&\dfrac{7}{2}x_2&+&\dfrac{7}{2}x_3&=&7\\
\end{matrix}\right.\end{gather*}

\item
Разделим 2-ю строку на коэффициент при $x_2$ (-1)
\begin{gather*}\left\{\begin{matrix}
x_1&+&\dfrac{3}{2}x_2&+&\dfrac{1}{2}x_3&=&5\\[1ex]
&&x_2&-&4x_3&=&-11\\[1ex]
&-&\dfrac{7}{2}x_2&+&\dfrac{7}{2}x_3&=&7\\
\end{matrix}\right.\end{gather*}

\item
Умножим 2-ю строку на коэффициент при 
$x_2~\left({-\frac{7}{2}}\right)$ в 3-ем уравнении
\begin{gather*}\begin{matrix}
&-&\dfrac{7}{2}x_2&+&14x_3&=&\dfrac{77}{2}\\
\end{matrix}\end{gather*}
и вычтем результат умножения из 3-й строки
\begin{gather*}\left\{\begin{matrix}
x_1&+&\dfrac{3}{2}x_2&+&\dfrac{1}{2}x_3&=&5\\[1ex]
&&x_2&-&4x_3&=&-11\\[1ex]
&&&-&\dfrac{21}{2}x_3&=&-\dfrac{63}{2}\\
\end{matrix}\right.\end{gather*}

\item
Разделим 3-ю строку на коэффициент при 
$x_3~\left({-\frac{21}{2}}\right)$
\begin{gather*}\left\{\begin{matrix}
x_1&+&\dfrac{3}{2}x_2&+&\dfrac{1}{2}x_3&=&5\\[1ex]
&&x_2&-&4x_3&=&-11\\[1ex]
&&&&x_3&=&3\\
\end{matrix}\right.\end{gather*}
\end{enumerate}

\emph{Обратный ход метода Гаусса.}

Последовательно определяем неизвестные $x_3,x_2,x_1$:
\begin{gather*}
\begin{array}{rl}
x_3&=3\\
x_2&=-11+4x_3=-11+4\cdot3=1\\
x_1&=5-\dfrac{3}{2}x_2-\dfrac{1}{2}x_3=5-\dfrac{3}{2}\cdot1-\dfrac{1}{2}\cdot3=2
\end{array}
\end{gather*}

\emph{Ответ}: решение системы линейных уравнений:
$x_1=2$, $x_2=1$, $x_3=3$.

Проведём \emph{проверку решения} системы уравнений
методом прямой подстановки. Для этого подставим 
найденный вектор неизвестных $\vec{x}=(2,1,3)$ 
в исходную систему уравнений:
\begin{gather*}\left\{\begin{matrix}
2\cdot 2&+&3\cdot 1&+&1\cdot 3&=&10\\
4\cdot 2&+&5\cdot 1&+&6\cdot3&=&31\\
3\cdot 2&+&1&+&5\cdot3&=&22\\
\end{matrix}\right.\end{gather*}

Проводим арифметические действия и получаем тождества 
для каждого уравнения системы:
\begin{gather*}
\begin{matrix}
\left\{\begin{matrix}
4&+&3&+&3&=&10\\
8&+&5&+&18&=&31\\
6&+&1&+&15&=&22\\
\end{matrix}\right.
&\implies&
\left\{\begin{matrix}
10&=&10\\
31&=&31\\
22&=&22\\
\end{matrix}\right.
\end{matrix}
\end{gather*}

%\end{document}
