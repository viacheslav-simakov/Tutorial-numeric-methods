\newpage
%
%	Численное интегрирование
%
\section{Численное интегрирование}
Если функция $f(x)$ непрерывна на отрезке $x\in[a,b]$ и 
известна ее первообразная $F(x)$, то определенный интеграл 
от этой функции в пределах от $a$ до $b$ может быть вычислен 
по формуле Ньютона -- Лейбница:
\begin{gather*}
\int\limits_{a}^{b}f(x)dx=F(b)-F(a),
\end{gather*}
где $F^{\prime}(x)=f(x)$ -- 
первообразная подынтегральной функции $f(x)$.

Однако во многих случаях первообразная функция $F(x)$ 
не может быть найдена с помощью элементарных средств 
или является слишком сложной, поэтому 
вычисление определенного интеграла 
может быть затруднительным или даже практически невозможным. 

Кроме того, на практике подынтегральная функция $f(x)$ часто 
задается таблично и тогда само понятие первообразной теряет смысл. 
Аналогичные вопросы возникают при вычислении кратных  
интегралов. Поэтому важное значение имеют приближенные 
и в первую очередь численные методы вычисления 
определенных интегралов. 

\emph{Задача численного интегрирования} функции заключается 
в вычислении значения определенного интеграла на основании ряда  
значений подынтегральной функции $f(x)$.

Обычный прием численного вычисления интеграла состоит 
в том, что данную функцию $f(x)$ на рассматриваемом отрезке 
$x\in[a, b]$ заменяют интерполирующей или аппроксимирующей  
функцией $\varphi(x)$ простого вида (например, полиномом), 
а затем приближенно полагают:
\begin{gather*}
\int\limits_{a}^{b}f(x)dx\approx \int\limits_{a}^{b}\varphi(x)dx
\end{gather*}

Далее рассматриваются способы приближенного вычисления 
определенных интегралов вида:
\begin{gather*}
I=\int\limits_{a}^{b}\varphi(x)dx,
\end{gather*}
основанные на замене интеграла конечной суммой:
\begin{gather*}
I\approx\sum\limits_{i=0}^{n}c_{i}\cdot\varphi(x_i),
\end{gather*}
где $c_{i}$ -- числовые коэффициенты квадратурной формулы; 
$x_i$ -- узлы квадратурной формулы, т.е. точки отрезка 
$[a, b], (i= 0,1,\cdots,n)$.

На основании свойств определенных интегралов, 
$I$ можно представить в виде суммы интегралов
по частичным отрезкам:
\begin{gather*}
\int\limits_{a}^{b}f(x)dx=
\sum\limits_{i=1}^{n}\int\limits_{x_{i-1}}^{x_i}f(x)dx
\end{gather*}

Поэтому, для построения формулы численного интегрирования 
на всем отрезке $[a,b]$ достаточно построить квадратурную формулу 
на частичном отрезке $[x_{i-1},x_i]$ для интеграла:
\begin{gather*}
S_i=\int\limits_{x_{i-1}}^{x_i}f(x)dx
\end{gather*}

%
% Формула прямоугольников
%
\subsection{Формула прямоугольников}
В методе прямоугольниковов на частичном отрезке 
подынтегральная функция заменяется полиномом нулевой степени,
то есть константу:
\begin{gather*}
f(x)\approx L_0(x)=\const
\end{gather*}

С геометрической точки зрения, в методе прямоугольников
площадь криволинейной трапеции (интеграл от функции) на
частичном отрезке заменяется площадью прямоугольника,
ширина которого будет определяться расстоянием между 
соответствующими соседними узлами интегрирования,
а высота -- значением подынтегральной функции в этих узлах.

В зависимости от выбора узла сетки $\{x_i\}$ для аппроксимации 
подынтегральной функции $f(x)$ на частичном отрезке 
$[x_{i-1},x_{i}]$ различают левую и правую формулы прямоугольников:
если в качестве значения аппроксимирующего полинома
выбирается значение подынтегральной функции 
на левом конце отрезка $L_0\approx f(x_{i-1})=y_{i-1}$, то
справедлива левая формула прямоугольников:
\begin{gather*}
S^{-}_i\approx\int\limits_{x_{i-1}}^{x_i}L_0(x)dx=
y_{i-1}\cdot(x_{i}-x_{i-1}),
\end{gather*}
а если значение аппроксимирующего полинома
соответствует значению подынтегральной функции 
на правом конце частичного отрезка $L_0\approx f(x_{i})=y_{i}$, то
справедлива правая формула прямоугольников:
\begin{gather*}
S^{+}_i\approx\int\limits_{x_{i-1}}^{x_i}L_0(x)dx=
y_{i}\cdot(x_{i}-x_{i-1}),
\end{gather*}
%
% Функция построения графика
%
\newcommand\FigInt[3]{
\begin{tikzpicture}[baseline]
\begin{axis}[% оси координат
xlabel=$x$,ylabel=$f(x)$,
%xmin=-1,xmax=1.5,
ymin=0,%ymax=2.25,
xtick={#1,#2},xticklabels={$x_{i-1}$,$x_i$},
ytick={0,cubic(#1),cubic(#2)},yticklabels={0,$y_{i-1}$,$y_i$},
width=8cm,% ширина графика
% определение функции
declare function={
cubic(\x)=0.5*(\x-2)*(\x-1)*(\x+1) + 1;
ya = cubic(#1);
yb = cubic(#2);
lagrange(\x) = (\x-#2)/(#1-#2)*cubic(#1) + (\x-#1)/(#2-#1)*cubic(#2);
},
]
% f(x)
\addplot[name path=F,
%pattern=north east lines,pattern color=darkred!35,
fill=red!15,
mark=ball,mark size=3pt,mark indices={1,50},
mark options={ball color=red!50,thin},
draw=darkred,thick,samples=50,domain=#1:#2]{cubic(x)}
\closedcycle;
% Ln(x)
\addplot[draw=none,thin,opacity=0.25,fill=darkblue!50,
domain=#1:#2,samples=50]{#3}
\closedcycle;
\end{axis}
\end{tikzpicture}
}
%
% График прямоугольников
%
\begin{figure}[H]\centering
\FigInt{-0.7}{2}{cubic(-0.7)}
%
\hskip 10pt
%
\FigInt{-0.7}{2}{cubic(2)}
\end{figure}

%
% Формула трапеций
%
\subsection{Формула трапеций}
Квадратурная \emph{формула трапеций} является следствием замены 
на частичном отрезке подынтегральной функции
интерполяционным полиномом первой степени $f(x)\approx L_1(x)$,
построенным по множеству узлов сетки $\{x_{i-1}, x_i\}$:
\begin{gather*}
L_1(x)=\dfrac{x-x_i}{x_{i-1}-x_i}\cdot y_{i-1} + \dfrac{x-x_{i-1}}{x_i-x_{i-1}}\cdot y_i.
\end{gather*}

Интегрирование интерполяционного полинома Лагранжа 
на частичном отрезке определяет формулу трапеций:
\begin{gather*}
S_i\approx\int\limits_{x_{i-1}}^{x_i}L_1(x)dx=
\dfrac{y_{i}+y_{i-1}}{2}\cdot(x_{i}-x_{i-1})
\end{gather*}
% график
\begin{figure}[H]\centering
\FigInt{-0.7}{2}{lagrange(x)}
\end{figure}
%
% Формула Симпсона
%
\subsection{Формула Симпсона}
На частичном отрезке $[x_{i-1},x_{i}]$ квадратурная 
\emph{формула Симпсона} является следствием 
аппроксимации подынтегральной функции 
$f(x)$ интерполяционным полиномом Лагранжа 
второй степени $f(x)\approx L_2(x)$, который построен
по узлам сетки $\{x_{i-1}, x_{i-1/2}, x_{i}\}$:
\begin{gather*}
\begin{matrix}
L_{2}(x)&=&\dfrac
{ (x-x_{i})\cdot(x-x_{i+1}) }
{ (x_{i}-x_{i-1})\cdot(x_{i+1}-x_{i-1}) } \cdot y_{i-1}& + \\
\\
&+&\dfrac
{ (x-x_{i-1})\cdot(x_{i+1}-x) }
{ (x_{i}-x_{i-1})\cdot(x_{i+1}-x_{i}) } \cdot y_{i}& + \\
\\
&+&\dfrac
{ (x-x_{i-1})\cdot(x-x_{i}) }
{ (x_{i+1}-x_{i-1})\cdot(x_{i+1}-x_{i}) } \cdot y_{i+1}
\end{matrix},
\end{gather*}
где $x_{i-1/2}$ -- узел вспомогательной сетки,
расположенный между узлами основной сетки
$x_{i-1}<x_{i-1/2}<x_{i}$

Выражение для полинома Лагранжа в каноническом виде:
\begin{gather*}
L_{2}(x)=c_0 + c_1\cdot x + c_2\cdot x^2,
\end{gather*}
где $c_0$, $c_1$, $c_2$ -- коэффициенты при 
соответствующих степенях $x$ интерполяционного полинома 
Лагранжа $L_2(x)$ в пределах частичного отрезка 
$[x_{i-1}, x_{i+1}]$.

Интегрирование интерполяционного полинома Лагранжа $L_2(x)$ 
на частичном отрезке $x\in[x_{i-1}, x_{i+1}]$ определяет формулу Симпсона:
\begin{gather*}
S_i\approx\int\limits_{x_{i-1}}^{x_{i+1}}L_2(x)dx=
c_0\cdot(x_{i+1}-x_{i-1})+
c_1\cdot\dfrac{x_{i+1}^2-x_{i-1}^2}{2}+
c_2\cdot\dfrac{x_{i+1}^3-x_{i-1}^3}{3}.
\end{gather*}

%
%	Численное интегрирования функции заданной таблично
%
\subsection{Численное интегрирования функции заданной таблично}
На множестве узлов сетки $\{x_i\}$ определены 
значения некоторой функции $\{y_i\}$:
\begin{center}
\begin{tabular}{l *{5}{l}}
\toprule
$x$&-3,31&0,31&1,32&2,47&3,50\\
\midrule
$f(x)$&2,45&4,03&-3,61&4,50&3,1\\
\bottomrule
\end{tabular}
\end{center}

\begin{enumerate}[leftmargin=0pt]
\item
Построим график функции $f(x)$ заданной таблично.
% *******************************
%	График функций
%
\begin{center}
\begin{tikzpicture}
\begin{axis}[
xlabel = {$x$},	% подпись оси x
ylabel = {$y(x)$},% подпись оси y
xmin=-4, xmax=4,
ymin=-4, ymax=6,
]
\addplot[name path=A, thick, color=orange, mark=*, mark size=3pt, mark options={fill=orange, draw=black, solid}] coordinates 
{(-3.31,2.45) (0.31,4.03) (1.32,-3.61) (2.47,4.50) (3.50, 3.1)};
\end{axis}
\end{tikzpicture}
\end{center}
% *******************************

Численное значение интеграла -- это площадь криволинейной трапеции,
ограниченной линиями графика и осью абсцисс $Ox$ 
(выделенная область на графике).

\item
Рассмотрим \textbf{метод трапеций} для нахождения численного значения интеграла
функции $f(x)$ заданной таблично на отрезке $x\in[-3,31; 3,50]$.

Разобъем весь отрезок интегрирования на частичные отрезки 
\begin{gather*}
[x_0,x_4]=[x_0,x_1] \cup [x_1,x_2] \cup [x_2,x_3] \cup [x_3,x_4]
\end{gather*}

На каждом частичном отрезке квадратурная формула является следствием 
замены подынтегральной функции $f(x)$ интерполяционным полиномом Лагранжа 
первой степени $f(x)\approx L_1(x)$, построенным но узлам $x_{i-1}, x_i$, 
т.е. прямой соединяющей два соседних узла.

% *******************************
%	График функций
%
\begin{center}
\begin{tikzpicture}
\begin{axis}[
every axis/.style={color=black, solid, thick},
xlabel = {$x$},		% подпись оси x
ylabel = {$f(x)$},	% подпись оси y
xmin=-4, xmax=4,
ymin=-4, ymax=6,
]
\addplot[name path=A, thick, color=orange, mark=*, mark size=3pt, mark options={fill=orange, draw=black, solid}] coordinates 
{(-3.31,2.45) (0.31,4.03) (1.32,-3.61) (2.47,4.50) (3.50, 3.1)};
%\path [name path=B] (\pgfkeysvalueof{/pgfplots/xmin},0) -- (\pgfkeysvalueof{/pgfplots/xmax},0);
\path [name path=B] (axis cs: -3.31,0) -- (axis cs: 3.5,0);
\addplot [orange!20] fill between [of=A and B, soft clip={domain=-3.31:3.5},];
\end{axis}
\end{tikzpicture}
\end{center}
% *******************************

Определим длину частичных отрезков:
\begin{gather*}
\begin{matrix}
h_1=&(x_1-x_0)=&0,31-(-3,31)=3,62\\
h_2=&(x_2-x_1)=&1,32-0,31=1,01\\ 
h_3=&(x_3-x_2)=&2,47-1,32=1,15\\
h_4=&(x_4-x_3)=&3,50-2,47=1,03\\
\end{matrix}
\end{gather*}

По методу трапеций, определим значение интеграла на каждом частичном отрезке:
\begin{gather*}
\begin{matrix}
I_1=&\dfrac{f(x_0)+f(x_1)}{2}\cdot h_1=\dfrac{2,45+4,03}{2}\cdot3,62=11,73\\[1em]
I_2=&\dfrac{f(x_1)+f(x_2)}{2}\cdot h_2=\dfrac{4,03-3,61}{2}\cdot1,01=0,21\\[1em]
I_3=&\dfrac{f(x_2)+f(x_3)}{2}\cdot h_3=\dfrac{-3,61+4,50}{2}\cdot1,15=0,51\\[1em]
I_4=&\dfrac{f(x_3)+f(x_4)}{2}\cdot h_4=\dfrac{4,50+3,10}{2}\cdot1,03=3,91\\
\end{matrix}
\end{gather*}

Определим интеграл $I$ на всем отрезке интегрирования $[-3,31; 3,50]$, 
воспользовавшись свойством аддитивности интеграла:
\begin{gather*}
I=I_1+I_2+I_3+I_4=11,73+0,21+0,51+3,91=16,37
.\end{gather*}

\item
Рассмотрим \textbf{метод Симпсона} для нахождения численное значение интеграла
функции $f(x)$ заданной таблично на отрезке $x\in[-3,31; 3,50]$.

Разобъем весь отрезок интегрирования на частичные отрезки:
\begin{gather*}
[x_0,x_4]=[x_0,x_2] \cup [x_2,x_4]
\end{gather*}
На каждом из двух отрезков построим интерполяционный полинома Лагранжа $L_2(x)$
\item
В пределах первого частичного отрезка $[-3,31; 1,32]$
построим полином Лагранжа $L_2(x)$ по узлам интерполяции
$x_0=-3,31; x_1=0,31; x_2=1,32$:
\begin{gather*}
\begin{matrix}
L^{(1)}_2(x)&=&\dfrac{(x-0,31)(x-1,32)}{((-3,31-0,31)(-3,31-1,32)}\cdot2,45&+\\[1em]
&+&\dfrac{(x-(-3,31))(x-1,32)}{(0,31-(-3,31))(0,31-1,32)}\cdot4,03&+\\[1em]
&+&\dfrac{(x-(-3,31))(x-0,31)}{(1,32-(-3,31))(1,32-0,31)}\cdot(-3,61)&\\
\end{matrix}
\end{gather*}

В результате алгебраических преобразований получим:
\begin{gather*}
L^{(1)}_2(x)=-1,73\cdot x^2-4,74\cdot x+5,66
\end{gather*}

Определим интеграл от интерполяционного полинома Лагранжа $L^{(1)}_2(x)$ на первом частичном отрезке:
\begin{gather*}
I_1=\int\limits_{-3,31}^{1,32} L^{(1)}_2(x)dx=
\int\limits_{-3,31}^{1,32} \left(-1,73\cdot x^2-4,74\cdot x+5,66\right)dx=25,88
\end{gather*}

% *******************************
%	График функций
%
\begin{center}
\begin{tikzpicture}
\begin{axis}[
every axis/.style={color=black, solid, thick},
xlabel = {$x$},		% подпись оси x
ylabel = {$f(x)$},	% подпись оси y
xmin=-4, xmax=4,
ymin=-4, ymax=10,
]
\addplot[color=orange, only marks, mark=*, mark size=3pt, mark options={fill=orange, draw=black, solid}] coordinates 
{(-3.31,2.45) (0.31,4.03) (1.32,-3.61) (2.47,4.50) (3.50, 3.1)};
%\path [name path=B] (\pgfkeysvalueof{/pgfplots/xmin},0) -- (\pgfkeysvalueof{/pgfplots/xmax},0);
\addplot[name path=A, thick, color=orange, domain=-3.31:1.32] {-1.73*x^2-4.74*x+5.66};
\draw[color=orange] (axis cs:-0.6,8.2) node [right] {$L^{(1)}_2(x)$};
\path [name path=B] (axis cs: -3.31,0) -- (axis cs: 1.32,0);
\addplot [orange!20] fill between [of=A and B, soft clip={domain=-3.31:1.32},];
\end{axis}
\end{tikzpicture}
\end{center}
% *******************************

\item
В пределах второго частичного отрезка $[1,32;3,50]$
построим полином Лагранжа $L_2(x)$ по узлам интерполяции
$x_2=1,32; x_3=2,47;x_4=3,50$:
\begin{gather*}
\begin{matrix}
L^{(2)}_2(x)&=&\dfrac{(x-2,47)(x-3,50)}{(1,32-2,47)(1,32-3,50)}\cdot(-3,61)&+\\[1em]
&+&\dfrac{(x-1,32)(x-3,50)}{(2,47-1,32)(2,47-3,50)}\cdot4,50&+\\[1em]
&+&\dfrac{(x-1,32)(x-2,47)}{(3,50-1,32)(3,50-2,47)}\cdot3,10\\
\end{matrix}
\end{gather*}

После тривиальных алгебраических преобразований:
\begin{gather*}
L^{(2)}_2(x)=-3,87\cdot x^2+21,76\cdot x-25,56
\end{gather*}

Определим интеграл от интерполяционного полинома Лагранжа $L^{(2)}_2(x)$ на втором частичном отрезке:
\begin{gather*}
I_2=\int\limits_{1,32}^{3,50} L^{(2)}_2(x)dx=
\int\limits_{1,32}^{3,50} \left(-3,87\cdot x^2+21,76\cdot x-25,56\right)dx=6,13
\end{gather*}

% *******************************
%	График функций
%
\begin{center}
\begin{tikzpicture}
\begin{axis}[
every axis/.style={color=black, solid, thick},
xlabel = {$x$},		% подпись оси x
ylabel = {$f(x)$},	% подпись оси y
xmin=-4, xmax=4,
ymin=-4, ymax=10,
]
\addplot[color=orange, only marks, mark=*, mark size=3pt, mark options={fill=orange, draw=black, solid}] coordinates 
{(-3.31,2.45) (0.31,4.03) (1.32,-3.61) (2.47,4.50) (3.50, 3.1)};
%\path [name path=B] (\pgfkeysvalueof{/pgfplots/xmin},0) -- (\pgfkeysvalueof{/pgfplots/xmax},0);
\addplot[name path=A, thick, color=orange, domain=-3.31:1.32] {-1.73*x^2-4.74*x+5.66};
\draw[color=orange] (axis cs:-0.6,8.2) node [right] {$L^{(1)}_2(x)$};
\path [name path=B] (axis cs: -3.31,0) -- (axis cs: 1.32,0);
\addplot [orange!20] fill between [of=A and B, soft clip={domain=-3.31:1.32},];
% второй отрезок
\addplot[name path=C, thick, color=orange, domain=1.32:3.50] {-3.87*x^2+21.76*x-25.56};
\draw[color=orange] (axis cs:2,6) node [right] {$L^{(2)}_2(x)$};
\path [name path=D] (axis cs: 1.32,0) -- (axis cs: 3.50,0);
\addplot [orange!20] fill between [of=C and D, soft clip={domain=1.32:3.50},];
\end{axis}
\end{tikzpicture}
\end{center}
% *******************************
\item
Определим интеграл всем отрезке $[-3,31; 3,50]$ воспользовавшись свойством аддитивности интеграла:
\begin{gather*}
I=I_1+I_2=25,88+6,13=32,01
\end{gather*}

\item
Сравнивая численные значения интегралов рассчитанные по методу трапеций и Симпсона,
можно сделать вывод о том, что значение интегралов существенно различаются:
определенный интеграл рассчитанный по методу Симпсона в 1,96 больше, чем по методу трапеций.
\end{enumerate}
%\end{document}
