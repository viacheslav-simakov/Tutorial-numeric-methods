\newpage
%
%	Численное интегрирование
%
\section{Численное интегрирование}
Если функция $f(x)$ непрерывна на отрезке $x\in[a,b]$ и 
известна ее первообразная $F(x)$, то определенный интеграл 
от этой функции в пределах от $a$ до $b$ может быть вычислен 
по формуле Ньютона -- Лейбница:
\begin{gather*}
\int\limits_{a}^{b}f(x)dx=F(b)-F(a),
\end{gather*}
где $F^{\prime}(x)=f(x)$ -- 
первообразная подынтегральной функции $f(x)$.

Численное значение интеграла -- это площадь криволинейной 
трапеции, ограниченной линиями графика функции 
и осью абсцисс $Ox$ (выделенная область на рисунке \ref{fig:INT}).

\begin{figure}[H]\centering
\begin{tikzpicture}
\def\xa{-1.6}
\def\xb{2}
\begin{axis}[% оси координат
xlabel=$x$,ylabel=$f(x)$,
xtick={\xa,\xb},xticklabels={$a$,$b$},
ytick={0},yticklabels={0}]
% f(x)
\addplot[name path=A,ball darkblue,mark=none,
samples=50,domain=\xa:\xb]{0.5*(\x-2)*(\x-1)*(\x+1) + 1};
\path[name path=B] (axis cs: \xa,0) -- (axis cs: \xb,0);
\addplot[blue!15] fill between [of=A and B, soft clip={domain=\xa:\xb},];
\end{axis}
\end{tikzpicture}
\caption{Геометрический смысл определенного интеграла}
\label{fig:INT}
\end{figure}

Однако во многих случаях первообразная функция $F(x)$ 
не может быть найдена с помощью элементарных средств 
или является слишком сложной, поэтому 
вычисление определенного интеграла 
может быть затруднительным или даже практически невозможным. 

Кроме того, на практике подынтегральная функция $f(x)$ часто 
задается таблично и тогда само понятие первообразной теряет смысл. 
Аналогичные вопросы возникают при вычислении кратных  
интегралов. Поэтому важное значение имеют приближенные 
и в первую очередь численные методы вычисления 
определенных интегралов. 

\emph{Задача численного интегрирования} функции заключается 
в вычислении значения определенного интеграла на основании ряда  
значений подынтегральной функции $f(x)$.

Обычный прием численного вычисления интеграла состоит 
в том, что данную функцию $f(x)$ на рассматриваемом отрезке 
$x\in[a, b]$ заменяют интерполирующей или аппроксимирующей  
функцией $\varphi(x)$ простого вида (например, полиномом), 
а затем приближенно полагают:
\begin{gather*}
\int\limits_{a}^{b}f(x)dx\approx \int\limits_{a}^{b}\varphi(x)dx
\end{gather*}

Далее рассматриваются способы приближенного вычисления 
определенных интегралов вида:
\begin{gather*}
I=\int\limits_{a}^{b}\varphi(x)dx,
\end{gather*}
основанные на замене интеграла конечной суммой:
\begin{gather*}
I\approx\sum\limits_{i=0}^{n}c_{i}\cdot\varphi(x_i),
\end{gather*}
где $c_{i}$ -- числовые коэффициенты квадратурной формулы; 
$x_i$ -- узлы квадратурной формулы, т.е. точки отрезка 
$[a, b], (i= 0,1,\cdots,n)$.

На основании свойств определенных интегралов, 
$I$ можно представить в виде суммы интегралов
по частичным отрезкам:
\begin{gather*}
\int\limits_{a}^{b}f(x)dx=
\sum\limits_{i=1}^{n}\int\limits_{x_{i-1}}^{x_i}f(x)dx
\end{gather*}

Поэтому, для построения формулы численного интегрирования 
на всем отрезке $[a,b]$ достаточно построить квадратурную формулу 
на частичном отрезке $[x_{i-1},x_i]$ для интеграла:
\begin{gather*}
S_i=\int\limits_{x_{i-1}}^{x_i}f(x)dx
\end{gather*}

%
% Формула прямоугольников
%
\subsection{Формула прямоугольников}
В методе прямоугольников на частичном отрезке 
подынтегральная функция заменяется полиномом нулевой степени,
то есть константу:
\begin{gather*}
f(x)\approx L_0(x)=\const
\end{gather*}

С геометрической точки зрения, в методе прямоугольников
площадь криволинейной трапеции (интеграл от функции) на
частичном отрезке заменяется площадью прямоугольника,
ширина которого будет определяться расстоянием между 
соответствующими соседними узлами интегрирования,
а высота -- значением подынтегральной функции в этих узлах.

В зависимости от выбора узла сетки $\{x_i\}$ для аппроксимации 
подынтегральной функции $f(x)$ на частичном отрезке 
$[x_{i-1},x_{i}]$ различают левую и правую формулы прямоугольников:
если в качестве значения аппроксимирующего полинома
выбирается значение подынтегральной функции 
на левом конце отрезка $L_0\approx f(x_{i-1})=y_{i-1}$
(рисунок \ref{fig:IntRect}), то справедлива 
левая формула прямоугольников:
\begin{gather*}
S^{-}_i\approx\int\limits_{x_{i-1}}^{x_i}L_0(x)dx=
y_{i-1}\cdot(x_{i}-x_{i-1}),
\end{gather*}
а если значение аппроксимирующего полинома
соответствует значению подынтегральной функции 
на правом конце частичного отрезка $L_0\approx f(x_{i})=y_{i}$
(рисунок \ref{fig:IntRect}), то справедлива 
правая формула прямоугольников :
\begin{gather*}
S^{+}_i\approx\int\limits_{x_{i-1}}^{x_i}L_0(x)dx=
y_{i}\cdot(x_{i}-x_{i-1}),
\end{gather*}
%
% Функция построения графика
%
\newcommand\FigInt[3]{
\begin{tikzpicture}[baseline]
\begin{axis}[% оси координат
xlabel=$x$,
name=GRAPH,
ylabel=$f(x)$,
%xmin=-1,xmax=1.5,
ymin=0,%ymax=2.25,
xtick={#1,#2},xticklabels={$x_{i-1}$,$x_i$},
ytick={0,cubic(#1),cubic(#2)},yticklabels={0,$y_{i-1}$,$y_i$},
width=7.5cm,% ширина графика
% определение функции
declare function={
cubic(\x)=0.5*(\x-2)*(\x-1)*(\x+1) + 1;
% полином Лагранжа L1
lagrange(\x) = (\x-#2)/(#1-#2)*cubic(#1) + (\x-#1)/(#2-#1)*cubic(#2);
% промежуточная точка
xc = #1 + (1-0.4)*(#2-#1)/2;
% полином Лагранжа L2
Lagrange(\x) = 
(\x-xc)*(\x-#2)/(xc-#1)/(#2-#1)*cubic(#1) +
(\x-#1)*(#2-\x)/(xc-#1)/(#2-xc)*cubic(xc) +
(\x-#1)*(\x-xc)/(#2-#1)/(#2-xc)*cubic(#2);
},
]
% f(x)
\addplot[name path=F,ball darkblue,fill=blue!10,
mark indices={1,50},samples=50,domain=#1:#2]{cubic(x)}
\closedcycle;
% Ln(x)
\addplot[draw=none,thin,opacity=0.25,fill=red!30,
domain=#1:#2,samples=50] #3 \closedcycle;
\end{axis}
% дополнительно
%\draw[thick,red] ($(GRAPH.south)-(0,0em)$) node {a};
\draw[thick,red] ($(current bounding box.south)-(0,1em)$) node {a)};
\end{tikzpicture}
}
%
% График прямоугольников
%
\begin{figure}[H]\centering
\FigInt{-0.7}{2}{ {cubic(-0.7)}
node[pos=0.7,color=darkred,opacity=1,above] {$L_0(x)$}
}
%
\hskip 10pt
%
\FigInt{-0.7}{2}{ {cubic(2)}
node[pos=0.8,color=darkred,opacity=1,above] {$L_0(x)$}
}\\
%\linebreak
a)\hspace{6cm}b)
\caption{График подынтегральной функции $f(x)$
и аппроксимирующего полинома $L_0(x)$ на частичном отрезке
для формулы прямоугольников}
\label{fig:IntRect}
\end{figure}

%
% Формула трапеций
%
\subsection{Формула трапеций}
Квадратурная \emph{формула трапеций} является следствием замены 
на частичном отрезке подынтегральной функции
интерполяционным полиномом первой степени $f(x)\approx L_1(x)$,
построенным по множеству узлов сетки $\{x_{i-1}, x_i\}$:
\begin{gather*}
L_1(x)=\dfrac{x-x_i}{x_{i-1}-x_i}\cdot y_{i-1} + \dfrac{x-x_{i-1}}{x_i-x_{i-1}}\cdot y_i.
\end{gather*}

Интегрирование интерполяционного полинома Лагранжа 
на частичном отрезке определяет формулу трапеций:
\begin{gather*}
S_i\approx\int\limits_{x_{i-1}}^{x_i}L_1(x)dx=
\dfrac{y_{i}+y_{i-1}}{2}\cdot(x_{i}-x_{i-1})
\end{gather*}
% график
\begin{figure}[H]\centering
\FigInt{-0.7}{2}{ {lagrange(x)} 
node[pos=0.8,color=darkred,opacity=1,above] {$L_1(x)$}
}
\caption{График подынтегральной функции $f(x)$
и аппроксимирующего полинома $L_1(x)$ на частичном отрезке
для формулы трапеций}
\label{fig:IntTrapez}
\end{figure}
%
% Формула Симпсона
%
\subsection{Формула Симпсона}
На частичном отрезке $[x_{i-1},x_{i}]$ квадратурная 
\emph{формула Симпсона} является следствием 
аппроксимации подынтегральной функции 
$f(x)$ интерполяционным полиномом Лагранжа 
второй степени $f(x)\approx L_2(x)$, который построен
по узлам сетки $\{x_{i-1}, x_{i-1/2}, x_{i}\}$:
\begin{gather*}
\begin{matrix}
L_{2}(x)&=&\dfrac
{ (x-x_{i-1/2})\cdot(x-x_{i}) }
{ (x_{i-1/2}-x_{i-1})\cdot(x_{i}-x_{i-1}) } \cdot y_{i-1}& + \\
\\
&+&\dfrac
{ (x-x_{i-1})\cdot(x_{i}-x) }
{ (x_{i-1/2}-x_{i-1})\cdot(x_{i}-x_{i-1/2}) } \cdot y_{i-1/2}& + \\
\\
&+&\dfrac
{ (x-x_{i-1})\cdot(x-x_{i-1/2}) }
{ (x_{i}-x_{i-1})\cdot(x_{i}-x_{i-1/2}) } \cdot y_{i}
\end{matrix},
\end{gather*}
где $x_{i-1/2}$ -- узел вспомогательной сетки,
расположенный между узлами основной сетки
$x_{i-1}<x_{i-1/2}<x_{i}$

Выражение для полинома Лагранжа в каноническом виде:
\begin{gather*}
L_{2}(x)=c_0 + c_1\cdot x + c_2\cdot x^2,
\end{gather*}
где $c_0$, $c_1$, $c_2$ -- коэффициенты при 
соответствующих степенях $x$ интерполяционного полинома 
Лагранжа $L_2(x)$ в пределах частичного отрезка 
$[x_{i-1}, x_{i+1}]$.

Интегрирование интерполяционного полинома Лагранжа $L_2(x)$ 
на частичном отрезке $x\in[x_{i-1}, x_{i+1}]$ определяет формулу Симпсона:
\begin{gather*}
S_i\approx\int\limits_{x_{i-1}}^{x_{i+1}}L_2(x)dx=
c_0\cdot(x_{i+1}-x_{i-1})+
c_1\cdot\dfrac{x_{i+1}^2-x_{i-1}^2}{2}+
c_2\cdot\dfrac{x_{i+1}^3-x_{i-1}^3}{3}.
\end{gather*}

% график
\begin{figure}[H]\centering
\FigInt{-0.7}{2}{ {Lagrange(x)} 
node[pos=0.7,color=darkred,opacity=1,above right] {$L_2(x)$}
}
\caption{График подынтегральной функции $f(x)$ и
аппроксимирующего полинома $L_2(x)$ на частичном отрезке
для формулы Симпсона}
\label{fig:IntSimpson}
\end{figure}

%***********************************
%
%	Численное интегрирования функции заданной таблично
%
%***********************************
\subsection{Численное интегрирования функции заданной таблично}
На множестве узлов сетки $\{x_i\}$ определены 
значения некоторой функции $\{y_i\}=f(x_i)$:
\begin{table}[H]
\vspace{-0.5\baselineskip}
\caption{Таблично заданная функциональная зависимость}
\begin{tabular*}{\textwidth}{%
l@{\extracolsep{\fill}}*{5}{r}p{0.25cm}}
\toprule
$i$&$0$&$1$&$2$&$3$&$4$\\
\midmidrule
$x_i$&$-3.31$&$0.31$&$1.32$&$2.47$&$3.50$\\
\addlinespace% дополнительный пробел
$y_i$&$2.45$&$4.03$&$-3.61$&$4.50$&$3.10$\\
\bottomrule
\end{tabular*}
\end{table}

\begin{enumerate}
% Стиль графиков
\pgfplotsset{%width=8cm,
xmin=-4,xmax=4,xtick={-4,-2,0,2,4},
ymin=-6,ymax=6,ytick={-6,-3,0,3,6},
}
\item
Построим график функции $f(x)$ заданной таблично.
% *******************************
%	График функций
%
\begin{figure}[H]\centering
\begin{tikzpicture}
\begin{axis}
\addplot[name path=A,ball darkblue,smooth] coordinates 
{(-3.31,2.45) (0.31,4.03) (1.32,-3.61) (2.47,4.50) (3.50, 3.1)};
\end{axis}
\end{tikzpicture}
\end{figure}
% *******************************
\item
Воспользуемся левой и правой формулами прямоугольников
для нахождения  численного значения интеграла 
от функции $f(x)$, заданной таблично на отрезке $x\in[x_0,x_4]$.
Для этого разобьем весь отрезок интегрирования 
на частичные отрезки: 
\begin{gather*}
[x_0,x_4]=[x_0,x_1] \cup [x_1,x_2] \cup [x_2,x_3] \cup [x_3,x_4]
\end{gather*}
%	График функций
\begin{figure}[H]\centering
% левые прямоугольники
\begin{tikzpicture}[baseline]
\begin{axis}
\addplot[name path=A,const plot mark left,ball darkblue] coordinates 
{(-3.31,2.45) (0.31,4.03) (1.32,-3.61) (2.47,4.50) (3.50, 3.1)};
%\path [name path=B] (\pgfkeysvalueof{/pgfplots/xmin},0) -- (\pgfkeysvalueof{/pgfplots/xmax},0);
\path[name path=B] (axis cs: -3.31,0) -- (axis cs: 3.5,0);
\addplot[blue!15] fill between [of=A and B, soft clip={domain=-3.31:3.5}];
\end{axis}
% правые прямоугольники
\end{tikzpicture}
\begin{tikzpicture}[baseline]
\begin{axis}
\addplot[name path=A,const plot mark right,ball darkblue] coordinates 
{(-3.31,2.45) (0.31,4.03) (1.32,-3.61) (2.47,4.50) (3.50, 3.1)};
\path[name path=B] (axis cs: -3.31,0) -- (axis cs: 3.5,0);
\addplot[blue!15] fill between [of=A and B, soft clip={domain=-3.31:3.5}];
\end{axis}
\end{tikzpicture}
\caption{Использование квадратурных формул 
левых и правых прямоугольников}
\end{figure}

На каждом частичном отрезке квадратурная формула 
является следствием замены подынтегральной функции 
$f(x)$ интерполяционным полиномом 
нулевой степени $f(x)\approx L_0(x)=\const$, 
построенным но узлам $\{x_{i-1}, x_{i}\}$.

По методу прямоугольников, определим значение 
интеграла на каждом частичном отрезке
(левые прямоугольники):
\begin{gather*}
\begin{array}{lclllll}
S^{-}_1&=&y_0\cdot(x_1-x_0)&=&2.45\cdot(0.31-(-3.31))&\approx&8.87\\
S^{-}_2&=&y_1\cdot(x_2-x_1)&=&4.03\cdot(1.32-0.31)&\approx&4.07\\
S^{-}_3&=&y_2\cdot(x_3-x_2)&=&-3.61\cdot(2.47-1.32)&\approx&-4.15\\
S^{-}_4&=&y_3\cdot(x_4-x_3)&=&4.5\cdot(3.50-2.47)&\approx&4.64\\
\end{array}
\end{gather*}

(правые прямоугольники):
\begin{gather*}
\begin{array}{*7l}
S^{+}_1&=&y_1\cdot(x_1-x_0)&=&4.03\cdot(0.31-(-3.31))&\approx&14.59\\
S^{+}_2&=&y_2\cdot(x_2-x_1)&=&-3.61\cdot(1.32-0.31)&\approx&-3.65\\
S^{+}_3&=&y_3\cdot(x_3-x_2)&=&4.50\cdot(2.47-1.32)&\approx&5.18\\
S^{+}_4&=&y_4\cdot(x_4-x_3)&=&3.10\cdot(3.50-2.47)&\approx&3.19\\
\end{array}
\end{gather*}

Значение интегралов $I^{-}$ и $I^{+}$ на всем отрезке 
интегрирования $[x_0,x_4]$: 
\begin{gather*}
\begin{array}{*7l}
I^{-}&=&S^{-}_1+S^{-}_2+S^{-}_3+S^{-}_4&=&
8.87+4.07-4.15+4.64&=&13.43\\
I^{+}&=&S^{+}_1+S^{+}_2+S^{+}_3+S^{+}_4&=&
14.59-3.65+5.18+3.19&=&19.31
\end{array}
\end{gather*}
 
\item
Рассмотрим \emph{метод трапеций} для нахождения 
численного значения интеграла от функции $f(x)$,
заданной таблично на отрезке $x\in[x_0,x_4]$.
Разобьем весь отрезок интегрирования на частичные отрезки: 
\begin{gather*}
[x_0,x_4]=[x_0,x_1] \cup [x_1,x_2] \cup [x_2,x_3] \cup [x_3,x_4]
\end{gather*}

На каждом частичном отрезке квадратурная формула 
является следствием замены подынтегральной функции 
$f(x)$ интерполяционным полиномом Лагранжа 
первой степени $f(x)\approx L_1(x)$, 
построенным но узлам $\{x_{i-1}, x_{i}\}$, 
т.е. прямой соединяющей два соседних узла.

% *******************************
%	График функций
%
\begin{figure}[H]\centering
\begin{tikzpicture}
\begin{axis}
\addplot[name path=A,
thick,draw=darkblue,mark=ball,mark size=3pt,
mark options={ball color=darkblue!50,thin,draw=darkblue}
] coordinates 
{(-3.31,2.45) (0.31,4.03) (1.32,-3.61) (2.47,4.50) (3.50, 3.1)};
%\path [name path=B] (\pgfkeysvalueof{/pgfplots/xmin},0) -- (\pgfkeysvalueof{/pgfplots/xmax},0);
\path[name path=B] (axis cs: -3.31,0) -- (axis cs: 3.5,0);
\addplot[blue!15] fill between [of=A and B, soft clip={domain=-3.31:3.5},];
\end{axis}
\end{tikzpicture}
\caption{Использование квадратурных формул трапеций}
\end{figure}
% *******************************

По методу трапеций, определим значение интеграла на каждом частичном отрезке:
\begin{gather*}
%\renewcommand*{\arraystretch}{2}
\begin{array}{*7l}
S_1&=&\dfrac{y_1+y_0}{2}\cdot(x_1-x_0)&=&
\dfrac{4.03+2.45}{2}\cdot(0.31-(-3.31))&\approx&11.73\\[1em]
S_2&=&\dfrac{y_2+y_1}{2}\cdot(x_2-x_1)&=&
\dfrac{-3.61+4.03}{2}\cdot(1.32-0.31)&\approx&0.21\\[1em]
S_3&=&\dfrac{y_3+y_2}{2}\cdot(x_3-x_2)&=&
\dfrac{4.50-3.61}{2}\cdot(2.47-1.32)&\approx&0.51\\[1em]
S_4&=&\dfrac{y_4+y_3}{2}\cdot(x_4-x_3)&=&
\dfrac{3.10+4.50}{2}\cdot(3.50-2.47)&\approx&3.91
\end{array}
\end{gather*}

Определим интеграл $I$ на всем отрезке интегрирования 
$[x_0,x_4]$, воспользовавшись свойством аддитивности 
интеграла:
\begin{gather*}
I=S_1+S_2+S_3+S_4=11.73+0.21+0.51+3.91=16.37.
\end{gather*}

\item
Рассмотрим \emph{метод Симпсона} для нахождения 
численное значение интеграла от функции $f(x)$,
заданной таблично на отрезке $x\in[x_0,x_4]$.

Разделим всё множество узлов сетки $\{x_i\}$, в которых
известны значения функции $\{y_i\}$, на основные 
и вспомогательные узлы:
\begin{table}[H]
\vspace{-0.5\baselineskip}
\caption{Таблично заданная функциональная зависимость}
\begin{tabular*}{\textwidth}{%
l@{\extracolsep{\fill}}*{5}{r}p{0.25cm}}
\toprule
$i$&$0$&$1-1/2$&$1$&$1+1/2$&$2$\\
\midmidrule
$x_i$&$-3.31$&$0.31$&$1.32$&$2.47$&$3.50$\\
\addlinespace% дополнительный пробел
$y_i$&$2.45$&$4.03$&$-3.61$&$4.50$&$3.10$\\
\bottomrule
\end{tabular*}
\end{table}
Разобьем весь отрезок интегрирования на частичные отрезки:
\begin{gather*}
[x_0,x_2]=[x_0,x_1] \cup [x_1,x_2].
\end{gather*}

В пределах первого частичного отрезка $[x_0,x_1]$
построим интерполяционный полином Лагранжа $L_2(x)$
по узлам сетки $x_0=-3.31$, $x_{1-1/2}=0.31$, $x_1=1.32$:
\begin{gather*}
\begin{matrix}
L_2(x)&=&\dfrac{(x-0.31)(x-1.32)}{((-3.31-0.31)(-3.31-1.32)}\cdot2.45&+\\[1em]
&+&\dfrac{(x-(-3.31))(x-1.32)}{(0.31-(-3.31))(0.31-1.32)}\cdot4.03&+\\[1em]
&+&\dfrac{(x-(-3.31))(x-0.31)}{(1.32-(-3.31))(1.32-0.31)}\cdot(-3.61)&\\
\end{matrix}
\end{gather*}

После алгебраических преобразований запишем 
интерполяционный полином в каноническом виде:
\begin{gather*}
L_2(x)=5.66-4.74\cdot x-1.73\cdot x^2
\end{gather*}

Определим интеграл от интерполяционного полинома 
$L_2(x)$ на первом частичном отрезке:
\begin{gather*}
I_1=\int\limits_{x_0}^{x_1} L_2(x)dx=
\int\limits_{-3.31}^{1.32}
\left(5.66-4.74\cdot x-1.73\cdot x^2\right)dx=25.88
\end{gather*}

% *******************************
%	График функций
\begin{center}
\begin{tikzpicture}
\begin{axis}[ymax=10]
\addplot[only marks,ball darkblue] coordinates 
{(-3.31,2.45) (0.31,4.03) (1.32,-3.61) (2.47,4.50) (3.50, 3.1)};
%\path [name path=B] (\pgfkeysvalueof{/pgfplots/xmin},0) -- (\pgfkeysvalueof{/pgfplots/xmax},0);
\addplot[name path=A,ball darkblue,mark=none,domain=-3.31:1.32]
{-1.73*x^2-4.74*x+5.66} node[pos=0.4,right] {$L_2(x)$};
\path[name path=B] (axis cs: -3.31,0) -- (axis cs: 1.32,0);
\addplot[darkblue!15] fill between [of=A and B, soft clip={domain=-3.31:1.32}];
\end{axis}
\end{tikzpicture}
\end{center}
% *******************************

В пределах второго частичного отрезка $[x_1,x_2]$
построим интерполяционный полином Лагранжа $L_2(x)$ 
по узлам сетки $x_1=1.32$, $x_{1+1/2}=2.47$, $x_2=3.50$:
\begin{gather*}
\begin{matrix}
L_2(x)&=&\dfrac{(x-2.47)(x-3.50)}{(1.32-2.47)(1.32-3.50)}\cdot(-3.61)&+\\[1em]
&+&\dfrac{(x-1.32)(x-3.50)}{(2.47-1.32)(2.47-3.50)}\cdot4.50&+\\[1em]
&+&\dfrac{(x-1.32)(x-2.47)}{(3.50-1.32)(3.50-2.47)}\cdot3.10\\
\end{matrix}
\end{gather*}

После тривиальных алгебраических преобразований:
\begin{gather*}
L_2(x)=-25.56+21.76\cdot x-3.87\cdot x^2
\end{gather*}

Определим интеграл от интерполяционного полинома 
$L_2(x)$ на втором частичном отрезке:
\begin{gather*}
I_2=\int\limits_{x_1}^{x_2} L_2(x)dx=
\int\limits_{1.32}^{3.50} \left(-25.56+21.76\cdot x-3.87\cdot x^2\right)dx=6.13
\end{gather*}

% *******************************
%	График функций
\begin{center}
\begin{tikzpicture}
\begin{axis}[ymax=10]
% данные
\addplot[only marks,ball darkblue] coordinates 
{(-3.31,2.45) (0.31,4.03) (1.32,-3.61) (2.47,4.50) (3.50, 3.1)};
% Ox
\path [name path=Ox] (axis cs: -3.31,0) -- (axis cs: 3.50,0);
% первый отрезок
\addplot[name path=A,thick,color=darkblue,domain=-3.31:1.32]
{-1.73*x^2-4.74*x+5.66} node[pos=0.4,right] {$L_2(x)$};
\addplot[darkblue!15] fill between [of=A and Ox, 
soft clip={domain=-3.31:1.32}];
% второй отрезок
\addplot[name path=C,thick,color=darkblue,domain=1.32:3.50]
{-3.87*x^2+21.76*x-25.56} node[pos=0.8,above] {$L_2(x)$};
\addplot[darkblue!15] fill between [of=C and Ox, 
soft clip={domain=1.32:3.50}];
\end{axis}
\end{tikzpicture}
\end{center}
% *******************************

Определим интеграл всем отрезке $[x_0,x_2]$ 
воспользовавшись свойством аддитивности интеграла:
\begin{gather*}
I=I_1+I_2=25.88+6.13=32.01
\end{gather*}

\item
Сравнивая численные значения определенного интеграла 
рассчитанные по методам прямоугольников, трапеций и Симпсона,
можно сделать вывод о том, что рассчитанные значения
различаются.
\begin{table}[H]
\vspace{-0.5\baselineskip}
\caption{Численные значения интегралов}
\begin{tabular*}{\textwidth}{%
l@{\extracolsep{\fill}}lp{3cm}}
\toprule
Метод интегрирования&Значение интеграла\\
\midmidrule
Левых прямоугольников&$13.43$\\
Правых прямоугольников&$19.31$\\
Трапеций&$16.37$\\
Симпсона&$32.01$\\
%\addlinespace% дополнительный пробел
\bottomrule
\end{tabular*}
\end{table}
Значение определенного интеграла от функции заданной таблично,
рассчитанное по методу Симпсона является наибольшим, а 
значение рассчитанное по методу левых прямоугольников 
-- наименьшее.
\end{enumerate}
