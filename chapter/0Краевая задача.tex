% !TEX TS-program = xelatex
% !TEX encoding = UTF-8
%%%%%%%%%%%%%%%%%%%%%%%%%%%%%%
%
% Преамбула
%
%	РАЗМЕР ШРИФТА:
%		\tiny
%		\scriptsize
%		\footnotesize	
%		\small 
%%%	\normalsize	= нормальный размер шрифта
%		\large
%		\Large
%		\huge
%		\Huge
%--------------------------------------------
%	\textrm{текст}	\textsf{текст}	\texttt{текст}
%	\textmd{текст}	\textbf{текст}	\textup{текст}
%	\textit{текст}	\textsl{текст}	\textsc{текст}
%%%%%%%%%%%%%%%%%%%%%%%%%%%%%%%%%%%%%%%%%%%%%
%
%	НАЧАЛО ПРЕАМБУЛЫ
%
\documentclass[14pt,a4paper]{extreport}
%
% языковые пакеты
%
\usepackage[T2A]{fontenc}
\usepackage[utf8]{inputenc}
%\usepackage[cp1251]{inputenc}
%
%	Размеры страницы
%
\usepackage[% https://ctan.org/pkg/geometry
	a4paper,% размер страницы
	mag=1000,
	left=2.5cm,
	right=1.5cm,
	top=2cm,
	bottom=2cm,
	headsep=0.7cm,
	footskip=0.7cm
]{geometry}
% --------------------------------------------
% *** межстрочный интервал ***
% стандартный пропуск строки означает коэффициент 1,2
% (например, высота шрифта 10pt, пропуск базовой строки 12pt).
% Умножьте на \linespread, чтобы вы получили 1,25 * 1,2 = 1,5 
% то есть половину.
% --------------------------------------------
%\linespread{1.5}
%
% МЕЖСТРОЧНЫЙ ИНТЕРВАЛ
%
% межстрочный интервал
\usepackage{setspace}
% полуторный интервал
\onehalfspacing
%\usepackage{indentfirst}
%
% ОТСТУП АБЗАЦА СЛЕВА
%
\setlength{\parindent}{1.25cm}
%
% НУМЕРАЦИЯ СТРАНИЦ 
% http://tug.ctan.org/tex-archive/macros/latex/contrib/fancyhdr/fancyhdr.pdf
%
\usepackage{fancyhdr}% https://www.ctan.org/pkg/fancyhdr
\pagestyle{empty}
\renewcommand{\headrulewidth}{0pt}
\renewcommand{\footrulewidth}{0pt}
%\lhead{\empty}\chead{\empty}\rhead{\empty}
%\lfoot{\empty}\cfoot{\empty}\rfoot{\arabic{page}}
%
% https://latex.org/forum/viewtopic.php?t=8952
%
\makeatletter
\renewcommand{\ps@plain}{%
\renewcommand\@oddhead{}%
\renewcommand\@evenhead{}%
\renewcommand\@oddfoot{\hfil\normalfont\textrm{\thepage}}%
\renewcommand\@evenfoot{\hfil\normalfont\textrm{\thepage}}%
}
\makeatother
%%%%%%%%%%%%%%%%%%%%%%%%%%%%%%%%%%%%%%%%%%%%%
%
% ПАКЕТ МНОГОЯЗЫКОВОЙ ВЁРСТКИ *** XeTeX ***
% https://ru.wikibooks.org/wiki/LaTeX/polyglossia
%
\usepackage{polyglossia}
% устанавливает главный язык документа
\setdefaultlanguage[spelling=modern]{russian}
\setmainlanguage{russian}
% устанавливает второй язык документа
\setotherlanguage{english}
% задаёт свойства шрифтов по умолчанию
\defaultfontfeatures{Ligatures={TeX},Renderer=Basic}
% задаёт основные шрифты документа
\setmainfont{Times New Roman}
\setromanfont{Times New Roman} 
\setsansfont{Arial} 
\setmonofont{Courier New}
% семейство кириллических шрифтов
\newfontfamily{\cyrillicfont}{Times New Roman} 
\newfontfamily{\cyrillicfontrm}{Times New Roman}
\newfontfamily{\cyrillicfonttt}{Courier New}
\newfontfamily{\cyrillicfontsf}{Arial}
%
% *** Новые Цвета ***
%
\usepackage{xcolor}% https://www.ctan.org/pkg/xcolor
\colorlet{darkgreen}{green!40!black}% темно-зеленый
\colorlet{darkred}{red!80!black}% темно-красный
\colorlet{darkorange}{orange!75!black}% темно-оранжевый
\colorlet{darkblue}{blue!70!black}% темно-синий
%
% Unicode-шрифты ДЛЯ ФОРМУЛ
%
\usepackage[intlimits]{amsmath}
\usepackage{amssymb,amsfonts}
% Расширенная матрица
% https://tex.stackexchange.com/questions/2233/whats-the-best-way-make-an-augmented-coefficient-matrix
% \begin{pmatrix}[cc|c]
%   1 & 2 & 3\\
%   4 & 5 & 9
% \end{pmatrix}
\makeatletter
\renewcommand*\env@matrix[1][*\c@MaxMatrixCols c]{%
	\hskip -\arraycolsep
	\let\@ifnextchar\new@ifnextchar
	\array{#1}}
\makeatother
%
% ШРИФТ Times в ФОРМУЛАХ как основной
%
%\usepackage[varg,cmbraces,cmintegrals]{newtxmath}
%
% прямые греческие буквы
%
%\usepackage{upgreek}
%
% Experimental Unicode mathematical typesetting
%
%\usepackage{unicode-math}
%\usepackage[bold-style=TeX]{unicode-math}
\usepackage[% https://ctan.org/pkg/unicode-math
	bold-style=upright%
]{unicode-math}
%
% ССЫЛКИ
%
\usepackage[% https://ctan.org/pkg/hyperref
	unicode=true,%
	bookmarksopen=true,%
	bookmarksnumbered=true,%
	colorlinks,%
	linkcolor=black,%
	urlcolor=darkblue,%
	citecolor=blue%
]{hyperref}
%
% Принудительное размещения рисунка:
% используйте пакет 'float', а затем [H] опцию,
% например, \begin{figure}[H] ... \end{figure}
\usepackage{float}
%%%%%%%%%%%%%%%%%%%%%%%%%%%%%%%%%%%%%%%%%%%%%
%
% ЗАГОЛОВКИ
%
%\usepackage[pagestyles]{titlesec}
\usepackage{titlesec}% https://www.ctan.org/pkg/titlesec
%
% \titleformat{command}
%		[shape]
%		{format}
%		{label}
%		{sep}
%		{before-code}
%		[after-code]
%
%	1) Глава (chapter)
%
\titleformat{\chapter}[hang]% shape
{\bfseries\normalsize\centering}% format
{\hspace{\parindent}\arabic{chapter}}% label
{0.5em}% sep
{}% before-code
[]% after-code
\titleformat*{\section}%
{\bfseries\normalsize}% format
%
% 2) Раздел (section)
%
\titleformat{\section}[hang]% shape
{\bfseries\normalsize}% format
{\hspace{1.25cm}\arabic{section}}% label
{0.5em}% sep
{}% before-code
[]% after-code
%
% 3) Подраздел (subsection)
%
\titleformat{\subsection}[hang]% shape
{\bfseries\normalsize}% format
{\hspace{1.25cm}\arabic{section}.\arabic{subsection}}% label
{0.5em}% sep
{}% before-code
[]% after-code
%
% 4) ПодПодраздел (subsubsection)
%
\titleformat{\subsubsection}[hang]% shape
{\bfseries\normalsize}% format
{\hspace{1.25cm}\arabic{section}.\arabic{subsection}.\arabic{subsubsection}}% label
{0.5em}% sep
{}% before-code
[]% after-code
%
% \titlespacing{command}
%		{before-sep}
%		{after-sep}
%		[right-sep]
%
\titlespacing{\chapter}
{0pt}% before-sep
{0pt}% after-sep
{0pt}% right-sep
\titlespacing{\section}
{0pt}% before-sep
{0pt}% after-sep
{0pt}% right-sep
\titlespacing{\subsection}
{0pt}% before-sep
{0pt}% after-sep
{0pt}% right-sep
\titlespacing{\subsubsection}
{0pt}% before-sep
{0pt}% after-sep
{0pt}% right-sep
%%%%%%%%%%%%%%%%%%%%%%%%%%%%%%%%%%%%%%%%%%%%
%
% СОДЕРЖАНИЕ
%
%%%%%%%%%%%%%%%%%%%%%%%%%%%%%%%%%%%%%%%%%%%%
%\addto\captionsrussian{%
%\renewcommand{\contentsname}{%
%\vspace{-5ex}%
%\begin{center}СОДЕРЖАНИЕ\end{center}%
%\vspace{-5ex}}%
%}
%\addto\captionsrussian{%
%  \renewcommand{\figurename}{Fig.}%
%}
\makeatletter
\renewcommand{\tableofcontents}{%
\newpage%
\noindent{\centering\textbf{СОДЕРЖАНИЕ}\par}%
\vspace{\baselineskip}% пустая строка
\@starttoc{toc}\par}%
\makeatother
% глубина детализации
\setcounter{tocdepth}{3}
\setcounter{secnumdepth}{3}
% переименование формата счетчика
\renewcommand{\thesection}{\arabic{section}}
%\renewcommand{\thesubsection}{\arabic{subsection}}
%	
% \titlecontents{section}
%		[left]
%		{above-code}
%		{numbered-entry-format}
%		{numberless-entry-format}
%		{filler-page-format}
%		[below-code]
%
\usepackage{titletoc}
%	Глава (chapter)
\titlecontents{chapter}
[0pt]% left
{}%above-code
{\thechapter\;}% numbered-entry-format
{}% numberless-entry-format
{\thepage}% filler-page-format
[]% below-code
%
% Раздел (section)
\titlecontents{section}
[0pt]% left
{\normalsize}% above-code
{\thecontentslabel\;}% numbered-entry-format
{}% numberless-entry-format
{(\thepage)}% filler-page-format
[]% below-code
%
% Подраздел (subsection)
\titlecontents{subsection}
[0pt]% left
{}%above-code
{\thecontentslabel\;}% numbered-entry-format
{}% numberless-entry-format
{\thepage}% filler-page-format
[]% below-code
% ПодПодраздел (subsubsection)
\titlecontents{subsubsection}
[0pt]% left [0pt]
{}%above-code
{\thecontentslabel\;}% numbered-entry-format
{}% numberless-entry-format
{\thepage}% filler-page-format
[]% below-code
%
% \dottedcontents{section}
%		[left]
%		{above-code}
%		{label width}
%		{leader width}
\dottedcontents{chapter}%
[1.5em]% left
{}% above-code
{1.5em}% label width
{0.5em}% leader width
\dottedcontents{section}%
[1.0em]% left [1.5em]
{}% above-code
{1.0em}% label width [1.5em]
{0.5em}% leader width
\dottedcontents{subsection}%
[1.5em]% left
{}% above-code
{1.5em}% label width
{0.5em}% leader width
\dottedcontents{subsubsection}%
[2.5em]% left
{}% above-code
{2.5em}% label width
{0.5em}% leader width
%
% Chapter without a pagebreak
%
\makeatletter 
\renewcommand\chapter{\newpage\par%
\thispagestyle{plain}% \global\@topnum\z@
\@afterindentfalse\secdef\@chapter\@schapter}
\makeatother %%%%% <---- Starting chapter without a pagebreak
%
% ВВЕДЕНИЕ
%
\newenvironment{Introduction}{%
%\chapter*{\vspace{-2ex}ВВЕДЕНИЕ}\par%
%\newpage\begin{center}\textbf{ВВЕДЕНИЕ}\end{center}\par
\newpage\phantomsection%
{\centering\textbf{ВВЕДЕНИЕ}\par}%
\addcontentsline{toc}{chapter}{ВВЕДЕНИЕ}%
% пустая линия
%\vspace{\baselineskip}
}{\newpage}
%
% ЗАКЛЮЧЕНИЕ
%
\newenvironment{Conclusion}{%
%\lfoot{\empty}\cfoot{\empty}\rfoot{\arabic{page}}%
\newpage\phantomsection%
%\chapter*{\vspace{-\baselineskip}ЗАКЛЮЧЕНИЕ}\par%
{\centering\textbf{ЗАКЛЮЧЕНИЕ}\par}%
\addcontentsline{toc}{chapter}{ЗАКЛЮЧЕНИЕ}%
% пустая линия
%\vspace{\baselineskip}
}{\newpage}
%
% ПРИЛОЖЕНИЕ
%
\newenvironment{Appendix}{%
%\lfoot{\empty}\cfoot{\empty}\rfoot{\arabic{page}}%
\newpage\phantomsection%
%\chapter*{\vspace{-\baselineskip}ЗАКЛЮЧЕНИЕ}\par%
{\centering\textbf{ПРИЛОЖЕНИЕ}\par}%
\addcontentsline{toc}{chapter}{ПРИЛОЖЕНИЕ}%
% пустая линия
%\vspace{\baselineskip}
}{\newpage}
%
% СПИСОК ИСПОЛЬЗОВАННЫХ ИСТОЧНИКОВ
%
% Пакет поддерживает сжатые, 
% отсортированные списки цитирования
% https://www.ctan.org/pkg/cite
\usepackage{cite}
% формат номера источника
% Заменяем библиографию в квадратных скобках
% http://ftp.tug.org/TUGboat/tb30-1/tb94mori.pdf
% 
\makeatletter%
\renewcommand*{\@biblabel}[1]{\hfill#1\;}
\makeatother
% окружение
\newenvironment{References}[1]{%
\newpage\phantomsection
\addcontentsline{toc}{chapter}{%
СПИСОК ИСПОЛЬЗОВАННЫХ ИСТОЧНИКОВ}
\renewcommand{\bibname}{%
СПИСОК ИСПОЛЬЗОВАННЫХ ИСТОЧНИКОВ}
\begin{thebibliography}{#1}
% пустая линия
\vspace{\baselineskip}
% интервал между библиографическими источниками
\setlength{\itemsep}{0pt}
\setlength{\parskip}{0pt}
}{\end{thebibliography}}
%%%%%%%%%%%%%%%%%%%%%%%%%%%%%%%%%%%%%%%%%%%%%
%
% ТАБЛИЦЫ
%
% Пакет повышает качество таблиц в LaTeX, 
% предоставляя дополнительные команды
% https://www.ctan.org/pkg/booktabs
\usepackage{booktabs}%
% \specialrule{wd}{abovespace}{belowspace}
% горизонтальная линия с отступами сверху и снизу
\renewcommand\midrule{\specialrule{0.5pt}{1ex}{1ex}}
\renewcommand\toprule{\specialrule{1pt}{0ex}{1ex}}
\renewcommand\bottomrule{\specialrule{1pt}{1ex}{0ex}}
%\newcommand\tabsrule{\specialrule{0.5pt}{1ex}{1ex}}
\newcommand\midmidrule{
\specialrule{0.5pt}{1ex}{0.1ex}
\specialrule{0.5pt}{0.1ex}{1ex}
}
% межстрочный интервал в ТАБЛИЦАХ
\renewcommand\arraystretch{1.2}
%%%%%%%%%%%%%%%%%%%%%%%%%%%%%%%%%%%%%%%%%%%%%
%
% СПИСКИ
%
\usepackage[inline]{enumitem}% https://ctan.org/pkg/enumitem
\setlist[enumerate,itemize]{
	left=0pt,
	align=left,
	leftmargin=0pt,
	label = \arabic*),
	leftmargin=*,
	labelsep=1ex,
	itemindent=0pt,
	nosep
}
%%%%%%%%%%%%%%%%%%%%%%%%%%%%%%%%%%%%%%%%%%%%%
%
%	Цветной прямоугольник с текстом
%
\usepackage{tcolorbox}% https://www.ctan.org/pkg/tcolorbox
% Установка опций по умолчанию
\tcbset{
	notitle,
	titlebox=invisible,
	size=title,
	width=\textwidth,
	boxsep=0em,
	left=1ex,
	right=1ex,
	toptitle=0mm,
	top=1ex,
	toprule=0mm,
	bottom=1ex,
	bottomrule=0mm,
	boxrule=0mm,
	arc=0mm,
	colback=orange!10,
}
%%%%%%%%%%%%%%%%%%%%%%%%%%%%%%%%%%%%%%%%%%%%%
%
% TikZ
% https://tex.stackexchange.com/questions/131293/arguments-for-tikz-style
%
\usepackage{tikz}
\usetikzlibrary{%
	pgfplots.groupplots,%
	backgrounds,%
	calc,%
	decorations.pathmorphing,%
	decorations.markings,
	shapes,% геометрические фигуры
	arrows.meta,% стрелки разной формы
	er,% построение диаграмм
	patterns% штриховка областей
}
\tikzset{
	background rectangle/.style={% стиль фона
		fill=olive!10%
	},%
	font=\normalsize%
}
%
% PGF
% http://elib.ict.nsc.ru/jspui/bitstream/ICT/1488/1/pgf-ru-all-method.pdf
%
\usepackage{%
	pgfplots,%
	pgfplotstable%
}
% Последовательность графичеких слоев
% \pgfsetlayers{%
%	background,%
%	pre main,%
%	axis grid,% 
%	axis ticks,% 
%	axis lines,%
%	axis tick labels,%
%	main,%
%	axis descriptions,%
%	axis foreground%
%}
\usepgfplotslibrary{fillbetween}
\usepgflibrary{plotmarks}
% Установка стилей графика
\pgfplotsset{
%	compat=1.9,
	width=8cm,
	every axis/.append style={%
		thick,
		tick style={
			black,
			semithick,
		},
	}
}
%
% НУМЕРАЦИЯ формул, рисунков
%
\renewcommand{\theequation}{\arabic{equation}}
\renewcommand{\thefigure}{\arabic{figure}}
%%%%%%%%%%%%%%%%%%%%%%%%%%%%%%%%%%%%%%%%%%%%%
%
% ПОДПИСИ рисуноков, таблиц
%
\usepackage[% https://www.ctan.org/pkg/caption
	format=plain,% Печатает подписи как обычный абзац
	labelsep=endash,%
	singlelinecheck=false,% отключить центрирование однострочной подписи
	belowskip=0pt,%
	margin={0pt,0pt},%
	indention=0cm,
%	font=onehalfspacing,% полуторный интервал
]{caption}
%	рисунки
\captionsetup[figure]{%
	name=Рисунок,%
	position=below,%
	justification=centering,% центрирование подписи
	font=onehalfspacing,% полуторный интервал
}
%	таблицы
\captionsetup[table]{%
	name=Таблица,%
	position=above,%
%	justification=justified,
	font=onehalfspacing,% полуторный интервал
}
%\captionsetup{belowskip=0pt,margin={0pt,0pt}}


% *** отступы ***
% вертикальный промежуток перед и после объектов,
% местоположение которых соответствует ключу h.
% Имеет естественную длину 12 pt.
\setlength{\intextsep}{\baselineskip}
% вертикальный промежуток между текстом и соответственно 
% одно- и двухколоночными объектами, местоположение 
% которых соответствует ключам t или b.
% Имеет естественную длину 20 pt.
\setlength{\textfloatsep}{\baselineskip}
% вертикальный промежуток между соответственно 
% одно- и двухколоночными объектами, местоположение 
% которых соответствует ключам t или b. 
% Имеет естественную длину 12 pt.
\setlength{\floatsep}{\baselineskip}
% отступ перед названием 
\setlength{\abovecaptionskip}{0.5\baselineskip}
% отступ после названиея
\setlength{\belowcaptionskip}{0ex}
%
% ОТСТУПЫ В ФОРМУЛАХ
%
%\expandafter\def\expandafter\normalsize\expandafter{%
%\normalsize
%\setlength\abovedisplayskip{0.5\baselineskip}
%\setlength\belowdisplayskip{0.5\baselineskip}
%\setlength\abovedisplayshortskip{0.5\baselineskip}
%\setlength\belowdisplayshortskip{0.5\baselineskip}
%}
\AtBeginDocument{%
\abovedisplayskip=0.5\baselineskip
\abovedisplayshortskip=0.5\baselineskip
\belowdisplayskip=0.5\baselineskip
\belowdisplayshortskip=0.5\baselineskip
% отступ перед названием 
%\abovecaptionskip=0.5\baselineskip
% отступ после названиея
%\belowcaptionskip=0\baselineskip
}
%
% Пакет mhchem предоставляет команды для
% набора химических молекулярных формул и уравнений.
\usepackage[version=4,arrows=font]{mhchem}
%\usepackage{expl3,calc}
% длина стрелок
\ExplSyntaxOn
\keys_define:nn { mhchem }
{
arrow-min-length .code:n =
% default is 2em
\cs_set:Npn \__mhchem_arrow_options_minLength:n { {#1} }
}
\ExplSyntaxOff
\mhchemoptions{arrow-min-length=1em}
%
%	Пакет предоставляет макросы для управления строками - 
%	тестирования содержимого строки, извлечения подстрок, 
%	подстановки подстрок и предоставления чисел, 
%	таких как длина строки, позиция или количество 
%	повторов подстроки.
%	https://www.ctan.org/pkg/xstring
%
%	\usepackage{xstring}

%
%	Вставка страниц из PDF-файла
%	Этот пакет упрощает включение внешних многостраничных 
%	PDF-документов в LaTеX документы
% 
\usepackage{pdfpages}
%%%%%%%%%%%%%%%%%%%%%%%%%%%%%%%%%%%%%%%%%%%%%
%
%	МОИ НОВЫЕ КОМАНДЫ
%
% Макрос Существует?
% https://tex.stackexchange.com/questions/164188/ifundefined-actually-defines-macros
\def\ifexists#1{\expandafter\ifx\csname#1\endcsname\relax 0\else 1\fi}
% Pure text from TeX
% https://tex.stackexchange.com/questions/567286/pdfstringdef-turns-accented-characters-into-octal-escape-sequence
\ExplSyntaxOn
\cs_set_eq:NN\textpurify\text_purify:n
\ExplSyntaxOff
% alert
\newcommand{\alert}[2][]{%
#1{\emph{\textcolor{darkred}{#2}}}%
}
% alertx
\newcommand{\alertx}[2][\texttt]{%
#1{\textbf{\textcolor{darkred}{#2}}}%
}
% проверка макроса
\newcommand{\Isdefined}[1]{%
\ifx#1\undefined\relax%
\alertx{\backslash def\detokenize{#1}undefined}%
\else{#1}\fi%
}
% Чистый текст макроса
\newcommand{\MacroTextPurify}[1]{%
\ifx#1\undefined\relax%
\alertx{\backslash def\detokenize{#1}undefined}%
\else\textpurify{#1}%
\fi}
% Пол студента
\newcommand{\detGender}[3]{%
\ifx#1\undefined{для студентов}%
\else%
\ifcase#1\relax{#2}% Gender=0 (женский)
\or{#3}% Gender=1 (мужской)
\else\alertx{Gender=\Gender{ unknow}}%
\fi%
\fi%
}
% пустая строка
\def\emptyline{\par\vspace{\baselineskip}}
% символ '\'
\def\backslash{\char`\\}
% постоянная
\DeclareMathOperator{\const}{const}
% Полужирное начертание для векторов
\newcommand\vect[1]{\mathbfit{#1}}
%\let\vec=\mathbf
% норма
\newcommand{\norma}[1]{\left\lVert#1\right\rVert}
% абсолютное значение
\newcommand{\abs}[1]{\left\lvert#1\right\rvert}
% скалярное произведение векторов
\newcommand{\dotvec}[2]{(\vec{#1},\vec{#2})}
% обыкновенная производная, например $\diff{N_d^{+}}{x}$
\newcommand\diff[2]{ \dfrac{\mathrm{d}#1}{\mathrm{d}#2} }
% обыкновенная производная второго порядка, например $\diff{N_d^{+}}{x}$
\newcommand\diffdiff[2]{ \dfrac{\mathrm{d}^2 #1}{\mathrm{d} #2^2} }
% частная производная, например $\pdiff{N_d^{+}}{x}$
\newcommand\pdiff[2]{ \dfrac{\partial #1}{\partial #2} }
% 1/2
\newcommand\onehalf{ \nicefrac{1}{2} }
% Случайное число \RandInt{min}{max}
\newcommand{\RandInt}[2]{\pgfmathrandominteger{\rndint}{#1}{#2}\rndint}
% Интеграл
\newcommand\intf[4][x]{ \int\limits_{#2}^{#3}{#4}\,\mathrm{d}#1 }
%%%%%%%%%%%%%%%%%%%%%%%%%%%%%%%%%%%%%%%%%%%%%
%
% ТИТУЛЬНЫЙ ЛИСТ
%
% \TitlePage[название документа] или \TitlePage
\newcommand{\TitlePage}[1][\undefined]{
{\centering% начало центрирования
МИНОБРНАУКИ РОССИИ\\
Федеральное государственное бюджетное образовательное учреждение
высшего образования\\
\textbf{<<САРАТОВСКИЙ НАЦИОНАЛЬНЫЙ ИССЛЕДОВАТЕЛЬСКИЙ\\
ГОСУДАРСТВЕННЫЙ УНИВЕРСИТЕТ\\
ИМЕНИ Н.Г. ЧЕРНЫШЕВСКОГО>>}
\emptyline
Кафедра материаловедения,\par
технологии и управления качеством
\emptyline
% название документа (DocumentTitle)
\ifnum1=\ifexists{SubTitle}\MakeUppercase{\SubTitle}\emptyline\fi
%\ifx#1\undefined\relax% не определено
%\else\MakeUppercase{#1}\emptyline%
%\fi
% ЗАГЛАВИЕ
\textbf{\MakeUppercase{\Isdefined{\TITLE}}}
\emptyline
по дисциплине <<\Isdefined{\ModuleTitle}>>\\
% для студентов/студентки/студента
\detGender{\Gender}{студентки}{студента}
\Isdefined{\NoCourse}{ курса }\Isdefined{\NoGroup}{ группы}\\
направления подготовки{ \Isdefined\ProgramCode }
<<\Isdefined{\ProgramTitle}>>
(профиль <<\Isdefined{\ProgramProfile}>>),\\
\Isdefined{\Department}\\% факультет
% Ф.И.О. (родительный падеж)
\ifx\FullNameGenetive\undefined\relax%
%\end{center}%
\else{\FullNameGenetive}%
% пустая строка
\emptyline
% добавляет заполняющее вертикальное пространство
\vfill
% таблица БАРС
\begin{tabular}[l]{b{5cm} b{2.5cm} c}
\toprule
Результат&Баллы&ВСЕГО\\
\midmidrule
Выполнение&&\Isdefined{\Exec}\\
\midrule
Оформление&&\Isdefined{\Polygraphy}\\
\midrule
Устный отчет&&\Isdefined{\OralReport}\\
\bottomrule
\end{tabular}%
% пустая строка
\par\emptyline
% ПОДПИСЬ
\raggedright\noindent
Преподаватель\\профессор, д.т.н., доцент
\hspace{1em}\rule{6.5cm}{0.5pt} 
\hspace{1em} В.В. Симаков\par
\emptyline
\fi%
}\par% окончание центрирования
}
%%%%%%%%%%%%%%%%%%%%%%%%%%%%%%%%%%%%%%%%%%%%%
%
% Колонтитулы
%
% 'Титульный лист'
\fancypagestyle{titlepage}{
\setcounter{page}{1}
% clear all header and footer fields
\fancyhf{}
% линии верхнего и нижнего колонтитулов
\renewcommand{\headrulewidth}{0pt}%
\renewcommand{\footrulewidth}{0pt}%
\cfoot{Саратов~\the\year} % город год
}
% 'Задание'
\fancypagestyle{taskpages}{
% clear all header and footer fields
\fancyhf{}
% нумерация римскими цифрами
%\pagenumbering{roman}
% нумерация арабскими цифрами
\pagenumbering{arabic}
% линии верхнего и нижнего колонтитулов
\renewcommand{\headrulewidth}{0pt}%
\renewcommand{\footrulewidth}{0pt}%
% нижний колонтитул
\lfoot{Задание~
\detGender{\Gender}{получила}{получил}
\hspace{0.25cm}\rule{4cm}{0.5pt}\hspace{0.25cm}%
\Isdefined{\Signature}}% подпись
%\rfoot{\roman{page}}% номера страниц
}
% 'Содержание документа'
\fancypagestyle{bodypages}{
\fancyhf{}
\pagestyle{fancy}
% линии верхнего и нижнего колонтитулов
\renewcommand{\headrulewidth}{0pt}%
\renewcommand{\footrulewidth}{0pt}%
\pagenumbering{arabic}
\rfoot{\arabic{page}}% номера страниц
}
%%%%%%%%%%%%%%%%%%%%%%%%%%%%%%%%%%%%%%%%%%%%%
%
% ЗАДАНИЕ
%
% \TitleTask или \TitleTask['на выполнение чего?']
\newcommand{\TitleTask}[1][лабораторной работы]{%
\newpage
% нумерация страниц с №1
\setcounter{page}{1}
%\newpage\phantomsection
\begin{center}
\textbf{ЗАДАНИЕ}\par
% добавление пункта в СОДЕРЖАНИЕ
%\addcontentsline{toc}{chapter}{Задание на выполнение #1}
%по дисциплине <<\Isdefined{\@ModuleTitle}>>
по дисциплине <<\MacroTextPurify{\ModuleTitle}>>
на выполнение #1 на тему 
<<\textbf{\MacroTextPurify{\TITLE}}>>
\end{center}
\par
}

\usepackage{siunitx}
\newcommand\vect[1]{{\mathrm{#1}}}
% Пропустить \LINESKIP строк
\def\LINESKIP{4}
%%%%%%%%%%%%%%%%%%%%%%%%%%%%%%
%
%	Новое окружение ВАРИАНТ
%
\newenvironment{variant}[2]{% environment name
\ifx\LINESKIP\undefined\relax
% Экзаменационный билет
%\def\ModuleTitle{Численные методы в материаловедение}
\def\ModuleTitle{Численные методы в менеджменте}
\begin{center}
\textbf{Экзаменационный билет № \arabic{page}}\\
по дисциплине <<\MacroTextPurify{\ModuleTitle}>>
\end{center}%
\else% Пропустить строки
\,\vspace{\LINESKIP\baselineskip}
\fi\par
% Сбросить счётчики
\setcounter{table}{0}
\setcounter{figure}{0}
\setcounter{equation}{0}
%
% ЗАДАНИЕ
%
Необходимо решить краевую задачу для установившегося
процесса теплообмена металлического образца, находящегося
в тепловом контакте с:
\par\vspace{\baselineskip}
% Порядок выполнения работы
\textbf{Порядок выполнения работы}
\begin{enumerate}
\item
1
\end{enumerate}
}{%
\par\vspace{\baselineskip}
%
% Рекомендуемые источники
%
\begin{tcolorbox}[tcbref]
\begin{enumerate}
\item
\href{https://youtu.be/1lHyAyHi7eo?list=RDCMUCrpcPMjPil0xtuuYqaoaS1Q}
{Решение краевых задач методом конечных разностей}\\
\url{https://youtu.be/1lHyAyHi7eo?list=RDCMUCrpcPMjPil0xtuuYqaoaS1Q}
%\item
%\href{}
%{}\\
%\url{}
\end{enumerate}
\end{tcolorbox}
\fi
\newpage}
%%%%%%%%%%%%%%%%%%%%%%%%%%%%%%
%
% НАЧАЛО ДОКУМЕНТА
%
\tikzset{
	sample/.style={
		draw=darkred,
		line width=0.35mm,
		fill=red!5,
	},
	clamp/.style={
		draw=black,
		line width=0.35mm,%thick,
		pattern=north east lines,
		pattern color=gray!50,
		decoration={
			snake,
			amplitude=0.75mm,
			segment length=10mm,
			post length=0mm,
			pre length=0mm,
		},
	},
	size/.style={
		font=\small,
		sloped,
		midway,
		above
	},
}
\begin{document}
%\input{../@TeX.Gauss.log}
\begin{figure}[H]\centering
\begin{tikzpicture}
\def\sizeDiameter{25};
\def\sizeLength{150};
\def\sizeClamp{18};
% сетка
%\draw[draw=gray!25,step=5mm,very thin] (-50mm,0mm) grid (50mm,100mm);
% ОБРАЗЕЦ
\def\Diameter{20mm};% диаметр
\def\Length{100mm};% длина
% КРЕПЛЕНИЕ
\def\ClampWidth{50mm};% длина
\def\ClampHeight{15mm};% высота
% положение крепления
\def\ClampBottom{30mm};
% начало отсчета
\coordinate (O) at (0mm,0mm);
% стержень
\path (O) -- ++(-0.5*\Diameter,0mm) coordinate(x);
\draw[sample] (x) 
-- ++(\Diameter,0mm) 
-- ++(0mm,\Length) 
-- ++(-\Diameter,0mm) 
-- cycle;
% ось OX
\path (O) -- ++(0mm,-2mm) coordinate(x);
\draw[very thin,draw=gray,loosely dashdotted]
(x) -- ++(0mm,\Length) -- ++(0mm,4mm);
% размер/диаметр
\path (O) -- ++(-0.5*\Diameter,\Length)coordinate(x0) -- ++(0mm,5mm) coordinate(x);
\draw[<->,thin] (x) -- ++(\Diameter,0mm)
node[size]{$\diameter{\sizeDiameter}$};
\draw (x0) -- ++(0mm,7mm);
\path (x0) -- ++(\Diameter,0mm)coordinate(x0);
\draw (x0) -- ++(0mm,7mm);
%% размер/длина
\path (O) -- ++(0.5*\Diameter,0mm)coordinate(x0) -- ++(\ClampWidth,0mm) -- ++(20mm,0mm)coordinate(x);
\draw[<->] (x) -- ++ (0mm,\Length)
node[size]{$\sizeLength$};
\draw[thin] (x0) -- ++(\ClampWidth,0mm) -- ++(22mm,0mm);
\path (x0) -- ++(0mm,\Length)coordinate(x0);
\draw[thin] (x0) -- ++(\ClampWidth,0mm) -- ++(22mm,0mm);
% крепление
\path (O) -- ++(-0.5*\Diameter,\ClampBottom) coordinate(x);
\draw[clamp] (x) 
-- ++(-\ClampWidth,0mm) 
decorate{-- ++(0mm,\ClampHeight)}
-- ++(\ClampWidth,0mm) 
-- cycle;
\path (x) -- ++ (\Diameter,0mm) coordinate(x);
\draw[clamp] (x) 
-- ++(\ClampWidth,0mm) coordinate(x)
decorate{-- ++(0mm,\ClampHeight)}
-- ++(-\ClampWidth,0mm)
-- cycle;
% положение крепления
\path (O) -- ++(0.5*\Diameter,0mm) -- ++(\ClampWidth,0mm)coordinate(x);
\draw[<->] (x) -- ++ (0mm,\ClampBottom)
node[size]{$25$};
% размер/ширина
\path (O) -- ++(0mm,\ClampBottom) -- ++(0.5*\Diameter,0mm) -- ++(\ClampWidth,0mm)coordinate(x0) -- ++(10mm,0mm)coordinate(x);
\draw[<->] (x) -- ++ (0mm,\ClampHeight)
node[size]{$\sizeClamp$};
\draw[thin] (x0) -- ++(12mm,0mm);
\path (x0) -- ++(0mm,\ClampHeight) coordinate(x0);
\draw[thin] (x0) -- ++(12mm,0mm);
%
% ПОДПИСИ
%
\path (O) -- ++(-0.5*\Diameter,\ClampBottom) -- ++(-\ClampWidth,\ClampHeight)coordinate(x);
\path (x) -- ++(\ClampWidth,0mm) node[size]{вакуумный уплотнитель};
\path (O) -- ++(-0.5*\Diameter,\ClampBottom) -- ++(0mm,\ClampHeight)coordinate(x0);
\path (O) -- ++(-0.5*\Diameter,\Length)coordinate(x);
\path (x0) -- (x) node[size,darkred]{токоввод};
\end{tikzpicture}
\caption{Эскиз закрепленного токоввода в вакуумном уплотнителе}
\label{fig:draft}
\end{figure}

Определить стационарное распределение температуры 
в токовводе, который закреплен вакуумным уплотнителем
(рисунок \ref{fig:draft}).

\begin{table}[H]\
\caption{Геометрические и технические параметры токоввода}
\label{tab:setup}\small
\begin{tabular*}{\textwidth}{l@{\extracolsep{\fill}}*2{l}l}
\toprule
Наименование параметра&Обозначение&Величина\\
\midmidrule
Материал токоввода&Cu&медь\\
Длина токоввода&
$L$&\SI[mode=text]{150}{\milli\meter}\\
Диаметр токоввода&
$\diameter$&\SI[mode=text]{15}{\milli\meter}\\
Сила электрического тока через токоввод&
$I$&\SI[mode=text]{15}{\ampere}\\
Толщина уплотнителя&
$d$&\SI[mode=text]{10}{\milli\meter}\\
\bottomrule
\end{tabular*}
\end{table}

\newpage
Метод Гаусса решения систем линейных уравнений.
\begin{enumerate}
\item
Разделим каждую строку матрицы на значение её
элемента в первом столбце, т.е. первую строку делим на $2$,
вторую на $4$, третью на 3:
\begin{gather*}
\begin{pmatrix}[ccc|c]
\textcolor{red}{2}&3&1&10\\
\textcolor{red}{4}&5&6&31\\
\textcolor{red}{3}&1&5&22\\
\end{pmatrix}
\;\to\;% 1 строка
\begin{pmatrix}[ccc|c]
\mathbf{1}&\frac{3}{2}&\frac{1}{2}&\frac{10}{2}\\
4&5&6&31\\
3&1&5&22\\
\end{pmatrix}
\;\to\;% 2 строка
\begin{pmatrix}[ccc|c]
\mathbf{1}&\frac{3}{2}&\frac{1}{2}&\frac{10}{2}\\
\mathbf{1}&\frac{5}{4}&\frac{6}{4}&\frac{31}{4}\\
3&1&5&22\\
\end{pmatrix}
\;\to\;% 3 строка
\begin{pmatrix}[ccc|c]
\mathbf{1}&\frac{3}{2}&\frac{1}{2}&\frac{10}{2}\\
\mathbf{1}&\frac{5}{4}&\frac{6}{4}&\frac{31}{4}\\
\mathbf{1}&\frac{1}{3}&\frac{5}{3}&\frac{22}{3}\\
\end{pmatrix}
\end{gather*}

Вычитаем из второй и третьей строк матрицы её первую строку:
\begin{gather*}
\begin{pmatrix}[ccc|c]
1&\frac{3}{2}&\frac{1}{2}&\frac{10}{2}\\
1&\frac{5}{4}&\frac{6}{4}&\frac{31}{4}\\
1&\frac{1}{3}&\frac{5}{3}&\frac{22}{3}\\
\end{pmatrix}
\;\to\;% 2 строка
\begin{pmatrix}[ccc|c]
1&\frac{3}{2}&\frac{1}{2}&\frac{10}{2}\\
\mathbf{0}&-\frac{1}{4}&1&\frac{11}{4}\\
1&\frac{1}{3}&\frac{5}{3}&\frac{22}{3}\\
\end{pmatrix}
\;\to\;% 3 строка
\begin{pmatrix}[ccc|c]
1&\frac{3}{2}&\frac{1}{2}&\frac{10}{2}\\
\mathbf{0}&-\frac{1}{4}&1&\frac{11}{4}\\
\mathbf{0}&-\frac{7}{6}&\frac{7}{6}&\frac{7}{3}\\
\end{pmatrix}
\end{gather*}

\item
Разделим вторую строку матрицы на $-\frac{1}{4}$, третью строку на $-\frac{7}{6}$:
\begin{gather*}
% 2 строка
\begin{pmatrix}[ccc|c]
1&\frac{3}{2}&\frac{1}{2}&\frac{10}{2}\\
0&-\frac{1}{4}&1&\frac{11}{4}\\
0&-\frac{7}{6}&\frac{7}{6}&\frac{7}{3}\\
\end{pmatrix}
\;\to\;% 3 строка
\begin{pmatrix}[ccc|c]
1&\frac{3}{2}&\frac{1}{2}&\frac{10}{2}\\
0&\mathbf{1}&-4&-11\\
0&\mathbf{1}&-1&-2\\
\end{pmatrix}
\end{gather*}

Вычитаем из третьей строки матрицы её вторую строку:
\begin{gather*}
\begin{pmatrix}[ccc|c]
1&\frac{3}{2}&\frac{1}{2}&\frac{10}{2}\\
0&1&-4&-11\\
0&1&-1&-2\\
\end{pmatrix}
\;\to\;% 3 строка
\begin{pmatrix}[ccc|c]
1&\frac{3}{2}&\frac{1}{2}&\frac{10}{2}\\
0&1&-4&-11\\
0&\mathbf{0}&3&9\\
\end{pmatrix}
\end{gather*}

\item
Разделим третью строку матрицы на $3$:
\begin{gather*}
\begin{pmatrix}[ccc|c]
1&\frac{3}{2}&\frac{1}{2}&\frac{10}{2}\\
0&1&-4&-11\\
0&0&3&9\\
\end{pmatrix}
\;\to\;% 3 строка
\begin{pmatrix}[ccc|c]
1&\frac{3}{2}&\frac{1}{2}&\frac{10}{2}\\
0&1&-4&-11\\
0&0&\mathbf{1}&3\\
\end{pmatrix}
\end{gather*}
\end{enumerate}

Обратный ход метода Гаусса\par
\begin{enumerate}
\item
Из третьего уравнения системы (третья строка матрицы)
определяем неизвестное $x_3$:
\begin{gather*}
x_3=3
\end{gather*}
\item
Из второго уравнения системы (вторая строка матрицы)
определяем неизвестное $x_2$:
\begin{gather*}
x_2-4\,x_3=-11\;\to\;
x_2=4\,x_3-11\\
x_2=4\cdot3-11=1
\end{gather*}
\item
Из первого уравнения системы (первая строка матрицы)
определяем неизвестное $x_1$:
\begin{gather*}
x_1+\frac{3}{2}\,x_2+\frac{1}{2}\,x_3=\frac{10}{2}\;\to\;
x_1=-\frac{3}{2}\,x_2-\frac{1}{2}\,x_3+5\\[1ex]
x_1=-\frac{3}{2}\cdot{1}-\frac{1}{2}\cdot{3}+5=2
\end{gather*}

Таким образом, найдено решение системы:
\begin{gather*}
\vec{\mathring{x}}=\begin{pmatrix}2\\1\\3\end{pmatrix}
\end{gather*}
\end{enumerate}


\subsection{Метод Гаусса с выбором главного элемента}
\begin{enumerate}
\item
Выбираем максимальный по модулю элемент в матрице 
(главный элемент): вторая строка, третий столбец
(\textcolor{red}{выделен красным цветом}).
\begin{gather*}
\begin{pmatrix}[ccc|c]
2&3&1&10\\
4&5&\textcolor{red}{6}&31\\
3&1&5&22\\
\end{pmatrix}
\end{gather*}
Делим каждую строку матрицы на значение элемента
матрицы в столбце главного элемента, т.е.
первую строку делим на $1$, вторую строку на $6$,
а третью строку на $5$:
\begin{gather*}
\begin{pmatrix}[ccc|c]
2&3&1&10\\
4&5&\textcolor{red}{6}&31\\
3&1&5&22\\
\end{pmatrix}
\;\to\;% 1 строка
\begin{pmatrix}[ccc|c]
2&3&\mathbf{1}&10\\
4&5&\textcolor{red}{6}&31\\
3&1&5&22\\
\end{pmatrix}
\;\to\;% 2 строка
\begin{pmatrix}[ccc|c]
2&3&\mathbf{1}&10\\
\frac{4}{6}&\frac{5}{6}&\textcolor{red}{\mathbf{1}}&\frac{31}{6}\\
3&1&5&22\\
\end{pmatrix}
\;\to\;% 3 строка
\begin{pmatrix}[ccc|c]
2&3&\mathbf{1}&10\\
\frac{4}{6}&\frac{5}{6}&\textcolor{red}{\mathbf{1}}&\frac{31}{6}\\
\frac{3}{5}&\frac{1}{5}&\mathbf{1}&\frac{22}{5}\\
\end{pmatrix}
\end{gather*}

Вычитаем из первой и третьей строки вторую строку:
\begin{gather*}
\begin{pmatrix}[ccc|c]
2&3&1&10\\
\frac{4}{6}&\frac{5}{6}&\textcolor{red}{1}&\frac{31}{6}\\
\frac{3}{5}&\frac{1}{5}&1&\frac{22}{5}\\
\end{pmatrix}
\;\to\;% 1 строка
\begin{pmatrix}[ccc|c]
\frac{4}{3}&\frac{13}{6}&\mathbf{0}&\frac{29}{6}\\
\frac{4}{6}&\frac{5}{6}&\textcolor{red}{1}&\frac{31}{6}\\
\frac{3}{5}&\frac{1}{5}&1&\frac{22}{5}\\
\end{pmatrix}
\;\to\;% 3 строка
\begin{pmatrix}[ccc|c]
\frac{4}{3}&\frac{13}{6}&\mathbf{0}&\frac{29}{6}\\
\frac{4}{6}&\frac{5}{6}&\textcolor{red}{1}&\frac{31}{6}\\
-\frac{1}{15}&-\frac{19}{30}&\mathbf{0}&-\frac{23}{30}\\
\end{pmatrix}
\end{gather*}

\item
Исключаем из рассмотрения строку с текущим главным 
элементом (вторую) и выбираем новый главный элемент матрицы 
(первая строка, второй столбец): 
\begin{gather*}
\begin{pmatrix}[ccc|c]
\frac{4}{3}&\textcolor{red}{\frac{13}{6}}&0&\frac{29}{6}\\
\frac{4}{6}&\frac{5}{6}&\textcolor{red}{1}&\frac{31}{6}\\
-\frac{1}{15}&-\frac{19}{30}&0&-\frac{23}{30}\\
\end{pmatrix}
\end{gather*}

Делим каждую строку матрицы на значение элемента
матрицы в столбце главного элемента, т.е.
первую строку делим на $\frac{13}{6}$, а третью строку на $-\frac{19}{30}$:
\begin{gather*}
\begin{pmatrix}[ccc|c]
\frac{4}{3}&\textcolor{red}{\frac{13}{6}}&0&\frac{29}{6}\\
\frac{4}{6}&\frac{5}{6}&\textcolor{red}{1}&\frac{31}{6}\\
-\frac{1}{15}&-\frac{19}{30}&0&-\frac{23}{30}\\
\end{pmatrix}
\;\to\;% 1 строка
\begin{pmatrix}[ccc|c]
\frac{8}{13}&\textcolor{red}{\mathbf{1}}&0&\frac{29}{13}\\
\frac{4}{6}&\frac{5}{6}&\textcolor{red}{1}&\frac{31}{6}\\
\frac{2}{19}&\mathbf{1}&0&\frac{23}{19}\\
\end{pmatrix}
\end{gather*}

Вычитаем из третьей строки матрицы её первую строку:
\begin{gather*}
\begin{pmatrix}[ccc|c]
\frac{8}{13}&\textcolor{red}{1}&0&\frac{29}{13}\\
\frac{4}{6}&\frac{5}{6}&\textcolor{red}{1}&\frac{31}{6}\\
\frac{2}{19}&1&0&\frac{23}{19}\\
\end{pmatrix}
\;\to\;% 3 строка
\begin{pmatrix}[ccc|c]
\frac{8}{13}&\textcolor{red}{1}&0&\frac{29}{13}\\
\frac{4}{6}&\frac{5}{6}&\textcolor{red}{1}&\frac{31}{6}\\
-\frac{126}{247}&\mathbf{0}&0&-\frac{252}{247}\\
\end{pmatrix}
\end{gather*}

\item
Исключаем из рассмотрения строку с текущим главным 
элементом (первую) и выбираем новый главный 
элемент матрицы (третья строка, первый столбец): 
\begin{gather*}
\begin{pmatrix}[ccc|c]
\frac{8}{13}&\textcolor{red}{1}&0&\frac{29}{13}\\
\frac{4}{6}&\frac{5}{6}&\textcolor{red}{1}&\frac{31}{6}\\
\textcolor{red}{-\frac{126}{247}}&0&0&-\frac{252}{247}\\
\end{pmatrix}
\end{gather*}

Делим каждую строку матрицы на значение элемента
матрицы в столбце главного элемента, т.е.
третью строку делим на $-\frac{126}{247}$:
\begin{gather*}
\begin{pmatrix}[ccc|c]
\frac{8}{13}&\textcolor{red}{1}&0&\frac{29}{13}\\
\frac{4}{6}&\frac{5}{6}&\textcolor{red}{1}&\frac{31}{6}\\
\textcolor{red}{-\frac{126}{247}}&0&0&-\frac{252}{247}\\
\end{pmatrix}
\;\to\;% 3 строка
\begin{pmatrix}[ccc|c]
\frac{8}{13}&\textcolor{red}{1}&0&\frac{29}{13}\\
\frac{4}{6}&\frac{5}{6}&\textcolor{red}{1}&\frac{31}{6}\\
\textcolor{red}{\mathbf{1}}&0&0&2\\
\end{pmatrix}
\end{gather*}
\end{enumerate}

Обратный ход метода Гаусса\par
Определим неизвестные из уравнений системы 
в обратном порядке следования номеров столбцов главных элементов,
т.е. $1\to2\to3$:
\begin{enumerate}
\item
Из третьего уравнения системы определим неизвестное $x_1$:
\begin{gather*}
x_1=2
\end{gather*}
\item
Из первого уравнения системы определим неизвестное $x_2$:
\begin{gather*}
\frac{8}{13}\,x_1+x_2=\frac{29}{13}\;\to\;
x_2=-\frac{8}{13}\,x_1+\frac{29}{13}\\[1ex]
x_2=-\frac{8}{13}\cdot2+\frac{29}{13}=1
\end{gather*}
\item
Из второго уравнения системы определим неизвестное $x_3$:
\begin{gather*}
\frac{4}{6}\,x_1+\frac{5}{6}\,x_2+x_3=\frac{31}{6}\;\to\;
x_3=-\frac{4}{6}\,x_1-\frac{5}{6}\,x_2+\frac{31}{6}\\[1ex]
x_3=-\frac{4}{6}\cdot2-\frac{5}{6}\cdot1+\frac{31}{6}=3
\end{gather*}
Таким образом, найдено решение системы линейных уравнений:
\begin{gather*}
\vec{\mathring{x}}=\begin{pmatrix}2\\1\\3\end{pmatrix}
\quad\text{или}\quad\vect{\mathring{x}}=(2,1,3)^\mathrm{T}\\
\end{gather*}
\end{enumerate}

Проведём \emph{проверку решения} системы уравнений
методом прямой подстановки. Для этого подставим 
найденный вектор неизвестных 
$\vec{\mathring{x}}=(2,1,3)^\mathrm{T}$ 
в исходную систему уравнений:
\begin{gather*}
\begin{pmatrix}
2&3&1\\
4&5&6\\
3&1&5\\
\end{pmatrix}
\cdot
\begin{pmatrix}
2\\
1\\
3\\
\end{pmatrix}
=
\begin{pmatrix}
2\cdot2+3\cdot1+1\cdot3\\
4\cdot2+5\cdot1+6\cdot3\\
3\cdot2+1\cdot1+5\cdot3\\
\end{pmatrix}
=
\begin{pmatrix}
10\\
31\\
22\\
\end{pmatrix}
\quad\to\quad
\begin{pmatrix}
10\\
31\\
22\\
\end{pmatrix}
=\begin{pmatrix}
10\\
31\\
22\\
\end{pmatrix}
\end{gather*}
%%%%%%%%%%%%%%%%%%%%%%%%%%%%%%
%
% КОНЕЦ ДОКУМЕНТА
%
\end{document}