\pgfdeclarelayer{pre main}
\pgfdeclarelayer{background}
\pgfdeclarelayer{foreground}
\pgfsetlayers{background,pre main,main,foreground} 
%
%	Задача Коши для систем дифференциальных уравнений
%
\newpage
\section{Задача Коши для систем 
обыкновенных дифференциальных уравнений}
При рассмотрении физических явлений и процессов часто 
не удается найти непосредственную взаимосвязь между 
величинами, характеризующими эволюционный, 
т.е. изменяющийся во времени, процесс. 
Однако во многих случаях можно установить связь между 
искомыми характеристиками изучаемого явления (функциями) 
и скоростями их изменения относительно других переменных, 
т.е. найти уравнения, в которые входят производные от
неизвестных функций. Такие уравнения называют 
\emph{дифференциальными}.

Обыкновенными дифференциальными 
уравнениями можно описать задачи движения системы 
взаимодействующих материальных точек, химической кинетики, 
электрических цепей, сопротивления материалов 
(например, статический прогиб упругого стержня) и многие другие.

Кроме того, ряд важных задач для уравнений в частных производных 
сводится к задачам для обыкновенных дифференциальных 
уравнений. Например, многомерная задача допускающая разделение 
пространственных переменных (например, задачи 
на нахождение собственных колебаний упругих балок 
и мембран простейшей формы, определение спектра 
собственных значений энергии частицы 
в сферически-симметричном поле), 
или если решение многомерной задачи зависит только от 
некоторой комбинации переменных (автомодельные решения), 
то задача нахождения решения уравнений в частных производных 
сводится к задачам на собственные значения 
для обыкновенных дифференциальных уравнений.
Поэтому, методы решения обыкновенных дифференциальных 
уравнений занимает важное место среди прикладных 
задач физики, химии и техники.

Задача Коши обычно возникает при анализе процессов, 
определяемых дифференциальным законом эволюции и 
начальным состоянием (начальным условием) 
и для системы обыкновенных дифференциальных уравнений 
эта задача формулируется в виде системы уравнений:
\begin{gather}\label{eq:ODE_SYS}
\diff{\vect{u}}{t} = \vect{f}(t, \vect{u}),
\quad
\vect{u}(0)=\mathring{\vect{u}},
\end{gather}
где $\vect{u}(t)$ -- неизвестные функции, которые подлежат 
определению;
$\vect{f}(t, \vect{u})$ -- известные функции, 
зависящие от времени и неизвестных функций;
$\mathring{\vect{u}}$ -- \emph{начальные условия}, 
т.е. значения неизвестных функций в начальный момент времени 
$(t=0)$. 

Система обыкновенных дифференциальных 
уравнений первого порядка и начальные условия
\eqref{eq:ODE_SYS} в развернутом виде могут быть 
записаны как:
\begin{gather}\label{eq:ODE_IC}
\left\{\begin{matrix}
\diff{u_1}{t}&=&f_1(t, u_1,u_2,\cdots,u_n),&u_1(0)&=&\mathring u_1\\[1em]
\diff{u_2}{t}&=&f_2(t, u_1,u_2,\cdots,u_n),&u_2(0)&=&\mathring u_2\\[1em]
\hdotsfor{1}&=&\hdotsfor{1}&\hdotsfor{1}&=&\dots\\[1em]
\diff{u_n}{t}&=&f_n(t, u_1,u_2,\cdots,u_n),&u_n(0)&=&\mathring u_n
\end{matrix}\right.,
\end{gather}
где $n$ -- количество дифференциальных уравнений 
в системе \eqref{eq:ODE_SYS}.

Точное решение систем дифференциальных уравнений 
вида \eqref{eq:ODE_IC}, которые описывают 
многообразие прикладных задач, может быть 
получено лишь в исключительных случаях. 
Поэтому возникает необходимость 
\emph{приближенного решения} таких задач. 
В настоящее время создано и разработано 
значительное число приближенных методов решения 
дифференциальных уравнений, каждый из которых имеет 
свои преимущества и недостатки.

%
% Метод Эйлера
%
\emptyline
\subsection{Метод Эйлера решения задачи Коши}
Будем полагать, что решение задачи Коши \eqref{eq:ODE_IC} существует, 
единственно и обладает необходимыми свойствами гладкости.

Введем временную сетку, т.е. будем
рассматривать изменения неизвестных функций 
только в заданные моменты времени:
\begin{gather*}
\vect{t}=\{t_j\},\quad j=0,1,2,\dots,m,
\end{gather*}
где $j$ -- номер временного интервала;
$\Delta t_{j+1} = (t_{j+1}-t_j)$ -- шаг сетки, т.е. временной интервал 
между двумя последовательными моментами времени;
$m$ -- количество узлов временной сетки.

Основная идея метода Эйлера заключается в предположении, 
о том что неизвестные функции $\vect{u}(t)$ изменяются линейно
в интервале $[t_j,t_{j+1}]$ между двумя соседними узлами 
временной сетки и интерполяция неизвестных функций 
проводится полиномом первого порядка $\vect{L}_1(t)$:
\begin{gather*}
\vect{u}(t)\approx\vect{L}_1(t)=
\dfrac{t-t_{j+1}}{t_j-t_{j+1}}\cdot\vect{u}(t_j)+
\dfrac{t-t_j}{t_{j+1}-t_j}\cdot\vect{u}(t_{j+1}).
\end{gather*}

Производная от неизвестной функции приближенно 
аппроксимируется выражением вида:
\begin{gather}
\diff{\vect{u}}{t}\approx\vect{L}^{\prime}_1(t) =
\dfrac{\vect{u}(t_{j+1})-\vect{u}(t_j)}{t_{j+1}-t_j},
\end{gather}
где $t_{j+1}$ и $t_j$ -- два последовательных момента времени.

Тогда систему дифференциальных уравнений первого порядка 
\eqref{eq:ODE_SYS} приближенно можно записать в виде:
\begin{gather}\label{eq:ODE_NUM}
\dfrac{\vect{u}(t_{j+1})-\vect{u}(t_{j})}{\Delta t_{j+1}}\approx
\vect{f}\left(t_j, \vect{u}(t_{j})\right).
\end{gather}

Относительно неизвестных $\vect{u}(t_{j+1})$ 
это система линейных алгебраических уравнений и 
решение системы \eqref{eq:ODE_NUM} находится явным образом 
по рекуррентным формулам:
\begin{gather}\label{eq:ODE_EULER}
\vect{u}(t_{j+1})=\vect{u}(t_{j})+\Delta t_{j+1}\cdot\vect{f}\left(t_j, \vect{u}(t_{j})\right),
\quad
\vect{u}(0)=\mathring{\vect{u}}.
\end{gather}

Метод Эйлера является простейшим численным методом 
решения задачи Коши. Блок-схема алгоритма метода Эйлера
представлена на рисунке \ref{fig:ODE_IC:scheme Euler}. 
К недостаткам метода можно отнести малую точность и 
систематическое накопление ошибок.

Для простоты рассмотрим только одно дифференциальное уравнение 
с единственным начальным условием:
\begin{gather*}
y(t_{j+1})=y(t_{j})+\Delta t_{j+1}\cdot f\left(t_j, y(t_{j})\right),
\quad y(0)=\mathring y
\end{gather*}

На рисунке \ref{fig:EULER} представлена графическая иллюстрация 
метода Эйлера численного решения (маркеры) задачи Коши 
для одного дифференциального уравнения первого порядка
(сплошной линией представлено точное решение).
% *******************************
%	Ломаная Эйлера
%
\begin{figure}[H]\centering
\begin{tikzpicture}
\begin{axis}[% оси координат
xlabel=\empty,ylabel=\empty,
xmin=-0.5,xmax=8,xtick={0,1,4,7},xticklabels={$0$,$t_1$,$t_j$,$t_n$},
ymin=-0.25,ymax=3,ytick={0,1,2.08,2.59},yticklabels={$y(0)$,$y(t_1)$,$y(t_j)$,$y(t_n)$},
]
\addplot[name path=A,very thick,red,domain=0:7,samples=50]{ln(1+x)};
\addplot[name path=B,ball darkblue,thin]coordinates{(0,0)(1,1)(2,1.5)(3,1.83)(4,2.08)(5,2.28)(6,2.45)(7,2.59)};
\addplot[blue!5] fill between [of=A and B];
\end{axis}
%\draw[thick,red] ($(GRAPH.south)-(0,2em)$) node {a};
\end{tikzpicture}
\caption{Иллюстрация метода Эйлера:\linebreak
неизвестная кривая выделена красным цветом, 
а ее полигональная аппроксимация (ломаная Эйлера) -- синим}
\label{fig:EULER}
\end{figure}

%
% *** Алгоритм метода Эйлера
%
\begin{figure}[H]\centering
\begin{tikzpicture}[
font=\small,%
start chain=going below,%
node distance=10mm,%
>={Straight Barb[angle=45:1.5mm 1]},%
]
% начало
\node[beginendnode,on chain,join]{начало};
% начальное условие
\node[IOnode,on chain,join]
{$\vect{t}=\left\{0,t_1,t_2,\ldots,t_m\right\}$};
\node[IOnode,on chain,join]
{$\vect{u}(0)=\left\{u_1(0),u_2(0),\ldots,u_n(0)\right\}$};
% начало цикла
\node[rectnode,on chain,join]{$j=0$};
% интервал времени
\node[rectnode,on chain,join](loop)
{$\Delta{t}_{j+1}=t_{j+1}-t_j$};
% решение дифференциального уравнения
\node[rectnode,on chain,join]
{$\vect{u}(t_{j+1})=\vect{u}(t_j)+
\Delta t_{j+1}\cdot\vect{f}\left(t_j, \vect{u}(t_{j})\right)$};
% условие
\node[ifthenelsenode,on chain,join](compare){$j<m$};
\draw(compare.east) node[above right]{да};
\draw(compare.south) node[below right]{нет};
% инкремент
\begin{scope}[start branch=b,node distance=25mm]
\node[rectnode,on chain=going right,join](jloop){$j=j+1$};
\draw[->,thick](jloop.north) |- (loop.east);
\end{scope}
% решение
\begin{scope}[node distance=12mm]
\node[IOnode,on chain,join](output)
{$\vect{u}(\vect{t})=\{u_1(\vect{t}),u_2(\vect{t}),\ldots,u_n(\vect{t})\}$};
\end{scope}
% конец
\node[beginendnode,on chain,join]{конец};
\end{tikzpicture}
\caption{Блок-схема алгоритма численного решения
системы дифференциальных уравнений методом Эйлера}
\label{fig:ODE_IC:scheme Euler}
\end{figure}

%
%	Оценка погрешности решения задачи Коши
%
\subsection{Оценка погрешности решения задачи Коши}
Интегрирование системы дифференциальных уравнений 
\eqref{eq:ODE_SYS} по временной переменной $t$ 
с учетом начальных условий:
\begin{gather}\label{eq:ODE_SYS_INT}
\vect{v}(t)=\vect{u}(0)+\int\limits_{0}^{t}\vect{f}(\xi, \vect{v})\,\mathrm{d}\xi.
\end{gather}

Уравнение \eqref{eq:ODE_SYS_INT} является 
интегральным уравнением для неизвестной функции $\vect{v}(t)$,
а его решение эквивалентно решению задачи Коши 
\eqref{eq:ODE_SYS}, что можно проверить прямой подстановкой
\eqref{eq:ODE_SYS_INT} в \eqref{eq:ODE_SYS}.

На временной сетке $\{t_j\}$ интеграл в правой части равенства 
\eqref{eq:ODE_SYS_INT} приближенно вычисляется по 
\emph{формуле трапеций}:
\begin{gather}\label{eq:ODE_SYS_INT_TRAP}
\int\limits_{0}^{t_{j+1}}\vect{f}(\xi, \vect{v})\,\mathrm{d}\xi
\approx\sum\limits_{k=0}^{j}
\dfrac{
\vect{f}\left(t_{k+1},\vect{v}(t_{k+1})\right)+
\vect{f}\left(t_{k},\vect{v}(t_{k})\right)
}{2}\cdot(t_{k+1}-t_k).
\end{gather}

Воспользовавшись \eqref{eq:ODE_SYS_INT_TRAP}, 
выражение для решения интегрального уравнения 
\eqref{eq:ODE_SYS_INT} можно записать 
в рекуррентной форме:
\begin{gather}\label{eq:ODE_INT_EQ}
\vect{v}(t_{j+1})=\vect{v}(t_{j})+
\dfrac{
\vect{f}\left(t_{j+1},\vect{v}(t_{j+1})\right)+
\vect{f}\left(t_{j},\vect{v}(t_{j})\right)
}{2}\cdot\Delta t_{j+1}.
\end{gather}

Для определения приближенного значения решения 
интегрального уравнения \eqref{eq:ODE_INT_EQ}
могут быть использованы значения неизвестных функций 
${\vect{u}}(t_{j})$, рассчитанные по методу Эйлера
\eqref{eq:ODE_EULER} на $j$-ом временном слое:
\begin{gather}
\vect{v}(t_{j+1})\approx\vect{v}(t_{j})+
\dfrac{
\vect{f}\left(t_{j+1},{\vect{u}}(t_{j+1})\right)+
\vect{f}\left(t_{j},{\vect{u}}(t_{j})\right)
}{2}\cdot\Delta t_{j+1}.
\end{gather}

В качестве предельной абсолютной погрешности 
приближенного решения $\vect{u}(t_{j})$ 
задачи Коши \eqref{eq:ODE_SYS} можно принять 
какую-либо норму:
\begin{gather}
\vect{\epsilon}(t_j)=\vect{v}(t_j) - \vect{u}(t_j)
\end{gather}

Контроль точности приближенного решения 
может вестись покомпонентно или по норме.
Для различных компонент решения задачи $\vect{u}(t_j)$ 
могут использоваться различные допустимые значения 
погрешности.
Контроль точности по норме означает, что контролируется 
некоторая определенная норма оценки погрешности
(рисунок \ref{fig:ODE_IC:scheme error Euler}):
\begin{gather*}
\norma{\,\vect{\epsilon}\,}_{\infty}=\max_{i=1..n} \abs{\epsilon_i}
\quad
\norma{\,\vect{\epsilon}\,}_1=\sum\limits_{i=1}^n \abs{\epsilon_i},
\quad
\norma{\,\vect{\epsilon}\,}_2=\sqrt{\sum\limits_{i=1}^n \epsilon_i^2}
\end{gather*}

%
% *** Алгоритм оценки погрешности метода Эйлера
%
\begin{figure}[H]\centering
\begin{tikzpicture}[
font=\small,%
start chain=going below,%
node distance=10mm,%
>={Straight Barb[angle=45:1.5mm 1]},shorten >=0.5pt,%
]
% соединитель
\node[beginendnode,on chain,join]{начало};
% начальное условие
\node[IOnode,on chain,join]
{$\vect{t}=\left\{0,t_1,t_2,\ldots,t_m\right\}$};
\node[IOnode,on chain,join]
{$\vect{u}(\vect{t})=\left\{u_1(\vect{t}),u_2(\vect{t}),\ldots,u_n(\vect{t})\right\}$};
% начало цикла
\node[rectnode,on chain,join]{$j=0$};
% интервал времени
\node[rectnode,on chain,join](loop){$\Delta{t}_{j+1}=t_{j+1}-t_j$};
% решение интегрального уравнения
\node[rectnode,on chain,join]
{$\vect{v}(t_{j+1})=\vect{v}(t_{j})+
\dfrac{
\vect{f}\left(t_{j+1},{\vect{u}}(t_{j+1})\right)+
\vect{f}\left(t_{j},{\vect{u}}(t_{j})\right)
}{2}\cdot\Delta t_{j+1}$};
% погрешность
\node[rectnode,on chain,join]
{$\vect{\epsilon}(t_j)=\vect{v}(t_j) - \vect{u}(t_j)$};
% конец цикла
\node[ifthenelsenode,on chain,join](compare){$j<m$};
\draw(compare.east) node[above right]{да};
\draw(compare.south) node[below right]{нет};
\begin{scope}[start branch=b,node distance=40mm]
\node[rectnode,on chain,join](increment){$j=j+1$};
\draw[->,thick] (increment.north) |- (loop.east);
\end{scope}
% решение
\begin{scope}[node distance=15mm]
\node[IOnode,on chain,join]
{$\vect{\epsilon}(\vect{t})=\left\{
\epsilon_1(\vect{t}),\epsilon_2(\vect{t}),\ldots,\epsilon_n(\vect{t})
\right\}$};
\end{scope}
% конец
\node[beginendnode,on chain,join]{конец};
\end{tikzpicture}
\caption{Блок-схема для вычисления погрешности метода Эйлера}
\label{fig:ODE_IC:scheme error Euler}
\end{figure}

%
%	Численное решение задачи Коши методом Эйлера
%
\subsection{Численное решение задачи Коши методом Эйлера}
Применяя метод Эйлера, найдем решение задачи Коши
системы дифференциальных уравнений:
\begin{gather}\label{eq:ODE_MY}
\left\{\begin{matrix}[lclrcl]
\diff{u_1}{t}&=&0.2\cdot t+u_2,&u_1(0)&=&1\\[1em]
\diff{u_2}{t}&=&-\dfrac{u_1}{2},&u_2(0)&=&0
\end{matrix}\right.,
\end{gather}
в пределах отрезка $t\in[0, 10]$ на равномерной сетке с количеством временных интервалов $n=5$.

Введем обозначения
\begin{gather*}
\left\{\begin{array}{rcl}
f_1(t)&=&{0,2}\cdot{t}+u_2(t)\\[1em]
f_2(t)&=&-\dfrac{u_1(t)}{2}
\end{array}\right.,
\end{gather*}
где $f_1$ и $f_2$ -- функции, стоящие в правых частях 
дифференциальных уравнений системы \eqref{eq:ODE_MY}:

Рекуррентные соотношения \eqref{eq:ODE_EULER} 
для решения задачи Коши \eqref{eq:ODE_MY} методом Эйлера:
\begin{gather}\label{eq:ODE_SYS_RR}
\left\{\begin{matrix}
u_1(t_{j+1})&=&u_1(t_{j})+\Delta{t}\cdot f_1(t_j),&u_1(0)&=&1\\
u_2(t_{j+1})&=&u_2(t_{j})+\Delta{t}\cdot f_2(t_j),&u_2(0)&=&0
\end{matrix}\right.,
\end{gather}
где $\Delta{t}=(t_{j+1}-t_j)$ -- временной шаг метода Эйлера, 
т.е. интервал между двумя последовательными моментами времени

Решение системы интегральных уравнений \eqref{eq:ODE_SYS_INT} 
определяется рекуррентными соотношениями:
\begin{gather}\label{eq:ODE_SYS_INT_RR}
\left\{\begin{matrix}
\tilde{u}_1(t_{j+1})&=&\tilde{u}_1(t_{j})+\dfrac{f_1(t_{j})+f_1(t_{j+1})}{2}\cdot\Delta{t},&\tilde{u}_1(0)&=&1\\[1em]
\tilde{u}_2(t_{j+1})&=&\tilde{u}_2(t_{j})+\dfrac{f_2(t_{j})+f_2(t_{j+1})}{2}\cdot\Delta{t},&\tilde{u}_2(0)&=&0
\end{matrix}\right.
.\end{gather}

Определим временной шаг метода Эйлера, зная длину временного отрезка (``время наблюдения``) 
и количество интервалов:
\begin{gather*}
\Delta{t}=\dfrac{T-t_0}{n} = \dfrac{10-0}{5}=2,
\end{gather*}
где $t_0=0$ -- начальный момент времени;
$T=10$ -- максимальное время (``время наблюдения``).

Введем по переменной $t$ равномерную сетку с шагом $\tau=2$:
\begin{gather*}
\vect{t}=\{t_j\}=\{0,2,4,6,8,10\}
\end{gather*}

Последовательно определяем приближенное решение 
задачи Коши \eqref{eq:ODE_MY} методом Эйлера,
используя рекуррентные соотношения \eqref{eq:ODE_SYS_RR}.
\begin{enumerate}
\item
Определим значения неизвестных функций $u_1$ и $u_2$ в точке $t_1=2$:
\begin{gather*}\begin{array}{rcl}
f_1(0)&=&0.2\cdot0+u_2(0)=0.2\cdot0+0=0\\[1em]
f_2(0)&=&-\dfrac{u_1(0)}{2}=-\dfrac{1}{2}=-0.5
\end{array}\end{gather*}
%
\begin{gather*}
\left\{\begin{array}{rcl}
u_1(2)&=&u_1(0)+2\cdot{f_1(0)}=1+2\cdot0=1\\
u_2(2)&=&u_2(0)+2\cdot{f_2(0)}=0+2\cdot(-0.5)=-1
\end{array}\right..
\end{gather*}
\item
Определим значения неизвестных функций $u_1$ и $u_2$ в точке $t_2=4$:
\begin{gather*}\begin{array}{rcl}
f_1(2)&=&0.2\cdot2+u_2(2)=0.2\cdot2+(-1)=-0.6\\[1em]
f_2(2)&=&-\dfrac{u_1(2)}{2}=-\dfrac{1}{2}=-0.5
\end{array}\end{gather*}
%
\begin{gather*}
\left\{\begin{array}{rcl}
u_1(4)&=&u_1(2)+2\cdot{f_1(2)}=1+2\cdot(-0.6)=-0.2\\
u_2(4)&=&u_2(2)+2\cdot{f_2(2)}=-1+2\cdot(-0.5)=-2
\end{array}\right..
\end{gather*}
\item
Определим значения неизвестных функций $u_1$ и $u_2$ в точке $t_3=6$:
\begin{gather*}
\begin{array}{rcl}
f_1(4)&=&0.2\cdot4+u_2(4)=0.2\cdot4+(-2)=-1.2\\[1em]
f_2(4)&=&-\dfrac{u_1(4)}{2}=-\dfrac{-0.2}{2}=0.1
\end{array}\end{gather*}
%
\begin{gather*}
\left\{\begin{array}{rcl}
u_1(6)&=&u_1(4)+2\cdot{f_1(4)}=-0.2+2\cdot(-1.2)=-2.6\\
u_2(6)&=&u_2(4)+2\cdot{f_2(4)}=-2+2\cdot(0.1)=-1.8
\end{array}\right..
\end{gather*}
\item
Определим значения неизвестных функций $u_1$ и $u_2$ в точке $t_4=8$:
\begin{gather*}\begin{array}{rcl}
f_1(6)&=&0.2\cdot6+u_2(6)=0.2\cdot6+(-1.8)=-0.6\\[1em]
f_2(6)&=&-\dfrac{u_1(6)}{2}=-\dfrac{-2.6}{2}=1.3
\end{array}\end{gather*}
%
\begin{gather*}
\left\{\begin{array}{rcl}
u_1(8)&=&u_1(6)+2\cdot{f_1(4)}=-2.6+2\cdot(-0.6)=-3.8\\
u_2(8)&=&u_2(6)+2\cdot{f_2(6)}=-1.8+2\cdot(1.3)=0.8
\end{array}\right..
\end{gather*}
\item
Определим значения неизвестных функций $u_1$ и $u_2$ в точке $t_5=10$:
\begin{gather*}\begin{array}{rcl}
f_1(8)&=&0.2\cdot8+u_2(8)=0.2\cdot8+0.8=2.4\\[1em]
f_2(8)&=&-\dfrac{u_1(8)}{2}=-\dfrac{-3.8}{2}=1.9
\end{array}\end{gather*}
%
\begin{gather*}
\left\{\begin{array}{rcl}
u_1(10)&=&u_1(10)+2\cdot{f_1(4)}=-3.8+2\cdot(2.4)=1\\
u_2(10)&=&u_2(8)+2\cdot{f_2(8)}=0.8+2\cdot(1.9)=4.6
\end{array}\right..
\end{gather*}
\end{enumerate}

На рисунке \ref{fig:u(t)} представлено решение задачи Коши 
системы дифференциальных уравнений \eqref{eq:ODE_MY}.
% *******************************
%	График функций
%
\begin{figure}\centering
\begin{tikzpicture}
\begin{axis}[
xlabel = {$t$},	% подпись оси x
ylabel = {$u_1, u_2$},	% подпись оси y
xmin=-1, xmax=11, xtick={0,2,4,6,8,10}, %xticklabels={$0$,$t_1$,,$t_j$,,$t_N$},
ymin=-6, ymax=6, ytick={-6,-4,-2,0,2,4,6}, %yticklabels={$y^{(0)}$,$y^{(1)}$,,$y^{(j)}$,,$y^{(N)}$},
legend pos={north west}]
\addplot[ball darkgreen]coordinates{(0,1) (2,1) (4,-0.2) (6,-2.6) (8,-3.8) (10,1)};
\addlegendentry{$u_1$};
\addplot[ball darkred]coordinates{(0,0) (2,-1) (4,-2) (6,-1.8) (8,0.8) (10,4.6)};
\addlegendentry{$u_2$};
\end{axis}
\end{tikzpicture}
\caption{Зависимость неизвестных функций от времени}
\label{fig:u(t)}
\end{figure}
% *******************************

%
%	Приближенное решение интегрального уравнения
%
Последовательно определяем приближенное решение 
интегрального уравнения \eqref{eq:ODE_MY},
используя рекуррентные соотношения \eqref{eq:ODE_SYS_INT_RR}.
\begin{enumerate}
\item
Определим значения неизвестных функций 
$\tilde{u}_1$ и $\tilde{u}_2$ в точке $t_1=2$:
\begin{gather*}
\left\{\begin{array}{rcl}
\tilde{u}_1(2)&=&\tilde{u}_1(0)+\tau\cdot\dfrac{f_1(0)+f_1(2)}{2}
=1+2\cdot\dfrac{0+(-0.6)}{2}=0.4\\[1em]
\tilde{u}_2(2)&=&\tilde{u}_2(0)+\tau\cdot\dfrac{f_2(0)+f_2(2)}{2}
=0+2\cdot\dfrac{-0.5+(-0.5)}{2}=-1
\end{array}\right..
\end{gather*}
\item
Определим значения неизвестных функций 
$\tilde{u}_1$ и $\tilde{u}_2$ в точке $t_2=4$:
\begin{gather*}
\left\{\begin{array}{rcl}
\tilde{u}_1(4)&=&\tilde{u}_1(2)+\tau\cdot\dfrac{f_1(2)+f_1(4)}{2}
=0.4+2\cdot\dfrac{(-0.6)+(-1.2)}{2}=-2.4\\[1em]
\tilde{u}_2(4)&=&\tilde{u}_2(2)+\tau\cdot\dfrac{f_2(2)+f_2(4)}{2}
=-1+2\cdot\dfrac{-0.5+0.1}{2}=-1.4
\end{array}\right..
\end{gather*}
\item
Определим значения неизвестных функций
$\tilde{u}_1$ и $\tilde{u}_2$ в точке $t_3=6$:
\begin{gather*}
\left\{\begin{array}{rcl}
\tilde{u}_1(6)&=&\tilde{u}_1(4)+\tau\cdot\dfrac{f_1(4)+f_1(6)}{2}
=-2.4+2\cdot\dfrac{-1.2+(-0.6)}{2}=-4.2\\[1em]
\tilde{u}_2(6)&=&\tilde{u}_2(4)+\tau\cdot\dfrac{f_2(4)+f_2(6)}{2}
=-1.4+2\cdot\dfrac{0.1+1.3}{2}=0
\end{array}\right..
\end{gather*}
\item
Определим значения неизвестных функций 
$\tilde{u}_1$ и $\tilde{u}_2$ в точке $t_4=8$:
\begin{gather*}
\left\{\begin{array}{rcl}
\tilde{u}_1(8)&=&\tilde{u}_1(6)+\tau\cdot\dfrac{f_1(6)+f_1(8)}{2}
=-4.2+2\cdot\dfrac{-0.6+2.4}{2}=-2.4\\[1em]
\tilde{u}_2(8)&=&\tilde{u}_2(6)+\tau\cdot\dfrac{f_2(6)+f_2(8)}{2}
=0+2\cdot\dfrac{1.3+1.9}{2}=3.2
\end{array}\right.
.\end{gather*}
\item
Определим значения неизвестных функций 
$\tilde{u}_1$ и $\tilde{u}_2$ в точке $t_5=10$:
\begin{gather*}\begin{array}{rcl}
f_1(10)&=&{0.2}\cdot10+u_2(10)={0.2}\cdot8+4.6=6.2\\[1em]
f_2(10)&=&-\dfrac{u_1(10)}{2}=-\dfrac{1}{2}=-0.5
\end{array}\end{gather*}
%
\begin{gather*}
\left\{\begin{array}{rcl}
\tilde{u}_1(10)&=&\tilde{u}_1(8)+\tau\cdot\dfrac{f_1(8)+f_1(10)}{2}
=-2.4+2\cdot\dfrac{2.4+6.2}{2}=6.2\\[1em]
\tilde{u}_2(10)&=&\tilde{u}_2(8)+\tau\cdot\dfrac{f_2(8)+f_2(10)}{2}
=3.2+2\cdot\dfrac{1.9+(-0.5)}{2}=4.6
\end{array}\right..
\end{gather*}
\end{enumerate}

На рисунке \ref{fig:uu(t)} представлены решения задачи Коши \eqref{eq:ODE_SYS} и
интегрального уравнения \eqref{eq:ODE_SYS_INT}, 
рассчитанные в различные моменты времени.

В таблице \ref{tab:error} и на рисунке \ref{fig:error} представлены значения
предельной абсолютной погрешности приближенного решения задачи Коши 
для различных моментов времени.
% *******************************
%	График функций
%
\begin{figure}[H]\centering
\begin{tikzpicture}
\begin{axis}[
%xlabel = {$t$},	% подпись оси x
ylabel = {$u_1, \tilde{u}_1$},	% подпись оси y
xmin=-1, xmax=11, xtick={0,2,4,6,8,10}, %xticklabels={$0$,$t_1$,,$t_j$,,$t_N$},
ymin=-6, ymax=7, ytick={-6,-4,-2,0,2,4,6}, %yticklabels={$y^{(0)}$,$y^{(1)}$,,$y^{(j)}$,,$y^{(N)}$},
legend pos={north west}]
\addplot[ball darkgreen]coordinates{(0,1) (2,1) (4,-0.2) (6,-2.6) (8,-3.8) (10,1)};
\addlegendentry{$u_1$};
\addplot[thick, mark=*, mark size=3pt, mark options={fill=white, draw=black, solid}]
coordinates {(0,1) (2,0.4) (4,-2.4) (6,-4.2) (8,-2.4) (10,6.2)};
\addlegendentry{$\tilde{u}_1$};
\end{axis}
\end{tikzpicture}
% *******************************
%	График функций
%
\begin{tikzpicture}
\begin{axis}[
xlabel = {$t$},	% подпись оси x
ylabel = {$u_2, \tilde{u}_2$},	% подпись оси y
xmin=-1, xmax=11, xtick={0,2,4,6,8,10}, %xticklabels={$0$,$t_1$,,$t_j$,,$t_N$},
ymin=-3, ymax=5, ytick={-2,0,2,4}, %yticklabels={$y^{(0)}$,$y^{(1)}$,,$y^{(j)}$,,$y^{(N)}$},
legend pos={north west}]
\addplot[ball darkred]coordinates{(0,0) (2,-1) (4,-2) (6,-1.8) (8,0.8) (10,4.6)};
\addlegendentry{$u_2$};
\addplot[thick, mark=*, mark size=3pt, mark options={fill=white, draw=black, solid}]
coordinates{(0,0) (2,-1) (4,-1.4) (6,0) (8,3.2) (10,4.6)};
\addlegendentry{$\tilde{u}_2$};
\end{axis}
\end{tikzpicture}
\caption{Приближенное решение задачи Коши и соответствующего интегрального уравнения}
\label{fig:uu(t)}
\end{figure}

\begin{table}[H]
\caption{Предельная абсолютная погрешность 
приближенного решения задачи Коши \eqref{eq:ODE_MY}}
\label{tab:error}
\small
\begin{tabular*}{\textwidth}
{@{\extracolsep{\fill}}*{7}{p{1.5cm}}}
%\begin{tabular*}{\\textwidth}{p{8cm}p{5cm}r}
\toprule
$i$&$0$&$1$&$2$&$3$&$4$&$5$\\
$t_i$&$0$&$2$&$4$&$6$&$8$&$10$\\
\midrule
\multicolumn{7}{l}{Задача Коши}\\
\midmidrule
$u_1(t_i)$&$1$&$1$&$-0.2$&$-2.6$&$-3.8$&$1$\\
$u_2(t_i)$&$0$&$-1$&$-2$&$-1.8$&$0.8$&$4.6$\\
\midrule
\multicolumn{7}{l}{Интегральное уравнение}\\
\midmidrule
$\tilde{u}_1(t_i)$&$1$&$0.4$&$-2.4$&$-4.2$&$-2.4$&$6.2$\\
$\tilde{u}_2(t_i)$&$0$&$-1$&$-1.4$&$0$&$3.2$&$4.6$\\
\midrule
\multicolumn{7}{l}{Абсолютная погрешность 
$\vect{\epsilon}=|\tilde{\vect{u}}-\vect{u}|$}\\
\midmidrule
$\epsilon_1$&$0$&$0.6$&$2.2$&$1.6$&$1.4$&$5.2$\\
$\epsilon_2$&$0$&$0$&$0.6$&$1.8$&$2.4$&$0$\\
\bottomrule
\end{tabular*}
\end{table}

Из рисунка \ref{fig:error} видно, что максимальная 
предельная абсолютная погрешность для $u_1(t)$ 
составляет $\epsilon_1=5.2$, 
а для функции $u_2(t)$ -- $\epsilon_2=2.4$.

% *******************************
%	График погрешности
%
\begin{figure}[H]\centering
\begin{tikzpicture}
\begin{axis}[
%enlargelimits=true,
ybar,bar width=18pt,
nodes near coords,
xlabel = {Время, $t$},
ylabel = {Абсолютная погрешность, $\epsilon$},	% подпись оси y
xmin=1, xmax=11, xtick={0,2,4,6,8,10},
ymin=-0.5, ymax=6, ytick={0,2,4,6},
legend pos={north west},
xtick align=inside,ytick align=inside,]
\addplot[thin, blue, fill=blue!35]
coordinates{(2,0.6) (4,2.2) (6,1.6) (8,1.4) (10,5.2)};
\addlegendentry{$\epsilon_1$};
\addplot[thin, red, fill=red!35]
coordinates{(2,0) (4,0.6) (6,1.8) (8,2.4) (10,0)};
\addlegendentry{$\epsilon_2$};
\end{axis}
\end{tikzpicture}
\caption{Предельная абсолютная погрешность приближенного решения задачи Коши \eqref{eq:ODE_MY}}
\label{fig:error}
\end{figure}
