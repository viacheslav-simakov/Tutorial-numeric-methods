% !TEX TS-program = xelatex
% !TEX encoding = UTF-8
%%%%%%%%%%%%%%%%%%%%%%%%%%%%%%
% нумерация 1a
% https://newbedev.com/draw-arrows-outside-of-content-area-of-tikz
%
% Номерация подрисунков
%https://tex.stackexchange.com/questions/123019/subfigure-numbering-%in-subfigure-environment-subcaption-package
% Преамбула
%
%	РАЗМЕР ШРИФТА:
%		\tiny
%		\scriptsize
%		\footnotesize	
%		\small 
%%%	\normalsize	= нормальный размер шрифта
%		\large
%		\Large
%		\huge
%		\Huge
%--------------------------------------------
%	\textrm{текст}	\textsf{текст}	\texttt{текст}
%	\textmd{текст}	\textbf{текст}	\textup{текст}
%	\textit{текст}	\textsl{текст}	\textsc{текст}
%%%%%%%%%%%%%%%%%%%%%%%%%%%%%%%%%%%%%%%%%%%%%
%
%	НАЧАЛО ПРЕАМБУЛЫ
%
\documentclass[14pt,a4paper]{extreport}
%
% языковые пакеты
%
\usepackage[T2A]{fontenc}
\usepackage[utf8]{inputenc}
%\usepackage[cp1251]{inputenc}
%
%	Размеры страницы
%
\usepackage[% https://ctan.org/pkg/geometry
	a4paper,% размер страницы
	mag=1000,
	left=2.5cm,
	right=1.5cm,
	top=2cm,
	bottom=2cm,
	headsep=0.7cm,
	footskip=0.7cm
]{geometry}
% --------------------------------------------
% *** межстрочный интервал ***
% стандартный пропуск строки означает коэффициент 1,2
% (например, высота шрифта 10pt, пропуск базовой строки 12pt).
% Умножьте на \linespread, чтобы вы получили 1,25 * 1,2 = 1,5 
% то есть половину.
% --------------------------------------------
%\linespread{1.5}
%
% МЕЖСТРОЧНЫЙ ИНТЕРВАЛ
%
% межстрочный интервал
\usepackage{setspace}
% полуторный интервал
\onehalfspacing
%\usepackage{indentfirst}
%
% ОТСТУП АБЗАЦА СЛЕВА
%
\setlength{\parindent}{1.25cm}
%
% НУМЕРАЦИЯ СТРАНИЦ 
% http://tug.ctan.org/tex-archive/macros/latex/contrib/fancyhdr/fancyhdr.pdf
%
\usepackage{fancyhdr}% https://www.ctan.org/pkg/fancyhdr
\pagestyle{empty}
\renewcommand{\headrulewidth}{0pt}
\renewcommand{\footrulewidth}{0pt}
%\lhead{\empty}\chead{\empty}\rhead{\empty}
%\lfoot{\empty}\cfoot{\empty}\rfoot{\arabic{page}}
%
% https://latex.org/forum/viewtopic.php?t=8952
%
\makeatletter
\renewcommand{\ps@plain}{%
\renewcommand\@oddhead{}%
\renewcommand\@evenhead{}%
\renewcommand\@oddfoot{\hfil\normalfont\textrm{\thepage}}%
\renewcommand\@evenfoot{\hfil\normalfont\textrm{\thepage}}%
}
\makeatother
%%%%%%%%%%%%%%%%%%%%%%%%%%%%%%%%%%%%%%%%%%%%%
%
% ПАКЕТ МНОГОЯЗЫКОВОЙ ВЁРСТКИ *** XeTeX ***
% https://ru.wikibooks.org/wiki/LaTeX/polyglossia
%
\usepackage{polyglossia}
% устанавливает главный язык документа
\setdefaultlanguage[spelling=modern]{russian}
\setmainlanguage{russian}
% устанавливает второй язык документа
\setotherlanguage{english}
% задаёт свойства шрифтов по умолчанию
\defaultfontfeatures{Ligatures={TeX},Renderer=Basic}
% задаёт основные шрифты документа
\setmainfont{Times New Roman}
\setromanfont{Times New Roman} 
\setsansfont{Arial} 
\setmonofont{Courier New}
% семейство кириллических шрифтов
\newfontfamily{\cyrillicfont}{Times New Roman} 
\newfontfamily{\cyrillicfontrm}{Times New Roman}
\newfontfamily{\cyrillicfonttt}{Courier New}
\newfontfamily{\cyrillicfontsf}{Arial}
%
% *** Новые Цвета ***
%
\usepackage{xcolor}% https://www.ctan.org/pkg/xcolor
\colorlet{darkgreen}{green!40!black}% темно-зеленый
\colorlet{darkred}{red!80!black}% темно-красный
\colorlet{darkorange}{orange!75!black}% темно-оранжевый
\colorlet{darkblue}{blue!70!black}% темно-синий
%
% Unicode-шрифты ДЛЯ ФОРМУЛ
%
\usepackage[intlimits]{amsmath}
\usepackage{amssymb,amsfonts}
% Расширенная матрица
% https://tex.stackexchange.com/questions/2233/whats-the-best-way-make-an-augmented-coefficient-matrix
% \begin{pmatrix}[cc|c]
%   1 & 2 & 3\\
%   4 & 5 & 9
% \end{pmatrix}
\makeatletter
\renewcommand*\env@matrix[1][*\c@MaxMatrixCols c]{%
	\hskip -\arraycolsep
	\let\@ifnextchar\new@ifnextchar
	\array{#1}}
\makeatother
%
% ШРИФТ Times в ФОРМУЛАХ как основной
%
%\usepackage[varg,cmbraces,cmintegrals]{newtxmath}
%
% прямые греческие буквы
%
%\usepackage{upgreek}
%
% Experimental Unicode mathematical typesetting
%
%\usepackage{unicode-math}
%\usepackage[bold-style=TeX]{unicode-math}
\usepackage[% https://ctan.org/pkg/unicode-math
	bold-style=upright%
]{unicode-math}
%
% ССЫЛКИ
%
\usepackage[% https://ctan.org/pkg/hyperref
	unicode=true,%
	bookmarksopen=true,%
	bookmarksnumbered=true,%
	colorlinks,%
	linkcolor=black,%
	urlcolor=darkblue,%
	citecolor=blue%
]{hyperref}
%
% Принудительное размещения рисунка:
% используйте пакет 'float', а затем [H] опцию,
% например, \begin{figure}[H] ... \end{figure}
\usepackage{float}
%%%%%%%%%%%%%%%%%%%%%%%%%%%%%%%%%%%%%%%%%%%%%
%
% ЗАГОЛОВКИ
%
%\usepackage[pagestyles]{titlesec}
\usepackage{titlesec}% https://www.ctan.org/pkg/titlesec
%
% \titleformat{command}
%		[shape]
%		{format}
%		{label}
%		{sep}
%		{before-code}
%		[after-code]
%
%	1) Глава (chapter)
%
\titleformat{\chapter}[hang]% shape
{\bfseries\normalsize\centering}% format
{\hspace{\parindent}\arabic{chapter}}% label
{0.5em}% sep
{}% before-code
[]% after-code
\titleformat*{\section}%
{\bfseries\normalsize}% format
%
% 2) Раздел (section)
%
\titleformat{\section}[hang]% shape
{\bfseries\normalsize}% format
{\hspace{1.25cm}\arabic{section}}% label
{0.5em}% sep
{}% before-code
[]% after-code
%
% 3) Подраздел (subsection)
%
\titleformat{\subsection}[hang]% shape
{\bfseries\normalsize}% format
{\hspace{1.25cm}\arabic{section}.\arabic{subsection}}% label
{0.5em}% sep
{}% before-code
[]% after-code
%
% 4) ПодПодраздел (subsubsection)
%
\titleformat{\subsubsection}[hang]% shape
{\bfseries\normalsize}% format
{\hspace{1.25cm}\arabic{section}.\arabic{subsection}.\arabic{subsubsection}}% label
{0.5em}% sep
{}% before-code
[]% after-code
%
% \titlespacing{command}
%		{before-sep}
%		{after-sep}
%		[right-sep]
%
\titlespacing{\chapter}
{0pt}% before-sep
{0pt}% after-sep
{0pt}% right-sep
\titlespacing{\section}
{0pt}% before-sep
{0pt}% after-sep
{0pt}% right-sep
\titlespacing{\subsection}
{0pt}% before-sep
{0pt}% after-sep
{0pt}% right-sep
\titlespacing{\subsubsection}
{0pt}% before-sep
{0pt}% after-sep
{0pt}% right-sep
%%%%%%%%%%%%%%%%%%%%%%%%%%%%%%%%%%%%%%%%%%%%
%
% СОДЕРЖАНИЕ
%
%%%%%%%%%%%%%%%%%%%%%%%%%%%%%%%%%%%%%%%%%%%%
%\addto\captionsrussian{%
%\renewcommand{\contentsname}{%
%\vspace{-5ex}%
%\begin{center}СОДЕРЖАНИЕ\end{center}%
%\vspace{-5ex}}%
%}
%\addto\captionsrussian{%
%  \renewcommand{\figurename}{Fig.}%
%}
\makeatletter
\renewcommand{\tableofcontents}{%
\newpage%
\noindent{\centering\textbf{СОДЕРЖАНИЕ}\par}%
\vspace{\baselineskip}% пустая строка
\@starttoc{toc}\par}%
\makeatother
% глубина детализации
\setcounter{tocdepth}{3}
\setcounter{secnumdepth}{3}
% переименование формата счетчика
\renewcommand{\thesection}{\arabic{section}}
%\renewcommand{\thesubsection}{\arabic{subsection}}
%	
% \titlecontents{section}
%		[left]
%		{above-code}
%		{numbered-entry-format}
%		{numberless-entry-format}
%		{filler-page-format}
%		[below-code]
%
\usepackage{titletoc}
%	Глава (chapter)
\titlecontents{chapter}
[0pt]% left
{}%above-code
{\thechapter\;}% numbered-entry-format
{}% numberless-entry-format
{\thepage}% filler-page-format
[]% below-code
%
% Раздел (section)
\titlecontents{section}
[0pt]% left
{\normalsize}% above-code
{\thecontentslabel\;}% numbered-entry-format
{}% numberless-entry-format
{(\thepage)}% filler-page-format
[]% below-code
%
% Подраздел (subsection)
\titlecontents{subsection}
[0pt]% left
{}%above-code
{\thecontentslabel\;}% numbered-entry-format
{}% numberless-entry-format
{\thepage}% filler-page-format
[]% below-code
% ПодПодраздел (subsubsection)
\titlecontents{subsubsection}
[0pt]% left [0pt]
{}%above-code
{\thecontentslabel\;}% numbered-entry-format
{}% numberless-entry-format
{\thepage}% filler-page-format
[]% below-code
%
% \dottedcontents{section}
%		[left]
%		{above-code}
%		{label width}
%		{leader width}
\dottedcontents{chapter}%
[1.5em]% left
{}% above-code
{1.5em}% label width
{0.5em}% leader width
\dottedcontents{section}%
[1.0em]% left [1.5em]
{}% above-code
{1.0em}% label width [1.5em]
{0.5em}% leader width
\dottedcontents{subsection}%
[1.5em]% left
{}% above-code
{1.5em}% label width
{0.5em}% leader width
\dottedcontents{subsubsection}%
[2.5em]% left
{}% above-code
{2.5em}% label width
{0.5em}% leader width
%
% Chapter without a pagebreak
%
\makeatletter 
\renewcommand\chapter{\newpage\par%
\thispagestyle{plain}% \global\@topnum\z@
\@afterindentfalse\secdef\@chapter\@schapter}
\makeatother %%%%% <---- Starting chapter without a pagebreak
%
% ВВЕДЕНИЕ
%
\newenvironment{Introduction}{%
%\chapter*{\vspace{-2ex}ВВЕДЕНИЕ}\par%
%\newpage\begin{center}\textbf{ВВЕДЕНИЕ}\end{center}\par
\newpage\phantomsection%
{\centering\textbf{ВВЕДЕНИЕ}\par}%
\addcontentsline{toc}{chapter}{ВВЕДЕНИЕ}%
% пустая линия
%\vspace{\baselineskip}
}{\newpage}
%
% ЗАКЛЮЧЕНИЕ
%
\newenvironment{Conclusion}{%
%\lfoot{\empty}\cfoot{\empty}\rfoot{\arabic{page}}%
\newpage\phantomsection%
%\chapter*{\vspace{-\baselineskip}ЗАКЛЮЧЕНИЕ}\par%
{\centering\textbf{ЗАКЛЮЧЕНИЕ}\par}%
\addcontentsline{toc}{chapter}{ЗАКЛЮЧЕНИЕ}%
% пустая линия
%\vspace{\baselineskip}
}{\newpage}
%
% ПРИЛОЖЕНИЕ
%
\newenvironment{Appendix}{%
%\lfoot{\empty}\cfoot{\empty}\rfoot{\arabic{page}}%
\newpage\phantomsection%
%\chapter*{\vspace{-\baselineskip}ЗАКЛЮЧЕНИЕ}\par%
{\centering\textbf{ПРИЛОЖЕНИЕ}\par}%
\addcontentsline{toc}{chapter}{ПРИЛОЖЕНИЕ}%
% пустая линия
%\vspace{\baselineskip}
}{\newpage}
%
% СПИСОК ИСПОЛЬЗОВАННЫХ ИСТОЧНИКОВ
%
% Пакет поддерживает сжатые, 
% отсортированные списки цитирования
% https://www.ctan.org/pkg/cite
\usepackage{cite}
% формат номера источника
% Заменяем библиографию в квадратных скобках
% http://ftp.tug.org/TUGboat/tb30-1/tb94mori.pdf
% 
\makeatletter%
\renewcommand*{\@biblabel}[1]{\hfill#1\;}
\makeatother
% окружение
\newenvironment{References}[1]{%
\newpage\phantomsection
\addcontentsline{toc}{chapter}{%
СПИСОК ИСПОЛЬЗОВАННЫХ ИСТОЧНИКОВ}
\renewcommand{\bibname}{%
СПИСОК ИСПОЛЬЗОВАННЫХ ИСТОЧНИКОВ}
\begin{thebibliography}{#1}
% пустая линия
\vspace{\baselineskip}
% интервал между библиографическими источниками
\setlength{\itemsep}{0pt}
\setlength{\parskip}{0pt}
}{\end{thebibliography}}
%%%%%%%%%%%%%%%%%%%%%%%%%%%%%%%%%%%%%%%%%%%%%
%
% ТАБЛИЦЫ
%
% Пакет повышает качество таблиц в LaTeX, 
% предоставляя дополнительные команды
% https://www.ctan.org/pkg/booktabs
\usepackage{booktabs}%
% \specialrule{wd}{abovespace}{belowspace}
% горизонтальная линия с отступами сверху и снизу
\renewcommand\midrule{\specialrule{0.5pt}{1ex}{1ex}}
\renewcommand\toprule{\specialrule{1pt}{0ex}{1ex}}
\renewcommand\bottomrule{\specialrule{1pt}{1ex}{0ex}}
%\newcommand\tabsrule{\specialrule{0.5pt}{1ex}{1ex}}
\newcommand\midmidrule{
\specialrule{0.5pt}{1ex}{0.1ex}
\specialrule{0.5pt}{0.1ex}{1ex}
}
% межстрочный интервал в ТАБЛИЦАХ
\renewcommand\arraystretch{1.2}
%%%%%%%%%%%%%%%%%%%%%%%%%%%%%%%%%%%%%%%%%%%%%
%
% СПИСКИ
%
\usepackage[inline]{enumitem}% https://ctan.org/pkg/enumitem
\setlist[enumerate,itemize]{
	left=0pt,
	align=left,
	leftmargin=0pt,
	label = \arabic*),
	leftmargin=*,
	labelsep=1ex,
	itemindent=0pt,
	nosep
}
%%%%%%%%%%%%%%%%%%%%%%%%%%%%%%%%%%%%%%%%%%%%%
%
%	Цветной прямоугольник с текстом
%
\usepackage{tcolorbox}% https://www.ctan.org/pkg/tcolorbox
% Установка опций по умолчанию
\tcbset{
	notitle,
	titlebox=invisible,
	size=title,
	width=\textwidth,
	boxsep=0em,
	left=1ex,
	right=1ex,
	toptitle=0mm,
	top=1ex,
	toprule=0mm,
	bottom=1ex,
	bottomrule=0mm,
	boxrule=0mm,
	arc=0mm,
	colback=orange!10,
}
%%%%%%%%%%%%%%%%%%%%%%%%%%%%%%%%%%%%%%%%%%%%%
%
% TikZ
% https://tex.stackexchange.com/questions/131293/arguments-for-tikz-style
%
\usepackage{tikz}
\usetikzlibrary{%
	pgfplots.groupplots,%
	backgrounds,%
	calc,%
	decorations.pathmorphing,%
	decorations.markings,
	shapes,% геометрические фигуры
	arrows.meta,% стрелки разной формы
	er,% построение диаграмм
	patterns% штриховка областей
}
\tikzset{
	background rectangle/.style={% стиль фона
		fill=olive!10%
	},%
	font=\normalsize%
}
%
% PGF
% http://elib.ict.nsc.ru/jspui/bitstream/ICT/1488/1/pgf-ru-all-method.pdf
%
\usepackage{%
	pgfplots,%
	pgfplotstable%
}
% Последовательность графичеких слоев
% \pgfsetlayers{%
%	background,%
%	pre main,%
%	axis grid,% 
%	axis ticks,% 
%	axis lines,%
%	axis tick labels,%
%	main,%
%	axis descriptions,%
%	axis foreground%
%}
\usepgfplotslibrary{fillbetween}
\usepgflibrary{plotmarks}
% Установка стилей графика
\pgfplotsset{
%	compat=1.9,
	width=8cm,
	every axis/.append style={%
		thick,
		tick style={
			black,
			semithick,
		},
	}
}
%
% НУМЕРАЦИЯ формул, рисунков
%
\renewcommand{\theequation}{\arabic{equation}}
\renewcommand{\thefigure}{\arabic{figure}}
%%%%%%%%%%%%%%%%%%%%%%%%%%%%%%%%%%%%%%%%%%%%%
%
% ПОДПИСИ рисуноков, таблиц
%
\usepackage[% https://www.ctan.org/pkg/caption
	format=plain,% Печатает подписи как обычный абзац
	labelsep=endash,%
	singlelinecheck=false,% отключить центрирование однострочной подписи
	belowskip=0pt,%
	margin={0pt,0pt},%
	indention=0cm,
%	font=onehalfspacing,% полуторный интервал
]{caption}
%	рисунки
\captionsetup[figure]{%
	name=Рисунок,%
	position=below,%
	justification=centering,% центрирование подписи
	font=onehalfspacing,% полуторный интервал
}
%	таблицы
\captionsetup[table]{%
	name=Таблица,%
	position=above,%
%	justification=justified,
	font=onehalfspacing,% полуторный интервал
}
%\captionsetup{belowskip=0pt,margin={0pt,0pt}}


% *** отступы ***
% вертикальный промежуток перед и после объектов,
% местоположение которых соответствует ключу h.
% Имеет естественную длину 12 pt.
\setlength{\intextsep}{\baselineskip}
% вертикальный промежуток между текстом и соответственно 
% одно- и двухколоночными объектами, местоположение 
% которых соответствует ключам t или b.
% Имеет естественную длину 20 pt.
\setlength{\textfloatsep}{\baselineskip}
% вертикальный промежуток между соответственно 
% одно- и двухколоночными объектами, местоположение 
% которых соответствует ключам t или b. 
% Имеет естественную длину 12 pt.
\setlength{\floatsep}{\baselineskip}
% отступ перед названием 
\setlength{\abovecaptionskip}{0.5\baselineskip}
% отступ после названиея
\setlength{\belowcaptionskip}{0ex}
%
% ОТСТУПЫ В ФОРМУЛАХ
%
%\expandafter\def\expandafter\normalsize\expandafter{%
%\normalsize
%\setlength\abovedisplayskip{0.5\baselineskip}
%\setlength\belowdisplayskip{0.5\baselineskip}
%\setlength\abovedisplayshortskip{0.5\baselineskip}
%\setlength\belowdisplayshortskip{0.5\baselineskip}
%}
\AtBeginDocument{%
\abovedisplayskip=0.5\baselineskip
\abovedisplayshortskip=0.5\baselineskip
\belowdisplayskip=0.5\baselineskip
\belowdisplayshortskip=0.5\baselineskip
% отступ перед названием 
%\abovecaptionskip=0.5\baselineskip
% отступ после названиея
%\belowcaptionskip=0\baselineskip
}
%
% Пакет mhchem предоставляет команды для
% набора химических молекулярных формул и уравнений.
\usepackage[version=4,arrows=font]{mhchem}
%\usepackage{expl3,calc}
% длина стрелок
\ExplSyntaxOn
\keys_define:nn { mhchem }
{
arrow-min-length .code:n =
% default is 2em
\cs_set:Npn \__mhchem_arrow_options_minLength:n { {#1} }
}
\ExplSyntaxOff
\mhchemoptions{arrow-min-length=1em}
%
%	Пакет предоставляет макросы для управления строками - 
%	тестирования содержимого строки, извлечения подстрок, 
%	подстановки подстрок и предоставления чисел, 
%	таких как длина строки, позиция или количество 
%	повторов подстроки.
%	https://www.ctan.org/pkg/xstring
%
%	\usepackage{xstring}

%
%	Вставка страниц из PDF-файла
%	Этот пакет упрощает включение внешних многостраничных 
%	PDF-документов в LaTеX документы
% 
\usepackage{pdfpages}
%%%%%%%%%%%%%%%%%%%%%%%%%%%%%%%%%%%%%%%%%%%%%
%
%	МОИ НОВЫЕ КОМАНДЫ
%
% Макрос Существует?
% https://tex.stackexchange.com/questions/164188/ifundefined-actually-defines-macros
\def\ifexists#1{\expandafter\ifx\csname#1\endcsname\relax 0\else 1\fi}
% Pure text from TeX
% https://tex.stackexchange.com/questions/567286/pdfstringdef-turns-accented-characters-into-octal-escape-sequence
\ExplSyntaxOn
\cs_set_eq:NN\textpurify\text_purify:n
\ExplSyntaxOff
% alert
\newcommand{\alert}[2][]{%
#1{\emph{\textcolor{darkred}{#2}}}%
}
% alertx
\newcommand{\alertx}[2][\texttt]{%
#1{\textbf{\textcolor{darkred}{#2}}}%
}
% проверка макроса
\newcommand{\Isdefined}[1]{%
\ifx#1\undefined\relax%
\alertx{\backslash def\detokenize{#1}undefined}%
\else{#1}\fi%
}
% Чистый текст макроса
\newcommand{\MacroTextPurify}[1]{%
\ifx#1\undefined\relax%
\alertx{\backslash def\detokenize{#1}undefined}%
\else\textpurify{#1}%
\fi}
% Пол студента
\newcommand{\detGender}[3]{%
\ifx#1\undefined{для студентов}%
\else%
\ifcase#1\relax{#2}% Gender=0 (женский)
\or{#3}% Gender=1 (мужской)
\else\alertx{Gender=\Gender{ unknow}}%
\fi%
\fi%
}
% пустая строка
\def\emptyline{\par\vspace{\baselineskip}}
% символ '\'
\def\backslash{\char`\\}
% постоянная
\DeclareMathOperator{\const}{const}
% Полужирное начертание для векторов
\newcommand\vect[1]{\mathbfit{#1}}
%\let\vec=\mathbf
% норма
\newcommand{\norma}[1]{\left\lVert#1\right\rVert}
% абсолютное значение
\newcommand{\abs}[1]{\left\lvert#1\right\rvert}
% скалярное произведение векторов
\newcommand{\dotvec}[2]{(\vec{#1},\vec{#2})}
% обыкновенная производная, например $\diff{N_d^{+}}{x}$
\newcommand\diff[2]{ \dfrac{\mathrm{d}#1}{\mathrm{d}#2} }
% обыкновенная производная второго порядка, например $\diff{N_d^{+}}{x}$
\newcommand\diffdiff[2]{ \dfrac{\mathrm{d}^2 #1}{\mathrm{d} #2^2} }
% частная производная, например $\pdiff{N_d^{+}}{x}$
\newcommand\pdiff[2]{ \dfrac{\partial #1}{\partial #2} }
% 1/2
\newcommand\onehalf{ \nicefrac{1}{2} }
% Случайное число \RandInt{min}{max}
\newcommand{\RandInt}[2]{\pgfmathrandominteger{\rndint}{#1}{#2}\rndint}
% Интеграл
\newcommand\intf[4][x]{ \int\limits_{#2}^{#3}{#4}\,\mathrm{d}#1 }
%%%%%%%%%%%%%%%%%%%%%%%%%%%%%%%%%%%%%%%%%%%%%
%
% ТИТУЛЬНЫЙ ЛИСТ
%
% \TitlePage[название документа] или \TitlePage
\newcommand{\TitlePage}[1][\undefined]{
{\centering% начало центрирования
МИНОБРНАУКИ РОССИИ\\
Федеральное государственное бюджетное образовательное учреждение
высшего образования\\
\textbf{<<САРАТОВСКИЙ НАЦИОНАЛЬНЫЙ ИССЛЕДОВАТЕЛЬСКИЙ\\
ГОСУДАРСТВЕННЫЙ УНИВЕРСИТЕТ\\
ИМЕНИ Н.Г. ЧЕРНЫШЕВСКОГО>>}
\emptyline
Кафедра материаловедения,\par
технологии и управления качеством
\emptyline
% название документа (DocumentTitle)
\ifnum1=\ifexists{SubTitle}\MakeUppercase{\SubTitle}\emptyline\fi
%\ifx#1\undefined\relax% не определено
%\else\MakeUppercase{#1}\emptyline%
%\fi
% ЗАГЛАВИЕ
\textbf{\MakeUppercase{\Isdefined{\TITLE}}}
\emptyline
по дисциплине <<\Isdefined{\ModuleTitle}>>\\
% для студентов/студентки/студента
\detGender{\Gender}{студентки}{студента}
\Isdefined{\NoCourse}{ курса }\Isdefined{\NoGroup}{ группы}\\
направления подготовки{ \Isdefined\ProgramCode }
<<\Isdefined{\ProgramTitle}>>
(профиль <<\Isdefined{\ProgramProfile}>>),\\
\Isdefined{\Department}\\% факультет
% Ф.И.О. (родительный падеж)
\ifx\FullNameGenetive\undefined\relax%
%\end{center}%
\else{\FullNameGenetive}%
% пустая строка
\emptyline
% добавляет заполняющее вертикальное пространство
\vfill
% таблица БАРС
\begin{tabular}[l]{b{5cm} b{2.5cm} c}
\toprule
Результат&Баллы&ВСЕГО\\
\midmidrule
Выполнение&&\Isdefined{\Exec}\\
\midrule
Оформление&&\Isdefined{\Polygraphy}\\
\midrule
Устный отчет&&\Isdefined{\OralReport}\\
\bottomrule
\end{tabular}%
% пустая строка
\par\emptyline
% ПОДПИСЬ
\raggedright\noindent
Преподаватель\\профессор, д.т.н., доцент
\hspace{1em}\rule{6.5cm}{0.5pt} 
\hspace{1em} В.В. Симаков\par
\emptyline
\fi%
}\par% окончание центрирования
}
%%%%%%%%%%%%%%%%%%%%%%%%%%%%%%%%%%%%%%%%%%%%%
%
% Колонтитулы
%
% 'Титульный лист'
\fancypagestyle{titlepage}{
\setcounter{page}{1}
% clear all header and footer fields
\fancyhf{}
% линии верхнего и нижнего колонтитулов
\renewcommand{\headrulewidth}{0pt}%
\renewcommand{\footrulewidth}{0pt}%
\cfoot{Саратов~\the\year} % город год
}
% 'Задание'
\fancypagestyle{taskpages}{
% clear all header and footer fields
\fancyhf{}
% нумерация римскими цифрами
%\pagenumbering{roman}
% нумерация арабскими цифрами
\pagenumbering{arabic}
% линии верхнего и нижнего колонтитулов
\renewcommand{\headrulewidth}{0pt}%
\renewcommand{\footrulewidth}{0pt}%
% нижний колонтитул
\lfoot{Задание~
\detGender{\Gender}{получила}{получил}
\hspace{0.25cm}\rule{4cm}{0.5pt}\hspace{0.25cm}%
\Isdefined{\Signature}}% подпись
%\rfoot{\roman{page}}% номера страниц
}
% 'Содержание документа'
\fancypagestyle{bodypages}{
\fancyhf{}
\pagestyle{fancy}
% линии верхнего и нижнего колонтитулов
\renewcommand{\headrulewidth}{0pt}%
\renewcommand{\footrulewidth}{0pt}%
\pagenumbering{arabic}
\rfoot{\arabic{page}}% номера страниц
}
%%%%%%%%%%%%%%%%%%%%%%%%%%%%%%%%%%%%%%%%%%%%%
%
% ЗАДАНИЕ
%
% \TitleTask или \TitleTask['на выполнение чего?']
\newcommand{\TitleTask}[1][лабораторной работы]{%
\newpage
% нумерация страниц с №1
\setcounter{page}{1}
%\newpage\phantomsection
\begin{center}
\textbf{ЗАДАНИЕ}\par
% добавление пункта в СОДЕРЖАНИЕ
%\addcontentsline{toc}{chapter}{Задание на выполнение #1}
%по дисциплине <<\Isdefined{\@ModuleTitle}>>
по дисциплине <<\MacroTextPurify{\ModuleTitle}>>
на выполнение #1 на тему 
<<\textbf{\MacroTextPurify{\TITLE}}>>
\end{center}
\par
}

\usepackage{subcaption}
\usepackage{pgffor}
\usetikzlibrary{matrix,chains,shapes}
% https://latex.net/tikz-3dplot/
\usepackage{tikz-3dplot}
%%%%%%%%%%%%%%%%%%%%%%%%%%%%%%
% Стиль графиков
\pgfplotsset{
width=9cm,% размер графиков
xlabel=$x$,ylabel=$f(x)$,
every axis/.style={% оси координат
	font=\small,% размер шрифта
	color=black,solid,thick,
	xtick style={semithick,black},
	ytick style={semithick,black},
	grid=major,% сетка
	major grid style={thin,dashed,color=gray!25,},
},
every node near coord/.append style={
font=\footnotesize,%
%font=\small,%
color=black,%
%fill=olive!10,%
above,
yshift=+2pt,%
/pgf/number format/fixed,%
/pgf/number format/fixed zerofill,%
/pgf/number format/precision=1,%
},
}
% стиль графиков
\pgfplotsset{
compat=1.15,
ball darkred/.style={
	thick,
	mark=ball,
	color=darkred,
	mark size=3pt,
	mark options={% маркеры
		thin,
		draw=darkred,
		ball color=darkred!50,
	},
},
ball darkblue/.style={
	thick,
	mark=ball,
	color=darkblue,
	mark size=3pt,
	mark options={% маркеры
		thin,
		draw=darkblue,
		ball color=darkblue!50,
	},
},
ball darkgreen/.style={
	thick,
	mark=ball,
	color=darkgreen,
	mark size=3pt,
	mark options={% маркеры
		thin,
		draw=darkgreen,
		ball color=darkgreen!50,
	},
},
}
\pgfplotsset{
% define the layers you need.
% (Don't forget to add `main' somewhere in that list!!)
	layers/my layer set/.define layer set={
		background,
		main,
		foreground
	}{
		% you could state styles here which should be moved to
		% corresponding layers, but that is not necessary here.
		% That is why we don't state anything here
	},
        % activate the newly created layer set
        set layers=my layer set,
}
%\addplot [red,
%	% and with `on layer' you can state the layer where the
%	% plot should be drawn on
%	on layer=foreground,
%	] {x};

\pgfdeclarelayer{pre main}
\pgfdeclarelayer{background}
\pgfdeclarelayer{foreground}
\pgfsetlayers{background,pre main,main,foreground} 
% Декорация линий
\tikzset{
	% стрелки
	FARROW/.style={
		arrows={-Straight Barb[angle=45:1.5mm 1]	},
		shorten >= 0pt,
	},
	% декорация линий
	curve line/.style={
		decoration={
			snake,
			amplitude=1.5mm,%0.75mm,
			segment length=7.5mm,
			post length=0mm,
			pre length=0mm,
		},
	},
	% соединительные стрелки
	every join/.style={->,black,thick},
	pointnode/.style={
		join=by -,
		circle,
		fill=red,
		minimum size=3pt,
		inner sep=0pt,
	},
	circlenode/.style={
		draw,thick,fill=blue!10,line width=0.25mm,
		shape=circle,
		inner sep=4pt,
	},
	rectnode/.style={
		draw,line width=0.25mm,
		shape=rectangle,
%		rounded corners=2pt, 
		inner sep=8pt,
	},
	srectnode/.style={
		draw,line width=0.25mm,
		shape=rectangle split,
%		rounded corners=2pt, 
		inner sep=8pt,
	},
	rrectnode/.style={
		draw,line width=0.25mm,
		shape=rounded rectangle,
%		rounded corners=2pt, 
		inner sep=8pt,
	},
	beginendnode/.style={
		draw,line width=0.25mm,
		shape=rounded rectangle,
%		rounded corners=2pt, 
		inner sep=8pt,
	},
	datanode/.style={
		draw,line width=0.25mm,
		shape=	trapezium,
		trapezium left angle=70,
		trapezium right angle=110, 
		text centered,
		inner sep=8pt,
	},
	% входные/выходные данные
	IOnode/.style={
		draw,line width=0.25mm,
		shape=	trapezium,
		trapezium left angle=70,
		trapezium right angle=110, 
		text centered,
		inner sep=8pt,
	},
	ifthenelsenode/.style={
		draw,line width=0.25mm,
		shape=diamond,
		aspect=2,
		text centered,
		inner sep=4pt,
	},
	startloopnode/.style={
		draw,text width=4cm,align=center,line width=0.25mm,
		shape=chamfered rectangle,
		chamfered rectangle corners={north west,north east},
		minimum height=2em,
		minimum width=3.5cm,
	},
	endloopnode/.style={
		draw,text width=4cm,align=center,line width=0.25mm,
		shape=chamfered rectangle,
		chamfered rectangle corners={south west,south east},
		minimum height=2em,
		minimum width=3.5cm,
	},
	signalnode/.style={
		draw,line width=0.25mm,
		shape=chamfered rectangle,
		chamfered rectangle angle=45,
		chamfered rectangle xsep=1cm,
		inner sep=4pt,
	},
}
            
\begin{document}
%
% ТИТУЛЬНЫЙ ЛИСТ
%
\pagestyle{titlepage}
{\centering% начало центрирования
МИНОБРНАУКИ РОССИИ\\
Федеральное государственное бюджетное образовательное учреждение
высшего образования\\
\textbf{<<САРАТОВСКИЙ НАЦИОНАЛЬНЫЙ ИССЛЕДОВАТЕЛЬСКИЙ\\
ГОСУДАРСТВЕННЫЙ УНИВЕРСИТЕТ\\
ИМЕНИ Н.Г. ЧЕРНЫШЕВСКОГО>>}
\emptyline
Кафедра материаловедения,\par
технологии и управления качеством
\emptyline
% АВТОРЫ
\textbf{Синёв И.В., Симаков В.В.}
% ЗАГЛАВИЕ
\emptyline
\textbf{МЕТОДЫ ЧИСЛЕННОГО АНАЛИЗА В МАТЕРИАЛОВЕДЕНИИ}\\
\emptyline
Учебное пособие
\emptyline
для студентов 2 курса\\
направления подготовки бакалавриата\\
22.03.01 <<Материаловедение и технологии материалов>>,\\
профиля подготовки
<<Нанотехнологии, диагностика и синтез\linebreak
современных материалов>>,\\
института физики
\par
% добавляет заполняющее вертикальное пространство
\vfill
}\par% окончание центрирования
% авторы
\newpage
\cfoot{} %
{\centering{Авторы}\par}

\textbf{Синёв Илья Владимирович} -- к.ф.-м.н., доцент, 
доцент кафедры материаловедения, технологии и управления качеством,
Саратовский национальный исследовательский
государственный университет имени Н.Г. Чернышевского.

\textbf{Симаков Вячеслав Владимирович} -- д.т.н., доцент, 
профессор кафедры материаловедения, технологии и управления качеством,
Саратовский национальный исследовательский
государственный университет имени Н.Г. Чернышевского.

%\emptyline
%В учебном пособии рассматриваются основные методы решения 
%систем линейных алгебраических уравнений (прямые и итерационные), 
%нелинейных уравнений, построения полиномов Лагранжа и Ньютона, 
%определения собственных чисел и векторов, 
%численного интегрирования и дифференцирования. 
%Строятся решения задачи Коши методами Эйлера
%, Рунге-Кутты, Адамса; изучаются методы Ритца, моментов, 
%наименьших квадратов решения обыкновенных  дифференциальных  
%уравнений  с  граничными  условиями.  
%Излагаются алгоритмы  решения  прикладных  инженерных  задач  
%с  использованием  вычислительной техники, описываются способы 
%оценки погрешностей получаемых решений.

%
% колонтитулы
%
\newpage
\pagestyle{bodypages}
\newpage
\setcounter{page}{2}
% СОДЕРЖАНИЕ
\tableofcontents

%
% ВВЕДЕНИЕ
%
\begin{Introduction}

Численные методы являются основой решения комплексных задач, 
возникающих в ходе развития материаловедения и новейших 
областей техники и технологии.
Использование ЭВМ при решении научных и технических задач 
позволяет провести детальное математическое моделирование 
процессов и систем, которое существенно сокращает потребность 
в натурных экспериментах, а в ряде случаев может их заменить. 

Важным аспектом алгоритмов численных методов 
является гарантирование нахождения приближенного решения 
с заданной точностью за конечное число математических операций. 
При этом для численных методов характерна множественность 
возможных методов решения.

В пособии излагаются основы численных методов решения задач
алгебры, математического анализа, оптимизации, 
обыкновенных дифференциальных уравнений. 
Рассмотрены алгоритмы и возможности их 
применения для приближенного решения различных 
классов математических задач. В пособии изложен материал, 
необходимый студентам для приобретения навыков 
применения численных методов для решения 
задач теоретического и прикладного характера
в области профессиональной деятельности.
Наибольшее внимание уделяется фундаментальным разделам 
численных методов -- численному решению систем линейных 
алгебраических уравнений, методам оптимизации и разностным 
методам решения задач математической физики. 

Теоретический материал изложен сжато, но при этом 
большое внимание уделено рекомендациям по практическому 
применению рассмотренных алгоритмов. 
Изложение материала проиллюстрировано рядом примеров 
направленных на развитие практических навыков применения 
численных методов для решения конкретных задач.
Предполагается, что при изучении данного курса
студенты знакомы с основами высшей математики и 
владеют навыками программирования.

\end{Introduction}

%\end{document}

% Метод Гаусса
%% Выделение цветом
\pgfkeys{/main/.code=\fcolorbox{gray}{yellow}{$#1$}}
\pgfkeys{/one/.code=\textcolor{darkred}{1}}
\pgfkeys{/zero/.code=\textcolor{gray!50}{0}}
%\pgfkeys{/main=hi!}% пример использования
%
%	Метод Гаусса решения систем линейных уравнений
%
\newpage
\section{Метод Гаусса решения систем линейных уравнений}
К решению систем линейных алгебраических уравнений сводится подавляющее 
большинство задач вычислительной математики и многие численные методы основаны 
на решении систем линейных уравнений вида:
\begin{gather}\label{eq:LS}
\left\{
\begin{matrix}
a_{11}\cdot x_1&+&a_{12}\cdot x_2&+&\dots&+&a_{1m}\cdot x_m&=&b_1\\
a_{21}\cdot x_1&+&a_{22}\cdot x_2&+&\dots&+&a_{2m}\cdot x_m&=&b_2\\
a_{31}\cdot x_1&+&a_{32}\cdot x_2&+&\dots&+&a_{3m}\cdot x_m&=&b_3\\
\hdotsfor{9}\\
a_{m1}\cdot x_1&+&a_{m2}\cdot x_2&+&\dots&+&a_{mm}\cdot x_m&=&b_m
\end{matrix}
\right.,
\end{gather}
где $x_1,x_2,\dots,x_m$ --
неизвестные, которые необходимо определить;
$\{a_{ij}\}$ и \linebreak$b_1,b_2,\dots,b_m$ -- 
известные числовые коэффициенты левой и правой 
частей системы уравнений, соответственно.

Решение системы линейных алгебраических уравнений 
\eqref{eq:LS} представляет совокупность $m$ 
действительных или мнимых чисел $\{s_1,s_2,\dots,s_m\}$, 
таких что их соответствующая подстановка вместо 
$\{x_1,x_2,\dots,x_m\}$ в систему обращает все её 
уравнения в тождества:
\begin{gather*}
\left\{
\begin{matrix}
a_{11}\cdot s_1&+&a_{12}\cdot s_2&+&\dots&+&a_{1m}\cdot s_m&\equiv&b_1\\
a_{21}\cdot s_1&+&a_{22}\cdot s_2&+&\dots&+&a_{2m}\cdot s_m&\equiv&b_2\\
a_{31}\cdot s_1&+&a_{32}\cdot s_2&+&\dots&+&a_{3m}\cdot s_m&\equiv&b_3\\
\hdotsfor{9}\\
a_{m1}\cdot s_1&+&a_{m2}\cdot s_2&+&\dots&+&a_{mm}\cdot s_m&\equiv&b_m
\end{matrix}
\right.,
\end{gather*}

Система линейных алгебраических уравнений \eqref{eq:LS}
может быть представлена в более компактной 
матричной форме:
\begin{gather}\label{eq:LSM}
\mathbf{A}\cdot\vect{x}=\vect{b},
\end{gather}
где $\mathbf{A}$ -- квадратная матрица $m\times m$,
$\vect{x}$ и $\vect{b}$ -- искомый вектор неизвестных и
заданный вектор размерности $1\times m$, правых частей
системы уравнений:
\begin{gather*}
\mathbf{A}=
\begin{pmatrix}
a_{11}&a_{12}&\cdots&a_{1m}\\
a_{21}&a_{22}&\cdots&a_{2m}\\
\vdots&\vdots&\ddots&\vdots\\
a_{m1}&a_{m2}&\cdots&a_{mm}\\
\end{pmatrix},
\quad
\vect{x}=\begin{pmatrix}x_1\\x_2\\\vdots\\x_m\end{pmatrix},
\quad
\vect{b}=\begin{pmatrix}b_1\\b_2\\\vdots\\b_m\end{pmatrix}.
\end{gather*}

Теорема Кронекера--Капелли устанавливает необходимое и 
достаточное условие совместности системы 
линейных алгебраических уравнений посредством 
свойств матричных представлений: 
система совместна тогда и только тогда, когда ранг её матрицы 
совпадает с рангом расширенной матрицы,
полученной путем добавления столбца правых частей $\vect{b}$
матрице системы $\mathbf{A}$: 
\begin{gather}\label{eq:LSAM}
(\mathbf{A}\,\vert\,\vect{b})=
\begin{pmatrix}[cccc|c]
a_{11}&a_{12}&\cdots&a_{1m}&b_1\\
a_{21}&a_{22}&\cdots&a_{2m}&b_2\\
\vdots&\vdots&\ddots&\vdots&\vdots\\
a_{m1}&a_{m2}&\cdots&a_{mm}&b_m\\
\end{pmatrix}
\end{gather}

Преимущество расширенной матрицы заключается в 
возможности выполнения тех же операций с вектором 
правых частей системы уравнений, что и со строками
матрицы.

Предполагается, что определитель матрицы $\mathbf{A}$ 
отличен от нуля $\det\mathbf{A}\ne0$,
так что решение $\vect{x}=(x_1,x_2,\dots,x_m)^\mathrm{T}$
существует и единственно.
Систему линейных уравнений можно решить 
по крайней мере двумя способами:
либо воспользовавшись \emph{формулами Крамера}, 
либо методом последовательного исключения неизвестных 
(\emph{методом Гаусса}).
При больших порядка матрицы $m$ способ Крамера, 
основанный на вычислении определителей,
требует порядка $m!$ арифметических действий, 
в то время как метод Гаусса -- $O(m^3)$ действий.

Для большинства вычислительных задач характерным 
является большой порядок матрицы 
$\mathbf{A}$ ($m\approx10^2\dots10^5$),
поэтому метод Гаусса в различных вариантах широко 
используется при решении задач линейной алгебры на ЭВМ.

%
%	Прямой ход метода Гаусса
%
\emptyline
\subsection{Прямой ход метода Гаусса}
Метод Гаусса состоит в последовательном исключении 
неизвестных $x_i$ из системы линейных уравнений
\eqref{eq:LSAM}.
Например, предположим, что $a_{11}\ne0$, 
тогда разделим первое уравнение системы на $a_{11}$, 
и в результате получим:
\begin{gather*}
\begin{pmatrix}[cccc|c]
1&c_{12}&\cdots&c_{1m}&y_1\\
a_{21}&a_{22}&\cdots&a_{2m}&b_2\\
\vdots&\vdots&\ddots&\vdots&\vdots\\
a_{m1}&a_{m2}&\cdots&a_{mm}&b_m\\
\end{pmatrix},
\end{gather*}
где $c_{1j}$ и $y_1$ -- нормированные коэффициенты 
1-ой строки и правой части 1-го уравнения, соответственно:
\begin{gather*}
c_{1j}=\dfrac{a_{1j}}{a_{11}}\quad (j=2,3,\ldots,m),\quad
y_1=\dfrac{b_1}{a_{11}}.
\end{gather*}

Последовательно умножим первое уравнение системы на $a_{i1}$ и 
вычтем полученное уравнение из каждого $i$-го уравнения системы 
$i=2,3,\dots,m$. В результате получим следующую матрицу:
\begin{gather*}
\begin{pmatrix}[crcr|r]
1&c_{12}&\cdots&c_{1m}&y_1\\
0&a_{22}-a_{21}\cdot c_{12}&\cdots&a_{2m}-a_{21}\cdot c_{1m}&b_2-a_{21}\cdot y_1\\
\vdots&\vdots&\ddots&\vdots&\vdots\\
0&a_{m2}-a_{m1}\cdot c_{12}&\cdots&a_{mm}-a_{m1}\cdot c_{1m}&b_m-a_{m1}\cdot y_1\\
\end{pmatrix}.
\end{gather*}

Запишем полученную матрицу в более компактном виде:
\begin{gather*}
\begin{pmatrix}[cccc|r]
1&c_{12}&\cdots&c_{1m}&y_1\\
0&a_{22}^{(1)}&\cdots&a_{2m}^{(1)}&b_2^{(1)}\\
\vdots&\vdots&\ddots&\vdots&\vdots\\
0&a_{m2}^{(1)}&\cdots&a_{mm}^{(1)}&b_m^{(1)}\\
\end{pmatrix},
\end{gather*}
где $a_{ij}^{(1)}$ и $b_i^{(1)}$
-- модифицированные коэффициенты при неизвестных и правой части,
соответственно.
\begin{gather*}
a_{ij}^{(1)}=(a_{ij}-a_{i1}\cdot c_{1j}), \quad 
b_i^{(1)}=(b_i-a_{i1}\cdot y_{1}), \quad
(i,j=2,3,\dots,m)
\end{gather*}

Если $a_{22}^{(1)}\ne0$, то в модифицированной системе 
аналогично можно исключить неизвестное $x_2$. 
Для этого можно разделить второе уравнение системы 
на коэффициент при второй неизвестной $a_{22}^{(1)}$,
и в результате получим:
\begin{gather*}
\begin{pmatrix}[cccc|r]
1&c_{12}&\cdots&c_{1m}&y_1\\
0&1&\cdots&c_{2m}&y_2\\
\vdots&\vdots&\ddots&\vdots&\vdots\\
0&a_{m2}^{(1)}&\cdots&a_{mm}^{(1)}&b_m^{(1)}\\
\end{pmatrix},
\end{gather*}
где $c_{2j}$ и $y_2$ -- нормированные коэффициенты 
2-ой строки и правой части 2-го уравнения, соответственно.
\begin{gather*}
c_{2j}=\dfrac{a_{2j}^{(1)}}{a_{22}^{(1)}}
\quad(j=3,4,\ldots,m),\quad
y_2=\dfrac{b_1^{(1)}}{a_{22}^{(1)}}.
\end{gather*}

Последовательно умножим второе уравнение системы 
на $a_{i2}^{(1)}$ и вычтем полученное уравнение 
из каждого $i$-го уравнения системы $i=3,4,\ldots,m$.
В результате расширенная матрица примет вид:
\begin{gather*}
\begin{pmatrix}[crcr|r]
1&c_{12}&\cdots&c_{1m}&y_1\\
0&1&\cdots&c_{2m}&y_2\\
\vdots&\vdots&\ddots&\vdots&\vdots\\
0&0&\cdots&a_{mm}^{(1)}-a_{m2}^{(1)}\cdot c_{2m}&b_m^{(1)}-a_{m2}^{(1)}\cdot y_2\\
\end{pmatrix}
\end{gather*}

или в более компактном виде:
\begin{gather*}
\begin{pmatrix}[cccc|r]
1&c_{12}&\cdots&c_{1m}&y_1\\
0&1&\cdots&c_{2m}&y_2\\
\vdots&\vdots&\ddots&\vdots&\vdots\\
0&0&\cdots&a_{mm}^{(2)}&b_m^{(2)}\\
\end{pmatrix}
\end{gather*}
где $a_{ij}^{(2)}$ и $b_i^{(2)}$
-- повторно модифицированные коэффициенты при неизвестных 
и правой части, соответственно:
\begin{gather*}
a_{ij}^{(2)}=(a_{ij}^{(1)}-a_{i2}^{(1)}\cdot c_{2j}),\quad
b_i^{(2)}=(b_i^{(1)}-a_{i2}^{(1)}\cdot y_{2}),\quad
(i,j=3,4,\dots,m)
\end{gather*}

Исключая таким же образом неизвестные $x_3,x_4,\dots,x_m$, 
исходная система линейных уравнений приводится 
к эквивалентному виду:
\begin{gather}\label{eq:UMatrix}
\begin{pmatrix}[cccc|r]
1&c_{12}&\cdots&c_{1m}&y_1\\
0&1&\cdots&c_{2m}&y_2\\
\vdots&\vdots&\ddots&\vdots&\vdots\\
0&0&\cdots&1&y_m\\
\end{pmatrix}
\end{gather}

%
%	Обратный ход метода Гаусса
%
\emptyline
\subsection{Обратный ход метода Гаусса}
Обратный ход заключается в нахождении неизвестных 
$x_1,x_2,\dots,x_m$ полученной эквивалентной системы 
в прямом ходе метода Гаусса.
Поскольку расширенная матрица системы \eqref{eq:UMatrix}
имеет треугольный вид, 
то можно последовательно найти все неизвестные 
$x_m,x_{m-1}, \dots,x_1$:
\begin{gather*}
\left\{\begin{array}{lclclcl}
x_m&=&y_m\\
x_{m-1}&=&y_{m-1}&-&c_{m-1,m}\cdot x_m\\
x_{m-2}&=&y_{m-2}&-&c_{m-2,m-1}\cdot x_{m-1}&-&c_{m-2,m}\cdot x_m\\
\hdotsfor{1}&=&\hdotsfor{5}\\
x_1&=&y_1&-&\sum\limits_{j=2}^{m} c_{1j}\cdot x_j\\
\end{array}\right.
\end{gather*}

Общие формулы обратного хода имеют вид:
\begin{gather*}
x_m=y_m,\quad 
x_i=y_i - \sum\limits_{j=i+1}^{m} c_{ij}\cdot x_j,
\quad i=(m-1,m-2,\dots,1)
\end{gather*}

%
%	Пример решения методом Гаусса
%
\emptyline
\subsection{Численное решение системы линейных 
алгебраических уравнений методом Гаусса}
Представим систему линейных алгебраических уравнений
в матричном виде и запишем расширенную матрицу этой системы,
полученную путем добавления к матрице системы
столбца правой части уравнений:
\begin{gather*}
\left\{\begin{matrix}
2x_1&+&3x_2&+&x_3&=&10\\
4x_1&+&5x_2&+&6x_3&=&31\\
3x_1&+&x_2&+&5x_3&=&22\\
\end{matrix}\right.
\quad\iff\quad
\begin{pmatrix}[ccc|c]
\pgfkeys{/main=2}&3&1&10\\
4&5&6&31\\
3&1&5&22\\
\end{pmatrix}
\end{gather*}

% Прямой ход метода Гаусса
\emph{Прямой ход метода Гаусса.}
\begin{enumerate}
\item
Разделим каждую строку матрицы на значение её
элемента в первом столбце, т.е. первую строку делим на $2$,
вторую на $4$, третью на 3:
\begin{gather*}
\begin{pmatrix}[ccc|c]
\pgfkeys{/main=2}&3&1&10\\
4&5&6&31\\
3&1&5&22\\
\end{pmatrix}\quad\to\quad
% результат
\begin{pmatrix}[ccc|c]
\pgfkeys{/main=1}&\frac{3}{2}&\frac{1}{2}&\frac{10}{2}\\
\pgfkeys{/one}&\frac{5}{4}&\frac{6}{4}&\frac{31}{4}\\
\pgfkeys{/one}&\frac{1}{3}&\frac{5}{3}&\frac{22}{3}\\
\end{pmatrix}
\end{gather*}

Вычитаем из второй и третьей строк матрицы её первую строку:
\begin{gather*}
\begin{pmatrix}[ccc|c]
\pgfkeys{/main=1}&\frac{3}{2}&\frac{1}{2}&\frac{10}{2}\\
1&\frac{5}{4}&\frac{6}{4}&\frac{31}{4}\\
1&\frac{1}{3}&\frac{5}{3}&\frac{22}{3}\\
\end{pmatrix}\quad\to\quad
% результат
\begin{pmatrix}[ccc|c]
\pgfkeys{/main=1}&\frac{3}{2}&\frac{1}{2}&\frac{10}{2}\\
\pgfkeys{zero}&-\frac{1}{4}&1&\frac{11}{4}\\
\pgfkeys{zero}&-\frac{7}{6}&\frac{7}{6}&\frac{7}{3}\\
\end{pmatrix}
\end{gather*}

\item
Разделим вторую строку матрицы на $-\frac{1}{4}$, 
третью строку на $-\frac{7}{6}$:
\begin{gather*}
\begin{pmatrix}[ccc|c]
\pgfkeys{/main=1}&\frac{3}{2}&\frac{1}{2}&\frac{10}{2}\\
\pgfkeys{zero}&\pgfkeys{/main={-\frac{1}{4}}}&1&\frac{11}{4}\\
\pgfkeys{zero}&-\frac{7}{6}&\frac{7}{6}&\frac{7}{3}\\
\end{pmatrix}\quad\to\quad
% результат
\begin{pmatrix}[ccc|c]
\pgfkeys{/main=1}&\frac{3}{2}&\frac{1}{2}&\frac{10}{2}\\
\pgfkeys{zero}&\pgfkeys{/main=1}&-4&-11\\
\pgfkeys{zero}&\pgfkeys{/one}&-1&-2\\
\end{pmatrix}
\end{gather*}

Вычитаем из третьей строки матрицы её вторую строку:
\begin{gather*}
\begin{pmatrix}[ccc|c]
\pgfkeys{/main=1}&\frac{3}{2}&\frac{1}{2}&\frac{10}{2}\\
\pgfkeys{zero}&\pgfkeys{/main=1}&-4&-11\\
\pgfkeys{zero}&1&-1&-2\\
\end{pmatrix}\quad\to\quad
% результат
\begin{pmatrix}[ccc|c]
\pgfkeys{/main=1}&\frac{3}{2}&\frac{1}{2}&\frac{10}{2}\\
\pgfkeys{zero}&\pgfkeys{/main=1}&-4&-11\\
\pgfkeys{zero}&\pgfkeys{zero}&3&9\\
\end{pmatrix}
\end{gather*}

\item
Разделим третью строку матрицы на $3$:
\begin{gather*}
\begin{pmatrix}[ccc|c]
1&\frac{3}{2}&\frac{1}{2}&\frac{10}{2}\\
\pgfkeys{zero}&1&-4&-11\\
\pgfkeys{zero}&\pgfkeys{zero}&3&9\\
\end{pmatrix}\quad\to\quad
% результат
\begin{pmatrix}[ccc|c]
\pgfkeys{/main=1}&\frac{3}{2}&\frac{1}{2}&\frac{10}{2}\\
\pgfkeys{zero}&\pgfkeys{/main=1}&-4&-11\\
\pgfkeys{zero}&\pgfkeys{zero}&\pgfkeys{/main=1}&3\\
\end{pmatrix}
\end{gather*}
\end{enumerate}

\emph{Обратный ход метода Гаусса}.\par
Последовательно определяем неизвестные
в обратном порядке их следования
$x_3\to x_2\to x_1$:
\begin{enumerate}
\item
Из третьего уравнения системы (третья строка матрицы)
определяем неизвестное $x_3$:
\begin{gather*}
x_3=3
\end{gather*}
\item
Из второго уравнения системы (вторая строка матрицы)
определяем неизвестное $x_2$:
\begin{gather*}
x_2-4\,x_3=-11\quad\to\quad
x_2=4\,x_3-11\\
x_2=4\cdot3-11=1
\end{gather*}
\item
Из первого уравнения системы (первая строка матрицы)
определяем неизвестное $x_1$:
\begin{gather*}
x_1+\frac{3}{2}\,x_2+\frac{1}{2}\,x_3=\frac{10}{2}\quad\to\quad
x_1=-\frac{3}{2}\,x_2-\frac{1}{2}\,x_3+5\\[1ex]
x_1=-\frac{3}{2}\cdot{1}-\frac{1}{2}\cdot{3}+5=2
\end{gather*}

Таким образом, найдено решение системы:
\begin{gather*}
\vect{s}=\begin{pmatrix}2\\1\\3\end{pmatrix}
\qquad\iff\qquad
\vect{s}=(2,1,3)^\mathrm{T}
\end{gather*}
\end{enumerate}

Проведём \emph{проверку решения} системы уравнений
методом прямой подстановки найденного 
вектора неизвестных $\vect{s}=(2,1,3)^\mathrm{T}$ 
в исходную систему уравнений:
\begin{gather*}
\mathbf{A}\cdot\vect{s}=\vect{b}
\quad\iff\quad
\begin{pmatrix}[ccc|c]
2&3&1&10\\
4&5&6&31\\
3&1&5&22\\
\end{pmatrix}
\cdot\begin{pmatrix}2\\1\\3\\\end{pmatrix}
\end{gather*}

После умножения расширенной матрицы
системы уравнений на вектор найденного решения
получим тождество:
\begin{gather*}
\begin{pmatrix}[c|c]
2\cdot2+3\cdot1+1\cdot3&10\\
4\cdot2+5\cdot1+6\cdot3&31\\
3\cdot2+1\cdot1+5\cdot3&22\\
\end{pmatrix}
=
\begin{pmatrix}[c|c]
10&10\\
31&31\\
22&22\\
\end{pmatrix}
\end{gather*}

%
%	Метод Гаусса с выбором главного элемента
%
\emptyline
\subsection{Метод Гаусса с выбором главного элемента}
На практике, часто может оказаться, что система \eqref{eq:LSM}
имеет единственное решение, хотя какой-либо 
из угловых миноров матрицы $\mathbf{A}$ равен нулю. 
Кроме того, заранее обычно неизвестно, 
все ли угловые миноры матрицы $\mathbf{A}$ отличны от нуля.
В этих случаях обычный метод Гаусса может оказаться 
\emph{непригодным}.
Избежать указанных трудностей позволяет метод Гаусса 
с выбором главного элемента.

\emph{Основная идея метода} состоит в том, чтобы 
на очередном шаге исключать не следующее по номеру неизвестное,
а то неизвестное, коэффициент при котором 
является \emph{наибольшим по модулю}.
Таким образом, в качестве ведущего элемента здесь выбирается 
\alert{главный}, т.е. наибольший по модулю элемент.
Поэтому, если $\det\mathbf{A}\ne0$, то в процессе вычислений 
не будет происходить деление на нуль.

На практике чаще всего применяется и метод Гаусса 
с выбором главного элемента по всей матрице, 
когда в качестве ведущего выбирается максимальный 
по модулю элемент \emph{среди всех элементов} матрицы системы.

\emptyline
\subsection{Численное решение системы линейных 
алгебраических уравнений методом Гаусса
с выбором главного элемента по всей матрице}
Рассмотрим на примере решение системы линейных уравнений
методом Гаусса с выбором главного элемента по всей
матрице системы:
\begin{gather*}
\left\{\begin{matrix}
2x_1&+&3x_2&+&x_3&=&10\\
4x_1&+&5x_2&+&6x_3&=&31\\
3x_1&+&x_2&+&5x_3&=&22\\
\end{matrix}\right.
\quad\iff\quad
\begin{pmatrix}[ccc|c]
2&3&1&10\\
4&5&6&31\\
3&1&5&22\\
\end{pmatrix}
\end{gather*}

\emph{Прямой ход метода Гаусса с выбором главного элемента}.
\begin{enumerate}
\item
Выбираем максимальный по модулю элемент в матрице 
(главный элемент) во второй строке, третьем столбце
(выделен цветом):
\begin{gather*}
\begin{pmatrix}[ccc|c]
2&3&1&10\\
4&5&\pgfkeys{/main=6}&31\\
3&1&5&22\\
\end{pmatrix}
\end{gather*}
Разделим каждую строку матрицы на значение элемента
матрицы в столбце главного элемента, т.е.
первую строку делим на $1$, вторую строку на $6$,
а третью строку на $5$:
\begin{gather*}
\begin{pmatrix}[ccc|c]
2&3&1&10\\
4&5&\pgfkeys{/main=6}&31\\
3&1&5&22\\
\end{pmatrix}\quad\to\quad
% результат
\begin{pmatrix}[ccc|c]
2&3&\pgfkeys{/one}&10\\
\frac{4}{6}&\frac{5}{6}&\pgfkeys{/main=1}&\frac{31}{6}\\
\frac{3}{5}&\frac{1}{5}&\pgfkeys{/one}&\frac{22}{5}\\
\end{pmatrix}
\end{gather*}

Вычитаем из первой и третьей строки вторую строку:
\begin{gather*}
\begin{pmatrix}[ccc|c]
2&3&1&10\\
\frac{4}{6}&\frac{5}{6}&\pgfkeys{/main=1}&\frac{31}{6}\\
\frac{3}{5}&\frac{1}{5}&1&\frac{22}{5}\\
\end{pmatrix}\quad\to\quad
% результат
\begin{pmatrix}[ccc|c]
\frac{4}{3}&\frac{13}{6}&\pgfkeys{/zero}&\frac{29}{6}\\
\frac{4}{6}&\frac{5}{6}&\pgfkeys{/main=1}&\frac{31}{6}\\
-\frac{1}{15}&-\frac{19}{30}&\pgfkeys{/zero}&-\frac{23}{30}\\
\end{pmatrix}
\end{gather*}

\item
Исключаем из рассмотрения строку с текущим главным 
элементом (вторую строку) и выбираем новый 
главный элемент матрицы (первая строка, второй столбец): 
\begin{gather*}
\begin{pmatrix}[ccc|c]
\frac{4}{3}&\pgfkeys{/main=\frac{13}{6}}&\pgfkeys{/zero}&\frac{29}{6}\\
\frac{4}{6}&\frac{5}{6}&\pgfkeys{/main=1}&\frac{31}{6}\\
-\frac{1}{15}&-\frac{19}{30}&\pgfkeys{/zero}&-\frac{23}{30}\\
\end{pmatrix}
\end{gather*}

Делим каждую строку матрицы на значение элемента
матрицы в столбце главного элемента, т.е.
первую строку делим на $\frac{13}{6}$, а третью строку на $-\frac{19}{30}$:
\begin{gather*}
\begin{pmatrix}[ccc|c]
\frac{4}{3}&\pgfkeys{/main=\frac{13}{6}}&\pgfkeys{/zero}&\frac{29}{6}\\
\frac{4}{6}&\frac{5}{6}&\pgfkeys{/main=1}&\frac{31}{6}\\
-\frac{1}{15}&-\frac{19}{30}&\pgfkeys{/zero}&-\frac{23}{30}\\
\end{pmatrix}\quad\to\quad
% результат
\begin{pmatrix}[ccc|c]
\frac{8}{13}&\pgfkeys{/main=1}&\pgfkeys{/zero}&\frac{29}{13}\\
\frac{4}{6}&\frac{5}{6}&\pgfkeys{/main=1}&\frac{31}{6}\\
\frac{2}{19}&\pgfkeys{/one}&\pgfkeys{/zero}&\frac{23}{19}\\
\end{pmatrix}
\end{gather*}

Вычитаем из третьей строки матрицы её первую строку:
\begin{gather*}
\begin{pmatrix}[ccc|c]
\frac{8}{13}&\pgfkeys{/main=1}&\pgfkeys{/zero}&\frac{29}{13}\\
\frac{4}{6}&\frac{5}{6}&\pgfkeys{/main=1}&\frac{31}{6}\\
\frac{2}{19}&1&\pgfkeys{/zero}&\frac{23}{19}\\
\end{pmatrix}\quad\to\quad
% результат
\begin{pmatrix}[ccc|c]
\frac{8}{13}&\pgfkeys{/main=1}&\pgfkeys{/zero}&\frac{29}{13}\\
\frac{4}{6}&\frac{5}{6}&\pgfkeys{/main=1}&\frac{31}{6}\\
-\frac{126}{247}&\pgfkeys{/zero}&\pgfkeys{/zero}&-\frac{252}{247}\\
\end{pmatrix}
\end{gather*}

\item
Исключаем из рассмотрения строку с текущим главным 
элементом (первую) и выбираем новый главный 
элемент матрицы (третья строка, первый столбец): 
\begin{gather*}
\begin{pmatrix}[ccc|c]
\frac{8}{13}&\pgfkeys{/main=1}&\pgfkeys{/zero}&\frac{29}{13}\\
\frac{4}{6}&\frac{5}{6}&\pgfkeys{/main=1}&\frac{31}{6}\\
\pgfkeys{/main=-\frac{126}{247}}&\pgfkeys{/zero}&\pgfkeys{/zero}&-\frac{252}{247}\\
\end{pmatrix}
\end{gather*}

Делим каждую строку матрицы на значение элемента
матрицы в столбце главного элемента, т.е.
третью строку делим на $-\frac{126}{247}$:
\begin{gather*}
\begin{pmatrix}[ccc|c]
\frac{8}{13}&\pgfkeys{/main=1}&\pgfkeys{/zero}&\frac{29}{13}\\
\frac{4}{6}&\frac{5}{6}&\pgfkeys{/main=1}&\frac{31}{6}\\
\pgfkeys{/main=-\frac{126}{247}}&\pgfkeys{/zero}&\pgfkeys{/zero}&-\frac{252}{247}\\
\end{pmatrix}\quad\to\quad
% результат
\begin{pmatrix}[ccc|c]
\frac{8}{13}&\pgfkeys{/main=1}&\pgfkeys{/zero}&\frac{29}{13}\\
\frac{4}{6}&\frac{5}{6}&\pgfkeys{/main=1}&\frac{31}{6}\\
\pgfkeys{/main=1}&\pgfkeys{/zero}&\pgfkeys{/zero}&2\\
\end{pmatrix}
\end{gather*}
\end{enumerate}

\emph{Обратный ход метода Гаусса 
с выбором главного элемента}.\par
Определим неизвестные из уравнений системы 
в обратном порядке следования номеров столбцов главных элементов,
т.е. $x_1\to x_2\to x_3$:
\begin{enumerate}
\item
Из третьего уравнения системы определим неизвестное $x_1$:
\begin{gather*}
x_1=2
\end{gather*}
\item
Из первого уравнения системы определим неизвестное $x_2$:
\begin{gather*}
\frac{8}{13}\,x_1+x_2=\frac{29}{13}\quad\to\quad
x_2=-\frac{8}{13}\,x_1+\frac{29}{13}\\[1ex]
x_2=-\frac{8}{13}\cdot2+\frac{29}{13}=1
\end{gather*}
\item
Из второго уравнения системы определим неизвестное $x_3$:
\begin{gather*}
\frac{4}{6}\,x_1+\frac{5}{6}\,x_2+x_3=\frac{31}{6}\quad\to\quad
x_3=-\frac{4}{6}\,x_1-\frac{5}{6}\,x_2+\frac{31}{6}\\[1ex]
x_3=-\frac{4}{6}\cdot2-\frac{5}{6}\cdot1+\frac{31}{6}=3
\end{gather*}
Таким образом, найдено решение системы линейных уравнений:
\begin{gather*}
\vect{s}=\begin{pmatrix}2\\1\\3\end{pmatrix}
\qquad\iff\qquad
\vect{s}=(2,1,3)^\mathrm{T}
\end{gather*}
\end{enumerate}

%\end{document}

% Интерполяция полиномом Лагранжа
%\newpage
\section{Интерполирование функций}
Задача интерполирования состоит в том, чтобы по 
известным значениям функции $f(x)$ в отдельных
точках отрезка восстановить её значения 
в остальных точках этого отрезка. Такая постановка 
задачи допускает множество решений.

Например, задача интерполирования возникает, 
в том случае, когда известны результаты измерения 
$y_i=f(x_i)$ некоторой физической величины $f$ 
в ограниченном количестве точек $x_i$ ($i=0,1,\dots,n$), 
а требуется оценить значения этой величины в других точках.

Интерполирование используется также, когда вычисление 
значений $f(x)$ трудоемко, например, значение искомой функции
может быть определено как решение сложной задачи, 
в которой $x$ играет роль параметра. При этом 
можно вычислить небольшую таблицу значений функции, 
но прямое нахождение функции при большом числе значений 
аргумента практически затруднительно или нецелесообразно. 

\emptyline
\subsection{Линейная интерполяция функции}
При \emph{линейной интерполяции} функция $f(x)$
на отрезке $x\in[a, b]$ заменяется обобщенным 
интерполяционным полиномом 
$p_n(x)$, который построен в виде линейной комбинации 
$(n+1)$ аналитических функций $\{\phi_i(x)\}$
\begin{gather}\label{eq:Interpolation_Polynom}
p_n(x)=c_0\cdot\phi_0(x)+c_1\cdot\phi_1(x)+\ldots+c_n\cdot\phi_n(x)
=\sum\limits_{i=0}^n c_i\cdot\phi_i(x),
\end{gather}
таким образом, чтобы значения полинома $p_n(x)$ 
в определённых точках отрезка $\{x_0,x_1,\dots,x_n\}$ 
(узлах сетки) совпадают со значениями функции 
в этих точках $\{y_0,y_1,\dots,y_n\}$ (условия сопряжения):
\begin{gather}\label{eq:Interpolation Conjugation}
\left\{\begin{matrix}
p_n(x_0)&=&y_0\\
p_n(x_1)&=&y_1\\
\hdotsfor{3}\\
p_n(x_n)&=&y_n\\
\end{matrix}\right.,\quad\iff\quad
% подробно
\left\{\begin{matrix}
\sum\limits_{i=0}^n c_i\cdot\phi_i(x_0)&=&y_0\\[1em]
\sum\limits_{i=0}^n c_i\cdot\phi_i(x_1)&=&y_1\\
\hdotsfor{3}\\
\sum\limits_{i=0}^n c_i\cdot\phi_i(x_n)&=&y_n\\
\end{matrix}\right..
\end{gather}

Из условий \eqref{eq:Interpolation Conjugation}, 
накладываемых на интерполяционный полином,
формулируется система линейных уравнений относительно 
неизвестных коэффициентов полинома ${c_0,c_1,\dots,c_n}$:
\begin{gather}\label{eq:Interpolation_LS}
\mathbf{A}\cdot\vect{c}=\vect{y},
\end{gather}
где $\mathbf{A}$ -- квадратная матрица $(n+1)\times(n+1)$,
$\vect{c}$ и $\vect{y}$ -- 
вектор неизвестных коэффициентов полинома $p_n(x)$
и вектор значений функции $f(x)$ в заданных точках $\{x_i\}$:
\begin{gather*}
\mathbf{A}=
\begin{pmatrix}
\phi_0(x_0)&\phi_1(x_0)&\cdots&\phi_n(x_0)\\
\phi_0(x_1)&\phi_1(x_1)&\cdots&\phi_n(x_1)\\
\vdots&\vdots&\ddots&\vdots\\
\phi_0(x_n)&\phi_1(x_n)&\cdots&\phi_n(x_n)\\
\end{pmatrix},
\quad
\vect{c}=\begin{pmatrix}c_0\\c_1\\\vdots\\c_n\end{pmatrix},
\quad
\vect{y}=\begin{pmatrix}y_0\\y_1\\\vdots\\y_n\end{pmatrix}.
\end{gather*}

Если среди узлов интерполяции $\{x_i\}$ нет совпадающих
($x_i\ne x_j$ для всех $i,j=0,1,\dots,n$) и 
определитель системы отличен от нуля $\det\mathrm{A}\ne0$,
то задача интерполяции имеет единственное решение, а
система функций $\{\phi_i(x)\}$ называется чебышевской. 
Поэтому при линейной интерполяции необходимо 
строить обобщенный полином $p_n(x)$ на основе 
\emph{чебышевской системы функций}.

Таким образом, для определения коэффициентов 
интерполяционного полинома 
\eqref{eq:Interpolation_Polynom} необходимо 
найти решение системы линейных уравнений
\eqref{eq:Interpolation_LS}, 
любыми аналитическими, приближенными
или численными методами, например, методом Гаусса.

Интерполирование не всегда дает удовлетворительное решение 
задачи о приближении функции \emph{с заданной точностью}
на данном промежутке, так как совпадение функции $f(x)$
с полиномом $p(x)$ в точках $x_i$ и $x_{i+1}$
не гарантирует малость величины
$\abs{f(x)-p(x)}$ на отрезке $[x_i,x_{i+1}]$.

%
% Интерполирование алгебраическими полиномами
%
\emptyline
\subsection{Интерполяция алгебраическими полиномами}
Задача интерполяции алгебраическими полиномами 
сводится к построению полинома степени $n$ по
чебышевской системе алгебраических функций
$\{1,x,x^2,\dots,x^n\}$:
\begin{equation}\label{eq:Interpolation_AP}
p_{n}(x)
=c_{0}+c_{1}\cdot{x}+c_{2}\cdot{x^2}+\ldots+c_{n}\cdot{x^n}
=\sum\limits_{i=0}^{n}c_i\cdot x^i.
\end{equation}

Определитель системы \eqref{eq:Interpolation_LS}
представляет собой определителем Вандермонда, 
который  отличен от нуля $\det\mathbf{A}\ne0$ , 
если среди точек $\{x_i\}$ нет совпадающих,
т.е. $x_i\ne x_j$ для всех $i,j=0,1,\dots,n$:
\begin{gather*}
\det\mathbf{A}=
\begin{vmatrix}
1&x_0&\cdots&x_0^n\\
1&x_1&\cdots&x_1^n\\
\vdots&\vdots&\ddots&\vdots\\
1&x_n&\cdots&x_n^n\\
\end{vmatrix}.
\end{gather*}

Выражение для коэффициентов алгебраического полинома
и вид самого полинома \eqref{eq:Interpolation_AP} 
можно записать различными способами.
Наиболее распространенная запись интерполяционного 
многочлена в форме Лагранжа и в форме Ньютона. 

%
%	Интерполяция функций полиномами Лагранжа
%
\emptyline
\subsection{Интерполяционный полином в форме Лагранжа}
Интерполяционная формула Лагранжа позволяет 
представить многочлен $L_{n}(x)$ в виде 
линейной комбинации значений функции $y(x)$ 
в узлах интерполирования $\{x_i\}$:
\begin{equation}\label{eq:Lagrange}
L_{n}(x)
=\lambda_{0}(x)\cdot{y_0}+\lambda_{1}(x)\cdot{y_1}+\dots+\lambda_{n}(x)\cdot{y_n}
=\sum\limits_{i=0}^n \lambda_i(x)\cdot y_i
\end{equation}
где $\lambda_0(x),\lambda_1(x),\dots,\lambda_n(x)$ -- произвольные неизвестные функции.

Для определения неизвестных функций $\lambda_i(x)$ 
из условий интерполирования следует:
\begin{equation*}\label{inter2}
\left\{\begin{matrix}
\lambda_0(x_0)\cdot{y_0}+\lambda_1(x_0)\cdot{y_1}+\dots+\lambda_n(x_0)\cdot{y_n}&=&y_0\\
\lambda_0(x_1)\cdot{y_0}+\lambda_1(x_1)\cdot{y_1}+\dots+\lambda_n(x_1)\cdot{y_n}&=&y_1\\
\hdotsfor{3}\\
\lambda_0(x_n)\cdot{y_0}+\lambda_1(x_n)\cdot{y_1}+\dots+\lambda_n(x_n)\cdot{y_n}&=&y_n\\
\end{matrix}\right.
\end{equation*}

Эта система уравнений имеет решение если выполняются условия:
\begin{equation*}
\label{uslovia_c}
\lambda_{i}(x_j)=\left\{\begin{matrix}
1, &x_j=x_{i}\\
0, &x_j\ne{x_{i}}
\end{matrix}\right.
\end{equation*}

Коэффициенты $\lambda_{i}(x)$ можно искать в виде 
многочленов степени $n$:
\begin{equation*}
\label{eq_c}
\left\{\begin{matrix}
\lambda_0(x)&=&\alpha_0\cdot(x-x_1)\cdot(x-x_2)
\cdot(x-x_2)\cdot&\ldots&\cdot(x-x_n)\\
\lambda_1(x)&=&\alpha_1\cdot(x-x_0)\cdot(x-x_2)
\cdot(x-x_3)\cdot&\ldots&\cdot(x-x_n)\\
\hdotsfor{5}\\
\lambda_n(x)&=&\alpha_n\cdot(x-x_0)\cdot(x-x_1)
\cdot(x-x_2)\cdot&\ldots&\cdot(x-x_{n-1})
\end{matrix}\right.
\end{equation*}

Определим неизвестные 
$\alpha_0, \alpha_1, \ldots, \alpha_n$ 
из условия для коэффициентов $\lambda_i(x)$:
\begin{equation*}
\left\{
\begin{matrix}
1&=&\alpha_0\cdot(x_0-x_1)\cdot(x_0-x_2)\cdot(x_0-x_2)\cdot&
\dots&\cdot(x_0-x_n)\\
1&=&\alpha_1\cdot(x_1-x_0)\cdot(x_1-x_2)\cdot(x_1-x_3)\cdot&
\dots&\cdot(x_1-x_n)\\
\hdotsfor{5}\\
1&=&\alpha_n\cdot(x_n-x_0)\cdot(x_n-x_1)\cdot(x_n-x_2)\cdot&
\dots&\cdot(x_n-x_{n-1})
\end{matrix}
\right.
\end{equation*}

Таким образом, коэффициенты $\lambda_{i}(x)$ 
интерполяционного многочлена\linebreak
Лагранжа находятся из соотношений:
\begin{equation*}
\left\{\begin{matrix}
\lambda_0(x)&=&\dfrac
{(x-x_1)\cdot(x-x_2)\cdot\dots\cdot(x-x_n)}
{(x_0-x_1)\cdot(x_0-x_2)\cdot\dots\cdot(x_0-x_n)}\\[1em]
\lambda_1(x)&=&\dfrac
{(x-x_0)\cdot(x-x_2)\cdot\dots\cdot(x-x_n)}
{(x_1-x_0)\cdot(x_1-x_2)\cdot\dots\cdot(x_1-x_n)}\\[1em]
\hdotsfor{3}\\[1em]
\lambda_n(x)&=&\dfrac
{(x-x_0)\cdot(x-x_1)\cdot\dots\cdot(x-x_{n-1})}
{(x_n-x_0)\cdot(x_n-x_1)\cdot\dots\cdot(x_n-x_{n-1})}
\end{matrix}\right.,
\end{equation*}
или в более компактной форме:
\begin{equation*}
\lambda_i(x)=\dfrac
{\prod\limits_{j \ne i}^n (x - x_j)}
{\prod\limits_{j \ne i}^n (x_i - x_j)}
\end{equation*}

Итак, интерполяционный многочлен Лагранжа 
\eqref{eq:Lagrange} имеет вид:
\begin{equation}\label{eq:Lagrange_Polynom}
L_{n}(x)=\sum\limits_{i=0}^n\dfrac
{\prod\limits_{j \ne i}^n(x-x_j) }
{\prod\limits_{j \ne i}^n(x_i-x_j)}\cdot{y_i}
\end{equation}

%
% Интерполирование таблично заданной функции
%
\emptyline
\subsection{Интерполяция функции заданной таблично}
Известно множество данных (узлов интерполяции)
$\{x_i\}$, в которых определены 
значения функции $y_i=f(x_i)$:
\begin{table}[h]
\vspace{-0.5\baselineskip}
\caption{Таблично заданная функциональная зависимость}
\label{tab:Interpolation_Data}
\begin{tabular*}{\textwidth}{%
l@{\extracolsep{\fill}}*{4}{r}p{0.25cm}}
\toprule
$i$&$0$&$1$&$2$&$3$\\
\midmidrule
$x_i$&$-0.76$&$-0.09$&$0.22$&$0.55$\\
\addlinespace% дополнительный пробел
$y_i$&$0.08$&$1.84$&$0.40$&$0.96$\\
\bottomrule
\end{tabular*}
\end{table}

Построим обобщенный интерполяционный полином $p_3(x)$
для таблично заданной функции исходя из чебышевской 
системы функций $\{1,x,e^{-x},e^x\}$:
\begin{gather*}
p_4(x) = c_0+c_1\cdot{x}+c_2\cdot e^{-x}+c_3\cdot e^x
\end{gather*}

\begin{enumerate}
\item
Составим расширенную матрица системы уравнений 
\eqref{eq:Interpolation_LS} для определения 
коэффициентов полинома $(c_0,c_1,c_2,c_3)^\mathrm{T}$:
\begin{gather*}
\begin{pmatrix}[cccc|c]
\phi_0(x_0)&\phi_1(x_0)&\phi_2(x_0)&\phi_3(x_0)&y_0\\
\phi_0(x_1)&\phi_1(x_1)&\phi_2(x_1)&\phi_3(x_1)&y_1\\
\phi_0(x_2)&\phi_1(x_2)&\phi_2(x_2)&\phi_3(x_2)&y_2\\
\phi_0(x_3)&\phi_1(x_3)&\phi_2(x_3)&\phi_3(x_3)&y_3\\
\end{pmatrix},\quad\text{здесь}\quad
\left\{\begin{matrix}
\phi_0(x)&=&1\\
\phi_1(x)&=&x\\
\phi_2(x)&=&e^{-x}\\
\phi_3(x)&=&e^x\\
\end{matrix}\right..
\end{gather*}
\item
Воспользуемся данными таблицы \ref{tab:Interpolation_Data}
и заполним числовыми значения элементы расширенной матрицы:
\begin{gather*}
\begin{pmatrix}[cccc|c]
1&-0.76&e^{0.76}&e^{-0.76}&0.08\\
1&-0.09&e^{0.09}&e^{-0.09}&1.84\\
1&0.22&e^{-0.22}&e^{0.22}&0.40\\
1&0.55&e^{-0.55}&e^{0.55}&0.96\\
\end{pmatrix}
\;\iff\;
\begin{pmatrix}[cccc|c]
1&-0.76&\fcolorbox{gray!50}{yellow}{2.138}&0.468&0.08\\
1&-0.09&1.094&0.914&1.84\\
1&0.22&0.803&1.246&0.40\\
1&0.55&0.577&1.733&0.96\\
\end{pmatrix}
\end{gather*}

\item
Решение системы линейных уравнений
найдем методом Гаусса с выбором главного элемента
в расширенной матрице (выделен цветом):
\begin{gather*}
\vect{c}=(-0.393,-81.472,-37.288,39.053)^\mathrm{T}
\end{gather*}

Следовательно, обобщенный интерполяционный полином
для функции заданной таблично можно записать в виде:
\begin{gather*}
\textcolor{darkblue}{
p_3(x) = -0.393-81.472\cdot{x}-37.288\cdot e^{-x}+39.053\cdot e^x
}
\end{gather*}

\item
В таблице \ref{tab:Interpolation_GPD} представлены 
данные расчета коэффициентов обобщенного интерполяционного 
полинома $c_i$, значений этого полинома в узлах сетки $p_3(x_i)$
и абсолютная погрешность интерполяции 
$\varepsilon_i=y_i-p_3(x_i)$.
%
% Таблица результатов
%
\vspace{-0.5\baselineskip}
\begin{table}[H]
\caption{Коэффициенты обобщенного интерполяционного 
полинома $c_i$, значения этого полинома в узлах сетки $p_3(x_i)$
и абсолютная погрешность интерполяции $\varepsilon_i$}
\label{tab:Interpolation_GPD}
\begin{tabular*}{\textwidth}{%
@{\extracolsep{\fill}}*{5}{r}p{2cm}}
\toprule
$i$&$0$&$1$&$2$&$3$\\
\midmidrule% @x,@y
$x_i$&$-0.76$&$-0.09$&$0.22$&$0.55$\\
$y_i$&$0.08$&$1.84$&$0.4$&$0.96$\\
\midrule% коэффициенты полинома
$c_i$&$-0.393$&$-81.472$&$-37.288$&$39.053$\\
\midrule% значение полинома и погрешность
$p_3(x_i)$&$0.057$&$1.832$&$0.422$&$0.973$\\
$\varepsilon_i$&$0.023$&$0.008$&$-0.022$&$-0.013$\\
\bottomrule
\end{tabular*}
\end{table}

\item
На рисунке \ref{fig:Interpolation_GP} представлена 
диаграмма рассеяния (разброса) данных 
функции заданной таблично $y_i=f(x_i)$ (маркеры) и 
результаты вычислений обобщенного интерполяционного 
полинома $p_3(x)$ (сплошная линия).
\begin{figure}[H]\centering
\begin{tikzpicture}
\begin{axis}[ylabel=$p_3(x)$,
xmin=-0.9,xmax=0.7,xtick={-0.8,-0.4,0,0.4},
ymin=-1,ymax=3]
% табличные данные
\addplot[ball darkblue,only marks] coordinates {(-0.76,0.08)(-0.09,1.84)(0.22,0.4)(0.55,0.96)};
% интерполяционный полином
\addplot[darkblue,domain=-0.8:0.6,samples=50]
{-0.393*1-81.472*x-37.288*exp(-x)+39.053*exp(x)}
node[pos=0.7,right] {$p_3(x)$};
\end{axis}
\end{tikzpicture}
\caption{График таблично заданной функции $y_i=f(x_i)$ (маркеры) 
и обобщенного интерполяционного полинома $p_3(x)$
(сплошная линия)}
\label{fig:Interpolation_GP}
\end{figure}
\end{enumerate}

%
% Таблица графика
%
% xmin = -0.8
% xmax = 0.6
% dx   = 0.028
%
\begin{table}[H]
\caption{Рассчётные значения обобщенного интерполяционного 
полинома в узлах сетки $p_3(x_i)$}
\label{tab:Interpolation Plot}
\small
\begin{tabular*}{\textwidth}{%
p{1cm}@{\extracolsep{\fill}}*{9}{r}}
\toprule
$i$&$0$&$1$&$2$&$\dots$&$\dots$&$\dots$&$47$&$48$&$49$\\
\midmidrule
$x_i$&$-0.800$&$-0.771$&$-0.743$&$\dots$&$\dots$&$\dots$&$0.543$&$0.571$&$0.600$\\
$y_i$&$-0.654$&$-0.128$&$0.330$&$\dots$&$\dots$&$\dots$&$0.920$&$1.145$&$1.419$\\
\bottomrule
\end{tabular*}
\end{table}


Построим интерполяционный полином 
в форме Лагранжа $L_3(x)$ на основе данных 
об узлах интерполяции $\{x_i\}$ 
и значений функции в этих узлах $\{y_i\}$:
\begin{gather*}
L_{3}(x)=\sum \limits_{i=0}^3\dfrac
{\prod\limits_{j \ne i}^3(x-x_j)}
{\prod\limits_{j \ne i}^3(x_i-x_j)}\cdot{y_i}
\end{gather*}

\begin{enumerate}
\item
Представим полином Лагранжа в развернутом виде:
\begin{gather*}
\begin{split}
L_{3}(x)=
&\;\dfrac{(x-x_1)\cdot(x-x_2)\cdot(x-x_3)}{(x_0-x_1)\cdot(x_0-x_2)\cdot(x_0-x_3)}\cdot y_0+\\[1ex]
&\;\dfrac{(x-x_0)\cdot(x-x_2)\cdot(x-x_3)}{(x_1-x_0)\cdot(x_1-x_2)\cdot(x_1-x_3)}\cdot y_1+\\[1ex]
&\;\dfrac{(x-x_0)\cdot(x-x_1)\cdot(x-x_3)}{(x_2-x_0)\cdot(x_2-x_1)\cdot(x_2-x_3)}\cdot y_2+\\[1ex]
&\;\dfrac{(x-x_0)\cdot(x-x_1)\cdot(x-x_2)}{(x_3-x_0)\cdot(x_3-x_1)\cdot(x_3-x_2)}\cdot y_3
\end{split}
\end{gather*}

\item
Воспользуемся численными данными об узлах интерполяции 
$\{x_i\}$ и известными значениями интерпретируемой функции 
в этих узлах $\{y_i\}$:
\begin{gather*}
\begin{split}
L_{3}(x)=
&\;\dfrac{(x-(-0.09))\cdot(x-0.22)\cdot(x-0.55)}{(-0.76-(-0.09))\cdot(-0.76-0.22)\cdot(-0.76-0.55)}\cdot0.08+\\[1ex]
&\;\dfrac{(x-(-0.76))\cdot(x-0.22)\cdot(x-0.55)}{(-0.09-(-0.76))\cdot(-0.09-0.22)\cdot(-0.09-0.55)}\cdot1.84+\\[1ex]
&\;\dfrac{(x-(-0.76))\cdot(x-(-0.09))\cdot(x-0.55)}{(0.22-(-0.76))\cdot(0.22-(-0.09))\cdot(0.22-0.55)}\cdot0.40+\\[1ex]
&\;\dfrac{(x-(-0.76))\cdot(x-(-0.09))\cdot(x-0.22)}{(0.55-(-0.76))\cdot(0.55-(-0.09))\cdot(0.55-0.22)}\cdot0.96
\end{split}
\end{gather*}

\item
Проведем необходимые арифметические действия:
\begin{gather*}
\begin{split}
L_{3}(x)=
&\;\dfrac{(x+0.09)\cdot(x-0.22)\cdot(x-0.55)}{(-0.67)\cdot(-0.98)\cdot(-1.31)}\cdot0.08+\\[1ex]
&\;\dfrac{(x+0.76)\cdot(x-0.22)\cdot(x-0.55)}{(0.67)\cdot(-0.31)\cdot(-0.64)}\cdot1.84+\\[1ex]
&\;\dfrac{(x+0.76)\cdot(x+0.09)\cdot(x-0.55)}{(0.98)\cdot(0.31)\cdot(-0.33)}\cdot0.40+\\[1ex]
&\;\dfrac{(x+0.76)\cdot(x+0.09)\cdot(x-0.22)}{(1.31)\cdot(0.64)\cdot(0.33)}\cdot0.96
\end{split}
\end{gather*}

или
\begin{gather*}
\begin{matrix}
L_{3}(x)=
&\;\dfrac{(x+0.09)\cdot(x-0.22)\cdot(x-0.55)}{-0.86}\cdot0.08+\\[1ex]
&\;\dfrac{(x+0.76)\cdot(x-0.22)\cdot(x-0.55)}{0.13}\cdot1.84+\\[1ex]
&\;\dfrac{(x+0.76)\cdot(x+0.09)\cdot(x-0.55)}{-0.10}\cdot0.40+\\[1ex]
&\;\dfrac{(x+0.76)\cdot(x+0.09)\cdot(x-0.22)}{0.28}\cdot0.96&
\end{matrix}
\end{gather*}

Продолжая делать упрощения окончательно получим:
\begin{gather*}
\begin{split}
L_{3}(x)=
&\;(x+0.09)\cdot(x-0.22)\cdot(x-0.55)\cdot(-0.09)+\\
&\;(x+0.76)\cdot(x-0.22)\cdot(x-0.55)\cdot13.84+\\
&\;(x+0.76)\cdot(x+0.09)\cdot(x-0.55)\cdot(-3.99)+\\
&\;(x+0.76)\cdot(x+0.09)\cdot(x-0.22)\cdot3.47
\end{split}
\end{gather*}

\item
Запишем выражение для интерполяционный полином Лагранжа
в каноническом виде:
\begin{gather*}\textcolor{darkred}{
L_{3}(x)=1.37 - 5.248\cdot x + 0.912\cdot x^2 + 13.23\cdot x^3
}
\end{gather*}

\item
В таблице \ref{tab:Interpolation_LPD} представлены 
данные расчета коэффициентов интерполяционного 
полинома Лагранжа $c_i$, значений этого полинома 
в узлах сетки $L_3(x_i)$ и абсолютная погрешность интерполяции 
$\varepsilon_i=y_i-L_3(x_i)$.
%
% Таблица
%
\begin{table}[H]
\caption{Коэффициенты интерполяционного 
полинома Лагранжа $c_i$, значения этого полинома 
в узлах сетки $L_3(x_i)$ и абсолютная погрешность 
интерполяции $\varepsilon_i$}
\label{tab:Interpolation_LPD}
\begin{tabular*}{\textwidth}{%
@{\extracolsep{\fill}}*{6}{r}}
\toprule
$i$&$x_i$&$y_i$&$c_i$&$L_3(x_i)$&$\varepsilon_i$\\
\midmidrule
0&$-0.76$&$0.08$&$1.37$&$0.078$&$0.002$\\
1&$-0.09$&$1.84$&$-5.248$&$1.840$&$0.000$\\
2&$0.22$&$0.40$&$0.912$&$0.400$&$0.000$\\
3&$0.55$&$0.96$&$13.23$&$0.961$&$-0.001$\\
\bottomrule
\end{tabular*}
\end{table}

\item
На рисунке \eqref{fig:Interpolation_LP} представлена 
диаграмму рассеяния (разброса) данных функции 
заданной таблично $y_i=f(x_i)$ (маркеры) и 
результаты вычислений интерполяционного 
полинома Лагранжа $L_3(x)$ (сплошная линия).
% *******************************
%	График функций
%
\begin{figure}[H]\centering
\begin{tikzpicture}
\begin{axis}[% оси координат
ylabel=$L_3(x)$,
xmin=-0.9,xmax=0.7,xtick={-0.8,-0.4,0,0.4},
ymin=-1, ymax=3,
]
% табличные данные
\addplot[ball darkred,only marks]
coordinates {(-0.76,0.08) (-0.09,1.84) (0.22,0.40) (0.55,0.96)};
% полином Лагранжа
\addplot[darkred,mark=none,domain=-0.8:0.6,samples=100] 
{1.37 - 5.248*x + 0.912*x^2 + 13.23*x^3}
node[pos=0.7,right] {$L_3(x)$};
\end{axis}
\end{tikzpicture}
\caption{График таблично заданной функции $y_i=f(x_i)$ (маркеры) 
и интерполяционного полинома Лагранжа $L_3(x)$
(сплошная линия)}
\label{fig:Interpolation_LP}
\end{figure}
% *******************************
\end{enumerate}

% Численное дифференцирование
%\newpage
%
%	Численное дифференцирование
%
\section{Численное дифференцирование}
Задача численного дифференцирования состоит 
в приближенном вычислении производных функции $y(x)$
по заданным в конечном числе точек $\{x_i\}$ 
значениям этой функции.

Численное дифференцирование применяется, 
если функцию $y(x)$ трудно или невозможно 
продифференцировать аналитически, например, если 
функция является таблично заданной, а также 
при решении дифференциальных уравнений 
разностными методами.

Многие формулы численного дифференцирования можно получить, 
используя интерполяционные формулы.
Для этого достаточно заменить функцию $y(x)$ 
интерполяционным полиномом Лагранжа $L_n(x)$ 
и вычислить производные этого многочлена, 
используя его явное представление.

Рассмотрим произвольную сетку $\{x_i\}$ и 
проведем интерполирование функции $y(x)$ в узлах сетки 
$x_{i-1} < x_{i} < x_{i+1}$ полиномом Лагранжа второго
порядка, приближенно полагая $y(x)\approx L_{2}(x)$
для $x\in[x_{i-1},x_{i+1}]$:
% *******************************
%	График функций
%
\begin{figure}[H]\centering
\begin{tikzpicture}
\begin{axis}[
%axis lines = middle,
every axis/.style={color=black, solid, thick},
xlabel = {\empty},		% подпись оси x
ylabel = {\empty},	% подпись оси y
xmin=-0.75, xmax=2,
ymin=-0.5, ymax=3,
%xtick style={thick, black}, 
xtick={-0.5,0,1.75}, xticklabels={$x_{i-1}$,$x_i$,$x_{i+1}$},
%ytick style={thick, black}, 
ytick={2.25,1,0.5625}, yticklabels={$y_{i-1}$,$y_i$,$y_{i+1}$},
grid=major,		
major grid style={color=black!20, dashed, thin},
]
\addplot [only marks,mark=ball,ball color=darkred!75,
mark size=4pt,mark options={draw=darkred,thin}]
coordinates {(-0.5,2.25) (0,1) (1.75,0.5625)};
\addplot [color=darkred, thick, domain=-0.5:1.75] 
{(x-1)^2} node[pos=0.75,above] {$L_2(x)$};
\end{axis}
\end{tikzpicture}
\end{figure}
% *******************************
\begin{equation*}
\label{approx2}
\begin{matrix}
L_{2}(x)
&=&\dfrac
{ (x-x_{i})\cdot(x-x_{i+1}) }
{ (x_{i-1}-x_{i})\cdot(x_{i-1}-x_{i+1}) } \cdot y_{i-1}& + \\
\\
&+&\dfrac
{ (x-x_{i-1})\cdot(x-x_{i+1}) }
{ (x_{i}-x_{i-1})\cdot(x_{i}-x_{i+1}) } \cdot y_{i}& + \\
\\
&+&\dfrac
{ (x-x_{i-1})\cdot(x-x_{i}) }
{ (x_{i+1}-x_{i-1})\cdot(x_{i+1}-x_{i}) } \cdot y_{i+1}
\end{matrix}
\end{equation*}
где $y_{i-1} = y(x_{i-1})$, $y_{i} = y(x_{i})$, $y_{i+1} = y(x_{i+1})$
-- значение функции $y(x)$ в узлах сетки.

%------------------------------------------------------------------
Первая производная многочлена Лагранжа $L_2(x)$:
\begin{gather*}
\begin{matrix}
L^{\prime}_{2}(x)
&=&\dfrac
{ 2x-x_{i}-x_{i+1} }
{ (x_{i-1}-x_{i})\cdot(x_{i-1}-x_{i+1}) } \cdot y_{i-1}& + \\
\\
&+&\dfrac
{ 2x-x_{i-1}-x_{i+1} }
{ (x_{i}-x_{i-1})\cdot(x_{i}-x_{i+1}) } \cdot y_{i}& + \\
\\
&+&\dfrac
{ 2x-x_{i-1}-x_{i} }
{ (x_{i+1}-x_{i-1})\cdot(x_{i+1}-x_{i}) } \cdot y_{i+1}
\end{matrix}
\end{gather*}

Это выражение можно принять за приближенное значение
первой производной  $y^{\prime}(x)$ 
в любой точке отрезка $[x_{i-1},x_{i+1}]$.
Например, в точке $x=x_{i}$ первая производная от функции 
$y(x)$ приближенно равна:
\begin{gather*}
\begin{matrix}
y^{\prime}(x_i)\approx L^{\prime}_{2}(x_i)
&=&\dfrac
{ x_{i}-x_{i+1} }
{ (x_{i-1}-x_{i})\cdot(x_{i-1}-x_{i+1}) } \cdot y_{i-1}& + \\
\\
&+&\dfrac
{ (x_i-x_{i-1}) + (x_{i}-x_{i+1}) }
{ (x_{i}-x_{i-1})\cdot(x_{i}-x_{i+1}) } \cdot y_{i}& + \\
\\
&+&\dfrac
{ x_{i}-x_{i-1} }
{ (x_{i+1}-x_{i-1})\cdot(x_{i+1}-x_{i}) } \cdot y_{i+1}
\end{matrix}
\end{gather*}

%------------------------------------------------------------------
Вторую производную полинома Лагранжа
можно принять за приближенное значение 
второй производной от функции $y(x)$ 
в любой точке отрезка $[x_{i-1}, x_{i+1}]$:
\begin{gather*}
\begin{matrix}
y^{\prime\prime}(x)\approx L^{\prime\prime}_{2}(x)
&=&\dfrac
{ 2 }
{ (x_{i-1}-x_{i})\cdot(x_{i-1}-x_{i+1}) } \cdot y_{i-1}& + \\
\\
&+&\dfrac
{ 2 }
{ (x_{i}-x_{i-1})\cdot(x_{i}-x_{i+1}) } \cdot y_{i}& + \\
\\
&+&\dfrac
{ 2 }
{ (x_{i+1}-x_{i-1})\cdot(x_{i+1}-x_{i}) } \cdot y_{i+1}
\end{matrix}
\end{gather*}

На \emph{равномерной сетке} $\{x_i\}$, расстояние между 
соседними узлами которой одинаково, выражения 
для первой и второй производной в точке $x=x_i$ упрощаются:
\begin{gather*}
y^{\prime}(x_i)\approx
\dfrac{y_{i+1}-y_{i-1}}{2h},
\qquad
y^{\prime\prime}(x_i)\approx
\dfrac{y_{i-1}-2y_{i}+y_{i+1}}{h^2},
\end{gather*}
где $h=(x_i-x_{i-1})=(x_{i+1}-x_i)$ -- шаг сетки.

Для приближенного вычисления производных более высоких порядков
$y^{(n)}(x)$ уже недостаточно полинома Лагранжа второго 
порядка $L_2(x)$. Поэтому необходимо использовать 
полиномы более высокого порядка, что приводит 
к увеличению числа узлов аппроксимации.

Следует отметить, что порядок погрешности аппроксимации 
производных от функции $y(x)$ зависит  как от порядка 
интерполяционного полинома, так и от расположения 
узлов сетки $\{x_i\}$.

%
%	Численного дифференцирование таблично заданной функции
%
\subsection{Численного дифференцирование функции заданной таблично}
% масиив абсцисс [x]
\def\Xarray{{-0.98,-0.76,-0.48,-0.09,0.22,0.55}}
% масиив ординат [y]
\def\Yarray{{4.11,4.83,5.13,5.01,5.13,6.11}}
% элемент массива
\newcommand\x[1]{\pgfmathparse{\Xarray[#1]} \pgfmathresult}
\newcommand\y[1]{\pgfmathparse{\Yarray[#1]} \pgfmathresult}
%
%	График функций
%
\newcommand\FigNumDiff[3]{
\begin{figure}[H]\centering
\begin{tikzpicture}
\begin{axis}[
xlabel = {$x$},		% подпись оси x
ylabel = {$y(x)$},	% подпись оси y
xmin=-1.5, xmax=1,
ymin=3.5, ymax=6.5,
xtick style={thick, black},
ytick style={thick, black},
grid=major,		
major grid style={color=black!20, dashed, thin},
]
% точки графика
\addplot[only marks,mark=*,mark size=3pt,
mark options={fill=gray!25,draw=darkred}] 
coordinates {#1};
% полином Лагранжа
#2;
% отрезок интерполирования
\addplot[only marks,mark=ball,mark size=3pt,
mark options={ball color=darkred!50,draw=darkred}] 
coordinates {#3};
\end{axis}
\end{tikzpicture}
\end{figure}}

Известно множество данных (узлов сетки) $\{x_i\}$
в которых определены значения функции $\{f(x_i)\}$:
\begin{table}[H]
\vspace{-0.5\baselineskip}
\caption{Таблично заданная функциональная зависимость
$y_i=f(x_i)$}
\label{tab:Num_Diff}
\begin{tabular*}{\textwidth}{%
ll@{\extracolsep{\fill}}*{9}{r}}
\toprule
$i$&&$0$&$1$&$2$&$3$&$4$&$5$&$6$&$7$&$8$\\
\midmidrule
$x_i$&&$-1.2$&$-0.98$&$-0.76$&$-0.48$&$-0.09$&$0.22$&$0.32$&$0.55$&$0.76$\\
\addlinespace% дополнительный пробел
$y_i$&&$3.78$&$4.11$&$4.83$&$5.13$&$5.01$&$5.13$&$5.73$&$6.11$&$5.92$\\
\bottomrule
\end{tabular*}
\end{table}

\begin{enumerate}
\item
Построим график функции $y(x)$, используя данные таблицы \ref{tab:Num_Diff}.
% *******************************
%	График функций
%
\begin{figure}[H]\centering
\begin{tikzpicture}
\begin{axis}[
xlabel = {$x$},		% подпись оси x
ylabel = {$y(x)$},	% подпись оси y
xmin=-1.5, xmax=1,
ymin=3.5, ymax=6.5,
xtick style={thick, black},
ytick style={thick, black},
grid=major,		
major grid style={color=black!20, dashed, thin},
]
\addplot[smooth,color=darkred,mark=ball,mark size=4pt,
mark options={draw=darkred,thin,ball color=darkred!75}]
coordinates 
{(-1.2,3.78) (-0.98,4.11) (-0.76,4.83) (-0.48,5.13) (-0.09,5.01) (0.22,5.13) (0.32,5.73) (0.55,6.11) (0.76,5.92)};
\end{axis}
\end{tikzpicture}
\end{figure}
% *******************************
% x1
\item
Аппроксимируем функцию $y(x)$ в узлах $\{x_{0},x_{1},x_{2}\}$
полиномом Лагранжа второго порядка $L_2(x)$, 
используя данные таблицы \ref{tab:Num_Diff}:
\begin{gather*}
\begin{matrix}
L_2(x)&=&\dfrac{(x-(-0.98))(x-(-0.76))}{(-1.20-(-0.98))(-1.20-(-0.76))}\cdot3.78&+\\[1em]
&+&\dfrac{(x-(-1.20))(x-(-0.76))}{(-0.98-(-1.20))(-0.98-(-0.76))}\cdot4.11&+\\[1em]
&+&\dfrac{(x-(-1.20))(x-(-0.98))}{(-0.76-(-1.20))(-0.76-(-0.98))}\cdot4.83
\end{matrix}
\end{gather*}
Проводя элементарные алгебраические преобразования полином Лагранжа 
в пределах отрезка $[x_0,x_2]$ имеет вид:
\begin{gather*}
L_2(x)=4.028925620\cdot x^2+10.28305785\cdot x+10.31801653
\end{gather*}
% *******************************
%	График функций
%
\FigNumDiff
{(-1.2,3.78) (-0.98,4.11) (-0.76,4.83) (-0.48,5.13) (-0.09,5.01) (0.22,5.13) (0.32,5.73) (0.55,6.11) (0.76,5.92)}
{% L2(x)
\addplot[color=darkred,very thick,samples=50,domain=-1.2:-0.76] 
{4.028925620*x^2+10.28305785*x+10.31801653}
node [pos=0.5,right] {$L_2(x)$};
}
{(-1.2,3.78) (-0.98,4.11) (-0.76,4.83)}

Определим первую и вторую производную функции $y(x)$ 
в точке $x_1=-0.98$:
\begin{gather*}
y^{\prime}(-0.98)\approx L_2^{\prime}(-0.98)=2.386363635\\
y^{\prime\prime}(-0.98)\approx L_2^{\prime\prime}(-0.98)=8.057851240
\end{gather*}
%
% x2
\item
Аппроксимация функции $y(x)$ в узлах $\{x_{1},x_{2},x_{3}\}$
полиномом Лагранжа второго порядка $L_2(x)$, 
используя данные таблицы \ref{tab:Num_Diff}:
\begin{gather*}
\begin{matrix}
L_2(x)&=&\dfrac{(x-(-0.76))(x-(-0.48))}{(-0.98-(-0.76))(-0.98-(-0.48))}\cdot4.11&+\\[1em]
&+&\dfrac{(x-(-0.98))(x-(-0.48))}{(-0.76-(-0.98))(-0.76-(-0.48))}\cdot4.83&+\\[1em]
&+&\dfrac{(x-(-0.98))(x-(-0.76))}{(-0.48-(-0.98))(-0.48-(-0.76))}\cdot5.13
\end{matrix}
\end{gather*}
Проводя элементарные алгебраические преобразования полином Лагранжа 
в пределах отрезка $[x_1,x_3]$ имеет вид:
\begin{gather*}
L_2(x)=-4.402597390\cdot x^2-4.387792189\cdot x+4.038218187
\end{gather*}
% График функций
\FigNumDiff
{(-1.2,3.78) (-0.09,5.01) (0.22,5.13) (0.32,5.73) (0.55,6.11) (0.76,5.92)}
{% L2(x)
\addplot[color=darkred, very thick, samples=50, domain=-0.98:-0.48] 
{-4.402597390*x^2-4.387792189*x+4.038218187}
node [pos=0.9,above left] {$L_2(x)$};
}
{(-0.98,4.11) (-0.76,4.83) (-0.48,5.13)}

Определим первую и вторую производную функции $y(x)$ 
в точке $x_2=-0.76$:
\begin{gather*}
y^{\prime}(-0.76)\approx L_2^{\prime}(-0.76)=2.304155844\\
y^{\prime\prime}(-0.76)\approx L_2^{\prime\prime}(-0.76)=-8.805194780
\end{gather*}
%
% x3
\item
Аппроксимация функции $y(x)$ в узлах $\{x_{2},x_{3},x_{4}\}$
полиномом Лагранжа второго порядка $L_2(x)$, 
используя данные таблицы \ref{tab:Num_Diff}:
\begin{gather*}
\begin{matrix}
L_2(x)&=&\dfrac{(x-(-0.48))(x-(-0.09))}{(-0.76-(-0.48))(-0.76-(-0.09))}\cdot4.83&+\\[1em]
&+&\dfrac{(x-(-0.76))(x-(-0.09))}{(-0.48-(-0.76))(-0.48-(-0.09))}\cdot5.13&+\\[1em]
&+&\dfrac{(x-(-0.76))(x-(-0.48))}{(-0.09-(-0.76))(-0.09-(-0.48))}\cdot5.01
\end{matrix}
\end{gather*}
Проводя элементарные алгебраические преобразования полином Лагранжа 
в пределах отрезка $[x_2,x_4]$ имеет вид:
\begin{gather*}
L_2(x)=-2.058389370\cdot x^2-1.480974249\cdot x+4.893385272
\end{gather*}
% График функций
\FigNumDiff
{(-1.2,3.78) (-0.98,4.11) (-0.76,4.83) (-0.48,5.13) (-0.09,5.01) (0.22,5.13) (0.32,5.73) (0.55,6.11) (0.76,5.92)}
{% L2(x)
\addplot[color=darkred,very thick, samples=50, domain=-0.76:-0.09] 
{-2.058389370*x^2-1.480974249*x+4.893385272}
node [pos=0.8,above] {$L_2(x)$};
}
{(-0.76,4.83) (-0.48,5.13) (-0.09,5.01)}

Определим первую и вторую производную функции $y(x)$
в точке $x_3=-0.48$:
\begin{gather*}
y^{\prime}(-0.48)\approx L_2^{\prime}(-0.48)=0.495079546\\
y^{\prime\prime}(-0.48)\approx L_2^{\prime\prime}(-0.48)=-4.116778740
\end{gather*}
%
%x4
\item
Апроксимацию функции $y(x)$ в узлах $\{x_{3},x_{4},x_{5}\}$
полиномом Лагранжа второго порядка $L_2(x)$, 
используя данные таблицы \ref{tab:Num_Diff}:
\begin{gather*}
\begin{matrix}
L_2(x)&=&
\dfrac{(x-(-0.09))(x-0.22)}{(-0.48-(-0.09))(-0.48-0.22)}\cdot5.13&+\\[1em]
&+&\dfrac{(x-(-0.48))(x-0.22)}{(-0.09-(-0.48))(-0.09-0.22)}\cdot5.01&+\\[1em]
&+&\dfrac{(x-(-0.48))(x-(-0.09))}{(0.22-(-0.48))(0.22-(-0.09))}\cdot5.13
\end{matrix}
\end{gather*}
Проводя элементарные алгебраические преобразования полином Лагранжа 
в пределах отрезка $[x_3,x_5]$ имеет вид:
\begin{gather*}
L_2(x)=0.9925558300\cdot x^2+0.2580645177\cdot x+5.025186105
\end{gather*}
% График функций
\FigNumDiff
{(-1.2,3.78) (-0.98,4.11) (-0.76,4.83) (0.32,5.73) (0.55,6.11) (0.76,5.92)}
{% L2(x)
\addplot[color=darkred,very thick,samples=50,domain=-0.48:0.22] 
{0.9925558300*x^2+0.2580645177*x+5.025186105}
node [pos=0.55,below] {$L_2(x)$};
}
{(-0.48,5.13) (-0.09,5.01) (0.22,5.13)}

Определим первую и вторую производную функции $y(x)$ 
в точке $x_4=-0.09$:
\begin{gather*}
y^{\prime}(-0.09)\approx L_2^{\prime}(-0.09)=0.0794044683\\
y^{\prime\prime}(-0.09)\approx L_2^{\prime\prime}(-0.09)=1.985111660
\end{gather*}
%
%x5
\item
Аппроксимация функции $y(x)$ в узлах $\{x_{4},x_{5},x_{6}\}$
полиномом Лагранжа второго порядка $L_2(x)$, 
используя данные таблицы \ref{tab:Num_Diff}:
\begin{gather*}
\begin{matrix}
L_2(x)&=&
\dfrac{(x-0.22)(x-0.32)}{(-0.09-0.22)(-0.09-0.32)}\cdot5.01&+\\[1em]
&+&\dfrac{(x-(-0.09))(x-0.32)}{(0.22-(-0.09))(0.22-0.32)}\cdot5.13&+\\[1em]
&+&\dfrac{(x-(-0.09))(x-0.22)}{(0.32-(-0.09))(0.32-0.22)}\cdot5.73
\end{matrix}
\end{gather*}
Проводя элементарные алгебраические преобразования полином Лагранжа 
в пределах отрезка $[x_4,x_6]$ имеет вид:
\begin{gather*}
L_2(x)=13.69000778\cdot x^2-1.392604236\cdot x+4.773776556
\end{gather*}
% *******************************
% График функций
\FigNumDiff
{(-1.2,3.78) (-0.98,4.11) (-0.76,4.83) (-0.48,5.13) (0.55,6.11) (0.76,5.92)}
{% L2(x)
\addplot[color=darkred,very thick,samples=50,domain=-0.09:0.32] 
{13.69000778*x^2-1.392604236*x+4.773776556}
node[pos=0.2,below right] {$L_2(x)$};
}
{(-0.09,5.01) (0.22,5.13) (0.32,5.73)}

Определим первую и вторую производную функции $y(x)$ 
в точке $x_5=0.22$:
\begin{gather*}
y^{\prime}(0.22)\approx L_2^{\prime}(0.22)=4.630999187\\
y^{\prime\prime}(0.22)\approx L_2^{\prime\prime}(0.22)=27.38001556
\end{gather*}
%x6
\item
Аппроксимация функции $y(x)$ в узлах $\{x_{5},x_{6},x_{7}\}$
полиномом Лагранжа второго порядка $L_2(x)$, 
используя данные таблицы \ref{tab:Num_Diff}:
\begin{gather*}
\begin{matrix}
L_2(x)&=&
\dfrac{(x-0.32)(x-0.55)}{(0.22-0.32)(0.22-0.55)}\cdot5.13&+\\[1em]
&+&\dfrac{(x-0.22)(x-0.55)}{(0.32-0.22)(0.32-0.55)}\cdot5.73&+\\[1em]
&+&\dfrac{(x-0.22)(x-0.32)}{(0.55-0.22)(0.55-0.32)}\cdot6.11
\end{matrix}
\end{gather*}
Проводя элементарные алгебраические преобразования полином Лагранжа 
в пределах отрезка $[x_5,x_7]$ имеет вид:
\begin{gather*}
L_2(x)=-13.17523062\cdot x^2+13.11462456\cdot x+2.882463758
\end{gather*}
% *******************************
%	График функций
%
\begin{figure}[H]\centering
\begin{tikzpicture}
\begin{axis}[
xlabel = {$x$},		% подпись оси x
ylabel = {$f(x)$},	% подпись оси y
xmin=-1.5, xmax=1,
ymin=3.5, ymax=6.5,
xtick style={thick, black},
ytick style={thick, black},
grid=major,		
major grid style={color=black!20, dashed, thin},
]
\addplot[only marks, mark=*, mark size=4pt, mark options={fill=orange, draw=black, solid}] coordinates 
{(-1.2,3.78) (-0.98,4.11) (-0.76,4.83) (-0.48,5.13) (-0.09,5.01) (0.22,5.13) (0.32,5.73) (0.55,6.11) (0.76,5.92)};
\addplot[color=orange, very thick, samples=50, domain=0.22:0.55] {
-13.17523062*x^2+13.11462456*x+2.882463758
};
\draw[color=orange] (axis cs: 0.55,5.5) node {$L_2(x)$};
\end{axis}
\end{tikzpicture}
\end{figure}
% *******************************
Определим первую и вторую производную функции $y(x)$ в точке $x_6=0.32$:
\begin{gather*}
y^{\prime}(0.32)\approx L_2^{\prime}(0.32)=4.682476963\\
y^{\prime\prime}(0.32)\approx L_2^{\prime\prime}(0.32)=-26.35046124
\end{gather*}
% x7
\item
Апроксимацию функции $y(x)$ в узлах $\{x_{6},x_{7},x_{8}\}$
полиномом Лагранжа второго порядка $L_2(x)$, 
используя данные таблицы \ref{tab:Num_Diff}:
\begin{gather*}
\begin{matrix}
L_2(x)&=&
\dfrac{(x-0.55)(x-0.76)}{(0.32-0.55)(0.32-0.76)}\cdot5.73&+\\[1em]
&+&\dfrac{(x-0.32)(x-0.76)}{(0.55-0.32)(0.55-0.76)}\cdot6.11&+\\[1em]
&+&\dfrac{(x-0.32)(x-0.55)}{(0.76-0.32)(0.76-0.55)}\cdot5.92
\end{matrix}
\end{gather*}
Проводя элементарные алгебраические преобразования полином Лагранжа 
в пределах отрезка $[x_6,x_8]$ имеет вид:
\begin{gather*}
L_2(x)=-5.811217790\cdot x^2+6.707933391\cdot x+4.178530017
\end{gather*}
% *******************************
%	График функций
%
\begin{figure}[H]\centering
\begin{tikzpicture}
\begin{axis}[
xlabel = {$x$},		% подпись оси x
ylabel = {$f(x)$},	% подпись оси y
xmin=-1.5, xmax=1,
ymin=3.5, ymax=6.5,
xtick style={thick, black},
ytick style={thick, black},
grid=major,		
major grid style={color=black!20, dashed, thin},
]
\addplot[only marks, mark=*, mark size=4pt, mark options={fill=orange, draw=black, solid}] coordinates 
{(-1.2,3.78) (-0.98,4.11) (-0.76,4.83) (-0.48,5.13) (-0.09,5.01) (0.22,5.13) (0.32,5.73) (0.55,6.11) (0.76,5.92)};
\addplot[color=orange, very thick, samples=50, domain=0.32:0.76] {
-5.811217790*x^2+6.707933391*x+4.178530017
};
\draw[color=orange] (axis cs: 0.65,5.65) node {$L_2(x)$};
\end{axis}
\end{tikzpicture}
\end{figure}
% *******************************
Определим первую и вторую производную функции $y(x)$ в точке $x_7=0.55$:
\begin{gather*}
y^{\prime}(0.55)\approx L_2^{\prime}(0.55)=0.315593822\\
y^{\prime\prime}(0.55)\approx L_2^{\prime\prime}(0.55)=-11.62243558
\end{gather*}
\item
Таким образом, определены значения первой $y^{\prime}(x_i)$ и
второй $y^{\prime\prime}(x_i)$ производной функции $y(x)$ 
в каждом внутреннем узле сетки $\{x_i\}$:
\begin{center}
\begin{tabular}{ l *{7}{l}}
\toprule
$x$&-0,98&-0,76&-0,48&-0,09&0,22&0,32&0,55\\
\midrule
$f^{\prime}(x)$&2,39&2,30&0,50&0,08&4,63&4,68&0,32\\
\midrule
$f^{\prime\prime}(x)$&8,06&-8,81&-4,12&1,99&27,38&-26,35&-11,62\\
\bottomrule
\end{tabular} 
\end{center}
% *******************************
%	График функций
%
\begin{center}
\begin{tikzpicture}
\begin{axis}[
xlabel = {$x$},		% подпись оси x
ylabel = {$\textcolor{blue}{y^{\prime}(x)}$},	% подпись оси y
xmin=-1.5, xmax=1,
ymin=-0.5, ymax=5,
xtick style={thick, black},
ytick style={thick, black},
grid=major,		
major grid style={color=black!20, dashed, thin},
]
\addplot[blue,mark=*, mark size=4pt, mark options={fill=blue!50, draw=black, solid}] coordinates 
{(-0.98,2.39) (-0.76,2.30) (-0.48,0.50) (-0.09,0.08) (0.22,4.63) (0.32,4.68) (0.55,0.32)};
\end{axis}
\end{tikzpicture}
\begin{tikzpicture}
\begin{axis}[
xlabel = {$x$},		% подпись оси x
ylabel = {$\textcolor{red}{y^{\prime\prime}(x)}$},	% подпись оси y
xmin=-1.5, xmax=1,
ymin=-30, ymax=30,
xtick style={thick, black},
ytick style={thick, black},
grid=major,		
major grid style={color=black!20, dashed, thin},
]
\addplot[red,mark=*, mark size=4pt, mark options={fill=red!75, draw=black, solid}] coordinates 
{(-0.98,8.06) (-0.76,-8.81) (-0.48,-4.12) (-0.09,1.99) (0.22,27.38) (0.32,-26.35) (0.55,-11.62)};
\end{axis}
\end{tikzpicture}
\end{center}

\end{enumerate}

%\end{document}
% Численное интегрирование
%\newpage
%
%	Численное интегрирование
%
\section{Численное интегрирование}
Если функция $f(x)$ непрерывна на отрезке $x\in[a,b]$ и 
известна ее первообразная $F(x)$, то определенный интеграл 
от этой функции в пределах от $a$ до $b$ может быть вычислен 
по формуле Ньютона -- Лейбница:
\begin{gather*}
\int\limits_{a}^{b}f(x)dx=F(b)-F(a),
\end{gather*}
где $F^{\prime}(x)=f(x)$ -- 
первообразная подынтегральной функции $f(x)$.

Численное значение интеграла -- это площадь криволинейной 
трапеции, ограниченной линиями графика функции 
и осью абсцисс $Ox$ (выделенная область на рисунке \ref{fig:INT}).

\begin{figure}[H]\centering
\begin{tikzpicture}
\def\xa{-1.6}
\def\xb{2}
\begin{axis}[% оси координат
xlabel=$x$,ylabel=$f(x)$,
xtick={\xa,\xb},xticklabels={$a$,$b$},
ytick={0},yticklabels={0}]
% f(x)
\addplot[name path=A,ball darkblue,mark=none,
samples=50,domain=\xa:\xb]{0.5*(\x-2)*(\x-1)*(\x+1) + 1};
\path[name path=B] (axis cs: \xa,0) -- (axis cs: \xb,0);
\addplot[blue!15] fill between [of=A and B, soft clip={domain=\xa:\xb},];
\end{axis}
\end{tikzpicture}
\caption{Геометрический смысл определенного интеграла}
\label{fig:INT}
\end{figure}

Однако во многих случаях первообразная функция $F(x)$ 
не может быть найдена с помощью элементарных средств 
или является слишком сложной, поэтому 
вычисление определенного интеграла 
может быть затруднительным или даже практически невозможным. 

Кроме того, на практике подынтегральная функция $f(x)$ часто 
задается таблично и тогда само понятие первообразной теряет смысл. 
Аналогичные вопросы возникают при вычислении кратных  
интегралов. Поэтому важное значение имеют приближенные 
и в первую очередь численные методы вычисления 
определенных интегралов. 

\emph{Задача численного интегрирования} функции заключается 
в вычислении значения определенного интеграла на основании ряда  
значений подынтегральной функции $f(x)$.

Обычный прием численного вычисления интеграла состоит 
в том, что данную функцию $f(x)$ на рассматриваемом отрезке 
$x\in[a, b]$ заменяют интерполирующей или аппроксимирующей  
функцией $\varphi(x)$ простого вида (например, полиномом), 
а затем приближенно полагают:
\begin{gather*}
\int\limits_{a}^{b}f(x)dx\approx \int\limits_{a}^{b}\varphi(x)dx
\end{gather*}

Далее рассматриваются способы приближенного вычисления 
определенных интегралов вида:
\begin{gather*}
I=\int\limits_{a}^{b}\varphi(x)dx,
\end{gather*}
основанные на замене интеграла конечной суммой:
\begin{gather*}
I\approx\sum\limits_{i=0}^{n}c_{i}\cdot\varphi(x_i),
\end{gather*}
где $c_{i}$ -- числовые коэффициенты квадратурной формулы; 
$x_i$ -- узлы квадратурной формулы, т.е. точки отрезка 
$[a, b], (i= 0,1,\cdots,n)$.

На основании свойств определенных интегралов, 
$I$ можно представить в виде суммы интегралов
по частичным отрезкам:
\begin{gather*}
\int\limits_{a}^{b}f(x)dx=
\sum\limits_{i=1}^{n}\int\limits_{x_{i-1}}^{x_i}f(x)dx
\end{gather*}

Поэтому, для построения формулы численного интегрирования 
на всем отрезке $[a,b]$ достаточно построить квадратурную формулу 
на частичном отрезке $[x_{i-1},x_i]$ для интеграла:
\begin{gather*}
S_i=\int\limits_{x_{i-1}}^{x_i}f(x)dx
\end{gather*}

%
% Формула прямоугольников
%
\subsection{Формула прямоугольников}
В методе прямоугольников на частичном отрезке 
подынтегральная функция заменяется полиномом нулевой степени,
то есть константу:
\begin{gather*}
f(x)\approx L_0(x)=\const
\end{gather*}

С геометрической точки зрения, в методе прямоугольников
площадь криволинейной трапеции (интеграл от функции) на
частичном отрезке заменяется площадью прямоугольника,
ширина которого будет определяться расстоянием между 
соответствующими соседними узлами интегрирования,
а высота -- значением подынтегральной функции в этих узлах.

В зависимости от выбора узла сетки $\{x_i\}$ для аппроксимации 
подынтегральной функции $f(x)$ на частичном отрезке 
$[x_{i-1},x_{i}]$ различают левую и правую формулы прямоугольников:
если в качестве значения аппроксимирующего полинома
выбирается значение подынтегральной функции 
на левом конце отрезка $L_0\approx f(x_{i-1})=y_{i-1}$
(рисунок \ref{fig:IntRect}), то справедлива 
левая формула прямоугольников:
\begin{gather*}
S^{-}_i\approx\int\limits_{x_{i-1}}^{x_i}L_0(x)dx=
y_{i-1}\cdot(x_{i}-x_{i-1}),
\end{gather*}
а если значение аппроксимирующего полинома
соответствует значению подынтегральной функции 
на правом конце частичного отрезка $L_0\approx f(x_{i})=y_{i}$
(рисунок \ref{fig:IntRect}), то справедлива 
правая формула прямоугольников :
\begin{gather*}
S^{+}_i\approx\int\limits_{x_{i-1}}^{x_i}L_0(x)dx=
y_{i}\cdot(x_{i}-x_{i-1}),
\end{gather*}
%
% Функция построения графика
%
\newcommand\FigInt[3]{
\begin{tikzpicture}[baseline]
\begin{axis}[% оси координат
xlabel=$x$,
name=GRAPH,
ylabel=$f(x)$,
%xmin=-1,xmax=1.5,
ymin=0,%ymax=2.25,
xtick={#1,#2},xticklabels={$x_{i-1}$,$x_i$},
ytick={0,cubic(#1),cubic(#2)},yticklabels={0,$y_{i-1}$,$y_i$},
width=7.5cm,% ширина графика
% определение функции
declare function={
cubic(\x)=0.5*(\x-2)*(\x-1)*(\x+1) + 1;
% полином Лагранжа L1
lagrange(\x) = (\x-#2)/(#1-#2)*cubic(#1) + (\x-#1)/(#2-#1)*cubic(#2);
% промежуточная точка
xc = #1 + (1-0.4)*(#2-#1)/2;
% полином Лагранжа L2
Lagrange(\x) = 
(\x-xc)*(\x-#2)/(xc-#1)/(#2-#1)*cubic(#1) +
(\x-#1)*(#2-\x)/(xc-#1)/(#2-xc)*cubic(xc) +
(\x-#1)*(\x-xc)/(#2-#1)/(#2-xc)*cubic(#2);
},
]
% f(x)
\addplot[name path=F,ball darkblue,fill=blue!10,
mark indices={1,50},samples=50,domain=#1:#2]{cubic(x)}
\closedcycle;
% Ln(x)
\addplot[draw=none,thin,opacity=0.25,fill=red!30,
domain=#1:#2,samples=50] #3 \closedcycle;
\end{axis}
% дополнительно
%\draw[thick,red] ($(GRAPH.south)-(0,0em)$) node {a};
\draw[thick,red] ($(current bounding box.south)-(0,1em)$) node {a)};
\end{tikzpicture}
}
%
% График прямоугольников
%
\begin{figure}[H]\centering
\FigInt{-0.7}{2}{ {cubic(-0.7)}
node[pos=0.7,color=darkred,opacity=1,above] {$L_0(x)$}
}
%
\hskip 10pt
%
\FigInt{-0.7}{2}{ {cubic(2)}
node[pos=0.8,color=darkred,opacity=1,above] {$L_0(x)$}
}\\
%\linebreak
a)\hspace{6cm}b)
\caption{График подынтегральной функции $f(x)$
и аппроксимирующего полинома $L_0(x)$ на частичном отрезке
для формулы прямоугольников}
\label{fig:IntRect}
\end{figure}

%
% Формула трапеций
%
\subsection{Формула трапеций}
Квадратурная \emph{формула трапеций} является следствием замены 
на частичном отрезке подынтегральной функции
интерполяционным полиномом первой степени $f(x)\approx L_1(x)$,
построенным по множеству узлов сетки $\{x_{i-1}, x_i\}$:
\begin{gather*}
L_1(x)=\dfrac{x-x_i}{x_{i-1}-x_i}\cdot y_{i-1} + \dfrac{x-x_{i-1}}{x_i-x_{i-1}}\cdot y_i.
\end{gather*}

Интегрирование интерполяционного полинома Лагранжа 
на частичном отрезке определяет формулу трапеций:
\begin{gather*}
S_i\approx\int\limits_{x_{i-1}}^{x_i}L_1(x)dx=
\dfrac{y_{i}+y_{i-1}}{2}\cdot(x_{i}-x_{i-1})
\end{gather*}
% график
\begin{figure}[H]\centering
\FigInt{-0.7}{2}{ {lagrange(x)} 
node[pos=0.8,color=darkred,opacity=1,above] {$L_1(x)$}
}
\caption{График подынтегральной функции $f(x)$
и аппроксимирующего полинома $L_1(x)$ на частичном отрезке
для формулы трапеций}
\label{fig:IntTrapez}
\end{figure}
%
% Формула Симпсона
%
\subsection{Формула Симпсона}
На частичном отрезке $[x_{i-1},x_{i}]$ квадратурная 
\emph{формула Симпсона} является следствием 
аппроксимации подынтегральной функции 
$f(x)$ интерполяционным полиномом Лагранжа 
второй степени $f(x)\approx L_2(x)$, который построен
по узлам сетки $\{x_{i-1}, x_{i-1/2}, x_{i}\}$:
\begin{gather*}
\begin{matrix}
L_{2}(x)&=&\dfrac
{ (x-x_{i-1/2})\cdot(x-x_{i}) }
{ (x_{i-1/2}-x_{i-1})\cdot(x_{i}-x_{i-1}) } \cdot y_{i-1}& + \\
\\
&+&\dfrac
{ (x-x_{i-1})\cdot(x_{i}-x) }
{ (x_{i-1/2}-x_{i-1})\cdot(x_{i}-x_{i-1/2}) } \cdot y_{i-1/2}& + \\
\\
&+&\dfrac
{ (x-x_{i-1})\cdot(x-x_{i-1/2}) }
{ (x_{i}-x_{i-1})\cdot(x_{i}-x_{i-1/2}) } \cdot y_{i}
\end{matrix},
\end{gather*}
где $x_{i-1/2}$ -- узел вспомогательной сетки,
расположенный между узлами основной сетки
$x_{i-1}<x_{i-1/2}<x_{i}$

Выражение для полинома Лагранжа в каноническом виде:
\begin{gather*}
L_{2}(x)=c_0 + c_1\cdot x + c_2\cdot x^2,
\end{gather*}
где $c_0$, $c_1$, $c_2$ -- коэффициенты при 
соответствующих степенях $x$ интерполяционного полинома 
Лагранжа $L_2(x)$ в пределах частичного отрезка 
$[x_{i-1}, x_{i+1}]$.

Интегрирование интерполяционного полинома Лагранжа $L_2(x)$ 
на частичном отрезке $x\in[x_{i-1}, x_{i+1}]$ определяет формулу Симпсона:
\begin{gather*}
S_i\approx\int\limits_{x_{i-1}}^{x_{i+1}}L_2(x)dx=
c_0\cdot(x_{i+1}-x_{i-1})+
c_1\cdot\dfrac{x_{i+1}^2-x_{i-1}^2}{2}+
c_2\cdot\dfrac{x_{i+1}^3-x_{i-1}^3}{3}.
\end{gather*}

% график
\begin{figure}[H]\centering
\FigInt{-0.7}{2}{ {Lagrange(x)} 
node[pos=0.7,color=darkred,opacity=1,above right] {$L_2(x)$}
}
\caption{График подынтегральной функции $f(x)$ и
аппроксимирующего полинома $L_2(x)$ на частичном отрезке
для формулы Симпсона}
\label{fig:IntSimpson}
\end{figure}

%***********************************
%
%	Численное интегрирования функции заданной таблично
%
%***********************************
\subsection{Численное интегрирования функции заданной таблично}
На множестве узлов сетки $\{x_i\}$ определены 
значения некоторой функции $\{y_i\}=f(x_i)$:
\begin{table}[H]
\vspace{-0.5\baselineskip}
\caption{Таблично заданная функциональная зависимость}
\begin{tabular*}{\textwidth}{%
l@{\extracolsep{\fill}}*{5}{r}p{0.25cm}}
\toprule
$i$&$0$&$1$&$2$&$3$&$4$\\
\midmidrule
$x_i$&$-3.31$&$0.31$&$1.32$&$2.47$&$3.50$\\
\addlinespace% дополнительный пробел
$y_i$&$2.45$&$4.03$&$-3.61$&$4.50$&$3.10$\\
\bottomrule
\end{tabular*}
\end{table}

\begin{enumerate}
% Стиль графиков
\pgfplotsset{%width=8cm,
xmin=-4,xmax=4,xtick={-4,-2,0,2,4},
ymin=-6,ymax=6,ytick={-6,-3,0,3,6},
}
\item
Построим график функции $f(x)$ заданной таблично.
% *******************************
%	График функций
%
\begin{figure}[H]\centering
\begin{tikzpicture}
\begin{axis}
\addplot[name path=A,ball darkblue,smooth] coordinates 
{(-3.31,2.45) (0.31,4.03) (1.32,-3.61) (2.47,4.50) (3.50, 3.1)};
\end{axis}
\end{tikzpicture}
\end{figure}
% *******************************
\item
Воспользуемся левой и правой формулами прямоугольников
для нахождения  численного значения интеграла 
от функции $f(x)$, заданной таблично на отрезке $x\in[x_0,x_4]$.
Для этого разобьем весь отрезок интегрирования 
на частичные отрезки: 
\begin{gather*}
[x_0,x_4]=[x_0,x_1] \cup [x_1,x_2] \cup [x_2,x_3] \cup [x_3,x_4]
\end{gather*}
%	График функций
\begin{figure}[H]\centering
% левые прямоугольники
\begin{tikzpicture}[baseline]
\begin{axis}
\addplot[name path=A,const plot mark left,ball darkblue] coordinates 
{(-3.31,2.45) (0.31,4.03) (1.32,-3.61) (2.47,4.50) (3.50, 3.1)};
%\path [name path=B] (\pgfkeysvalueof{/pgfplots/xmin},0) -- (\pgfkeysvalueof{/pgfplots/xmax},0);
\path[name path=B] (axis cs: -3.31,0) -- (axis cs: 3.5,0);
\addplot[blue!15] fill between [of=A and B, soft clip={domain=-3.31:3.5}];
\end{axis}
% правые прямоугольники
\end{tikzpicture}
\begin{tikzpicture}[baseline]
\begin{axis}
\addplot[name path=A,const plot mark right,ball darkblue] coordinates 
{(-3.31,2.45) (0.31,4.03) (1.32,-3.61) (2.47,4.50) (3.50, 3.1)};
\path[name path=B] (axis cs: -3.31,0) -- (axis cs: 3.5,0);
\addplot[blue!15] fill between [of=A and B, soft clip={domain=-3.31:3.5}];
\end{axis}
\end{tikzpicture}
\caption{Использование квадратурных формул 
левых и правых прямоугольников}
\end{figure}

На каждом частичном отрезке квадратурная формула 
является следствием замены подынтегральной функции 
$f(x)$ интерполяционным полиномом 
нулевой степени $f(x)\approx L_0(x)=\const$, 
построенным но узлам $\{x_{i-1}, x_{i}\}$.

По методу прямоугольников, определим значение 
интеграла на каждом частичном отрезке
(левые прямоугольники):
\begin{gather*}
\begin{array}{lclllll}
S^{-}_1&=&y_0\cdot(x_1-x_0)&=&2.45\cdot(0.31-(-3.31))&\approx&8.87\\
S^{-}_2&=&y_1\cdot(x_2-x_1)&=&4.03\cdot(1.32-0.31)&\approx&4.07\\
S^{-}_3&=&y_2\cdot(x_3-x_2)&=&-3.61\cdot(2.47-1.32)&\approx&-4.15\\
S^{-}_4&=&y_3\cdot(x_4-x_3)&=&4.5\cdot(3.50-2.47)&\approx&4.64\\
\end{array}
\end{gather*}

(правые прямоугольники):
\begin{gather*}
\begin{array}{*7l}
S^{+}_1&=&y_1\cdot(x_1-x_0)&=&4.03\cdot(0.31-(-3.31))&\approx&14.59\\
S^{+}_2&=&y_2\cdot(x_2-x_1)&=&-3.61\cdot(1.32-0.31)&\approx&-3.65\\
S^{+}_3&=&y_3\cdot(x_3-x_2)&=&4.50\cdot(2.47-1.32)&\approx&5.18\\
S^{+}_4&=&y_4\cdot(x_4-x_3)&=&3.10\cdot(3.50-2.47)&\approx&3.19\\
\end{array}
\end{gather*}

Значение интегралов $I^{-}$ и $I^{+}$ на всем отрезке 
интегрирования $[x_0,x_4]$: 
\begin{gather*}
\begin{array}{*7l}
I^{-}&=&S^{-}_1+S^{-}_2+S^{-}_3+S^{-}_4&=&
8.87+4.07-4.15+4.64&=&13.43\\
I^{+}&=&S^{+}_1+S^{+}_2+S^{+}_3+S^{+}_4&=&
14.59-3.65+5.18+3.19&=&19.31
\end{array}
\end{gather*}
 
\item
Рассмотрим \emph{метод трапеций} для нахождения 
численного значения интеграла от функции $f(x)$,
заданной таблично на отрезке $x\in[x_0,x_4]$.
Разобьем весь отрезок интегрирования на частичные отрезки: 
\begin{gather*}
[x_0,x_4]=[x_0,x_1] \cup [x_1,x_2] \cup [x_2,x_3] \cup [x_3,x_4]
\end{gather*}

На каждом частичном отрезке квадратурная формула 
является следствием замены подынтегральной функции 
$f(x)$ интерполяционным полиномом Лагранжа 
первой степени $f(x)\approx L_1(x)$, 
построенным но узлам $\{x_{i-1}, x_{i}\}$, 
т.е. прямой соединяющей два соседних узла.

% *******************************
%	График функций
%
\begin{figure}[H]\centering
\begin{tikzpicture}
\begin{axis}
\addplot[name path=A,
thick,draw=darkblue,mark=ball,mark size=3pt,
mark options={ball color=darkblue!50,thin,draw=darkblue}
] coordinates 
{(-3.31,2.45) (0.31,4.03) (1.32,-3.61) (2.47,4.50) (3.50, 3.1)};
%\path [name path=B] (\pgfkeysvalueof{/pgfplots/xmin},0) -- (\pgfkeysvalueof{/pgfplots/xmax},0);
\path[name path=B] (axis cs: -3.31,0) -- (axis cs: 3.5,0);
\addplot[blue!15] fill between [of=A and B, soft clip={domain=-3.31:3.5},];
\end{axis}
\end{tikzpicture}
\caption{Использование квадратурных формул трапеций}
\end{figure}
% *******************************

По методу трапеций, определим значение интеграла на каждом частичном отрезке:
\begin{gather*}
%\renewcommand*{\arraystretch}{2}
\begin{array}{*7l}
S_1&=&\dfrac{y_1+y_0}{2}\cdot(x_1-x_0)&=&
\dfrac{4.03+2.45}{2}\cdot(0.31-(-3.31))&\approx&11.73\\[1em]
S_2&=&\dfrac{y_2+y_1}{2}\cdot(x_2-x_1)&=&
\dfrac{-3.61+4.03}{2}\cdot(1.32-0.31)&\approx&0.21\\[1em]
S_3&=&\dfrac{y_3+y_2}{2}\cdot(x_3-x_2)&=&
\dfrac{4.50-3.61}{2}\cdot(2.47-1.32)&\approx&0.51\\[1em]
S_4&=&\dfrac{y_4+y_3}{2}\cdot(x_4-x_3)&=&
\dfrac{3.10+4.50}{2}\cdot(3.50-2.47)&\approx&3.91
\end{array}
\end{gather*}

Определим интеграл $I$ на всем отрезке интегрирования 
$[x_0,x_4]$, воспользовавшись свойством аддитивности 
интеграла:
\begin{gather*}
I=S_1+S_2+S_3+S_4=11.73+0.21+0.51+3.91=16.37.
\end{gather*}

\item
Рассмотрим \emph{метод Симпсона} для нахождения 
численное значение интеграла от функции $f(x)$,
заданной таблично на отрезке $x\in[x_0,x_4]$.

Разделим всё множество узлов сетки $\{x_i\}$, в которых
известны значения функции $\{y_i\}$, на основные 
и вспомогательные узлы:
\begin{table}[H]
\vspace{-0.5\baselineskip}
\caption{Таблично заданная функциональная зависимость}
\begin{tabular*}{\textwidth}{%
l@{\extracolsep{\fill}}*{5}{r}p{0.25cm}}
\toprule
$i$&$0$&$1-1/2$&$1$&$1+1/2$&$2$\\
\midmidrule
$x_i$&$-3.31$&$0.31$&$1.32$&$2.47$&$3.50$\\
\addlinespace% дополнительный пробел
$y_i$&$2.45$&$4.03$&$-3.61$&$4.50$&$3.10$\\
\bottomrule
\end{tabular*}
\end{table}
Разобьем весь отрезок интегрирования на частичные отрезки:
\begin{gather*}
[x_0,x_2]=[x_0,x_1] \cup [x_1,x_2].
\end{gather*}

В пределах первого частичного отрезка $[x_0,x_1]$
построим интерполяционный полином Лагранжа $L_2(x)$
по узлам сетки $x_0=-3.31$, $x_{1-1/2}=0.31$, $x_1=1.32$:
\begin{gather*}
\begin{matrix}
L_2(x)&=&\dfrac{(x-0.31)(x-1.32)}{((-3.31-0.31)(-3.31-1.32)}\cdot2.45&+\\[1em]
&+&\dfrac{(x-(-3.31))(x-1.32)}{(0.31-(-3.31))(0.31-1.32)}\cdot4.03&+\\[1em]
&+&\dfrac{(x-(-3.31))(x-0.31)}{(1.32-(-3.31))(1.32-0.31)}\cdot(-3.61)&\\
\end{matrix}
\end{gather*}

После алгебраических преобразований запишем 
интерполяционный полином в каноническом виде:
\begin{gather*}
L_2(x)=5.66-4.74\cdot x-1.73\cdot x^2
\end{gather*}

Определим интеграл от интерполяционного полинома 
$L_2(x)$ на первом частичном отрезке:
\begin{gather*}
I_1=\int\limits_{x_0}^{x_1} L_2(x)dx=
\int\limits_{-3.31}^{1.32}
\left(5.66-4.74\cdot x-1.73\cdot x^2\right)dx=25.88
\end{gather*}

% *******************************
%	График функций
\begin{center}
\begin{tikzpicture}
\begin{axis}[ymax=10]
\addplot[only marks,ball darkblue] coordinates 
{(-3.31,2.45) (0.31,4.03) (1.32,-3.61) (2.47,4.50) (3.50, 3.1)};
%\path [name path=B] (\pgfkeysvalueof{/pgfplots/xmin},0) -- (\pgfkeysvalueof{/pgfplots/xmax},0);
\addplot[name path=A,ball darkblue,mark=none,domain=-3.31:1.32]
{-1.73*x^2-4.74*x+5.66} node[pos=0.4,right] {$L_2(x)$};
\path[name path=B] (axis cs: -3.31,0) -- (axis cs: 1.32,0);
\addplot[darkblue!15] fill between [of=A and B, soft clip={domain=-3.31:1.32}];
\end{axis}
\end{tikzpicture}
\end{center}
% *******************************

В пределах второго частичного отрезка $[x_1,x_2]$
построим интерполяционный полином Лагранжа $L_2(x)$ 
по узлам сетки $x_1=1.32$, $x_{1+1/2}=2.47$, $x_2=3.50$:
\begin{gather*}
\begin{matrix}
L_2(x)&=&\dfrac{(x-2.47)(x-3.50)}{(1.32-2.47)(1.32-3.50)}\cdot(-3.61)&+\\[1em]
&+&\dfrac{(x-1.32)(x-3.50)}{(2.47-1.32)(2.47-3.50)}\cdot4.50&+\\[1em]
&+&\dfrac{(x-1.32)(x-2.47)}{(3.50-1.32)(3.50-2.47)}\cdot3.10\\
\end{matrix}
\end{gather*}

После тривиальных алгебраических преобразований:
\begin{gather*}
L_2(x)=-25.56+21.76\cdot x-3.87\cdot x^2
\end{gather*}

Определим интеграл от интерполяционного полинома 
$L_2(x)$ на втором частичном отрезке:
\begin{gather*}
I_2=\int\limits_{x_1}^{x_2} L_2(x)dx=
\int\limits_{1.32}^{3.50} \left(-25.56+21.76\cdot x-3.87\cdot x^2\right)dx=6.13
\end{gather*}

% *******************************
%	График функций
\begin{center}
\begin{tikzpicture}
\begin{axis}[ymax=10]
% данные
\addplot[only marks,ball darkblue] coordinates 
{(-3.31,2.45) (0.31,4.03) (1.32,-3.61) (2.47,4.50) (3.50, 3.1)};
% Ox
\path [name path=Ox] (axis cs: -3.31,0) -- (axis cs: 3.50,0);
% первый отрезок
\addplot[name path=A,thick,color=darkblue,domain=-3.31:1.32]
{-1.73*x^2-4.74*x+5.66} node[pos=0.4,right] {$L_2(x)$};
\addplot[darkblue!15] fill between [of=A and Ox, 
soft clip={domain=-3.31:1.32}];
% второй отрезок
\addplot[name path=C,thick,color=darkblue,domain=1.32:3.50]
{-3.87*x^2+21.76*x-25.56} node[pos=0.8,above] {$L_2(x)$};
\addplot[darkblue!15] fill between [of=C and Ox, 
soft clip={domain=1.32:3.50}];
\end{axis}
\end{tikzpicture}
\end{center}
% *******************************

Определим интеграл всем отрезке $[x_0,x_2]$ 
воспользовавшись свойством аддитивности интеграла:
\begin{gather*}
I=I_1+I_2=25.88+6.13=32.01
\end{gather*}

\item
Сравнивая численные значения определенного интеграла 
рассчитанные по методам прямоугольников, трапеций и Симпсона,
можно сделать вывод о том, что рассчитанные значения
различаются.
\begin{table}[H]
\vspace{-0.5\baselineskip}
\caption{Численные значения интегралов}
\begin{tabular*}{\textwidth}{%
l@{\extracolsep{\fill}}lp{3cm}}
\toprule
Метод интегрирования&Значение интеграла\\
\midmidrule
Левых прямоугольников&$13.43$\\
Правых прямоугольников&$19.31$\\
Трапеций&$16.37$\\
Симпсона&$32.01$\\
%\addlinespace% дополнительный пробел
\bottomrule
\end{tabular*}
\end{table}
Значение определенного интеграла от функции заданной таблично,
рассчитанное по методу Симпсона является наибольшим, а 
значение рассчитанное по методу левых прямоугольников 
-- наименьшее.
\end{enumerate}

% Нелинейные уравнения
%
%	Нелинейные уравнения
%
\newpage
\section{Решение нелинейных уравнений}
Пусть задана функция $f(x)$ действительного переменного и 
необходимо найти корни уравнения или, что то же самое, нули функции $f(x)$:
\begin{gather}\label{eq:NLE:f(x)=0}
f(x)=0.
\end{gather}

На примере алгебраического многочлена известно, что нули $f(x)$ могут быть 
как действительными, так и комплексными числами.
Поэтому \textit{более точная} постановка задачи состоит в нахождении корней уравнения,
расположенных в заданной области комплексной плоскости. 
Можно рассматривать также задачу о нахождении действительных корней уравнения,
которые расположены в пределах заданного отрезка $x\in[a,b]$.
% *******************************
%	График функций
%
\begin{figure}[H]\centering
\begin{tikzpicture}
\begin{axis}
[
xlabel={\empty},		% подпись оси x
ylabel={\empty},	% подпись оси y
xmin=-1.5, xmax=2.75, xtick={-1,0,1,1.5,2}, xticklabels={$x_1$,$\mathbf{a}$,$x_2$,$\mathbf{b}$,$x_3$},
ymin=-2, ymax=3, ytick={0}, yticklabels={$f(x)=0$},
]
\addplot[color=black,domain=-1.5:2.75, samples=2] {0};
\addplot[thick,color=darkred,domain=-1.25:2.5, samples=100] {(x-1)*(x+1)*(x-2)};
\addplot[only marks,ball darkred] coordinates {(-1,0) (1,0) (2,0)};
\draw[red] (axis cs:0,2.2) node[right] {$f(x)$};
\end{axis}
\end{tikzpicture}
\caption{График функции $y=f(x)$}
\label{fig:NLE:f(x)=0}
\end{figure}
На рисунке \eqref{fig:NLE:f(x)=0} представлены 
$x_1$, $x_2$ и $x_3$ -- действительные корни уравнения
\eqref{eq:NLE:f(x)=0}, т.е. $f(x_1)=0, f(x_2)=0, f(x_3)=0$

Задача нахождения корней уравнения $f(x)=0$ обычно решается в два этапа:
\begin{enumerate}
\item
На первом этапе изучается расположение корней (в общем случае на комплексной плоскости)
и проводится их разделение, т. е. \emph{выделяются области} в комплексной плоскости, 
\emph{содержащие только один корень}.
Кроме того, изучается вопрос о кратности корней.
Тем самым находятся некоторые начальные приближения для корней уравнения. 
\item
На втором этапе, \emph{используя заданное начальное приближение},
строится итерационный процесс, позволяющий \emph{уточнить значение отыскиваемого корня}.
\end{enumerate}

\begin{tcolorbox}
Следует отметить, что не существует каких-то общих регулярных приемов решения задачи 
о расположении корней произвольной функции $f(x)$.
\end{tcolorbox} 

Численные методы решения нелинейных уравнений являются,
как правило, итерационными методами, которые предполагают 
задание достаточно близких к искомому решению начальных данных.

%
%	Итервальный метод поиска корня уравнения
%
\emptyline
\subsection{Итервальный метод}
Итервальный метод поиска корня уравнения $f(x)=0$ 
состоит следующем:
\begin{enumerate}%[label=(\arabic*)]
\item
Область поиска корня $[a,b]$ разбивается на заранее заданное количество интервалов $N$:
\begin{gather*}
[a,b]=[a,x_1] \cup [x_1,x_2] \cup \cdots \cup [x_i,x_{i+1}] \cup \cdots [x_{N},b]
\end{gather*}
\item
Вычисляется таблица значений функции $\{f(x_i)\}$ на границах этих интервалов $\{x_i\}$.
\item
Проводится последовательный перебор таблицы значений функции $\{f(x_i)\}$
$(i=1,2,\cdots,N)$.
\item
Если при некотором $i$ значения функции $f(x_i)$ и $f(x_{i+1})$ имеют разные знаки,
то это означает, что на интервале $x\in(x_i, x_{i+1})$ 
имеет по крайней мере один действительный корень уравнения $f(x)=0$.
\item
В качестве новой, более узкой $\left(|x_i, x_{i+1}|<|a,b|\right)$, 
области поиска выбирается отрезок $(x_i, x_{i+1})$, т.е. полагают
\begin{gather*}
x_{i}=a,\quad x_{i+1}=b
\end{gather*}
и с помощью аналогичной процедуры (1) процесс происка корня уравнения $f(x)=0$ 
повторяют до тех пор пока, область поиска не станет меньше заранее заданной величины
$\epsilon$ (погрешности поиска корня уравнения):
\begin{gather*}
(b-a)<\epsilon
\end{gather*}
\end{enumerate}

% *******************************
%	График функций
%
\begin{figure}[H]\centering
\begin{tikzpicture}
\begin{axis}
[
%axis on top,
xlabel={\empty},	% подпись оси x
ylabel={\empty},	% подпись оси y
xmin=-0.5, xmax=2.15, xtick={0,0.75,1.25,1.75}, xticklabels={a,$x_i$,$x_{i+1}$,b},
ymin=-2, ymax=3, ytick={0}, yticklabels={$f(x)=0$},
]
\addplot[gray,thin,domain=-0.5:2.15, samples=2] {0};
\addplot[name path=A,color=black,domain=-0.35:1.95,samples=100] {(x-1)*(x+1)*(x-2)};
\addplot[only marks,ball darkred] coordinates {(0,2) (0.25,1.640625) (0.5,1.125) (0.75,0.546875)};
\draw[darkred] (axis cs:0.75,0.55) node[above right] {$f(x_i)>0$};
\addplot[only marks,ball darkblue] coordinates {(1.25,-0.421875) (1.5,-0.625) (1.75,-0.515625)};
\draw[darkblue] (axis cs:1.2,-0.5) node[below left] {$f(x_{i+1})<0$};
%
\path [name path=B] (axis cs: 0,-2) -- (axis cs: 1.75,-2);
\addplot[gray!15] fill between [of=A and B, soft clip={domain=0:1.75}];
\addplot[lime!20] fill between [of=A and B, soft clip={domain=0.75:1.25},];
\end{axis}
\end{tikzpicture}
\caption{Иллюстрация итервального метода поиска корня уравнения}
%$[a,b]$ -- начальная область поиска корня уравнения $f(x)=0$;\\
%$[x_i,x_{i+1}]$ -- интервал, содержащий корень уравнения.
\end{figure}

%
%	Метод бисекции
%
\subsection{Метод бисекции}
Метод бисекции основан на теореме \emph{Больцано-Коши}
(теорема о промежуточном значении):
если непрерывная функция $f(x)$, определённая 
на вещественном интервале $[a,b]$,
принимает два различных значения $f(a)\ne f(b)$, 
тогда существует такое $c\in[a,b]$,
что эта функция в этой точке принимает промежуточное значение 
$f(a) \leqslant f(c) \leqslant f(b)$.

Следствие теоремы Больцано-Коши (теорема о нуле 
непрерывной функции): 
если функция $f(x)$ непрерывна на некотором отрезке $[a,b]$ 
и на концах этого отрезка принимает значения $f(a)$ и $f(b)$ 
противоположных знаков, то существует точка $x_0$, 
в которой значение функции равно нулю $f(x_0)=0$.

Если непрерывная функция \emph{строго монотонна}
на отрезке $[a,b]$, т.е. для любого $\forall x\in[a,b]$ выполняется условие 
$f^\prime(x)>0$ либо $f^\prime(x)<0$, то в соответствие 
со следствие теоремы Больцано-Коши в пределах отрезка 
$[a,b]$ сущетсвует \textit{единственный} корень уравнения $f(x)=0$.

\alert{Метод бисекции} (деления пополам) является регулярным способом поиска 
действительного корня уравнения $f(x)=0$, однако для реализации этого метода
необходимо \textit{правильно выбрать область поиска}, т.е. начальный отрезок $[a,b]$ 
на концах которого функция $f(x)$ принимает значения разных знаков 
$f(a)\cdot f(b)<0$ и в пределах этого отрезка строго монотонна 
$f^{\prime}(x)<0$ либо $f^{\prime}(x)>0$.

Алгоритм метода деления отрезка пополам (метод бисекции):
\begin{enumerate}%[label=(\arabic*)]
\item
Область поиска корня уравнения отрезок $[a,b]$ делится пополам:
\begin{gather*}
c=\dfrac{a+b}{2}
\end{gather*}
\item
Вычисляется значение функции в середине отрезка $f(c)$.
\item
Проводится сравнение знаков функции в середине отрезка $f(c)$
и, например, на левом конце отрезка $f(a)$:
\begin{enumerate}
\item
если $f(a)\cdot f(c)<0$, функция $f(x)$ на концах отрезка $[a,c]$
принимает значения разных знаков, следовательно, искомый корень уравнения
$f(x)=0$ находится внутри отрезка $[a,c]$,
поэтому правый конец отрезка ``переносится`` в его середину.
\item
если $f(a)\cdot f(c)>0$, функция $f(x)$ на концах отрезка $[a,c]$
принимает значения одного знака, следовательно, искомый корень уравнения
$f(x)=0$ находится внутри отрезка $[c,b]$,
поэтому левый конец отрезка ``переносится`` в его середину.
\end{enumerate}
\begin{gather*}
\mathrm{sign}\left(f(a)\cdot f(c)\right)=
\begin{cases}<0,&b=c\\>0,&a=c\end{cases}
\end{gather*}
Таким образом, область поиска корня уравнения $f(x)=0$ ``сужается наполовину``.
\item
Процесс вычислений (1)--(3) повторятся до тех пока, 
длина вновь полученного интервала $[a,b]$ станет меньше 
заранее заданного числа $\epsilon$ 
(погрешности поиска корня уравнения):
\begin{gather*}
(b-a)<\epsilon
\end{gather*}
\end{enumerate}

В качестве корня уравнения $x_0$ приближенно принимаются середину
последнего полученного интервала $[a,b]$.

% *******************************
%	График функций
%
\begin{figure}[H]\centering
\begin{tikzpicture}
\begin{axis}
[
	xlabel = {\empty},		% подпись оси x
	ylabel = {\empty},	% подпись оси y
	xmin=-0.5, xmax=3.5, xtick={0,1.5,3}, xticklabels={a,c,b},
	ymin=-15, ymax=25, ytick={0}, yticklabels={$f(x)=0$},
]
\path[name path=B] (axis cs: 0,-15) -- (axis cs: 3,-15);
\addplot[gray,thin,domain=-0.5:3.5, samples=2] {0};
\addplot[name path=A,thick, color=black,domain=0:3, samples=100] {(x+2)*(x-2)*(x-5)};
\addplot[orange!15] fill between [of=A and B, soft clip={domain=1.5:3}];
% f(a)
\addplot[only marks,ball darkred] coordinates {(0,20)};
\draw[red] (axis cs:0,22) node[right] {$f(a)>0$};
%\addplot[thin,dashed,color=black] coordinates {(0,-15) (0,20)};
% f(c)
\addplot[only marks,ball darkred] coordinates {(1.5,6.125)};
\draw[red] (axis cs:1.5,8) node[right] {$f(c)>0$};
%\addplot[thin,dashed,color=black] coordinates {(1.5,-15) (1.5,6.125)};
% f(b)
\addplot[only marks,ball darkblue] coordinates {(3,-10)};
\draw[blue] (axis cs:3,-12) node[left] {$f(b)<0$};
%\addplot[thin,dashed,color=black] coordinates {(3,-15) (3,-10)};
%\addplot [olive!10] fill between [of=A and B, soft clip={domain=0:1.5},];
\end{axis}
\end{tikzpicture}
\caption{Иллюстрация метода бисекции (деления отрезка пополам):
$f(a)\cdot f(c)>0$, поэтому новая область поиска корня отрезок $[c,b]$}
\end{figure}

Следует отметить, что если условие строгой монотонности 
для функции $f(x)$ на отрезке $[a,b]$ не выполняется и на этом отрезке 
имеется несколько корней, то указанный итерационный процесс 
сойдется к одному из корней, 
но \emph{заранее неизвестно, к какому именно}.


% *** Алгоритм поиска минимума методом градиентного спуска
% https://tex.stackexchange.com/questions/370704/drawing-a-scheme
\begin{figure}[H]\centering
\begin{tikzpicture}[
font=\small,
start chain=going below,node distance=10mm,every join/.style=->,%
>={Straight Barb[angle=45:1.5mm 1]},shorten >=0.5pt,%
]
% соединитель
\node[rrectnode,on chain,join]{старт};
% данные
\node[datanode,on chain,join]
{$a, b, \epsilon$};
% grad(f)
\node[rectnode,on chain,join](c)
{$c=\dfrac{a+b}{2}$};
% f(xa)*f(c)
\node[ifthenelsenode,on chain,join](compare)
{$f(a) \cdot f(c)$};
% ошибка
\node[ifthenelsenode,on chain,join](error)
{$(b-a)\leqslant\epsilon$};
% нет
\begin{scope}[start branch=b,node distance=15mm]
\node[rectnode,on chain=going right,join](no)
{$\vect{x}_0=\vect{x}_1$};
\draw(error.east) node[above right]{нет};
\draw[->,thick] (no.north) |- (c.east);
\end{scope}
% выход
\draw (error.south) node[below left]{да};
\begin{scope}[node distance=12mm]
\node[rrectnode,on chain,join]{стоп};
\end{scope}
\end{tikzpicture}
% *******************************************
\caption{Блок-схема алгоритма нахождения минимума\linebreak 
функции $f(\vect{x})$ многих переменных 
методом градиентного спуска}
\label{fig:GD:scheme gradf(x)->min}
\end{figure}

%
%	Метод выделения корней
%
\emptyline
\subsection{Метод выделения корней}
Один из недостатков интервального метода и метода бисекции 
является сходимость итерационного процесса к заранее неизвестному
корню уравнения $f(x)=0$. 
Этот недостаток можно устранить удалением уже найденного корня. 

Если $x_1$ простой корень уравнения $f(x)=0$ и функция $f(x)$ 
непрерывна по Липшицу, то вспомогательная функция
\begin{gather*}
g(x)=\dfrac{f(x)}{(x-x_1)}
\end{gather*}
непрерывна, причем все нули функций $f(x)$ и $g(x)$ совпадают, 
за исключением $x_1$, так как $g(x_1)\ne0$.

Поэтому найденный корень $x_1$ можно удалить, т.е. перейти 
в процессе поиска корня уравнения $f(x)=0$ от функции $f(x)$ к функции $g(x)$.
Тогда процесс нахождения остальных корней уравнения 
сведется к нахождению корней $g(x)=0$.

Когда найден какой-нибудь новый корень $x_2$ уравнения $g(x)=0$, 
то этот корень тоже можно удалить, вводя новую вспомогательную функцию:
\begin{gather*}
\varphi(x)=\dfrac{g(x)}{(x-x_2)}=\dfrac{f(x)}{(x-x_1)(x-x_2)}
\end{gather*}

Таким образом, можно последовательно найти все корни исходного уравнения $f(x)=0$.

В любом методе поиска корней уравнения $f(x)$ 
окончательные итерации вблизи определяемого корня 
рекомендуется делать не по функциям типа 
$g(x)$, а по исходной функции $f(x)$. 
Последние итерации, вычисленные по функции $g(x)$, 
используются при этом в качестве нулевого приближения.

%
%	Численное решение нелинейного уравнения методом бисекции
%
\emptyline
\subsection{Численное решение нелинейного уравнения методом бисекции}
На отрезке $x\in[-3,5]$ задана непрерывная функция:
\begin{gather*}
f(x)=\tanh(x)\cdot(1+\cos(x))-\dfrac{1}{2}
\end{gather*}

С помощью метода бисекции найдем первый положительный корень\\
$x_1>0$ нелинейного уравнения $f(x)=0$.
% *******************************
%	График функций
%
\begin{figure}[H]\centering
\begin{tikzpicture}
\begin{axis}
[
%	xlabel = {$x$},		% подпись оси x
%	ylabel = {$f(x)$},	% подпись оси y
	xmin=-4, xmax=6, xtick={-3,-1,0,1,3,5},
	ymin=-2, ymax=1, ytick={-2,-1.5,-1,-0.5,0,0.5,1},
]
\addplot[gray,thin,domain=-4:6, samples=2] {0};
\addplot[name path=A,thick,domain=-3:5, samples=100] {tanh(x)*(1+cos(deg(x)))-0.5};
\path [name path=B] (axis cs: -4,-2) -- (axis cs: 6,-2);
\addplot [orange!15] fill between [of=A and B, soft clip={domain=1:3}];
\end{axis}
\end{tikzpicture}
\caption{График функции $f(x)=\tanh(x)\cdot(1+\cos(x))-\dfrac{1}{2}$}
\label{fig:NLE-01}
\end{figure}

На основе анализа графика функции $f(x)$ (рисунок \ref{fig:NLE-01}) 
выбираем область поиска первого положительного корня
уравнения $(x>0)$, на границах которой функция $f(x)$ 
принимает значения разных знаков $f(a)\cdot f(b)<0$ 
и в пределах этого области строго монотонна 
$f^{\prime}(x)<0$.

Таким требованиям удовлетворяет отрезок $x\in[1,3]$ (выдененная область на графике),
так как функция на отрезке монотонна $f^{\prime}(x)<0$ и на концах отрезка 
принимает значения разных знаков:
\begin{gather*}
\begin{array}{rcl}
f(1)&=&\tanh(1)\cdot(1+\cos(1))-\dfrac{1}{2}\approx0.67>0\\[1em]
f(3)&=&\tanh(3)\cdot(1+\cos(3))-\dfrac{1}{2}\approx-0.49<0
\end{array}
\end{gather*}

Зададим погрешность поиска корня уравнения 
$\varepsilon=0.1$ и используем метод бисекции 
для поиска первого положительного корня уравнения:
\begin{enumerate}
\item
Область поиска корня уравнения отрезок $[1,3]$ делится пополам
и вычисляется значение функции в середине отрезка $f(c)$:
\begin{gather*}
c=\dfrac{1+3}{2}=2\\
f(c)=f(2)=\tanh(2)\cdot(1+\cos(2))-\dfrac{1}{2}\approx0.06285>0
\end{gather*}
Проводится сравнение знаков функции в середине отрезка $f(c)$
и на левом конце отрезка $f(a)$:
\begin{gather*}
f(a)\cdot f(c)=f(1)\cdot f(2)=0.67\cdot0.06285=0.0423>0
\end{gather*}
Следовательно, искомый корень уравнения находится внутри 
отрезка $[2,3]$.
Проведем сравнение длины отрезка и погрешности поиска корня:
\begin{gather*}
|a,b|=b-a=3-2=1>\varepsilon=0.1
\end{gather*}
Так как длина отрезка больше погрешности поиска, 
то итерационный процесс продолжаем.

% *******************************
%	График функций
%
\begin{center}
\begin{tikzpicture}
\begin{axis}
[
%	xlabel = {$x$},		% подпись оси x
%	ylabel = {$f(x)$},	% подпись оси y
	xmin=0.5, xmax=3.5, xtick={1,2,3},
	ymin=-1, ymax=1, %ytick={0},
]
\addplot[gray,thin,domain=0.5:3.5, samples=2] {0};
\addplot[name path=A,thick,domain=1:3, samples=100] {tanh(x)*(1+cos(deg(x)))-0.5};
\path [name path=B] (axis cs: 1,-1) -- (axis cs: 3,-1);
\addplot [orange!20] fill between [of=A and B, soft clip={domain=2:3}];
% f(a)
\addplot[only marks,ball darkred]coordinates {(1,.6731)}
node[above] {$f(a)$};
% f(c)
\addplot[only marks,ball darkred]coordinates {(2,.628505523e-1)}
node[above right] {$f(c)$};
% f(b)
\addplot[only marks,ball darkblue]coordinates {(3,-.4900419862)}
node[above] {$f(b)$};
\end{axis}
\end{tikzpicture}
\end{center}
% *****************************************
\item
Область поиска корня уравнения отрезок $[2,3]$ делится пополам
и вычисляется значение функции в середине отрезка $f(c)$:
\begin{gather*}
c=\dfrac{2+3}{2}=2.5\\
f(c)=f(2.5)=\tanh(2.5)\cdot(1+\cos(2.5))-\dfrac{1}{2}\approx-0.3038<0
\end{gather*}
Проводится сравнение знаков функции в середине отрезка $f(c)$
и на левом конце отрезка $f(a)$:
\begin{gather*}
f(a)\cdot f(c)=f(2)\cdot f(2.5)=0.6285\cdot(-0.3038)=-0.019094<0
\end{gather*}
Следовательно, искомый корень уравнения находится внутри отрезка $[2,2.5]$.
Проведем сравнение длины отрезка и погрешности поиска корня:
\begin{gather*}
|a,b|=b-a=2.5-2=0.5>\varepsilon=0.1
\end{gather*}
Так как длина отрезка больше погрешности поиска, то итерационный процесс продолжаем.
% *******************************
%	График функций
%
\begin{center}
\begin{tikzpicture}
\begin{axis}
[
%	xlabel = {$x$},		% подпись оси x
%	ylabel = {$f(x)$},	% подпись оси y
	xmin=1.75, xmax=3.25, xtick={2,2.5,3},
	ymin=-0.6, ymax=0.3, %ytick={0},
]
\addplot[gray,thin,domain=1.75:3.25, samples=2] {0};
\addplot[name path=A,thick,domain=2:3, samples=100] {tanh(x)*(1+cos(deg(x)))-0.5};
\path [name path=B] (axis cs: 2,-0.6) -- (axis cs: 3,-0.6);
\addplot [orange!20] fill between [of=A and B, soft clip={domain=2:2.5}];
% f(a)
\addplot[only marks,ball darkred]coordinates {(2,.628505523e-1)}
node[above] {$f(a)$};
% f(c)
\addplot[only marks,ball darkblue]coordinates {(2.5,-.3038054478)}
node[above right] {$f(c)$};
% f(b)
\addplot[only marks,ball darkblue]coordinates {(3,-.4900419862)}
node[above] {$f(b)$};
\end{axis}
\end{tikzpicture}
\end{center}
% *****************************************
\item
Область поиска корня уравнения отрезок $[2,2.5]$ делится пополам
и вычисляется значение функции в середине отрезка $f(c)$:
\begin{gather*}
c=\dfrac{2+2.5}{2}=2.25\\
f(c)=f(2.25)=\tanh(2.25)\cdot(1+\cos(2.25))-\dfrac{1}{2}\approx-0.13634<0
\end{gather*}
Проводится сравнение знаков функции в середине отрезка $f(c)$
и на левом конце отрезка $f(a)$:
\begin{gather*}
f(a)\cdot f(c)=f(2)\cdot f(2.25)=0.6285\cdot(-0.13634)=-0.00857<0
\end{gather*}
Следовательно, искомый корень уравнения находится внутри отрезка $[2,2.25]$.
Проведем сравнение длины отрезка и погрешности поиска корня:
\begin{gather*}
|a,b|=b-a=2.25-2=0.25>\varepsilon=0.1
\end{gather*}
Длина отрезка меньше погрешности поиска, поэтому итерационный процесс продолжаем.

% *******************************
%	График функций
%
\begin{center}
\begin{tikzpicture}
\begin{axis}
[
	xmin=1.9, xmax=2.6, xtick={2,2.25,2.5},
	ymin=-0.4, ymax=0.2, %ytick={0},
]
\addplot[gray,thin,domain=1.9:2.6, samples=2] {0};
\addplot[name path=A,thick,domain=2:2.5, samples=100] {tanh(x)*(1+cos(deg(x)))-0.5};
\path [name path=B] (axis cs: 2,-0.4) -- (axis cs: 3,-0.4);
\addplot [orange!20] fill between [of=A and B, soft clip={domain=2:2.25}];
% f(a)
\addplot[only marks,ball darkred] coordinates {(2,.628505523e-1)}
node[above] {$f(a)$};
% f(c)
\addplot[only marks,ball darkblue] coordinates {(2.25,-.1363440929)}
node[above right] {$f(c)$};
% f(b)
\addplot[only marks,ball darkblue] coordinates {(2.5,-.3038054478)}
node[above right] {$f(b)$};
\end{axis}
\end{tikzpicture}
\end{center}
% *****************************************
\item
Область поиска корня уравнения отрезок $[2,2.25]$ делится пополам
и вычисляется значение функции в середине отрезка $f(c)$:
\begin{gather*}
c=\dfrac{2+2.25}{2}=2.125\\
f(c)=f(2.125)=\tanh(2.125)\cdot(1+\cos(2.125))-\dfrac{1}{2}\approx-0.03959<0
\end{gather*}
Проводится сравнение знаков функции в середине отрезка $f(c)$
и на левом конце отрезка $f(a)$:
\begin{gather*}
f(a)\cdot f(c)=f(2)\cdot f(2.125)=0.6285\cdot(-0.03959)=-0.002488<0
\end{gather*}
Следовательно, искомый корень уравнения находится внутри отрезка $[2,2.125]$.
Проведем сравнение длины отрезка и погрешности поиска корня:
\begin{gather*}
|a,b|=b-a=2.125-2=0.125>\varepsilon=0.1
\end{gather*}
Длина отрезка меньше погрешности поиска, поэтому итерационный процесс продолжаем.

% *******************************
%	График функций
%
\begin{center}
\begin{tikzpicture}
\begin{axis}
[
	xmin=1.95, xmax=2.3, xtick={2,2.125,2.25},
	ymin=-0.2, ymax=0.2, %ytick={0},
]
\addplot[gray,thin,domain=1.95:2.3, samples=2] {0};
\addplot[name path=A,domain=2:2.25, samples=100] {tanh(x)*(1+cos(deg(x)))-0.5};
\path [name path=B] (axis cs: 2,-0.2) -- (axis cs: 2.25,-0.2);
\addplot [orange!20] fill between [of=A and B, soft clip={domain=2:2.125}];
% f(a)
\addplot[only marks,ball darkred] coordinates {(2,.628505523e-1)}
node[above] {$f(a)$};
% f(c)
\addplot[only marks,ball darkblue] coordinates {(2.125,-.395911619e-1)}
node[right,yshift=1ex] {$f(c)$};
% f(b)
\addplot[only marks,ball darkblue] coordinates {(2.25,-.1363440929)}
node[above,xshift=1ex] {$f(b)$};
\end{axis}
\end{tikzpicture}
\end{center}
% *****************************************
\item
Область поиска корня уравнения отрезок $[2,2.125]$ делится пополам
и вычисляется значение функции в середине отрезка $f(c)$:
\begin{gather*}
c=\dfrac{2+2.125}{2}=2.0625\\
f(c)=f(2.0625)=\tanh(2.0625)\cdot(1+\cos(2.0625))-\dfrac{1}{2}\approx0.01108>0
\end{gather*}
Проводится сравнение знаков функции в середине отрезка $f(c)$
и на левом конце отрезка $f(a)$:
\begin{gather*}
f(a)\cdot f(c)=f(2)\cdot f(2.0625)=0.6285\cdot0.01108=0.0007>0
\end{gather*}
Следовательно, искомый корень уравнения находится внутри отрезка\linebreak 
$[2.0625,2.125]$.
Проведем сравнение длины отрезка и погрешности поиска корня:
\begin{gather*}
|a,b|=b-a=2.125-2.0625=0.0625<\varepsilon=0.1
\end{gather*}
Длина отрезка меньше погрешности поиска, поэтому итерационный процесс завершаем.

% *******************************
%	График функций
%
\begin{center}
\begin{tikzpicture}
\begin{axis}
[
	xmin=1.985, xmax=2.14, xtick={2,2.0625,2.125},
	ymin=-0.1, ymax=0.1, ytick={-0.1,-0.05,0,0.05,0.1},% yticklabels={-0.1,,0,,0.1},
]
\addplot[gray,thin,domain=1.95:2.14, samples=2] {0};
\addplot[name path=A,domain=2:2.125, samples=100] {tanh(x)*(1+cos(deg(x)))-0.5};
\path [name path=B] (axis cs: 2,-0.1) -- (axis cs: 2.125,-0.1);
\addplot [orange!20] fill between [of=A and B, soft clip={domain=2.0625:2.125}];
% f(a)
\addplot[only marks,ball darkred] coordinates {(2,.628505523e-1)}
node[above] {$f(a)$};
% f(c)
\addplot[only marks,ball darkred] coordinates {(2.0625,.110785242e-1)}
node[above,xshift=1ex] {$f(c)$};
% f(b)
\addplot[only marks,ball darkblue] coordinates {(2.125,-.395911619e-1)}
node[above,xshift=1ex] {$f(b)$};
\end{axis}
\end{tikzpicture}
\end{center}
% *****************************************
\end{enumerate}

В качестве первого положительного корня $x_1$ уравнения $f(x)=0$ 
приближенно выберем середину последнего полученного интервала 
$[a,b]$ и для контроля определим значение функции 
в точке $f(x_1)$ приближенного корня уравнения:
\begin{gather*}
x_1=\dfrac{a+b}{2}=\dfrac{2.0625+2.125}{2}=2.09375\\
f(x_1)=f(2.09375)=\tanh(2.09375)\cdot(1+\cos(2.09375))-\dfrac{1}{2}\approx-0.01442
\end{gather*}

% *******************************
%	График функций
%
\begin{figure}[H]\centering
\begin{tikzpicture}
\begin{axis}
[
	xmin=-4, xmax=6, xtick={-3,0,2.09375,5},% xticklabels={-3,-1,0,1,$x_1$,3,5},
	ymin=-2, ymax=1, ytick={-2,-1,0,1},% yticklabels={-2,-1.5,-1,-0.5,0,0.5,1},
]
\addplot[name path=A,domain=-3:5, samples=100] {tanh(x)*(1+cos(deg(x)))-0.5};
\addplot[only marks,ball darkred] coordinates {(2.09375,-.144150229e-1)}
node[above right,fill=white] {$f(x_1)=0$};
\addplot[darkred,dashed,thick] coordinates {(2.09375,-2) (2.09375,-.144150229e-1)};
\end{axis}
\end{tikzpicture}
\caption{График функции $f(x)$ (\alert{сплошная линия}) и 
первый положительный корень $x_1=2.09375$ (маркер) 
уравнения $f(x)=0$}
\end{figure}


% Поиск экстремума функций многих переменных
%\input{chapter/7 - Minimum}
% Аппроксимация
%\newpage
%
%	Аппроксимация функция
%
\section{Аппроксимация функция}
Задача о приближении функции ставится следующим образом:
данную функцию $f(x)$ необходимо заменить 
обобщенным полиномом $p_m(x)$ заданного порядка $m$ 
так, чтобы отклонение (в известном смысле) функции $f(x)$ 
от обобщенного полинома $p_m(x)$ на указанном множестве 
$\vec{x}=\{x\}$ было наименьшим. 
При этом полином $p_m(x)$ в общем случае 
называется аппроксимирующим.

Если множество $\vec{x}$ состоит из отдельных точек 
$x\in\{x_0, x_1, x_2, \dots x_n\}$ (узлов),
то приближение называется \textit{точечным}.
Если $\vec{x}$ есть отрезок $x_a<x<x_b$, 
то приближение называется \textit{интегральным}. 
Для практики важным является приближение функций 
алгебраическими и тригонометрическими полиномами.

\subsection{Точечное квадратичное аппроксимирование функций}
На практике часто бывает, что заданный порядок $m$ 
приближающего полинома $p_m(x)$ меньше числа 
узлов аппроксимации ${m<n}$, в которых 
известно значение функции $y_i=f(x_i)$ ($i=0,1,2, \cdots, n$).
В этом случае обычно используют точечный 
метод наименьших квадратов и
рассматривается полином степени $m$ вида:
\begin{gather*}
p_m(x)=c_0+c_1\cdot{x}+c_2\cdot{x^2}+\dots+c_m\cdot{x^m}=
\sum\limits_{j=0}^{m}c_j\cdot{x^j}.
\end{gather*}

В качестве меры отклонения $\norma{r}$ полинома $p_m(x)$ 
от известной функции $y(x)$ на множестве точек 
$\{x_0, x_1, x_2,\cdots,x_n\}$, как правило, принимается 
сумма квадратов отклонений полинома от этой функции 
на заданной системе точек:
\begin{gather*}
\norma{r}=\sum_{i=0}^{n}\left(p_m(x_i)-y_i\right)^2
\end{gather*}

Следует отметить, что мера отклонения полинома 
от известной функции есть функция многих переменных
$\norma{r}=g(c_0, c_1, \dots, c_m)$, т.е. коэффициентов полинома
$c_i$ ($i=0,1,\dots,m$), которые необходимо подобрать так, 
чтобы величина меры отклонения была наименьшей 
$\norma{r}\to{\min}$.
Полученный полином называется аппроксимирующим 
для данной функции, а процесс построения этого полинома -- 
точечной квадратичной аппроксимацией или 
точечным квадратичным аппроксимированием функции. 

Для решения задачи точечного квадратичного аппроксимирования,
т.е. определения числовых значений всех коэффициентов 
полинома $p_m(x)$, необходимо найти \emph{положения минимума 
функции} многих переменных $\norma{r}$.

Определим частные производные от величины суммы квадратов отклонений и 
воспользовавшись условием экстремума функции многих переменных, 
составим систему уравнений вида:
\begin{gather*}
\pdiff{\norma{r}}{c_0}=
\pdiff{\norma{r}}{c_1}=
\pdiff{\norma{r}}{c_2}=\cdots=
\pdiff{\norma{r}}{c_m}=0
\end{gather*}

Для определения неизвестных коэффициентов полинома
$c_0, c_1, c_2,\dots, c_m$ необходимо решить систему 
$m+1$ уравнений с $m+1$ неизвестными: 
\begin{gather*}
\renewcommand*{\arraystretch}{1.5}
\left\{\begin{array}{lclcl}
\pdiff{\norma{r}}{a_0}&=&2\cdot\sum\limits_{i=0}^{n}\left(c_0+c_1\cdot{x_i}+c_2\cdot{x_i^2}+\ldots+c_m\cdot{x_i^m} - y_i\right)\cdot1&=&0\\
\pdiff{\norma{r}}{a_1}&=&2\cdot\sum\limits_{i=0}^{n}\left(c_0+c_1\cdot{x_i}+c_2\cdot{x_i^2}+\ldots+c_m\cdot{x_i^m} - y_i\right)\cdot{x_i}&=&0\\
\pdiff{\norma{r}}{a_2}&=&2\cdot\sum\limits_{i=0}^{n}\left(c_0+c_1\cdot{x_i}+c_2\cdot{x_i^2}+\ldots+c_m\cdot{x_i^m} - y_i\right)\cdot{x_i^2}&=&0\\
\hdotsfor{1}&=&\hdotsfor{1}&=&0\\
\pdiff{\norma{r}}{a_m}&=&2\cdot\sum\limits_{i=0}^{n}\left(c_0+c_1\cdot{x_i}+c_2\cdot{x_i^2}+\ldots+c_m\cdot{x_i^m} - y_i\right)\cdot{x_i^m}&=&0\\
\end{array}\right.
\end{gather*}

Таким образом, задача точечной квадратичной аппроксимации 
функции сводится к решению системы линейных уравнений 
относительно неизвестных -- коэффициентов полинома 
$\{c_0, c_1, c_2,\dots, c_m\}$:
\begin{gather*}
\begin{matrix}
\mathbf{A}\cdot\vec{c}=\vec{b}
&\text{или}&
\begin{pmatrix}
a_{00}&a_{01}&\cdots&a_{0m}\\
a_{10}&a_{11}&\cdots&a_{1m}\\
\vdots&\vdots&\ddots&\vdots\\
a_{m0}&a_{m1}&\cdots&a_{mm}\\
\end{pmatrix}
\cdot
\begin{pmatrix}c_0\\c_1\\\vdots\\c_m\end{pmatrix}
=\begin{pmatrix}b_0\\b_1\\\vdots\\b_m\end{pmatrix}
\end{matrix},\end{gather*}
где $\mathbf{A}=\{a_{k\ell}\}$ и $\vec{b}=\{b_k\}$ 
-- квадратная матрица и вектор правых частей 
системы линейных уравнений, соответственно:
\begin{gather*}
a_{k\ell}=\sum\limits_{i=0}^n x_i^k\cdot x_i^\ell,
\quad
b_{k}=\sum\limits_{i=0}^n x_i^k\cdot y_i,
\quad k,\ell=0,1,2,\dots,m
\end{gather*}

Если среди узлов сетки $\{x_i\}$ 
нет совпадающих, а также степень полинома 
меньше чем число узлов аппроксимации $m<n$, 
то определитель системы не равен нулю $\det\mathbf{A}\ne0$.
Следовательно, эта система имеет единственное решение 
$\{\mathring{c}_0, \mathring{c}_1, \mathring{c}_2,\dots, \mathring{c}_m\}$,
а полином $p_m(x)$ с такими коэффициентами $\mathring{c}_i$ 
будет обладать минимальным квадратичным отклонением 
$\norma{r}_{\min}$. 

%
%	Аппроксимирования функций полиномом второй степени $p_2(x)$
%
\subsection{Аппроксимирования функций полиномом
второй степени $p_2(x)$}
Известна таблица данных некоторой функциональной зависимости 
$y(x)$:
\begin{table}[H]
\vspace{-0.5\baselineskip}
\caption{Таблично заданная функциональная зависимость
$y_i=f(x_i)$}
\begin{tabular*}{\textwidth}{%
l@{\extracolsep{\fill}}*{5}{r}p{0.25cm}}
\toprule
$i$&$0$&$1$&$2$&$3$&$4$\\
\midmidrule
$x_i$&$-0.76$&$-0.48$&$-0.09$&$0.22$&$0.55$\\
\addlinespace% дополнительный пробел
$y_i$&$5.15$&$4.39$&$4.10$&$5.71$&$5.30$\\
\bottomrule
\end{tabular*}
\end{table}

Необходимо аппроксимировать функцию $\{y_i\}$,
заданную таблично, алгебраическим полиномом 
второй степени $p_2(x)$:
\begin{gather*}
p_2(x)=c_0 + c_1\cdot x + c_2\cdot x^2
\end{gather*}

\begin{enumerate}
\item
Построим меру отклонения полинома $p_2(x)$ 
от таблично заданной функции $y_i=f(x_i)$
на множестве точек $\{x_0, x_1, x_2, x_3, x_4\}$:
\begin{gather*}
\norma{r}=\sum_{i=0}^{4}\left(c_0+c_1\cdot{x_i}+c_2\cdot{x_i^2}-y_i\right)^2,
\end{gather*}
где $y_i=f(x_i)$ -- значение функции в точке $x_i$.

\item
Запишем меру отклонения $\norma{r}$ в явном виде 
на основе данных из условия задачи:
\begin{gather*}
\begin{split}
\norma{r}=
&\left(c_0 + c_1\cdot(-0.76) + c_2\cdot(-0.76)^2 - 5.15 \right)^2+\\
+&\left(c_0 + c_1\cdot(-0.48) + c_2\cdot(-0.48)^2 - 4.39 \right)^2+\\
+&\left(c_0 + c_1\cdot(-0.09) + c_2\cdot(-0.09)^2 - 4.10 \right)^2+\\
+&\left(c_0 + c_1\cdot(0.22) + c_2\cdot(0.22)^2 - 5.71 \right)^2+\\
+&\left(c_0 + c_1\cdot(0.55) + c_2\cdot(0.55)^2 - 5.30 \right)^2
\end{split}
\end{gather*}

\item
Определим частную производную от меры отклонений $\norma{r}$ 
по аргументу $c_0$ и приравняем её нулю:
\begin{gather*}
\begin{split}
\pdiff{\norma{r}}{c_0}=
&2\cdot\left(a_0 + a_1\cdot(-0.76) + a_2\cdot(-0.76)^2 - 5.15 \right)\cdot 1+\\
&2\cdot\left(a_0 + a_1\cdot(-0.48) + a_2\cdot(-0.48)^2 - 4.39 \right)\cdot 1+\\
&2\cdot\left(a_0 + a_1\cdot(-0.09) + a_2\cdot(-0.09)^2 - 4.10 \right)\cdot 1+\\
&2\cdot\left(a_0 + a_1\cdot(0.22) + a_2\cdot(0.22)^2 - 5.71 \right)\cdot 1+\\
&2\cdot\left(a_0 + a_1\cdot(0.55) + a_2\cdot(0.55)^2 - 5.30 \right)\cdot 1=0
\end{split}
\end{gather*}

Коэффициенты первой строки матрицы $\mathbf{A}$
и первый элемент вектора $\vec{b}$:
\begin{gather*}
\begin{array}{lcl}
a_{00}&=&1+1+1+1+1=5\\
a_{01}&=&(-0.76) + (-0.48) + (-0.09) + (0.22) + (0.55) = -0.56\\
a_{02}&=&(-0.76)^2 + (-0.48)^2 + (-0.09)^2 + (0.22)^2 + (0.55)^2=1.18\\
%
b_0&=&5.15 + 4.39 + 4.10 + 5.71 + 5.30=24.65
\end{array}
\end{gather*}

\item
Определим частную производную от меры отклонений 
$\norma{r}$ по аргументу $c_1$ и приравняем её нулю:
\begin{gather*}
\begin{split}
\pdiff{S}{c_1}=
&2\cdot\left(c_0 + c_1\cdot(-0.76) + c_2\cdot(-0.76)^2 - 5.15 \right)\cdot(-0.76)+\\
&2\cdot\left(c_0 + c_1\cdot(-0.48) + c_2\cdot(-0.48)^2 - 4.39 \right)\cdot(-0.48)+\\
&2\cdot\left(c_0 + c_1\cdot(-0.09) + c_2\cdot(-0.09)^2 - 4.10 \right)\cdot(-0.09)+\\
&2\cdot\left(c_0 + c_1\cdot(0.22) + c_2\cdot(0.22)^2 - 5.71 \right)\cdot(0.22)+\\
&2\cdot\left(c_0 + c_1\cdot(0.55) + c_2\cdot(0.55)^2 - 5.30 \right)\cdot(0.55)=0
\end{split}
\end{gather*}

Коэффициенты второй строки матрицы $\mathbf{A}$
и второй элемент вектора $\vec{b}$:
\begin{gather*}
\begin{array}{lcl}
c_{10}&=&(-0.76) + (-0.48) + (-0.09) + (0.22) + (0.55) = -0.56\\
c_{11}&=&(-0.76)^2 + (-0.48)^2 + (-0.09)^2 + (0.22)^2 + (0.55)^2=1.18\\
c_{12}&=&(-0.76)^3 + (-0.48)^3 + (-0.09)^3 + (0.22)^3 + (0.55)^3=-0.38\\
%
b_1&=&5.15\cdot(-0.76)+4.39\cdot(-0.48)+4.10\cdot(-0.09)+\\
&&5.71\cdot(0.22)+5.30\cdot(0.55)=-2.24
\end{array}
\end{gather*}

\item
Определим частную производную от меры отклонений 
$\norma{r}$ по аргументу $c_2$ и приравняем её нулю:
\begin{gather*}
\begin{split}
\pdiff{\norma{r}}{c_2}=
&2\cdot\left(c_0 + c_1\cdot(-0.76) + c_2\cdot(-0.76)^2 - 5.15 \right)\cdot(-0.76)^2+\\
+&2\cdot\left(c_0 + c_1\cdot(-0.48) + c_2\cdot(-0.48)^2 - 4.39 \right)\cdot(-0.48)^2+\\
+&2\cdot\left(c_0 + c_1\cdot(-0.09) + c_2\cdot(-0.09)^2 - 4.10 \right)\cdot(-0.09)^2+\\
+&2\cdot\left(c_0 + c_1\cdot(0.22) + c_2\cdot(0.22)^2 - 5.71 \right)\cdot(0.22)^2+\\
+&2\cdot\left(c_0 + c_1\cdot(0.55) + c_2\cdot(0.55)^2 - 5.30 \right)\cdot(0.55)^2=0
\end{split}
\end{gather*}

Коэффициенты третьей строки матрицы $\mathbf{A}$
и третий элемент вектора $\vec{b}$:
\begin{gather*}
\begin{array}{lcl}
c_{20}&=&(-0.76)^2 + (-0.48)^2 + (-0.09)^2 + (0.22)^2 + (0.55)^2=1.18\\
c_{21}&=&(-0.76)^3 + (-0.48)^3 + (-0.09)^3 + (0.22)^3 + (0.55)^3=-0.38\\
c_{22}&=&(-0.76)^4 + (-0.48)^4 + (-0.09)^4 + (0.22)^4 + (0.55)^4=0.49\\
%
b_2&=&5.15\cdot(-0.76)^2 +4.39\cdot(-0.48)^2 +4.10\cdot(-0.09)^2+\\
&&5.71\cdot(0.22)^2 +5.30\cdot(0.55)^2=5.94
\end{array}
\end{gather*}

\item
Таким образом, для определения неизвестных коэффициентов $c_0,c_1,c_2$
аппроксимирующего полинома $p_2(x)$ необходимо решить 
систему линейных алгебраических уравнений:
\begin{gather*}
\left\{\begin{matrix}
&5\cdot c_0&-&0.56\cdot c_1&+&1.18\cdot c_2&=&24.65\\
-&0.56\cdot c_0&+&1.18\cdot c_1&-&0.38\cdot c_2&=&-2.24\\
&1.18\cdot c_0&-&0.38\cdot c_1&+&0.49\cdot c_2&=&5.94\\
\end{matrix}\right.
\end{gather*}

\item
Решение этой системы линейных уравнений можно найти методом Гаусса:
\begin{gather*}
\left\{\begin{array}{lcl}
c_0&=&4.66\\
c_1&=&0.80\\
c_2&=&1.52
\end{array}\right.
\end{gather*}

Таким образом, аппроксимирующий полином имеет вид:
\begin{gather*}
p_2(x)=4.66 + 0.80\cdot x + 1.52\cdot x^2
\end{gather*}

\item
На одном графике представим диаграмму рассеяния 
(разброса) данных функции заданной таблично $y_i=f(x_i)$
(маркеры) и результаты вычислений
аппроксимирующего алгебраического полинома 
второго порядка $p_2(x)$ (сплошная линия).
% *******************************
%	График функций
%
\begin{figure}[H]\centering
\begin{tikzpicture}
\begin{axis}[% оси координат
ylabel={$p_2(x)$},
xmin=-1, xmax=0.7, xtick={-0.8,-0.4,0,0.4},
ymin=3.8, ymax=6,
]
\addplot[PlotDarkBlue,only marks]
coordinates {(-0.76,5.15) (-0.48,4.39) (-0.09,4.10) (0.22,5.71) (0.55,5.30)};
\addplot[PlotDarkBlue,mark=none,domain=-0.9:0.6, samples=50] 
{4.66 + 0.80*x + 1.52*x^2} node[pos=0.5,below right] {$p_2(x)$};
\end{axis}
\end{tikzpicture}
\caption{График таблично заданной функции $y_i=f(x_i)$ (маркеры) 
и аппроксимирующего алгебраического полинома $p_2(x)$
(сплошная линия)}
\end{figure}

\end{enumerate}

\end{document}
Получилась система n+1 уравнений с таким же количеством неизвестных аj, причем линейная относительно этих переменных. Эта система называется системой нормальных уравнений. Из ее решения находятся параметры аj аппроксимирующей функции, обеспечивающие minR, т.е. наилучшее возможное квадратичное приближение. Зная коэффициенты, можно (если нужно) вычислить и величину R (например, для сравнения различных аппроксимирующих функций). Следует помнить, что при изменении даже одного значения исходных данных (или пары значений хi, уi, или одного из них) все коэффициенты изменят в общем случае свои значения, так как они полностью определяются исходными данными. Поэтому при повторении аппроксимации с несколько изменившимися данными (например, вследствие погрешностей измерения, помех, влияния неучтенных факторов и т.п.) получится другая аппроксимирующая функция, отличающаяся коэффициентами. Обратим внимание на то, что коэффициенты аj полинома находятся из решения системы уравнений, т.е. они связаны между собой. Это приводит к тому, что если какой-то коэффициент вследствие его малости захочется отбросить, придется пересчитывать заново оставшиеся. Можно рассчитать количественные оценки тесноты связи коэффициентов. Существует специальная теория планирования экспериментов, которая

позволяет обосновать и рассчитать значения хi, используемые для аппроксимации, чтобы получить заданные свойства коэффициентов (несвязанность, минимальная дисперсия коэффициентов и т.д.) или аппроксимирующей функции (равная точность описания реальной зависимости в различных направлениях, минимальная дисперсия предсказания значения функции и т.д.).

В случае постановки другой задачи — найти аппроксимирующую функцию, обеспечивающую погрешность не хуже заданной, — необходимо подбирать и структуру этой функции. Эта задача значительно сложнее предыдущей (найти параметры аппроксимирующей функции заданной структуры, обеспечивающей наилучшую возможную погрешность) и решается в основном путем перебора различных функций и сравнения получающихся мер близости. Для примера на рис. 3.7 приведены для визуального сравнения исходная и аппроксимирующие функции с различной степенью полинома, т.е. функции с различной структурой. Не следует забывать, что с повышением точности аппроксимации растет и сложность функции (при полиномиальных аппроксимирующих функциях), что делает ее менее удобной при использовании.

Пример 3.1. В ходе проведения эксперимента были получены данные, представленные в таблице 3.1. Необходимо способом наименьших квадратов подобрать для заданных значений x и y квадратичную функцию . Построить на одной координатной плоскости экспериментальные данные и аппроксимирующую функцию.

Исходными данными для решения задачи является таблица наблюдений – набор значений независимых переменных и соответствующие им значения функции отклика. Число строк (узлов) таблично заданной функции m называют объемом выборки.

Форма уравнения выбирается исследователем в соответствии с поведением аппроксимируемой функции в области изменения независимых переменных. Результатом же решения задачи аппроксимации являются оценки коэффициентов этого уравнения. Очевидно, что коэффициенты уравнения следует подбирать так, чтобы рассчитываемые по уравнению значения функции отклика максимально близко совпадали с заданными в исходной таблице наблюдений.


http://ru.bmstu.wiki/Аппроксимация_функций,_моделирующих_сигналы
Математические модели сигналов, детально и точно описывающие определенные физические объекты и процессы, могут быть очень сложными и мало пригодными для практического использования, как при математическом анализе физических данных, так и в прикладных задачах, основанных на математическом моделировании КПС. Кроме того, практическая регистрация сигналов выполняется, как правило, с определенной погрешностью или с определенным уровнем шумов, которые по своим значениям могут быть выше теоретической погрешности прогнозирования сигналов при расчетах по сложным, хотя и очень точным формулам. Не имеет большого смысла и проектирование систем обработки и анализа сигналов по высокоточным формулам, если повышение точности расчетов не дает ощутимого эффекта в повышении точности обработки данных. Во всех этих условиях возникает задача аппроксимации – представления произвольных сложных функций 
f
(
x
)
 простыми и удобными для практического использования функциями 
φ
(
x
)
 таким образом, чтобы отклонение 
φ
(
x
)
 от 
f
(
x
)
 в области ее задания было наименьшим по определенному критерию приближения. Функции 
φ
(
x
)
 получили название функций аппроксимации.

Математика очень часто оперирует со специальными математическими функциями решения дифференциальных уравнений и интегралов, которые не имеют аналитических выражений и представляются табличными числовыми значениями 
y
i
 для дискретных значений независимых переменных 
x
i
. Аналогичными таблицами 
{
y
i
,
x
i
}
 могут представляться и экспериментальные данные. Точки, в которых определены дискретные значения функций или данных, называются узловыми. Однако на практике могут понадобиться значения данных величин совсем в других точках, отличных от узловых, или с другим шагом дискретизации аргументов. Возникающая при этом задача вычисления значений функции в промежутках между узами называется задачей интерполяции, за пределами семейства узловых точек вперед или назад по переменным – задачей экстраполяции или прогнозирования. Решение этих задач также обычно выполняется с использованием аппроксимирующих функций.

Сглаживание статистических данных или аппроксимация данных с учетом их статистических параметров относится к задачам регрессии, и рассматриваются в следующей теме. Как правило, при регрессионном анализе усреднение данных производится методом наименьших квадратов (МНК).

Все вышеперечисленные задачи относятся к задачам приближения сигналов и функций и имеют многовековую историю, в процессе которой сформировались классические математические методы аппроксимации, интерполяции, экстраполяции и регрессии функций. В рамках настоящего курса мы не будем углубляться в строгую математическую теорию этих операций. Все современные математические системы (Mathcad, MATLAB, Maple и пр.) имеют в своем составе универсальный аппарат выполнения таких операций, дающий пользователю возможность реализации любых практических задач по обработке данных без отвлечения на теоретические подробности их исполнения. В качестве основной математической системы для примеров использована система Mathcad.
% Задача Коши для систем обыкновенных дифференциальных уравнений
%\pgfdeclarelayer{pre main}
\pgfdeclarelayer{background}
\pgfdeclarelayer{foreground}
\pgfsetlayers{background,pre main,main,foreground} 
%
%	Задача Коши для систем дифференциальных уравнений
%
\newpage
\section{Задача Коши для систем 
обыкновенных дифференциальных уравнений}
При рассмотрении физических явлений и процессов часто 
не удается найти непосредственную взаимосвязь между 
величинами, характеризующими эволюционный, 
т.е. изменяющийся во времени, процесс. 
Однако во многих случаях можно установить связь между 
искомыми характеристиками изучаемого явления (функциями) 
и скоростями их изменения относительно других переменных, 
т.е. найти уравнения, в которые входят производные от
неизвестных функций. Такие уравнения называют 
\emph{дифференциальными}.

Обыкновенными дифференциальными 
уравнениями можно описать задачи движения системы 
взаимодействующих материальных точек, химической кинетики, 
электрических цепей, сопротивления материалов 
(например, статический прогиб упругого стержня) и многие другие.

Кроме того, ряд важных задач для уравнений в частных производных 
сводится к задачам для обыкновенных дифференциальных 
уравнений. Например, многомерная задача допускающая разделение 
пространственных переменных (например, задачи 
на нахождение собственных колебаний упругих балок 
и мембран простейшей формы, определение спектра 
собственных значений энергии частицы 
в сферически-симметричном поле), 
или если решение многомерной задачи зависит только от 
некоторой комбинации переменных (автомодельные решения), 
то задача нахождения решения уравнений в частных производных 
сводится к задачам на собственные значения 
для обыкновенных дифференциальных уравнений.
Поэтому, методы решения обыкновенных дифференциальных 
уравнений занимает важное место среди прикладных 
задач физики, химии и техники.

Задача Коши обычно возникает при анализе процессов, 
определяемых дифференциальным законом эволюции и 
начальным состоянием (начальным условием) 
и для системы обыкновенных дифференциальных уравнений 
эта задача формулируется в виде системы уравнений:
\begin{gather}\label{eq:ODE_SYS}
\diff{\vect{u}}{t} = \vect{f}(t, \vect{u}),
\quad
\vect{u}(0)=\mathring{\vect{u}},
\end{gather}
где $\vect{u}(t)$ -- неизвестные функции, которые подлежат 
определению;
$\vect{f}(t, \vect{u})$ -- известные функции, 
зависящие от времени и неизвестных функций;
$\mathring{\vect{u}}$ -- \emph{начальные условия}, 
т.е. значения неизвестных функций в начальный момент времени 
$(t=0)$. 

Система обыкновенных дифференциальных 
уравнений первого порядка и начальные условия
\eqref{eq:ODE_SYS} в развернутом виде могут быть 
записаны как:
\begin{gather}\label{eq:ODE_IC}
\left\{\begin{matrix}
\diff{u_1}{t}&=&f_1(t, u_1,u_2,\cdots,u_n),&u_1(0)&=&\mathring u_1\\[1em]
\diff{u_2}{t}&=&f_2(t, u_1,u_2,\cdots,u_n),&u_2(0)&=&\mathring u_2\\[1em]
\hdotsfor{1}&=&\hdotsfor{1}&\hdotsfor{1}&=&\dots\\[1em]
\diff{u_n}{t}&=&f_n(t, u_1,u_2,\cdots,u_n),&u_n(0)&=&\mathring u_n
\end{matrix}\right.,
\end{gather}
где $n$ -- количество дифференциальных уравнений 
в системе \eqref{eq:ODE_SYS}.

Точное решение систем дифференциальных уравнений 
вида \eqref{eq:ODE_IC}, которые описывают 
многообразие прикладных задач, может быть 
получено лишь в исключительных случаях. 
Поэтому возникает необходимость 
\emph{приближенного решения} таких задач. 
В настоящее время создано и разработано 
значительное число приближенных методов решения 
дифференциальных уравнений, каждый из которых имеет 
свои преимущества и недостатки.

%
% Метод Эйлера
%
\emptyline
\subsection{Метод Эйлера решения задачи Коши}
Будем полагать, что решение задачи Коши \eqref{eq:ODE_IC} существует, 
единственно и обладает необходимыми свойствами гладкости.

Введем временную сетку, т.е. будем
рассматривать изменения неизвестных функций 
только в заданные моменты времени:
\begin{gather*}
\vect{t}=\{t_j\},\quad j=0,1,2,\dots,m,
\end{gather*}
где $j$ -- номер временного интервала;
$\Delta t_{j+1} = (t_{j+1}-t_j)$ -- шаг сетки, т.е. временной интервал 
между двумя последовательными моментами времени;
$m$ -- количество узлов временной сетки.

Основная идея метода Эйлера заключается в предположении, 
о том что неизвестные функции $\vect{u}(t)$ изменяются линейно
в интервале $[t_j,t_{j+1}]$ между двумя соседними узлами 
временной сетки и интерполяция неизвестных функций 
проводится полиномом первого порядка $\vect{L}_1(t)$:
\begin{gather*}
\vect{u}(t)\approx\vect{L}_1(t)=
\dfrac{t-t_{j+1}}{t_j-t_{j+1}}\cdot\vect{u}(t_j)+
\dfrac{t-t_j}{t_{j+1}-t_j}\cdot\vect{u}(t_{j+1}).
\end{gather*}

Производная от неизвестной функции приближенно 
аппроксимируется выражением вида:
\begin{gather}
\diff{\vect{u}}{t}\approx\vect{L}^{\prime}_1(t) =
\dfrac{\vect{u}(t_{j+1})-\vect{u}(t_j)}{t_{j+1}-t_j},
\end{gather}
где $t_{j+1}$ и $t_j$ -- два последовательных момента времени.

Тогда систему дифференциальных уравнений первого порядка 
\eqref{eq:ODE_SYS} приближенно можно записать в виде:
\begin{gather}\label{eq:ODE_NUM}
\dfrac{\vect{u}(t_{j+1})-\vect{u}(t_{j})}{\Delta t_{j+1}}\approx
\vect{f}\left(t_j, \vect{u}(t_{j})\right).
\end{gather}

Относительно неизвестных $\vect{u}(t_{j+1})$ 
это система линейных алгебраических уравнений и 
решение системы \eqref{eq:ODE_NUM} находится явным образом 
по рекуррентным формулам:
\begin{gather}\label{eq:ODE_EULER}
\vect{u}(t_{j+1})=\vect{u}(t_{j})+\Delta t_{j+1}\cdot\vect{f}\left(t_j, \vect{u}(t_{j})\right),
\quad
\vect{u}(0)=\mathring{\vect{u}}.
\end{gather}

Метод Эйлера является простейшим численным методом 
решения задачи Коши. Блок-схема алгоритма метода Эйлера
представлена на рисунке \ref{fig:ODE_IC:scheme Euler}. 
К недостаткам метода можно отнести малую точность и 
систематическое накопление ошибок.

Для простоты рассмотрим только одно дифференциальное уравнение 
с единственным начальным условием:
\begin{gather*}
y(t_{j+1})=y(t_{j})+\Delta t_{j+1}\cdot f\left(t_j, y(t_{j})\right),
\quad y(0)=\mathring y
\end{gather*}

На рисунке \ref{fig:EULER} представлена графическая иллюстрация 
метода Эйлера численного решения (маркеры) задачи Коши 
для одного дифференциального уравнения первого порядка
(сплошной линией представлено точное решение).
% *******************************
%	Ломаная Эйлера
%
\begin{figure}[H]\centering
\begin{tikzpicture}
\begin{axis}[% оси координат
xlabel=\empty,ylabel=\empty,
xmin=-0.5,xmax=8,xtick={0,1,4,7},xticklabels={$0$,$t_1$,$t_j$,$t_n$},
ymin=-0.25,ymax=3,ytick={0,1,2.08,2.59},yticklabels={$y(0)$,$y(t_1)$,$y(t_j)$,$y(t_n)$},
]
\addplot[name path=A,very thick,red,domain=0:7,samples=50]{ln(1+x)};
\addplot[name path=B,ball darkblue,thin]coordinates{(0,0)(1,1)(2,1.5)(3,1.83)(4,2.08)(5,2.28)(6,2.45)(7,2.59)};
\addplot[blue!5] fill between [of=A and B];
\end{axis}
%\draw[thick,red] ($(GRAPH.south)-(0,2em)$) node {a};
\end{tikzpicture}
\caption{Иллюстрация метода Эйлера:\linebreak
неизвестная кривая выделена красным цветом, 
а ее полигональная аппроксимация (ломаная Эйлера) -- синим}
\label{fig:EULER}
\end{figure}

%
% *** Алгоритм метода Эйлера
%
\begin{figure}[H]\centering
\begin{tikzpicture}[
font=\small,%
start chain=going below,%
node distance=10mm,%
>={Straight Barb[angle=45:1.5mm 1]},%
]
% начало
\node[beginendnode,on chain,join]{начало};
% начальное условие
\node[IOnode,on chain,join]
{$\vect{t}=\left\{0,t_1,t_2,\ldots,t_m\right\}$};
\node[IOnode,on chain,join]
{$\vect{u}(0)=\left\{u_1(0),u_2(0),\ldots,u_n(0)\right\}$};
% начало цикла
\node[rectnode,on chain,join]{$j=0$};
% интервал времени
\node[rectnode,on chain,join](loop)
{$\Delta{t}_{j+1}=t_{j+1}-t_j$};
% решение дифференциального уравнения
\node[rectnode,on chain,join]
{$\vect{u}(t_{j+1})=\vect{u}(t_j)+
\Delta t_{j+1}\cdot\vect{f}\left(t_j, \vect{u}(t_{j})\right)$};
% условие
\node[ifthenelsenode,on chain,join](compare){$j<m$};
\draw(compare.east) node[above right]{да};
\draw(compare.south) node[below right]{нет};
% инкремент
\begin{scope}[start branch=b,node distance=25mm]
\node[rectnode,on chain=going right,join](jloop){$j=j+1$};
\draw[->,thick](jloop.north) |- (loop.east);
\end{scope}
% решение
\begin{scope}[node distance=12mm]
\node[IOnode,on chain,join](output)
{$\vect{u}(\vect{t})=\{u_1(\vect{t}),u_2(\vect{t}),\ldots,u_n(\vect{t})\}$};
\end{scope}
% конец
\node[beginendnode,on chain,join]{конец};
\end{tikzpicture}
\caption{Блок-схема алгоритма численного решения
системы дифференциальных уравнений методом Эйлера}
\label{fig:ODE_IC:scheme Euler}
\end{figure}

%
%	Оценка погрешности решения задачи Коши
%
\subsection{Оценка погрешности решения задачи Коши}
Интегрирование системы дифференциальных уравнений 
\eqref{eq:ODE_SYS} по временной переменной $t$ 
с учетом начальных условий:
\begin{gather}\label{eq:ODE_SYS_INT}
\vect{v}(t)=\vect{u}(0)+\int\limits_{0}^{t}\vect{f}(\xi, \vect{v})\,\mathrm{d}\xi.
\end{gather}

Уравнение \eqref{eq:ODE_SYS_INT} является 
интегральным уравнением для неизвестной функции $\vect{v}(t)$,
а его решение эквивалентно решению задачи Коши 
\eqref{eq:ODE_SYS}, что можно проверить прямой подстановкой
\eqref{eq:ODE_SYS_INT} в \eqref{eq:ODE_SYS}.

На временной сетке $\{t_j\}$ интеграл в правой части равенства 
\eqref{eq:ODE_SYS_INT} приближенно вычисляется по 
\emph{формуле трапеций}:
\begin{gather}\label{eq:ODE_SYS_INT_TRAP}
\int\limits_{0}^{t_{j+1}}\vect{f}(\xi, \vect{v})\,\mathrm{d}\xi
\approx\sum\limits_{k=0}^{j}
\dfrac{
\vect{f}\left(t_{k+1},\vect{v}(t_{k+1})\right)+
\vect{f}\left(t_{k},\vect{v}(t_{k})\right)
}{2}\cdot(t_{k+1}-t_k).
\end{gather}

Воспользовавшись \eqref{eq:ODE_SYS_INT_TRAP}, 
выражение для решения интегрального уравнения 
\eqref{eq:ODE_SYS_INT} можно записать 
в рекуррентной форме:
\begin{gather}\label{eq:ODE_INT_EQ}
\vect{v}(t_{j+1})=\vect{v}(t_{j})+
\dfrac{
\vect{f}\left(t_{j+1},\vect{v}(t_{j+1})\right)+
\vect{f}\left(t_{j},\vect{v}(t_{j})\right)
}{2}\cdot\Delta t_{j+1}.
\end{gather}

Для определения приближенного значения решения 
интегрального уравнения \eqref{eq:ODE_INT_EQ}
могут быть использованы значения неизвестных функций 
${\vect{u}}(t_{j})$, рассчитанные по методу Эйлера
\eqref{eq:ODE_EULER} на $j$-ом временном слое:
\begin{gather}
\vect{v}(t_{j+1})\approx\vect{v}(t_{j})+
\dfrac{
\vect{f}\left(t_{j+1},{\vect{u}}(t_{j+1})\right)+
\vect{f}\left(t_{j},{\vect{u}}(t_{j})\right)
}{2}\cdot\Delta t_{j+1}.
\end{gather}

В качестве предельной абсолютной погрешности 
приближенного решения $\vect{u}(t_{j})$ 
задачи Коши \eqref{eq:ODE_SYS} можно принять 
какую-либо норму:
\begin{gather}
\vect{\epsilon}(t_j)=\vect{v}(t_j) - \vect{u}(t_j)
\end{gather}

Контроль точности приближенного решения 
может вестись покомпонентно или по норме.
Для различных компонент решения задачи $\vect{u}(t_j)$ 
могут использоваться различные допустимые значения 
погрешности.
Контроль точности по норме означает, что контролируется 
некоторая определенная норма оценки погрешности
(рисунок \ref{fig:ODE_IC:scheme error Euler}):
\begin{gather*}
\norma{\,\vect{\epsilon}\,}_{\infty}=\max_{i=1..n} \abs{\epsilon_i}
\quad
\norma{\,\vect{\epsilon}\,}_1=\sum\limits_{i=1}^n \abs{\epsilon_i},
\quad
\norma{\,\vect{\epsilon}\,}_2=\sqrt{\sum\limits_{i=1}^n \epsilon_i^2}
\end{gather*}

%
% *** Алгоритм оценки погрешности метода Эйлера
%
\begin{figure}[H]\centering
\begin{tikzpicture}[
font=\small,%
start chain=going below,%
node distance=10mm,%
>={Straight Barb[angle=45:1.5mm 1]},shorten >=0.5pt,%
]
% соединитель
\node[beginendnode,on chain,join]{начало};
% начальное условие
\node[IOnode,on chain,join]
{$\vect{t}=\left\{0,t_1,t_2,\ldots,t_m\right\}$};
\node[IOnode,on chain,join]
{$\vect{u}(\vect{t})=\left\{u_1(\vect{t}),u_2(\vect{t}),\ldots,u_n(\vect{t})\right\}$};
% начало цикла
\node[rectnode,on chain,join]{$j=0$};
% интервал времени
\node[rectnode,on chain,join](loop){$\Delta{t}_{j+1}=t_{j+1}-t_j$};
% решение интегрального уравнения
\node[rectnode,on chain,join]
{$\vect{v}(t_{j+1})=\vect{v}(t_{j})+
\dfrac{
\vect{f}\left(t_{j+1},{\vect{u}}(t_{j+1})\right)+
\vect{f}\left(t_{j},{\vect{u}}(t_{j})\right)
}{2}\cdot\Delta t_{j+1}$};
% погрешность
\node[rectnode,on chain,join]
{$\vect{\epsilon}(t_j)=\vect{v}(t_j) - \vect{u}(t_j)$};
% конец цикла
\node[ifthenelsenode,on chain,join](compare){$j<m$};
\draw(compare.east) node[above right]{да};
\draw(compare.south) node[below right]{нет};
\begin{scope}[start branch=b,node distance=40mm]
\node[rectnode,on chain,join](increment){$j=j+1$};
\draw[->,thick] (increment.north) |- (loop.east);
\end{scope}
% решение
\begin{scope}[node distance=15mm]
\node[IOnode,on chain,join]
{$\vect{\epsilon}(\vect{t})=\left\{
\epsilon_1(\vect{t}),\epsilon_2(\vect{t}),\ldots,\epsilon_n(\vect{t})
\right\}$};
\end{scope}
% конец
\node[beginendnode,on chain,join]{конец};
\end{tikzpicture}
\caption{Блок-схема для вычисления погрешности метода Эйлера}
\label{fig:ODE_IC:scheme error Euler}
\end{figure}

%
%	Численное решение задачи Коши методом Эйлера
%
\subsection{Численное решение задачи Коши методом Эйлера}
Применяя метод Эйлера, найдем решение задачи Коши
системы дифференциальных уравнений:
\begin{gather}\label{eq:ODE_MY}
\left\{\begin{matrix}[lclrcl]
\diff{u_1}{t}&=&0.2\cdot t+u_2,&u_1(0)&=&1\\[1em]
\diff{u_2}{t}&=&-\dfrac{u_1}{2},&u_2(0)&=&0
\end{matrix}\right.,
\end{gather}
в пределах отрезка $t\in[0, 10]$ на равномерной сетке с количеством временных интервалов $n=5$.

Введем обозначения
\begin{gather*}
\left\{\begin{array}{rcl}
f_1(t)&=&{0,2}\cdot{t}+u_2(t)\\[1em]
f_2(t)&=&-\dfrac{u_1(t)}{2}
\end{array}\right.,
\end{gather*}
где $f_1$ и $f_2$ -- функции, стоящие в правых частях 
дифференциальных уравнений системы \eqref{eq:ODE_MY}:

Рекуррентные соотношения \eqref{eq:ODE_EULER} 
для решения задачи Коши \eqref{eq:ODE_MY} методом Эйлера:
\begin{gather}\label{eq:ODE_SYS_RR}
\left\{\begin{matrix}
u_1(t_{j+1})&=&u_1(t_{j})+\Delta{t}\cdot f_1(t_j),&u_1(0)&=&1\\
u_2(t_{j+1})&=&u_2(t_{j})+\Delta{t}\cdot f_2(t_j),&u_2(0)&=&0
\end{matrix}\right.,
\end{gather}
где $\Delta{t}=(t_{j+1}-t_j)$ -- временной шаг метода Эйлера, 
т.е. интервал между двумя последовательными моментами времени

Решение системы интегральных уравнений \eqref{eq:ODE_SYS_INT} 
определяется рекуррентными соотношениями:
\begin{gather}\label{eq:ODE_SYS_INT_RR}
\left\{\begin{matrix}
\tilde{u}_1(t_{j+1})&=&\tilde{u}_1(t_{j})+\dfrac{f_1(t_{j})+f_1(t_{j+1})}{2}\cdot\Delta{t},&\tilde{u}_1(0)&=&1\\[1em]
\tilde{u}_2(t_{j+1})&=&\tilde{u}_2(t_{j})+\dfrac{f_2(t_{j})+f_2(t_{j+1})}{2}\cdot\Delta{t},&\tilde{u}_2(0)&=&0
\end{matrix}\right.
.\end{gather}

Определим временной шаг метода Эйлера, зная длину временного отрезка (``время наблюдения``) 
и количество интервалов:
\begin{gather*}
\Delta{t}=\dfrac{T-t_0}{n} = \dfrac{10-0}{5}=2,
\end{gather*}
где $t_0=0$ -- начальный момент времени;
$T=10$ -- максимальное время (``время наблюдения``).

Введем по переменной $t$ равномерную сетку с шагом $\tau=2$:
\begin{gather*}
\vect{t}=\{t_j\}=\{0,2,4,6,8,10\}
\end{gather*}

Последовательно определяем приближенное решение 
задачи Коши \eqref{eq:ODE_MY} методом Эйлера,
используя рекуррентные соотношения \eqref{eq:ODE_SYS_RR}.
\begin{enumerate}
\item
Определим значения неизвестных функций $u_1$ и $u_2$ в точке $t_1=2$:
\begin{gather*}\begin{array}{rcl}
f_1(0)&=&0.2\cdot0+u_2(0)=0.2\cdot0+0=0\\[1em]
f_2(0)&=&-\dfrac{u_1(0)}{2}=-\dfrac{1}{2}=-0.5
\end{array}\end{gather*}
%
\begin{gather*}
\left\{\begin{array}{rcl}
u_1(2)&=&u_1(0)+2\cdot{f_1(0)}=1+2\cdot0=1\\
u_2(2)&=&u_2(0)+2\cdot{f_2(0)}=0+2\cdot(-0.5)=-1
\end{array}\right..
\end{gather*}
\item
Определим значения неизвестных функций $u_1$ и $u_2$ в точке $t_2=4$:
\begin{gather*}\begin{array}{rcl}
f_1(2)&=&0.2\cdot2+u_2(2)=0.2\cdot2+(-1)=-0.6\\[1em]
f_2(2)&=&-\dfrac{u_1(2)}{2}=-\dfrac{1}{2}=-0.5
\end{array}\end{gather*}
%
\begin{gather*}
\left\{\begin{array}{rcl}
u_1(4)&=&u_1(2)+2\cdot{f_1(2)}=1+2\cdot(-0.6)=-0.2\\
u_2(4)&=&u_2(2)+2\cdot{f_2(2)}=-1+2\cdot(-0.5)=-2
\end{array}\right..
\end{gather*}
\item
Определим значения неизвестных функций $u_1$ и $u_2$ в точке $t_3=6$:
\begin{gather*}
\begin{array}{rcl}
f_1(4)&=&0.2\cdot4+u_2(4)=0.2\cdot4+(-2)=-1.2\\[1em]
f_2(4)&=&-\dfrac{u_1(4)}{2}=-\dfrac{-0.2}{2}=0.1
\end{array}\end{gather*}
%
\begin{gather*}
\left\{\begin{array}{rcl}
u_1(6)&=&u_1(4)+2\cdot{f_1(4)}=-0.2+2\cdot(-1.2)=-2.6\\
u_2(6)&=&u_2(4)+2\cdot{f_2(4)}=-2+2\cdot(0.1)=-1.8
\end{array}\right..
\end{gather*}
\item
Определим значения неизвестных функций $u_1$ и $u_2$ в точке $t_4=8$:
\begin{gather*}\begin{array}{rcl}
f_1(6)&=&0.2\cdot6+u_2(6)=0.2\cdot6+(-1.8)=-0.6\\[1em]
f_2(6)&=&-\dfrac{u_1(6)}{2}=-\dfrac{-2.6}{2}=1.3
\end{array}\end{gather*}
%
\begin{gather*}
\left\{\begin{array}{rcl}
u_1(8)&=&u_1(6)+2\cdot{f_1(4)}=-2.6+2\cdot(-0.6)=-3.8\\
u_2(8)&=&u_2(6)+2\cdot{f_2(6)}=-1.8+2\cdot(1.3)=0.8
\end{array}\right..
\end{gather*}
\item
Определим значения неизвестных функций $u_1$ и $u_2$ в точке $t_5=10$:
\begin{gather*}\begin{array}{rcl}
f_1(8)&=&0.2\cdot8+u_2(8)=0.2\cdot8+0.8=2.4\\[1em]
f_2(8)&=&-\dfrac{u_1(8)}{2}=-\dfrac{-3.8}{2}=1.9
\end{array}\end{gather*}
%
\begin{gather*}
\left\{\begin{array}{rcl}
u_1(10)&=&u_1(10)+2\cdot{f_1(4)}=-3.8+2\cdot(2.4)=1\\
u_2(10)&=&u_2(8)+2\cdot{f_2(8)}=0.8+2\cdot(1.9)=4.6
\end{array}\right..
\end{gather*}
\end{enumerate}

На рисунке \ref{fig:u(t)} представлено решение задачи Коши 
системы дифференциальных уравнений \eqref{eq:ODE_MY}.
% *******************************
%	График функций
%
\begin{figure}\centering
\begin{tikzpicture}
\begin{axis}[
xlabel = {$t$},	% подпись оси x
ylabel = {$u_1, u_2$},	% подпись оси y
xmin=-1, xmax=11, xtick={0,2,4,6,8,10}, %xticklabels={$0$,$t_1$,,$t_j$,,$t_N$},
ymin=-6, ymax=6, ytick={-6,-4,-2,0,2,4,6}, %yticklabels={$y^{(0)}$,$y^{(1)}$,,$y^{(j)}$,,$y^{(N)}$},
legend pos={north west}]
\addplot[ball darkgreen]coordinates{(0,1) (2,1) (4,-0.2) (6,-2.6) (8,-3.8) (10,1)};
\addlegendentry{$u_1$};
\addplot[ball darkred]coordinates{(0,0) (2,-1) (4,-2) (6,-1.8) (8,0.8) (10,4.6)};
\addlegendentry{$u_2$};
\end{axis}
\end{tikzpicture}
\caption{Зависимость неизвестных функций от времени}
\label{fig:u(t)}
\end{figure}
% *******************************

%
%	Приближенное решение интегрального уравнения
%
Последовательно определяем приближенное решение 
интегрального уравнения \eqref{eq:ODE_MY},
используя рекуррентные соотношения \eqref{eq:ODE_SYS_INT_RR}.
\begin{enumerate}
\item
Определим значения неизвестных функций 
$\tilde{u}_1$ и $\tilde{u}_2$ в точке $t_1=2$:
\begin{gather*}
\left\{\begin{array}{rcl}
\tilde{u}_1(2)&=&\tilde{u}_1(0)+\tau\cdot\dfrac{f_1(0)+f_1(2)}{2}
=1+2\cdot\dfrac{0+(-0.6)}{2}=0.4\\[1em]
\tilde{u}_2(2)&=&\tilde{u}_2(0)+\tau\cdot\dfrac{f_2(0)+f_2(2)}{2}
=0+2\cdot\dfrac{-0.5+(-0.5)}{2}=-1
\end{array}\right..
\end{gather*}
\item
Определим значения неизвестных функций 
$\tilde{u}_1$ и $\tilde{u}_2$ в точке $t_2=4$:
\begin{gather*}
\left\{\begin{array}{rcl}
\tilde{u}_1(4)&=&\tilde{u}_1(2)+\tau\cdot\dfrac{f_1(2)+f_1(4)}{2}
=0.4+2\cdot\dfrac{(-0.6)+(-1.2)}{2}=-2.4\\[1em]
\tilde{u}_2(4)&=&\tilde{u}_2(2)+\tau\cdot\dfrac{f_2(2)+f_2(4)}{2}
=-1+2\cdot\dfrac{-0.5+0.1}{2}=-1.4
\end{array}\right..
\end{gather*}
\item
Определим значения неизвестных функций
$\tilde{u}_1$ и $\tilde{u}_2$ в точке $t_3=6$:
\begin{gather*}
\left\{\begin{array}{rcl}
\tilde{u}_1(6)&=&\tilde{u}_1(4)+\tau\cdot\dfrac{f_1(4)+f_1(6)}{2}
=-2.4+2\cdot\dfrac{-1.2+(-0.6)}{2}=-4.2\\[1em]
\tilde{u}_2(6)&=&\tilde{u}_2(4)+\tau\cdot\dfrac{f_2(4)+f_2(6)}{2}
=-1.4+2\cdot\dfrac{0.1+1.3}{2}=0
\end{array}\right..
\end{gather*}
\item
Определим значения неизвестных функций 
$\tilde{u}_1$ и $\tilde{u}_2$ в точке $t_4=8$:
\begin{gather*}
\left\{\begin{array}{rcl}
\tilde{u}_1(8)&=&\tilde{u}_1(6)+\tau\cdot\dfrac{f_1(6)+f_1(8)}{2}
=-4.2+2\cdot\dfrac{-0.6+2.4}{2}=-2.4\\[1em]
\tilde{u}_2(8)&=&\tilde{u}_2(6)+\tau\cdot\dfrac{f_2(6)+f_2(8)}{2}
=0+2\cdot\dfrac{1.3+1.9}{2}=3.2
\end{array}\right.
.\end{gather*}
\item
Определим значения неизвестных функций 
$\tilde{u}_1$ и $\tilde{u}_2$ в точке $t_5=10$:
\begin{gather*}\begin{array}{rcl}
f_1(10)&=&{0.2}\cdot10+u_2(10)={0.2}\cdot8+4.6=6.2\\[1em]
f_2(10)&=&-\dfrac{u_1(10)}{2}=-\dfrac{1}{2}=-0.5
\end{array}\end{gather*}
%
\begin{gather*}
\left\{\begin{array}{rcl}
\tilde{u}_1(10)&=&\tilde{u}_1(8)+\tau\cdot\dfrac{f_1(8)+f_1(10)}{2}
=-2.4+2\cdot\dfrac{2.4+6.2}{2}=6.2\\[1em]
\tilde{u}_2(10)&=&\tilde{u}_2(8)+\tau\cdot\dfrac{f_2(8)+f_2(10)}{2}
=3.2+2\cdot\dfrac{1.9+(-0.5)}{2}=4.6
\end{array}\right..
\end{gather*}
\end{enumerate}

На рисунке \ref{fig:uu(t)} представлены решения задачи Коши \eqref{eq:ODE_SYS} и
интегрального уравнения \eqref{eq:ODE_SYS_INT}, 
рассчитанные в различные моменты времени.

В таблице \ref{tab:error} и на рисунке \ref{fig:error} представлены значения
предельной абсолютной погрешности приближенного решения задачи Коши 
для различных моментов времени.
% *******************************
%	График функций
%
\begin{figure}[H]\centering
\begin{tikzpicture}
\begin{axis}[
%xlabel = {$t$},	% подпись оси x
ylabel = {$u_1, \tilde{u}_1$},	% подпись оси y
xmin=-1, xmax=11, xtick={0,2,4,6,8,10}, %xticklabels={$0$,$t_1$,,$t_j$,,$t_N$},
ymin=-6, ymax=7, ytick={-6,-4,-2,0,2,4,6}, %yticklabels={$y^{(0)}$,$y^{(1)}$,,$y^{(j)}$,,$y^{(N)}$},
legend pos={north west}]
\addplot[ball darkgreen]coordinates{(0,1) (2,1) (4,-0.2) (6,-2.6) (8,-3.8) (10,1)};
\addlegendentry{$u_1$};
\addplot[thick, mark=*, mark size=3pt, mark options={fill=white, draw=black, solid}]
coordinates {(0,1) (2,0.4) (4,-2.4) (6,-4.2) (8,-2.4) (10,6.2)};
\addlegendentry{$\tilde{u}_1$};
\end{axis}
\end{tikzpicture}
% *******************************
%	График функций
%
\begin{tikzpicture}
\begin{axis}[
xlabel = {$t$},	% подпись оси x
ylabel = {$u_2, \tilde{u}_2$},	% подпись оси y
xmin=-1, xmax=11, xtick={0,2,4,6,8,10}, %xticklabels={$0$,$t_1$,,$t_j$,,$t_N$},
ymin=-3, ymax=5, ytick={-2,0,2,4}, %yticklabels={$y^{(0)}$,$y^{(1)}$,,$y^{(j)}$,,$y^{(N)}$},
legend pos={north west}]
\addplot[ball darkred]coordinates{(0,0) (2,-1) (4,-2) (6,-1.8) (8,0.8) (10,4.6)};
\addlegendentry{$u_2$};
\addplot[thick, mark=*, mark size=3pt, mark options={fill=white, draw=black, solid}]
coordinates{(0,0) (2,-1) (4,-1.4) (6,0) (8,3.2) (10,4.6)};
\addlegendentry{$\tilde{u}_2$};
\end{axis}
\end{tikzpicture}
\caption{Приближенное решение задачи Коши и соответствующего интегрального уравнения}
\label{fig:uu(t)}
\end{figure}

\begin{table}[H]
\caption{Предельная абсолютная погрешность 
приближенного решения задачи Коши \eqref{eq:ODE_MY}}
\label{tab:error}
\small
\begin{tabular*}{\textwidth}
{@{\extracolsep{\fill}}*{7}{p{1.5cm}}}
%\begin{tabular*}{\\textwidth}{p{8cm}p{5cm}r}
\toprule
$i$&$0$&$1$&$2$&$3$&$4$&$5$\\
$t_i$&$0$&$2$&$4$&$6$&$8$&$10$\\
\midrule
\multicolumn{7}{l}{Задача Коши}\\
\midmidrule
$u_1(t_i)$&$1$&$1$&$-0.2$&$-2.6$&$-3.8$&$1$\\
$u_2(t_i)$&$0$&$-1$&$-2$&$-1.8$&$0.8$&$4.6$\\
\midrule
\multicolumn{7}{l}{Интегральное уравнение}\\
\midmidrule
$\tilde{u}_1(t_i)$&$1$&$0.4$&$-2.4$&$-4.2$&$-2.4$&$6.2$\\
$\tilde{u}_2(t_i)$&$0$&$-1$&$-1.4$&$0$&$3.2$&$4.6$\\
\midrule
\multicolumn{7}{l}{Абсолютная погрешность 
$\vect{\epsilon}=|\tilde{\vect{u}}-\vect{u}|$}\\
\midmidrule
$\epsilon_1$&$0$&$0.6$&$2.2$&$1.6$&$1.4$&$5.2$\\
$\epsilon_2$&$0$&$0$&$0.6$&$1.8$&$2.4$&$0$\\
\bottomrule
\end{tabular*}
\end{table}

Из рисунка \ref{fig:error} видно, что максимальная 
предельная абсолютная погрешность для $u_1(t)$ 
составляет $\epsilon_1=5.2$, 
а для функции $u_2(t)$ -- $\epsilon_2=2.4$.

% *******************************
%	График погрешности
%
\begin{figure}[H]\centering
\begin{tikzpicture}
\begin{axis}[
%enlargelimits=true,
ybar,bar width=18pt,
nodes near coords,
xlabel = {Время, $t$},
ylabel = {Абсолютная погрешность, $\epsilon$},	% подпись оси y
xmin=1, xmax=11, xtick={0,2,4,6,8,10},
ymin=-0.5, ymax=6, ytick={0,2,4,6},
legend pos={north west},
xtick align=inside,ytick align=inside,]
\addplot[thin, blue, fill=blue!35]
coordinates{(2,0.6) (4,2.2) (6,1.6) (8,1.4) (10,5.2)};
\addlegendentry{$\epsilon_1$};
\addplot[thin, red, fill=red!35]
coordinates{(2,0) (4,0.6) (6,1.8) (8,2.4) (10,0)};
\addlegendentry{$\epsilon_2$};
\end{axis}
\end{tikzpicture}
\caption{Предельная абсолютная погрешность приближенного решения задачи Коши \eqref{eq:ODE_MY}}
\label{fig:error}
\end{figure}

% Краевые задачи для обыкновенных дифференциальных уравнений
%%\input{../../preamble}
%\begin{document}
% разделитель в многострочных формулах
%\newcommand\eqsep{\\[1em]}
% интегралы
%\newcommand\intfLeft[1]{ \intf{ x_{i-1} }{ x_{i} }{#1} }
%\newcommand\intfleft[1]{ \intf{ x_{i-1/2} }{ x_{i} }{#1} }
%\newcommand\intfCenter[1]{ \intf{ x_{i-\onehalf} }{ x_{i+\onehalf} }{#1} }
%\newcommand\intfCenter[1]{ \intf{ x_{i-1/2} }{ x_{i+1/2} }{#1} }
%\newcommand\intfright[1]{ \intf{ x_{i} }{ x_{i+1/2} }{#1} }
%\newcommand\intfRight[1]{ \intf{ x_{i} }{ x_{i+1} }{#1} }

% СОДЕРЖАНИЕ
%\renewcommand\contentsname{\begin{center}СОДЕРЖАНИЕ\end{center}}
%\tableofcontents

%
%	Краевые задачи для обыкновенных дифференциальных уравнений
%
\newpage
\section{Краевые задачи для обыкновенных дифференциальных уравнений}

Исследование большого круга естественно-научных и  
инженерных проблем приводит к математическим задачам, относящимся к 
решению дифференциальных уравнений и граничных проблем для них,
интегральных и других функциональных уравнений.

В практике чаще всего встречаются задачи, 
для которых точное решение не может быть найдено или оно имеет 
невысокую эффективность. 
Поэтому приближенные методы решения задач граничных задач, например,
задач математической физики, в особенности метод сеток 
получили широкое распространение.

Основные \textit{достоинства приближенных методов} заключаются в том, 
что они \textit{являются универсальными и эффективными},
так как позволяют находить приближенное решение для широкого класса задач 
новейших областей науки, техники и технологий.
Как правило, такие задачи формулируются в виде основного 
дифференциального уравнения (системы дифференциальный уравнений)
и дополнительных (граничных) условий, которые
обеспечивают существование и единственность решения.

\begin{tcolorbox}[colback=orange!10]
Под краевой (граничной) задачей понимают задачу о нахождении решения 
дифференциального уравнения (системы дифференциальных уравнений),
удовлетворяющего краевым условиям на границе рассматриваемой области.
\end{tcolorbox}

В каждой краевой задаче искомое решение должно 
удовлетворять дифференциальному в рассматриваемой области $\Omega$,
а также некоторому дополнительному условию на границе $\Sigma$ области $\Omega$. 
Например, для обыкновенного дифференциального уравнения второго порядка:
\begin{gather}\label{eq:ODE_MAIN}
\diffdiff{y(x)}{x}=f\left(x,y,\diff{y(x)}{x}\right),
\end{gather}
где $y(x)$ -- искомая неизвестная функция;
$x$ -- независимое переменное; 
$f$ -- функция, определенная в некоторой замкнутой 
области $\Omega$ изменения своих аргументов.

% **************************************
%
%	ВИДЫ КРАЕВЫХ ЗАДАЧ
%
\emptyline
\subsection{Виды краевых условий}
В зависимости от вида краевого условия на границе $\Sigma=\{a,b\}$
различают три основных типа граничных задач:
задачу Дирихле, Неймана и смешанную краевую задачу 
(таблица \ref{tab:ODE_BC_TYPE}).
Необходимо отметить, что в отличие от задачи Коши, для 
которой при выполнении определенных условий 
гарантируется существование и единственность решения,
\textit{краевая задача} для того же дифференциального уравнения
\textit{может не иметь решения} или иметь несколько решений 
(в том числе и бесконечное множество решений). 

% ******* begin table *******
\begin{table}[H]
\caption{
Основные виды краевых условий
для обыкновенных дифференциальных уравнений 
второго порядка \eqref{eq:ODE_MAIN}
}
\label{tab:ODE_BC_TYPE}
\begin{tabular}{l l l l}
\toprule
№&Краевое условие&Граница $x=a$&Граница $x=b$\\
\midmidrule
1&Дирихле
&$y(a)=y_a$
&$y(b)=y_b$\\
\addlinespace[1ex]
2&Неймана
&$y^{\prime}(a)=y^{\prime}_a$
&$y^{\prime}(b)=y^{\prime}_b$\\
\addlinespace[1ex]
3&Смешанная
&$\kappa_a\cdot y^{\prime}(a)+\lambda_a\cdot y(a)=\mu_a$
&$\kappa_b\cdot y^{\prime}(b)+\lambda_b\cdot y(b)=\mu_b$\\
\addlinespace[1ex]
\bottomrule
\end{tabular}
\end{table}
% ******* end table *******

% *** Смешанная краевая задача ***
\begin{figure}\centering
%%\input{../../../preamble}
%\begin{document}

% *******************************
%
%	��������� ������� ������
%
% *******************************
\begin{center}
\begin{tikzpicture}[framed]
% ����������� �������
\draw[
chamfered rectangle,
chamfered rectangle xsep=4pt,
chamfered rectangle angle=60,
minimum height=3em,
minimum width=10em
] (0,0) node[thick,color=black,draw,color=black,fill=lime!10,] (G)
{$\kappa\cdot y^{\prime}(a)+\lambda\cdot y(a)=\mu$};
\node[above] at (G.north) {��������� ������};
% ������ �������
\draw[shape=rectangle,minimum height=3em, minimum width=5em]
($(G.south west)+(-3em,-5em)$)
node[thick,red,draw=red,fill=red!10] (D) {$y(a)=y_a$};
\node[below] at (D.south) {������ �������};
\node[right] at (D.east) {$y_a=\dfrac{\mu}{\lambda}$};
% ������ �������
\draw[shape=rectangle, minimum height=3em, minimum width=5em]
($(G.south east)+(3em,-5em)$)
node[thick,blue,draw=blue,fill=blue!10] (N) {$y^{\prime}(a)=y^{\prime}_a$};
\node[below] at (N.south) {������ �������};
\node[right] at (N.east) {$y^{\prime}_a=\dfrac{\mu}{\kappa}$};
% �������
\draw[-latex,very thick,red] (G.west) -| (D.north) node[left,midway] {$\kappa\equiv0$};
\draw[-latex,very thick,blue] (G.east) -| (N.north) node[right,midway] {$\lambda\equiv0$};
\end{tikzpicture}
\end{center}
% *******************************************
%\end{document}

\begin{tikzpicture}
% контрольная область
\draw[
chamfered rectangle,
chamfered rectangle xsep=4pt,
chamfered rectangle angle=60,
minimum height=3em,
minimum width=10em
] (0,0) node[thick,color=black,draw,color=black,fill=lime!10,] (G)
{$\kappa\cdot y^{\prime}(a)+\lambda\cdot y(a)=\mu$};
\node[above] at (G.north) {Смешанная задача};
% Задача Дирихле
\draw[shape=rectangle,minimum height=3em, minimum width=5em]
($(G.south west)+(-3em,-5em)$)
node[thick,red,draw=red,fill=red!10] (D) {$y(a)=y_a$};
\node[below] at (D.south) {Задача Дирихле};
\node[right] at (D.east) {$y_a=\dfrac{\mu}{\lambda}$};
% Задача Неймана
\draw[shape=rectangle, minimum height=3em, minimum width=5em]
($(G.south east)+(3em,-5em)$)
node[thick,blue,draw=blue,fill=blue!10] (N) {$y^{\prime}(a)=y^{\prime}_a$};
\node[below] at (N.south) {Задача Неймана};
\node[right] at (N.east) {$y^{\prime}_a=\dfrac{\mu}{\kappa}$};
% стрелки
\draw[-latex,very thick,red] (G.west) -| (D.north) node[left,midway] {$\kappa\equiv0$};
\draw[-latex,very thick,blue] (G.east) -| (N.north) node[right,midway] {$\lambda\equiv0$};
\end{tikzpicture}
% *******************************************
\caption{Виды краевых задач для дифференциальных уравнений\linebreak
второго порядка}
\label{fig:ODE_BC(Genaral)}
\end{figure}

С формальной точки зрения, задача Дирихле и Неймана являются
частными случаями смешанной краевой задачи 
(рисунок \ref{fig:ODE_BC(Genaral)}):
например, если $\kappa_a\equiv0$, то смешанная краевая задача
трансформируется в задачу Дирихле:
\begin{gather*}
0\cdot y^{\prime}(a)+\lambda_a\cdot y(a)=\mu_a,
\quad \Rightarrow \quad 
y(a) = \dfrac{\mu_a}{\lambda_a} =y_a
,\end{gather*}
а если полагать $\lambda_a\equiv0$, то формулируется 
задача Неймана:
\begin{gather*}
\kappa_a\cdot y^{\prime}(a)+0\cdot y(a)=\mu_a,
\quad \Rightarrow \quad 
y^{\prime}(a) = \dfrac{\mu_a}{\kappa_a} =y^{\prime}_a.
\end{gather*}

Поэтому для общности, далее будем рассматривать 
только смешанную краевую задачу.

% **************************************************
%
%	Краевые условия для обыкновенных дифференциальных уравнений
%
% **************************************************
\begin{figure}[H]\centering
\begin{subfigure}[b]{.5\linewidth}\centering
\begin{tikzpicture}
%
%	Задача Дирихле
%
\begin{axis}[width=8cm,
name=plota,
xlabel = {$x$},		% подпись оси x
ylabel = {$y(x)$},	% подпись оси y
xmin=-1, xmax=4, xtick={0,3}, xticklabels={$a$,$b$},
ymin=0, ymax=9, ytick={3,6}, yticklabels={\empty},
]
\addplot[thick,black,domain=0:3,samples=50] {x^2-2*x+3};
\addplot[
only marks,mark=*,mark size=3pt, mark options={fill=white,draw=red,thick},
nodes near coords,point meta=explicit symbolic,
node near coord style={above,yshift=0.5ex,color=red}%fill=olive!10},
] coordinates {(0,3) [$y_a$] (3,6) [$y_b$]};
\end{axis}
\end{tikzpicture}
\caption{задача Дирихле}
\end{subfigure}%
%
%	Задача Неймана
%
\begin{subfigure}[b]{.5\linewidth}\centering
\begin{tikzpicture}
\begin{axis}[width=8cm,
name=plotb, at={($(plota.east)+(3cm,0)$)}, anchor=west,
xlabel = {$x$},		% подпись оси x
ylabel = {$y(x)$},	% подпись оси y
xmin=-1, xmax=4, xtick={0,3}, xticklabels={$a$,$b$},
ymin=0, ymax=9, ytick={\empty}, %yticklabels={$y_a$,$y_b$},
]
\addplot[thick,black,domain=0:3,samples=50] {x^2-2*x+3};
\addplot[
only marks,mark=*,mark size=3pt,mark options={fill=white,draw=red,thick},
nodes near coords,	 point meta=explicit symbolic,
node near coord style={above,yshift=1ex,color=red}
] coordinates {(0,3) [$y^{\prime}_a$] (3,6) [$y^{\prime}_b$]};
% y'(a)
\addplot[thick,red,domain=-0.75:0.75,samples=2] {-2*x+3};
% y'(b)
\addplot[thick,red,domain=2.5:3.5,samples=2] {4*x-6};
\end{axis}
\end{tikzpicture}
\caption{задача Неймана}
\end{subfigure}
\caption{ Графическая иллюстрация постановки краевых задач\\
a) -- Дирихле; b) -- Неймана}
\end{figure}

%
%	Построение разностных схем методом баланса
%
\emptyline
\subsection{Построение разностных схем методом баланса}
Различные физические процессы (теплопроводности или диффузии,
колебаний, газодинамики и т. д.) характеризуются некоторыми 
интегральными законами сохранения (тепла, массы, количества движения, энергии и т. д.).
При выводе дифференциальных уравнений различных математических моделей
объектов или явлений отталкиваются от некоторого интегрального 
соотношения (уравнения баланса), выражающего закон сохранения для малого объема.
Дифференциальное уравнение получается из уравнения баланса 
при стягивании рассматриваемого объема к нулю 
в предположении существования непрерывных производных, 
входящих в уравнение.

Метод конечных разностей физически означает переход от непрерывной среды
к некоторой ее дискретной модели. При таком переходе естественно требовать, 
чтобы основные свойства физического процесса сохранялись, т.е.
выполнялись законы сохранения различных физичеких субстанций
(количестово тепла, массы, заряда, импульса и т.д.). 
Разностные схемы, выражающие на сетке законы сохранения физических субстанций, 
называют консервативными (или дивергентными). 
Законы сохранения для всей сеточной области (``интегральные законы сохранения``)
для консервативных схем должны быть алгебраическим следствием разностных уравнений.

Для получения консервативных разностных схем естественно исходить из уравнений баланса,
записанных для элементарных объемов (ячеек) сеточной области.
Входящие в эти уравнения баланса интегралы и производные следует заменить 
приближенными разностными выражениями.
В результате получаем однородную разностную схему.
Такой метод получения консервативных однородных разностных схем 
называется интегро-интерполяционным методом (методом баланса).

Рассмотрим краевую задачу, включающую 
обыкновенное дифференциальное уравнение второго порядка
и дополнительные (граничные) условия на концах отрезка $[0,\ell]$:
\begin{gather}
-\left(k(x)\cdot u^{\prime}(x)\right)^{\prime}+r(x)\cdot u(x)=g(x),\quad x\in[0,\ell]\label{eq:ODE_BC}\\
\left\{\begin{matrix}\label{eq:ODE_BC2}
-k(0)\cdot u^{\prime}(0)+\lambda_1\cdot u(0) = \mu_1\\
k(\ell)\cdot u^{\prime}(\ell)+\lambda_2\cdot u(\ell) = \mu_2
\end{matrix}\right.
,\end{gather}
где $u(x)$ -- искомая неизвестная функция;
$k(x)$, $r(x)$, $g(x)$ -- известные (заданные) функции,
удовлетворяющие условиям $k(x)>0$, $r(x)\geqslant0$;
$\lambda_1$, $\mu_1$ и $\lambda_2$, $\mu_2$ -- 
заданные (известные) числа;
$\ell$ -- размер области (длина отрезка).

\begin{tcolorbox}
Численные методы решения краевых задач основаны 
на замене дифференциальной задачи \eqref{eq:ODE_BC}, \eqref{eq:ODE_BC2}
системой линейных алгебраических уравнений, т.е. разностной схемой.

\emph{Разностная схема} -- совокупность разностных алгебраических уравнений, 
аппроксимирующих основное дифференциальное уравнение
и дополнительные (граничные) условия исходной дифференциальной задачи.
\end{tcolorbox}

%
%	ФИЗИЧЕСКАЯ ИНТЕРПРЕТАЦИЯ КРАЕВЫХ ЗАДАЧ
%
\emptyline
\subsubsection{Физическая интерпретация краевых задач}
Уравнение \eqref{eq:ODE_BC} можно интерпретировать как уравнение 
установившегося распределения температуры $u(x)$ в стержне длины $\ell$.
В такой постановке краевую задачу 
\eqref{eq:ODE_BC}, \eqref{eq:ODE_BC2} можно записать в виде:
\begin{gather}
w^{\prime}(x)+r(x)\cdot u(x)=g(x)\label{eq:ODE_BC0}\\
\left\{\begin{matrix}\label{eq:ODE_BC20}
w(0)+\lambda_1\cdot u(0) = \mu_1\\
-w(\ell)+\lambda_2\cdot u(\ell) = \mu_2
\end{matrix}\right.
,\end{gather}
где
$w(x)$ -- тепловой поток, величина которого в соответствие с 
\href{https://en.wikipedia.org/wiki/Thermal_conductivity}
{законом теплопроводности Фурье} пропорциональна 
градиенту температуры: % закон теплопроводности Фурье
\begin{gather}\label{eq:FourierLaw}
w(x)=-k(x)\cdot u^{\prime}(x)
;\end{gather}
$k(x)$ -- коэффициент температуропроводности материала стержня;
$r(x)\cdot u(x)$ -- мощность распределенных стоков тепла 
($r(x)<0$ источников тепла) вдоль боковой поверхности стержня;
$r(x)$ -- коэффициент конвекционного теплообмена боковой поверхности
стержня с окружающей средой;
$g(x)>0$ -- мощность распределенных источников ($g(x)<0$ стоков)
тепла внутри стержня.

Граничные условия \eqref{eq:ODE_BC20} определяют процесс теплообмена 
концов стержня с окружающей средой по закону
% закон конвекционного теплообмена Ньютона
\href{https://en.wikipedia.org/wiki/Newton\%27s_law_of_cooling}
{конвекционного теплообмена Ньютона} -- 
``Тепловой поток тепла $q$ через поверхность материального 
тела в окружающую среду прямо пропорционален разнице 
температур между поверхностью тела и окружающей средой``:
\begin{gather*}
q=\mathring{r}\cdot(u - \mathring{u}),
\end{gather*}
где $\mathring{r}$ --  коэффициент конвективного теплообмена;
$u$ -- температура поверхности тела;
$\mathring{u}$ -- температура окружающей среды.

Процессы выделения или поглощения тепла 
в твердом теле могут быть связаны с рядом явлений:
\begin{enumerate}
\item
выделение тепла ($g(x)>0$) при пропускании 
электрического тока через стержень (закон Джоуля-Ленца),
вследствие протекания в объеме стержня
каких-либо экзотермических химических реакций;
\item
поглощение тепла ($g(x)<0$) за счет 
термоэлектрических явлений (например, эффект Пельтье), а также
вследствие протекания эндотермических химических реакций
в объеме материала стержня.
\end{enumerate}

\end{document}

\begin{figure} % *** Однородный стержень ***
\begin{center}
%\input{../../../preamble}
%\begin{document}
%	���� ������ \GraphShiftToRight �� ���������, ��
%	���������� ������ \def\\GraphShiftToRight{3cm}
%
\ifx\GraphShiftToRight\undefined
	% ������ ������� ������� �� ������
	\def\GraphShiftToRight{3cm}
\fi
% **************************************************
%
%	���������� ��������
%
% **************************************************
\begin{tikzpicture}[framed]
\pgfplotsset{set layers=standard}
\pgfsetlayers{background,axis grid,main,axis ticks,axis lines, axis tick labels,axis descriptions,axis foreground}
\def\xbreak{2.5}
\def\xxbreak{6.5}
\def\kx{7}
\def\rx{4}
\begin{pgfonlayer}{main}
%
%	��������� �������������
%
% ������ ��� OY
\begin{axis}
[	width=7cm,
	name=plot1,
	my graph style,
	title={a)},
	const plot,
%	axis y line*=left,
	ylabel = {$k(x),~r(x)$},	% ������� ��� y
	xmin=-1, xmax=10, xtick={0,\xbreak,\xxbreak,9}, xticklabels={0,$a$,$b$,$\ell$},
	ymin=-1, ymax=10, ytick={\kx}, yticklabels={\empty},
%	\axisYcolor{blue}
]
% k(x)
\addplot[draw=none,fill=blue!5] coordinates {(0,0) (0,\kx) (9,\kx) (9,0)}\closedcycle;
\draw[very thick,draw=blue] (0,\kx) -- (9,\kx) node[midway,above,blue] {$k_1$};
% r(x)
\addplot[draw=none,fill=orange!20] coordinates {
(0,0) (0,\rx) (\xbreak,\rx) (\xbreak,0) (\xxbreak,0) (\xxbreak,\rx) (9,\rx) (9,0)}\closedcycle;
\draw[very thick,draw=red] (0,\rx) -- (\xbreak,\rx) node[midway,above,red] {$r_1$};
\draw[very thick,draw=red] (\xxbreak,\rx) -- (9,\rx) node[midway,above,red] {$r_1$};
\draw[very thick,draw=red] (\xbreak,0) -- (\xxbreak,0) node[midway,above,red] {$r_1=0$};
\end{axis}
%
%	��������� ������
%
\def\xbreak{3}
\def\xxbreak{6}
\def\kx{7}
\def\gx{4}
\begin{axis}
[	width=7cm,
	my graph style,
%	������ ������ 	 
	at={($(plot1.east)+(\GraphShiftToRight,0)$)},
	anchor=west,
	title={b)},
% ************
	const plot,
%	axis y line*=left,
	ylabel = {$k(x),~g(x)$},	% ������� ��� y
	xmin=-1, xmax=10, xtick={0,\xbreak,\xxbreak,9}, xticklabels={0,$a$,$b$,$\ell$},
	ymin=-1, ymax=10, ytick={\kx}, yticklabels={\empty},
%	\axisYcolor{blue}
]
% k(x)
\addplot[draw=none,fill=blue!5] coordinates {(0,0) (0,\kx) (9,\kx) (9,0)}\closedcycle;
\draw[very thick,draw=blue] (0,\kx) -- (9,\kx) node[midway,above,blue] {$k_1$};
% r(x)
\addplot[draw=none,fill=black!20] coordinates {
(0,0) (\xbreak,0) (\xbreak,\gx) (\xxbreak,\gx) (\xxbreak,0) (9,0)}\closedcycle;
\draw[very thick,draw=black] (0,0) -- (\xbreak,0) node[midway,above,black] {$g_1=0$};
\draw[very thick,draw=black] (\xxbreak,0) -- (9,0) node[midway,above,black] {$g_1=0$};
\draw[very thick,draw=black] (\xbreak,\gx) -- (\xxbreak,\gx) node[midway,above,black] {$g_1>0$};
\end{axis}
\end{pgfonlayer}{main}
\end{tikzpicture}
% ************************************************** 
%\end{document}
\caption{Примеры краевых задач с разрывными коэффициентами $r(x)$ и $g(x)$
дифференциального уравнения установившегося распределения температуры 
в однородном стержне длиной $\ell$:\\
(a) -- локальная теплоизоляция боковой поверхности стержня;\\
(b) -- локальный нагрев участка стержня}
\label{fig:ODE_BC(homo_material)}
\end{center}
\end{figure}

Необходимо отметить, что в зависимости от физического смысла постановки 
краевых задач коэффициенты дифференциального уравнения задачи 
могут быть и разрывными функциями 
(рисунки \ref{fig:ODE_BC(homo_material)} и \ref{fig:ODE_BC(hetero_material)}),
например:
\begin{enumerate}
\item
\textit{Однородный стержень} с локальной теплоизоляцией,
т.е. конвекционный теплообмен участка $[a,b]$
боковой поверхности стрежня с окружающей средой отсутствует
(рисунок \ref{fig:ODE_BC(homo_material)}~a):
\begin{gather*}
r(x)=\begin{cases}
r_1,\quad 0 \leqslant x \leqslant a\\
0,\quad a < x \leqslant b\\
r_1,\quad b < x \leqslant \ell
\end{cases}
,\end{gather*}
где $[a,b]$ -- теплоизолированный участок однородного стержня
($k(x)=k_1=\mathrm{const}$).

\item
\textit{Однородный стержень} с локальным нагревом, т.е.
на участке стержня $[a,b]$ происходит нагрев стержня 
за счет внешних источников тепловой энергии:
\begin{gather*}
g(x)=\begin{cases}
0,\quad 0 \leqslant x \leqslant a\\
g_1,\quad a < x \leqslant b\\
0,\quad b < x \leqslant \ell
\end{cases}
,\end{gather*}
где $[a,b]$ -- участок локального нагрева стержня 
внешними источниками тепловой энергии.

\item
\textit{Неоднородный стержень}, состоящий из двух разнородных материалов, 
которые отличаются коэффициентами температуропроводности $k_1\ne k_2$ и
конвекционного теплообмена $r_1\ne r_2$
(рисунок \ref{fig:ODE_BC(hetero_material)}~a):
\begin{gather*}
k(x)=\begin{cases}
k_1,\quad 0\leqslant x \leqslant c\\
k_2,\quad c< x \leqslant \ell
\end{cases}
,\quad
r(x)=\begin{cases}
r_1,\quad 0\leqslant x \leqslant c\\
r_2,\quad c< x \leqslant \ell
\end{cases}
,\end{gather*}
где $c$ -- точка контакта двух различных материалов;
$[0,c]$ и $(c,\ell]$ -- участки стержня, состоящие из разных материалов
с различными коэффициентами температуропроводности $k_1\ne k_2$ и
конвекционного теплообмена $r_1\ne r_2$.

\item
\textit{Неоднородный стержень}, с локальной теплоизоляцией,
т.е. конвекционный теплообмен на участке
боковой поверхности стрежня $[a,b]$ с окружающей средой отсутствует
(рисунок \ref{fig:ODE_BC(homo_material)}~b): 
\begin{gather*}
k(x)=\begin{cases}
k_1,\quad 0\leqslant x \leqslant c\\
k_2,\quad c< x \leqslant \ell
\end{cases}
,\quad
r(x)=\begin{cases}
r_1,\quad 0 \leqslant x \leqslant a\\
0,\quad a < x \leqslant b\\
r_2,\quad b < x \leqslant \ell
\end{cases}
,\end{gather*}
где $[a,b]$ -- теплоизолированный участок неоднородного стержня ($r\equiv0$).
\end{enumerate}

\begin{figure} % *** Неоднородный стержень ***
\begin{center}
%\input{../../../preamble}
%\begin{document}
%	���� ������ \GraphShiftToRight �� ���������, ��
%	���������� ������ \def\\GraphShiftToRight{3cm}
%
\ifx\GraphShiftToRight\undefined
	% ������ ������� ������� �� ������
	\def\GraphShiftToRight{3cm}
\fi
% **************************************************
%
%	������������ ��������
%
% **************************************************
\begin{tikzpicture}[framed]
% ������������������ ����������� �����
\pgfplotsset{set layers=standard}
\pgfsetlayers{background,axis grid,main,axis ticks, axis lines, axis tick labels,axis descriptions,axis foreground}
\begin{pgfonlayer}{main}
%
%	���������� ����������
%
\def\xbreak{4}
\def\ka{5.5}
\def\kb{7.5}
\def\ra{2}
\def\rb{4}
\begin{axis}
[	width=7cm,
	my graph style,
	name=plot1, title={a)},
	const plot,
	xlabel = {$x$},		% ������� ��� x
	ylabel = {$k(x),~r(x)$},	% ������� ��� y
	xmin=-1, xmax=10, xtick={0,\xbreak,9}, xticklabels={0,$c$,$\ell$},
	ymin=-1, ymax=10, ytick={\ka,\kb}, yticklabels={\empty},
]
% k(x)
\addplot[draw=none,fill=blue!5] coordinates {(0,\ka) (\xbreak,\ka) (\xbreak,\kb) (9,\kb)}\closedcycle;
\draw[very thick,draw=blue] (axis cs: 0,\ka) -- (axis cs: \xbreak,\ka) node[midway,above,blue] {$k_1$};
\draw[very thick,draw=blue] (axis cs: \xbreak,\kb) -- (axis cs: 9,\kb) node[midway,above,blue] {$k_2$};
% r(x)
\addplot[draw=none,fill=orange!20] coordinates {(0,\ra) (\xbreak,\ra) (\xbreak,\rb) (9,\rb)}\closedcycle;
\draw[very thick,draw=red] (axis cs: 0,\ra) -- (axis cs: \xbreak,\ra) node[midway,above,red] {$r_1$};
\draw[very thick,draw=red] (axis cs: \xbreak,\rb) -- (axis cs: 9,\rb) node[midway,above,red] {$r_2$};
\end{axis}
%
%	��������� ������
%
\def\xa{3}
\def\xc{5}
\def\xb{7}
\def\ka{5.5}
\def\kb{7.5}
\def\ra{2}
\def\rb{4}
\begin{axis}
[	width=7cm,
	my graph style,
%	������ ������ 	 
	at={($(plot1.east)+(\GraphShiftToRight,0)$)},
	anchor=west,
	title={b)},
% ************
	const plot,
	xlabel = {$x$},		% ������� ��� x
	ylabel = {$k(x),~r(x)$},	% ������� ��� y
	xmin=-1, xmax=10, xtick={0,\xa,\xc,\xb,9}, xticklabels={0,$a$,$c$,$b$,$\ell$},
	ymin=-1, ymax=10, ytick={\ka,\kb}, yticklabels={\empty},
]
% k(x)
\addplot[draw=none,fill=blue!5] coordinates {(0,\ka) (\xc,\ka) (\xc,\kb) (9,\kb)}\closedcycle;
\draw[very thick,draw=blue] (axis cs: 0,\ka) -- (axis cs: \xc,\ka) node[midway,above,blue] {$k_1$};
\draw[very thick,draw=blue] (axis cs: \xc,\kb) -- (axis cs: 9,\kb) node[midway,above,blue] {$k_2$};
% r(x)
\addplot[draw=none,fill=orange!20] coordinates {
(0,\ra) (\xa,\ra) (\xa,0) (\xb,0) (\xb,\rb) (9,\rb)
}\closedcycle;
\draw[very thick,draw=red] (axis cs: 0,\ra) -- (axis cs: \xa,\ra) node[midway,above,red] {$r_1$};
\draw[very thick,draw=red] (axis cs: \xb,\rb) -- (axis cs: 9,\rb) node[midway,above,red] {$r_2$};
\draw[very thick,draw=red] (axis cs: \xa,0) -- (axis cs: \xb,0) node[midway,above,red] {$r=0$};
\end{axis}
\end{pgfonlayer}{main}
\end{tikzpicture}
% **************************************************
%\end{document}
\caption{Примеры краевых задач с разрывными коэффициентами $k(x)$ и $r(x)$
дифференциального уравнения установившегося распределения температуры 
в неоднородном стержне длиной $\ell$:\\
(a) -- свободный теплообмен стержня с окружающей средой;\\
(b) -- теплоизолированный участок боковой поверхности стержня}
\label{fig:ODE_BC(hetero_material)}
\end{center}
\end{figure}

%
%	РАСЧЕТНАЯ СЕТКА
%
\subsubsection{Расчётная сетка}
Для перехода от дифференциальной краевой задачи 
к системе алгебраических уравнений воспользуемся 
\textit{разностным методом}.
Для этого на отрезке $[0,\ell]$ введем произвольную 
неравномерную сетку -- конечное упорядоченное 
множество точек $\{x_i\}$, принадлежащих этому отрезку:
\begin{gather*}
0=x_0<x_1<x_2<\cdots<x_i<\cdots<x_{N-1}<x_N=\ell,
\end{gather*}
где $x_i\in[0,\ell]$ -- узлы сетки;
$i=0,1,2\dots,N$ -- порядковый номер узла сетки;
$N$ -- количество узлов сетки.

\begin{figure} % *** График сетки ***
\begin{center}
%\input{../../../preamble}
%\begin{document}
%	���� ������ \GraphShiftToRight �� ���������, ��
%	���������� ������ \def\\GraphShiftToRight{3cm}
%
\ifx\GraphShiftToRight\undefined
	% ������ ������� ������� �� ������
	\def\GraphShiftToRight{3cm}
\fi
% ********************************************
%
%	������ �����
%
% ********************************************
\begin{tikzpicture}[framed]
\pgfplotsset{set layers=standard}
\pgfsetlayers{background,axis grid,main,axis ticks,axis lines, axis tick labels,axis descriptions,axis foreground}
% ���� �����
\def\xLeft{0.5}
\def\x{4}
\def\xRight{9.5}
% ������������� ����
\def\xleft{(\x+\xLeft)/2}
\def\xright{\x+\xRight)/2}
\begin{pgfonlayer}{main}
\begin{axis}
[	width=7cm, %height=10cm,
	my graph style,
	name=plot1,
	title={a)},
	xmin=0, xmax=10, xlabel={$x$}, 
	xtick={\xLeft,\xleft,\x,\xright,\xRight}, xticklabels={$x_{i-1}$,$x_{i-\onehalf}$,$x_i$,$x_{i+\onehalf}$,$x_{i+1}$},
	ymin=-3.5, ymax=10, ylabel={$k(x)$}, ytick={3,7}, yticklabels={\empty},
]
\addplot[
	const plot mark right,
	thick,black,
	mark=*, mark size=4pt, mark options={fill=lime,draw=black,solid,thick},
%	nodes near coords,
%	point meta=explicit symbolic,
%	node near coord style={midway,yshift=-1ex,black,fill=olive!10},
] coordinates { (\xLeft,3) (\x,3) (\xRight,7) };
\draw[thick,draw=black,fill=white] (axis cs:\x,7) circle[radius=4pt];
\path (axis cs:\xLeft,3) -- (axis cs:\x,3) node[midway,above,yshift=0.5ex,fill=olive!10] {$k_{i}$};
\path (axis cs:\x,7) -- (axis cs:\xRight,7) node[midway,above,yshift=0.5ex,fill=olive!10] {$k_{i+1}$};
% hi
\draw[draw=red,fill=red] (axis cs:\x,0.75) rectangle (axis cs:\xRight,1)
node[midway,above,red,yshift=0.25ex,fill=olive!10] {$h_{i+1}$};
% hi+1/2
\draw[draw=blue,fill=blue] (axis cs:{\xleft},-2.5) rectangle (axis cs:{\xright},-2.25) 
node[midway,above,blue,fill=olive!10,yshift=0.25ex] {$h_{i+\onehalf}$};
\end{axis}
%
% ������ ������
%
\begin{axis}
[	width=7cm, %height=10cm,
	my graph style,
	name=plot2,
	at={($(plot1.east)+(\GraphShiftToRight,0)$)},
	anchor=west,
	title={b)},
	xmin=0, xmax=10, xlabel={$x$}, 
	xtick={\xLeft,\xleft,\x,\xright,\xRight}, xticklabels={$x_{i-2}$,$x_{i-1}$,$x_i$,$x_{i+1}$,$x_{i+2}$},
	ymin=1, ymax=10, ylabel={$r(x)$}, ytick={3,7}, yticklabels={\empty},
]
\addplot[
	const plot mark right,
	thick,black,
	mark=*, mark size=4pt, mark options={fill=lime,draw=black,solid,thick},
] coordinates { (\xLeft,7) (\xleft,7) (\x,3) (\xright,3) (\xRight,7) };
\draw[thick,draw=black,fill=white] (axis cs:{\xleft},3) circle[radius=4pt];
\draw[thick,draw=black,fill=white] (axis cs:{\xright},7) circle[radius=4pt];
\path (axis cs:{\xLeft},7) -- (axis cs:{\xleft},7) node[midway,above,yshift=0.5ex,fill=olive!10] {$r_{i-1}$};
\path (axis cs:{\xleft},3) -- (axis cs:{\xright},3) node[midway,above,yshift=0.5ex,fill=olive!10] {$r_{i,i+1}=0$};
\path (axis cs:{\xright},7) -- (axis cs:{\xRight},7) node[midway,above,yshift=0.5ex,fill=olive!10] {$r_{i+2}$};
\end{axis}
\end{pgfonlayer}{main}
\end{tikzpicture}
% *****************************************************
%\end{document}
\caption{Схематическое изображение разрывных коэффициентов\\ 
(a) температуропроводности $k(x)$ и
(b) конвекционного теплообмена $r(x)$\\
в узлах расчётной сетки $\{x_i\}$\\
(точки разрыва коэффициентов указаны $\medcirc$ -- маркером)}
\label{fig:ODE_BC(mesh)}
\end{center}
\end{figure}

\begin{tcolorbox}
Если заданы коэффициенты дифференциального уравнения 
и \textit{известны их точки разрывов}, то всегда можно выбрать 
неравномерную сетку так, чтобы точки разрыва 
коэффициентов $k(x)$, $r(x)$, $g(x)$ были узлами расчетной сетки
(рисунок \ref{fig:ODE_BC(mesh)}).
\end{tcolorbox}

Выделим на отрезке $[0,\ell]$ подмножество промежуточных узлов $\{x_{i\pm\onehalf}\}$:
\begin{gather*}
x_{i\pm\onehalf}=\dfrac{x_i+x_{i\pm1}}{2},\quad i=1,2,\dots,N-1
,\end{gather*}
где $x_{i\pm\onehalf}$ -- середины отрезков $[x_{i-1}, x_i]$ и $[x_i, x_{i+1}]$, соответственно.

Расстояние между соседними узлами расчётной сетки $h_i$ (шаг сетки), 
как и расстояние между соседними промежуточными узлами, 
$\overline{h}_{i}$ (шаг промежуточной сетки) зависят от номера узла $i$ сетки
(рисунок \ref{fig:ODE_BC(mesh)}):
\begin{gather*}
\begin{array}{rcl}
h_i&=&x_i-x_{i-1},\\
\overline{h}_{i}&=&x_{i+\onehalf}-x_{i-\onehalf}=\frac{h_{i+1}+h_i}{2}
\end{array}.
\end{gather*}

Значения неизвестной функции $u(x)$ (температуры) рассмотрим в узлах сетки $\{x_i\}$
(рисунок \ref{fig:ODE_BC(u,w)}~a), 
а потоковую величину $w(x)$ (тепловой поток) -- в промежуточных узлах $\{x_{i\pm\onehalf}\}$ 
(рисунок \ref{fig:ODE_BC(u,w)}~b).

\begin{figure} % *** Распределение температуры u(x) и потока тепла w(x) ***
\begin{center}
%\input{../../../preamble}
%\begin{document}
%	���� ������ \GraphShiftToRight �� ���������, ��
%	���������� ������ \def\\GraphShiftToRight{3cm}
%
\ifx\GraphShiftToRight\undefined
	% ������ ������� ������� �� ������
	\def\GraphShiftToRight{3cm}
\fi
% *******************************
%
%	������ ������� u(x), w(x)
%
% *******************************
\begin{tikzpicture}[framed]
\pgfplotsset{set layers=standard}
\pgfsetlayers{background,axis grid,main,axis ticks, axis lines, axis tick labels,axis descriptions,axis foreground}
\begin{pgfonlayer}{main}
\def\xLeft{1}
\def\x{5.5}
\def\xRight{9}
\def\xleft{(\x+\xLeft)/2}
\def\xright{(\x+\xRight)/2}
\begin{axis}
[	width=7cm,
	my graph style,
	name=plot1, title={a)},
	xlabel = {$x$},	% ������� ��� x
	ylabel = {$u(x)$},	% ������� ��� y
	xmin=0, xmax=10, 
	xtick={\xLeft,\xleft,\x,\xright,\xRight}, 
	xticklabels={$x_{i-1}$,$x_{i-\onehalf}$,$x_{i}$,$x_{i+\onehalf}$,$x_{i+1}$},
	ymin=-1, ymax=12, ytick={1,6,9}, yticklabels={\empty},%yticklabels={$u(x_{i-1})$,$u(x_{i})$,$u(x_{i+1})$},
	const plot mark mid,
]
\addplot[
	thick, mark=*, mark size=4pt, mark options={fill=lime, draw=black, solid},
	nodes near coords,
	point meta=explicit symbolic,
	node near coord style={midway,above,yshift=1ex,black,fill=olive!10},
]
coordinates {(\xLeft,6) [$u_{i-1}$] (\x,1) [$u_i$] (\xRight,9) [$u_{i+1}$]};
\end{axis}
% ��������� �������� w(x)
\begin{axis}
[	width=7cm,
	my graph style,
%	������ ������ 	 
	at={($(plot1.east)+(\GraphShiftToRight,0)$)},
	anchor=west,
	title={b)},
% ************
	xlabel={$x$},
	ylabel={$w(x)$},	% ������� ��� y
	xmin=0, xmax=10, 
	xtick={\xLeft,\xleft,\x,\xright,\xRight}, 
	xticklabels={$x_{i-1}$,$x_{i-\onehalf}$,$x_{i}$,$x_{i+\onehalf}$,$x_{i+1}$},	
	ymin=0, ymax=10, ytick={3,7}, yticklabels={\empty},%yticklabels={$w(x_{i-\onehalf})$,$w(x_{i+\onehalf})$},
]
\addplot[const plot,thick,blue,thick] coordinates {(\xLeft,3) ({\x},7) (\xRight,7)};
\addplot[
	only marks,mark=*,mark size=4pt,mark options={fill=blue!20,draw=blue,solid,thick},
	nodes near coords,
	point meta=explicit symbolic,
	node near coord style={midway,above,yshift=1ex,blue,fill=olive!10},
]
coordinates {({\xleft},3) [$w_{i-\onehalf}$] ({\xright},7) [$w_{i+\onehalf}$]};
\end{axis}
\end{pgfonlayer}{main}
\end{tikzpicture}
% *******************************************
%\end{document}
\caption{Схематическое изображение распределения неизвестной функции $u(x)$
и потоковой величины $w(x)$ в пределах контрольной области $x\in[x_{i-1},x_{i+1}]$}
\label{fig:ODE_BC(u,w)}
\end{center}
\end{figure}

%
%	Разностная схема для дифференциального уравнения
%
\subsubsection{Разностная схема для дифференциального уравнения}
Для построения разностной схемы для краевой задачи 
\eqref{eq:ODE_BC0}, \eqref{eq:ODE_BC20}
воспользуемся интегро-интерполяционным методом
(или \textit{методом баланса}) построения разностных схем.

Проинтегрируем \textit{дифференциальное уравнение} \eqref{eq:ODE_BC0}
в пределах контрольной области $[x_{i-\onehalf},x_{i+\onehalf}]$:
\begin{gather*}
\intfCenter{w^{\prime}(x)}+\intfCenter{r(x)\cdot u(x)}=\intfCenter{g(x)}
\end{gather*}

Тогда уравнение теплового баланса для контрольной области 
запишется в виде:
\begin{gather}\label{eq:ODE_BC_CON}
w(x_{i+\onehalf})-w(x_{i-\onehalf})+\intfCenter{r(x)\cdot u(x)}=\intfCenter{g(x)}
\end{gather}

\begin{figure} % *** Баланс тепла ***
\begin{center}
%\input{../../../preamble}
%\begin{document}
% *******************************
%
%	������ �����
%
% *******************************
\begin{tikzpicture}[framed]
% ����������� �������
\draw[shape=rectangle, minimum height=2.5cm, minimum width=4.5cm] 
(0,0) node[thick,color=black,draw,color=black,fill=orange!10,] (A)
{$\int\limits_{x_{i-\onehalf}}^{x_{i+\onehalf}}g(x)dx$};
% ��������� ��������� �������
\draw[-Latex,ultra thick,blue] (A.east) -- +(3,0) node[above,midway] {$w(x_{i+\onehalf})$};
\draw[Latex-,ultra thick,blue] (A.west) -- +(-3,0) node[above,midway] {$w(x_{i-\onehalf})$};
\draw[-Latex,ultra thick,red] (A.north) -- +(0,3) node[right,midway] {$\int\limits_{x_{i-\onehalf}}^{x_{i+\onehalf}}r(x)\cdot u(x)dx$};
% ��� OX
\draw[very thick,black] ($(A.south west)+(-4,-1em)$) -- ($(A.south west) + (0,-1em)$);
\draw[ultra thick,orange] ($(A.south west)+(0,-1em)$) -- ($(A.south east) + (0,-1em)$);
\draw[->,very thick,black] ($(A.south east)+(0,-1em)$) -- ($(A.south east) + (4,-1em)$) node[below,yshift=-1ex,xshift=-1em] {$x$};
% ������� ������� (������������ ������ ��������������)
\draw[very thick,black] ($(A.south west)+(0,-0.75em)$) -- ($(A.south west) + (0,-1.25em)$) node[below] {$x_{i-\onehalf}$};
\draw[ultra thick,orange] ($(A.south)+(0,-0.75em)$) -- ($(A.south) + (0,-1.25em)$) node[below,black] {$x_{i}$};
\draw[very thick,black] ($(A.south east)+(0,-0.75em)$) -- ($(A.south east) + (0,-1.25em)$) node[below] {$x_{i+\onehalf}$};
\end{tikzpicture}
% *******************************************
%\end{document}

\caption{Иллюстрация баланса тепла \eqref{eq:ODE_BC_CON}
на участке стержня $[x_{i-\onehalf},x_{i+\onehalf}]$}
\label{fig:ODE_BC(heat_balance)}
\end{center}
\end{figure}

На рисунке \ref{fig:ODE_BC(heat_balance)} представлена 
графическая иллюстрация процессов переноса тепла, теплообмена
и тепловыделения (теплопоглощения) для контрольной области
$[x_{i-\onehalf},x_{i+\onehalf}]$. 
Рассмотрим физический смысл каждого члена
уравнения теплового баланса \eqref{eq:ODE_BC_CON}:
\begin{enumerate}
\item
первое слагаемое $w(x_{i-\onehalf})$ определяет количество тепла, 
``втекающего`` через сечение $x=x_{i-\onehalf}$;
\item
второе слагаемое $w(x_{i+\onehalf})$ это количество 
``вытекающего`` тепла через сечение $x=x_{i+\onehalf}$;
\item
третье слагаемое в левой части представляет собой количество тепла, 
отдаваемое стержнем внешней среде за счет конвекционного теплообмена 
на его боковой поверхности;
\item
правая честь \eqref{eq:ODE_BC_CON} соответствует количеству тепла, 
выделяющегося на отрезке $[x_{i-\onehalf},x_{i+\onehalf}]$ 
за счет распределенных источников тепла с плотностью $g(x)$.
\end{enumerate}

% теплопроводность
Для определения теплового потока в промежуточных 
узлах сетки $w(x_{i\pm\onehalf})$, воспользуемся 
закон теплопроводности Фурье \eqref{eq:FourierLaw}, 
из которого следует:
\begin{gather}
\notag
u^{\prime}(x)=-\dfrac{w(x)}{k(x)}\quad \Rightarrow \quad
\intf{ x_i }{ x_{i+1} }{ u^{\prime}(x) }
=-\intf{ x_i }{ x_{i+1} }{ \dfrac{w(x)}{k(x)} }
\eqsep
\label{eq:ODE_BC(FourierLaw)}
u(x_{i+1})-u({x_{i}})=-\intf{ x_i }{ x_{i+1} }{ \dfrac{w(x)}{k(x)} }
.\end{gather}

%
%	Приближенное значение интегралов
%
При построении разностной схемы,
в выражениях \eqref{eq:ODE_BC_CON} и \eqref{eq:ODE_BC(FourierLaw)}
необходимо вычислять определенные интегралы.
Однако, во многих практически важных приложениях 
первообразные подынтегральных функций 
не могут быть выражены в элементарных функциях.
Кроме того, коэффициенты краевой задачи 
могут быть заданы только в узлах расчетной сетки
$\{k_i\}$, $\{r_i\}$, $\{g_i\}$.
В этом случае для вычислений определенных интегралов
необходимо пользоваться численным методом интегрирования.

Для этого воспользуемся простейшими интерполяциями 
коэффициентов дифференциального уравнения $k(x)$, $r(x)$, $g(x)$
(рисунок \ref{fig:ODE_BC(mesh)}), неизвестной функции 
$u(x)$ (температуры) и ее потока $w(x)$ (тепловой поток) 
(рисунок \ref{fig:ODE_BC(u,w)}) в окрестности узлов $\{x_i\}$ в виде:
\begin{gather}\label{eq:ODE_BC(approx)}
\left\{\begin{array}{l l}
u(x_i)=\mathrm{const},\quad&x\in(x_{i-\onehalf},x_{i+\onehalf}) \\
w(x_{i+\onehalf})=\mathrm{const},\quad&x\in(x_i,x_{i+1})\\
k(x_i), r(x_i), g(x_i)=\mathrm{const},\quad&x\in(x_i,x_{i+1})
\end{array}\right.
\end{gather}

Тогда левосторонняя формула прямоугольников совпадает с точным значением интегралов
(рисунок \ref{fig:ODE_BC(Integral)}):
\begin{gather}\label{eq:ODE_BC(Integral)}
\intf{ x_i }{ x_{i+1} }{ y(x) }\approx y(x_i)\cdot (x_{i+1}-x_i)=y_i\cdot h_{i+1}
,\end{gather}
где $y_{i}=y(x_{i})$ -- значение функции в левом узле $x_{i-1}$ расчётной сетки.

\begin{figure} % *** Приближенное значение интегралов ***
\begin{center}
%\input{../../../preamble}
%\begin{document}
%
%	���� ������ \GraphShiftToRight �� ���������, ��
%	���������� ������ \def\GraphShiftToRight{3cm}
%
\ifx\GraphShiftToRight\undefined
	% ������ ������� ������� �� ������
	\def\GraphShiftToRight{3cm}
\fi
% ********************************************
%
%	��������� �� ����������� �������
%
% ********************************************
\begin{tikzpicture}[framed]
\pgfplotsset{set layers=standard}
\pgfsetlayers{background,axis grid,pre main,main,axis ticks,axis lines, axis tick labels,axis descriptions,axis foreground}
% ���� �����
\def\xLeft{0.5}
\def\x{4}
\def\xRight{9.5}
% k(x)
\def\yLeft{1}
\def\yRight{7}
% ������������� ����
\def\xleft{(\x+\xLeft)/2}
\def\xright{\x+\xRight)/2}
\begin{pgfonlayer}{main}
\begin{axis}
[	width=7cm, %height=10cm,
	my graph style,
	name=plot1,
	title={a)},
	xmin=0, xmax=10, xlabel={$x$}, 
	xtick={\xLeft,\xleft,\x,\xright,\xRight}, xticklabels={$x_{i-1}$,$x_{i-\onehalf}$,$x_i$,$x_{i+\onehalf}$,$x_{i+1}$},
	ymin=-3.5, ymax=10, ylabel={$k^{-1}(x)$}, ytick={\yLeft,\yRight}, yticklabels={\empty},
]
% ��������
\draw[draw=none,fill=orange!20] (axis cs: \x,-3.5) rectangle (axis cs: \xRight,\yRight);
\draw[thick,draw=black,fill=white] (axis cs:{\xright},-3.5) 
node[above] {$\int\limits_{x_{i}}^{x_{i+1}}\dfrac{dx}{k(x)}$};
\addplot[
	const plot mark right,
	thick,black,
	mark=*, mark size=4pt, mark options={fill=lime,draw=black,solid,thick},
%	nodes near coords,
%	point meta=explicit symbolic,
%	node near coord style={midway,yshift=-1ex,black,fill=olive!10},
] coordinates { (\xLeft,\yLeft) (\x,\yLeft) (\xRight,\yRight) };
\draw[thick,draw=black,fill=white] (axis cs:\x,\yRight) circle[radius=4pt];
\path (\xLeft,\yLeft) -- (\x,\yLeft) node[midway,above,fill=olive!10] {$k_{i}$};
\path (\x,\yRight) -- (\xRight,\yRight) node[midway,above,fill=olive!10] {$k_{i+1}$};
\end{axis}
%
% ������ ������
%
\def\yLeft{3}
\def\yRight{7}
\begin{axis}
[	width=7cm, %height=10cm,
	my graph style,
	name=plot2,
	at={($(plot1.east)+(\GraphShiftToRight,0)$)},
	anchor=west,
	title={b)},
	xmin=0, xmax=10, xlabel={$x$}, 
	xtick={\xLeft,\xleft,\x,\xright,\xRight}, xticklabels={$x_{i-1}$,$x_{i-\onehalf}$,$x_i$,$x_{i+\onehalf}$,$x_{i+1}$},
	ymin=-3, ymax=9, ylabel={$r(x)$}, ytick={\yLeft,\yRight}, yticklabels={\empty},
]
\draw[draw=none,fill=orange!20] (axis cs: {\xleft},-3) rectangle (axis cs: \x,\yLeft);
\draw[draw=none,fill=orange!20] (axis cs: \x,-3) rectangle (axis cs: {\xright},\yRight);
\addplot[
	const plot mark right,
	thick,black,
	mark=*, mark size=4pt, mark options={fill=lime,draw=black,solid,thick},
] coordinates { (\xLeft,\yLeft) (\x,\yLeft) (\xRight,\yRight) };
\draw[thick,draw=black,fill=white] (axis cs:{\x},\yRight) circle[radius=4pt];
\draw[thick,draw=black,fill=white] (axis cs:\x,-3) 
node[above,xshift=1.5ex] {$\int\limits_{x_{i-\onehalf}}^{x_{i+\onehalf}}r(x)dx$};
\path ({\xleft},\yLeft) -- (\x,\yLeft) node[midway,above] {$r_{i}$};
\path (\x,\yRight) -- ({\xright},\yRight) node[midway,above] {$r_{i+1}$};
\end{axis}
\end{pgfonlayer}{main}
\end{tikzpicture}
% *****************************************************
%\end{document}
\caption{Приближенное вычисление значения определенных интегралов
в уравнении баланса тепла \eqref{eq:ODE_BC_CON}}
\label{fig:ODE_BC(Integral)}
\end{center}
\end{figure}

Приближенное значение интеграла в \eqref{eq:ODE_BC(FourierLaw)}
выразим воспользовавшись кусочной аппроксимацией \eqref{eq:ODE_BC(approx)}
теплового потока $w(x)\approx w(x_{i+\onehalf})$ на интервале $(x_{i},x_{i+1})$
и левосторонней формулой прямоугольников 
\eqref{eq:ODE_BC(Integral)}:
\begin{gather}
\begin{array}{rcl}
u(x_{i+1})-u(x_{i})&\approx&-\dfrac{w(x_{i+\onehalf})}{k_{i+1}}\cdot h_{i+1}
\end{array}
,\end{gather}
и найдем поток тепла в промежуточных узлах сетки $w(x_{i+\onehalf})$:
\begin{gather}\label{eq:ODE_BC(w)}
w(x_{i+\onehalf})\approx -{k}_{i+1}\cdot\dfrac{u(x_{i+1})-u(x_{i})}{h_{i+1}}
\end{gather}
где $k_{i+1}=k(x_{i+1})$ -- значение коэффициента 
температуропроводности в узлах расчётной сетки $\{x_{i+1}\}$.

% Конвекционный теплообмен
Приближенное значение интегрального (суммарного) 
конвекционного теплового потока через  
боковую поверхность контрольной области
$[x_{i-\onehalf},x_{i+\onehalf}]$:
\begin{gather*}
\intfCenter{ r(x)\cdot u(x) }\approx u(x_i)\cdot\overline{r}_i
,\end{gather*}
где $\overline{r}_i$ -- интегральный коэффициент конвекционного теплообмена:
\begin{gather*}
\overline{r}_i=\intfCenter{ r(x) }
.\end{gather*}

Для определения интегрального коэффициента конвекционного теплообмена
$\overline{r}_i$ используем кусочную аппроксимацию \eqref{eq:ODE_BC(approx)}, 
а также условие расположения узлов сетки, таких что коэффициент 
$r(x)$ имеет \textit{точки разрыва только в узлах расчётной сетки}:
\begin{gather}\label{eq:ODE_BC(<r>)}
% <r>
\overline{r}_i=\bracketssquare{\hspace{1ex}\intfleft{ r(x) }+\intfright{ r(x) } }
=\dfrac{r_{i}h_i+r_{i+1}h_{i+1}}{2}
,\end{gather}
где $r_i=r(x_i)$ -- коэффициент конвекционного теплообмена
стержня с окружающей средов в узлах сетки $\{x_i\}$.

% Распределенные источники
Суммарное количество тепла $\overline{g}_i$,
выделяемого распределенными источниками
в пределах контрольной области $[x_{i-\onehalf},x_{i+\onehalf}]$,
определяется аналогично:
\begin{gather}\label{eq:ODE_BC(<g>)}
% <g>
\overline{g}_i=\intfCenter{ g(x) }\approx\dfrac{g_{i}h_i+g_{i+1}h_{i+1}}{2}
\end{gather}
где $g_i=g(x_i)$ -- плотность распределенных источников тепла 
в узлах сетки $\{x_i\}$.

Таким образом, баланс тепловой энергии на участке стержня 
$[x_{i-\onehalf}$, $x_{i+\onehalf}]$ с учетом соотношений
для переноса тепла за счет теплопроводности материала стержня
\eqref{eq:ODE_BC(w)}, конвекционного теплообмена с окружающей средой
\eqref{eq:ODE_BC(<r>)} и распределенных источников
\eqref{eq:ODE_BC(<g>)}, можно записать в виде системы разностных уравнений: 
\begin{gather}\label{eq:ODE_BC(FD)}
\left\{\begin{matrix}
w(x_{i+\onehalf}) - w(x_{i-\onehalf}) + \overline{r}_i\cdot u(x_i) = \overline{g}_i
\eqsep
w(x_{i+\onehalf})=-{k}_{i+1}\cdot\dfrac{u(x_{i+1})-u(x_{i})}{h_{i+1}}
\end{matrix}\right.
\end{gather} 
 
\begin{tcolorbox}%[colback=blue!10]
Система уравнений \eqref{eq:ODE_BC(FD)} по своему построению является разностным аналогом 
основного дифференциального уравнения \eqref{eq:ODE_BC0}.
Записывая уравнение \eqref{eq:ODE_BC(FD)} во всех узлах сетки, в которых оно определено
($i=1,2,\dots, N-1$), получим систему из $2N-1$ линейных алгебраических уравнений 
относительно $2N+1$ неизвестных $u(0), u(x_1),\dots, u(x_i),\dots, u(x_{N-1}),u(\ell)$.
Два недостающих уравнения получаются путем \textit{аппроксимации 
краевых условий} \eqref{eq:ODE_BC20}. 
\end{tcolorbox}

%
%	Разностная схема для краевых условий
%
\subsubsection{Разностная аппроксимация краевых условий}
Воспользуемся \textit{интегро-интерноляционным методом} и 
проинтегрируем основное уравнение \eqref{eq:ODE_BC0} 
вблизи левой границы рассматриваемой области $x=0$:
\begin{gather*}
\intf{0}{ x_{1/2} }{ w^{\prime}(x) }
+\intf{0}{ x_{1/2} }{ r(x)u(x) }
=\intf{0}{ x_{1/2} }{ g(x) }
\end{gather*}

Тогда уравнение теплового баланса близи границ области $x=0$ и $x=\ell$:
\begin{gather}\label{eq:ODE_BC(LEFT)}
w(x_{\onehalf})-w(0)+\overline{r}_0u(0)=\overline{g}_0
,\end{gather}
где
$w(x_{\onehalf})$ -- тепловой поток в первом промежуточном узле $x_{\onehalf}$;
$w(0)$ -- тепловой поток на левой границе области;
$\overline{r}_0$ и $\overline{g}_0$ -- интегральный коэффициент
конвекционного теплообмена стержня с окружающей средой 
и количество тепла выделяемое распределенными источникам 
на левой границе $x=0$:
\begin{gather}\label{eq:ODE_BC(r0,g0)}
\overline{r}_0=\intf{0}{ x_{1/2} }{ r(x) }=\dfrac{r_0h_1}{2},
\quad%\eqsep
\overline{g}_0=\intf{0}{ x_{1/2} }{ g(x) }=\dfrac{g_0h_1}{2}.
\end{gather}

Аналогично рассматривая правую границу области $x=\ell$,
уравнение теплового баланса можно записать в виде:
\begin{gather}\label{eq:ODE_BC(RIGHT)}
w(\ell)-w(x_{N-\onehalf})+\overline{r}_N u(\ell)=\overline{g}_N
,\end{gather}
где
$w(\ell)$ -- тепловой поток на правой границе области;
$w(x_{N-\onehalf})$ -- тепловой поток в последнем промежуточном узле $x_{N-\onehalf}$;
$\overline{r}_N$ и $\overline{g}_N$ -- интегральный коэффициент
конвекционного теплообмена стержня с окружающей средой 
и количество тепла выделяемое распределенными источникам 
на левой границе $x=\ell$:
\begin{gather}\label{eq:ODE_BC(rN,gN)}
\overline{r}_{N}=\intf{x_{N-1/2}}{ \ell }{ r(x) }=\dfrac{r_{N}h_N}{2},
\quad%\eqsep
\overline{g}_{N}=\intf{x_{N-1/2}}{ \ell }{ g(x) }=\dfrac{g_{N}h_N}{2}.
\end{gather}

%
%	РАЗНОСТНАЯ СХЕМА
%
Объединяя все разностные соотношения 
для дифференциального уравнения \eqref{eq:ODE_BC(FD)} 
и граничных условий \eqref{eq:ODE_BC20}, \eqref{eq:ODE_BC(LEFT)}, \eqref{eq:ODE_BC(RIGHT)}
получаем следующую разностную схему 
для граничной задачи \eqref{eq:ODE_BC0}, \eqref{eq:ODE_BC20}:
\begin{tcolorbox}
\begin{gather}\label{eq:ODE_BC(LIN_SYS)}
\left\{\begin{matrix}
w(x_{\onehalf})+\brackets{ \overline{r}_0+\lambda_1}u(0)=\overline{g}_0 + \mu_1
\eqsep
w(x_{i+\onehalf}) - w(x_{i-\onehalf}) + \overline{r}_i\cdot u(x_i) = \overline{g}_i
\eqsep
w(x_{i+\onehalf})=-{k}_{i+1}\cdot\dfrac{u(x_{i+1})-u(x_{i})}{h_{i+1}}
\eqsep
-w(x_{N-\onehalf})+\brackets{ \overline{r}_N+\lambda_2}u(\ell)=\overline{g}_N + \mu_2
\end{matrix}\right.
.\end{gather}
\end{tcolorbox}

\subsection{Метод решения систем линейных уравнений с матрицами специального вида}

Если исходной задачей является краевая задача для 
обыкновенного дифференциального уравнения, 
то соответствующую разностную схему можно решить с помощью метода прогонки.

В многомерном случае не существует столь же удобного 
и экономичного способа решения разностных уравнений, как метод 
прогонки. Поэтому возникает необходимость в развитии методов, 
специально предназначенных для решения многомерных разностных 
краевых задач. Мы будем рассматривать здесь лишь двумерные 
разностные задачи. 
Как и в общем случае систем линейных уравнений, 
методы решения разностных задач разделяются на прямые и итерационные. 
Итерационные методы являются более простыми, чем прямые, и в 
меньшей степени используют структуру матрицы. 
По этой причине для решения двумерных разностных уравнений 
первоначально использовались исключительно итерационные методы. 
Однако в случае разностных задач сходимость таких, например, методов, как 
метод простой итерации, Зейделя, верхней релаксации, 
весьма медленная. 
В настоящее время интенсивно развиваются и 
прямые методы решения двумерных разностных уравнений. 
Они применимы, как правило, к уравнениям с разделяющимися 
переменными, когда область изменения независимых переменных 
является прямоугольник. 

% *************************************************************
%
%	МЕТОД ПРОГОНКИ
%
% *************************************************************
\subsubsection{Метод прогонки для трехточечных уравнений}
Система уравнений \eqref{eq:ODE_BC_FD} представляет собой 
частный случай систем линейных алгебраических уравнений:
\begin{gather*}
\vec{A}\cdot\vec{u}=\vec{f}
\end{gather*}
с трехдиагональной матрицей $\vec{А}$, т.е. с матрицей, 
все элементы которой, не лежащие на главной и двух побочных диагоналях, равны нулю.
\begin{gather}\label{eq:ODE_BC_SYS}
\begin{pmatrix}
-c_0&b_0&0&\dots&\dots&\dots&0\\
a_1&-c_1&b_1&0&\dots&\dots&0\\
0&a_2&-c_2&b_2&0&\cdots&0\\
\dots&\dots&\dots&\dots&\dots&\dots&\dots\\
0&\dots&0&a_{N-2}&-c_{N-2}&b_{N-2}&0\\
0&\dots&\dots&0&a_{N-1}&-c_{N-1}&b_{N-1}\\
0&\dots&\dots&\dots&0&a_N&-c_N\\
\end{pmatrix}
\cdot
\begin{pmatrix}
u_0\\u_1\\u_2\\ \cdots\\u_{N-2}\\u_{N-1}\\u_{N}
\end{pmatrix}
=
\begin{pmatrix}
f_0\\f_1\\f_2\\ \cdots\\f_{N-2}\\f_{N-1}\\f_{N}
\end{pmatrix}
,\end{gather}
где $u_i=u(x_i)$ -- значение неизвестной функции в узлах сетки.

В общем случае системы линейных алгебраических уравнений 
с трехдиагональной матрицей имеют вид:
\begin{gather}\label{eq:ODE_BC_SOL}
a_i\cdot u_{i-1} - c_i\cdot u_{i} + b_i\cdot u_{i+1} = -f_i,\quad i=1,2,\dots,N-1
\end{gather}

Для численного решения систем с трехдиагональными матрицами 
применяется \textit{метод прогонки}, который представляет 
собой вариант метода последовательного исключения неизвестных.

Особенно широкое применение метод прогонки получил при решении 
систем разностных уравнений, возникающих при аппроксимации 
краевых задач для дифференциальных уравнений второго порядка.

Решение системы линейных уравнений \eqref{eq:ODE_BC_SOL}
будем искать в виде:
\begin{gather}\label{eq:ODE_ALPHA}
u_i=\alpha_{i+1}\cdot u_{i+1} + \beta_{i+1}
,\end{gather}
где $\alpha_{i+1}$ и $\beta_{i+1}$ -- неизвестные коэффициенты, 
которые необходимо определить.

Пользуясь уравнением \eqref{eq:ODE_ALPHA} и выражая $u_{i-1}$ получим:
\begin{gather}\label{eq:ODE_ALPHA2}
u_{i-1}=\alpha_{i}\cdot u_{i} + \beta_{i}=
\alpha_{i}~\alpha_{i+1}\cdot u_{i+1}+(\alpha_{i}~\beta_{i+1} + \beta_{i})
.\end{gather}

Подставляя соотношения для $u_{i-1}$ \eqref{eq:ODE_ALPHA2}
и $u_{i}$ \eqref{eq:ODE_ALPHA} в выражение \eqref{eq:ODE_BC_SOL} получим:
\begin{gather*}
a_i\cdot(\alpha_{i}~\alpha_{i+1}\cdot u_{i+1}+(\alpha_{i}~\beta_{i+1} + \beta_{i}))
-c_i\cdot(\alpha_{i+1}\cdot u_{i+1} + \beta_{i+1})
+b_i\cdot u_{i+1} = -f_i
\end{gather*}

Это уравнение будет выполнено для всех $i=1,2,\dots,N-1$, 
если потребовать равенство нулю всех коэффициентов 
при неизвестных $u_{i+1}$ и свободных членов:
\begin{gather*}
\begin{cases}
(a_i\cdot\alpha_{i}-c_i)\cdot\alpha_{i+1}=-b_i\\
(a_i\cdot\alpha_{i}-c_i)\cdot\beta_{i+1}=-(a_i\cdot\beta_i+f_i)
\end{cases}
\end{gather*}

Из последнего выражения получаем рекуррентные соотношения 
для определения значений неизвестных коэффициентов
$\alpha_{i+1}$ и $\beta_{i+1}$:
\begin{gather}\label{eq:ODE_ALPHA_BETA}
\alpha_{i+1}=\dfrac{b_i}{c_i-a_i\cdot\alpha_{i}},\quad
\beta_{i+1}=\dfrac{a_i\cdot\beta_i+f_i}{c_i-a_i\cdot\alpha_{i}}
\end{gather}

Соотношения \eqref{eq:ODE_ALPHA_BETA} представляют собой 
нелинейные разностные уравнения первого порядка.
Для их решения необходимо задать начальные значения 
$\alpha_1$ и $\beta_1$.
Эти начальные значения находим из краевого условия на левой
границе рассматриваемой области $x=0$:
\begin{gather*}
\begin{cases}
-c_0\cdot u_0+b_0\cdot u_1 = -f_0\\
u_0=\alpha_{1}\cdot u_{1} + \beta_{1}
\end{cases}
\quad\text{или}\quad
\begin{cases}
u_0=\dfrac{b_0}{c_0}\cdot u_1+\dfrac{f_0}{c_0}\\
u_0=\alpha_{1}\cdot u_{1} + \beta_{1}
\end{cases}
.\end{gather*}

Сопоставляя выражение для $u_0$ получим:
\begin{gather}\label{eq:ODE_ALPHA1_BETA1}
\alpha_{1}=\dfrac{b_0}{c_0},\quad \beta_{1}=\dfrac{f_0}{c_0}
\end{gather}

Нахождение коэффициентов $\alpha_{i+1}$ и $\beta_{i+1}$ 
по рекуррентным соотношениям 
\eqref{eq:ODE_ALPHA_BETA} и \eqref{eq:ODE_ALPHA1_BETA1}
называется \textit{прямой прогонкой}.
После того как прогоночные коэффициенты $\alpha_{i+1}$ и $\beta_{i+1}$
($i=1,2,\dots,N-1$), найдены, решение системы \eqref{eq:ODE_BC_SYS} 
находится по рекуррентному соотношению \eqref{eq:ODE_ALPHA},
если известно значение функции $u_N$ на правой границе области.
Неизвестное значение функции $u_N$ можно определить 
из краевого условия на правой границе области $x=\ell$:
\begin{gather}\label{eq:ODE_SYS_uN}
\begin{cases}
a_N\cdot u_{N-1}-c_N\cdot u_N = -f_N\\
u_{N-1}=\alpha_{N}\cdot u_{N} + \beta_{N}
\end{cases}
.\end{gather}

Из решения системы уравнений \eqref{eq:ODE_SYS_uN} 
определяется значение неизвестной функции $u_N$
на правой границе:
\begin{gather}\label{eq:ODE_uN}
u_N=\dfrac{a_N\cdot\beta_N+f_N}{c_N-a_N\cdot\alpha_N}
\end{gather}

Последовательное нахождение значений 
неизвестной функции $u(x)$ в узлах сетки 
по рекуррентному соотношению \eqref{eq:ODE_ALPHA}
называется \textit{обратной прогонкой}.


% *************************************************************
%
%	МЕТОД ПОТОКОВОЙ ПРОГОНКИ
%
% *************************************************************
\subsubsection{Потоковый вариант метода прогонки}
Потоковый вариант метода прогонки применяется при решении задач 
с сильно меняющимися коэффициентами в выражениях для потоков
неизвестных величин (элактрического заряда, количества тепла, количества жидкости и т.д.).
Например, в задачах гидродинамики с теплопроводностью и магнитной гидродинамики, 
коэффициенты теплопроводности и электропроводности могут сильно зависят 
от термодинамических параметров среды.
В случае тепловых задач в пределах рассматриваемых областей 
могут иметь место адиабатические участки, где теплопроводность отсутствует, а также 
изотермические участки с бесконечно высоким коэффициентом теплопроводности.
В задачах магнитной гидродинамики могут рассматриваться области 
с идеально проводящими и изолирующими участками.

Часто в таких задачах, помимо неизвестной функции (решения задачи), 
требуется найти еще и ее поток (например, электричества, тепла, жидкости и т.д.).
При решении разностных уравнений второго порядка, 
к которым сводятся разностные схемы для этих задач, 
с помощью метода обычной прогонки часто происходит значительная потеря точности. 
Последующее использование численного дифференцирования для вычисления потока 
приводит к неудовлетворительному результату. 
Избавиться от этого недостатка удается путем перехода к так называемому 
потоковому варианту метода прогонки. 

Предположим наличие между искомой функцией $u$ и ее потоком $w$ связи вида:
\begin{gather*}
\alpha\cdot u + \beta\cdot w = \gamma
\end{gather*}

Так как коэффициенты $\alpha$, $\beta$ и $\gamma$ определены с точностью 
до множителя, то на функции $\alpha$ и $\beta$ необходимо наложить дополнительное условие,
в зависимости от типа краевой задачи, характера коэффициента и т.д.
Например, иногда удобно считать $\alpha=1$ либо $\beta=1$
или полагать:
\begin{gather*}
\alpha + \beta\cdot C(x) = 1
,\end{gather*}
где $C(x)>0$ -- некоторая функция.
 
Будем искать решение системы уравнений \eqref{eq:ODE_BC(LIN_SYS)} в том же виде, 
в котором заданы краевые условия:
\begin{gather}
u(x_i) = \alpha_i\cdot w(x_{i+\onehalf}) + \beta_i
,\end{gather}
где $\alpha_i$ и $\beta_i$ -- неизвестные коэффициенты.
 
 




% **************************************************
%
%	Таблица коэффициентов системы разностных уравнений
%
\begin{table}[h]
\caption{
Выражения для коэффициентов системы разностных уравнений \eqref{eq:ODE_BC_FD}
краевой задачи \eqref{eq:ODE_BC0}, \eqref{eq:ODE_BC20}
}
\label{tab:ODE_BC_FD}
\begin{tabular}{p{1cm} p{5cm} p{6cm} p{3cm}}
\toprule
$i$&0&$1,2,3,\dots,N-1$&$N$\\
\midmidrule
$h_i$&&$(x_i-x_{i-1})$&$(x_{N}-x_{N-1})$\\
\midrule
% k
$\overline{k}_i$&&$\dfrac{2k_{i}k_{i-1}}{k_{i}+k_{i-1}}$&\\[1em]
% q
$\tilde{v}_i$&$v_0\cdot\dfrac{h_1}{2}$&$v_i\cdot\overline{h_i}$&\\[1em]
% f
$\tilde{g}_i$&$g_0\cdot\dfrac{h_1}{2}$&$g_i\cdot\overline{h_i}$&\\
\midrule
$a_i$&&$\dfrac{\overline{k}_i}{h_i}$&0\\[1em]
$b_i$&$a_1$&$a_{i+1}$&\\[1em]
$c_i$
&$a_1+\tilde{v}_0-\lambda_1$
&$(a_i+b_i)+\tilde{v}_i$
&1\\[1em]
%f
$f_i$
&$\tilde{g}_0+\mu_1$
&$\tilde{g}_i$
&$\mu_2$\\
\bottomrule
\end{tabular}
\end{table}
% **************************************
%
%	Разностная схема в МАТРИЧНОЙ ФОРМЕ
%
Запишем разностную схему в матричной форме:
\begin{gather}\label{eq:ODE_BC_FD}
\left\{\begin{array}{lclclcll}
&-&c_0\cdot u(0)&+&b_0\cdot u(x_{1})&=&-f_0\\
a_{1}\cdot u(0)&-&c_{1}\cdot u(x_{1})&+&b_{1}\cdot u(x_{2})&=&-f_{1}\\
\vdots&&\vdots&&\vdots&&\vdots\\
a_{i}\cdot u(x_{i-1})&-&c_{i}\cdot u(x_{i})&+&b_{i}\cdot u(x_{i+1})&=&-f_{i}\\
\vdots&&\vdots&&\vdots&&\vdots\\
a_{N-1}\cdot u(x_{N-2})&-&c_{N-1}\cdot u(x_{N-1})&+&b_{N-1}\cdot u(\ell)&=&-f_{N-1}\\
a_{N}\cdot u(x_{N-1})&-&c_N\cdot u(\ell)&&&=&-f_N
\end{array}\right.
,\end{gather}
где $a$, $c$, $b$ и $f$ -- коэффициенты при незвестных ${u(x_i)}$ 
и правая часть системы линейных уравнений, 
выражения для которых сведены в таблицу \eqref{tab:ODE_BC_FD}:


\begin{tcolorbox}
Алгоритм метода прогонки:
\begin{enumerate}
\item
Составить однородную разностную схему 
для решения исходной краевой задачи методом баланса 
(интегро-интерполяционным методом)
\item
На неравномерной сетке определить шаг $h_i$ сетки для каждого 
элементарного отрезка $[x_{i-1},x_{i}]$
\item
Вычислить все коэффициенты разностной схемы 
$\overline{k}_i$, $\overline{q}_i$, $\overline{f}_i$
краевой задачи во всех узлах сетки $\{x_i\}$ $(i=0,1,2,\dots,N)$
\item
Определить коэффициенты линейной системы уравнений
$a_i$, $b_i$, $c_i$ и $f_i$ $(i=0,1,2,\dots,N)$
полученной разностной схемы.
\item
Найти решение полученной системы линейных алгебраических уравнений
$\{u_i\}$ методом прогонки:
\begin{enumerate}[label=(\roman*)]
\item
Из краевого условия на левой границе области $x=0$
вычислить коэффициенты \eqref{eq:ODE_ALPHA1_BETA1}:
\begin{gather*}
\alpha_{1}=\dfrac{b_0}{c_0},\quad \beta_{1}=\dfrac{f_0}{c_0}
\end{gather*}
\item
Пользуясь рекуррентными соотношениями \eqref{eq:ODE_ALPHA_BETA},
определить значения неизвестных коэффициентов для $i=2,3,\dots,N-1$:
\begin{gather*}
\alpha_{i+1}=\dfrac{b_i}{c_i-a_i\cdot\alpha_{i}},\quad
\beta_{i+1}=\dfrac{a_i\cdot\beta_i+f_i}{c_i-a_i\cdot\alpha_{i}}
\end{gather*}
\item
Из краевого условия на правой границе области $x=\ell$ определить
значение неизвестной функции $u_N$ из соотношения \eqref{eq:ODE_uN}:
\begin{gather*}
u_N=\dfrac{a_N\cdot\beta_N+f_N}{c_N-a_N\cdot\alpha_N}
\end{gather*}
\item
Определить значения неизвестной функции в узлах сетки
пользуясь рекуррентными соотношениями \eqref{eq:ODE_ALPHA}:
\begin{gather*}
u_i=\alpha_{i+1}\cdot u_{i+1} + \beta_{i+1}
\end{gather*}
\end{enumerate}
\end{enumerate}
\end{tcolorbox}

%
%	ЧИСЛЕННОЕ РЕШЕНИЕ краевой задачи методом прогонки
%
\subsection{Численное решение краевой задачи методом прогонки}
Рассмотрим краевую задачу на отрезке $x\in[0,7]$:
\begin{gather*}
\left((3+\sin(x))\cdot u^{\prime}(x)\right)^{\prime}-|\cos(x)|\cdot u(x) + 0.5\cdot (x-1)^2=0\\
\left\{\begin{matrix}
u(0)=2\\
-k(7)\cdot u^{\prime}(7)+0,5\cdot u(7)=17,24
\end{matrix}\right.
\end{gather*}

Из данных краевой задачи можно заключить, что коэффициенты 
в дифференциальном уравнении являются функциями вида:
\begin{gather*}
k(x)=3+\sin(x),\quad q(x)=|\cos(x)|,\quad f(x)=0.5\cdot (x-1)^2
\end{gather*}

Для построения разностной схемы для решения краевой задачи,
воспользуемся интегро-интерполяционным методом (методом баланса).
\begin{enumerate}
\item
Разностный аналог дифференциального уравнения краевой задачи
получаем аналогично \eqref{eq:ODE_BC_CON2}:
\begin{gather*}
-\overline{k}_{i+1}\cdot\dfrac{u(x_{i+1})-u(x_{i})}{h_{i+1}}
+\overline{k}_i\cdot\dfrac{u(x_{i})-u(x_{i-1})}{h_i}
+\overline{q}_i\cdot\dfrac{h_{i+1}+h_i}{2}\cdot u(x_i)
=\overline{f}_i\cdot\dfrac{h_{i+1}+h_i}{2}% h_{i+\onehalf}
\end{gather*} 
\item
На левой границе области $x=0$ граничное условие имеет вид:
\begin{gather}\label{eq:ODE_BC_LEFT_MY}
u(0)=2
\end{gather}
\item
Вблизи правой границы на отрезке $x\in[x_{N-\onehalf},\ell]$
воспользуемся \textit{интегро-интерноляционным методом} и 
проинтегрируем основное уравнение \eqref{eq:ODE_BC0} задачи:
\begin{gather*}
\int\limits_{x_{N-\onehalf}}^{\ell}w^{\prime}(x)dx
+\int\limits_{x_{N-\onehalf}}^{\ell}q(x)\cdot u(x)dx
=\int\limits_{x_{N-\onehalf}}^{\ell}f(x)dx
,\end{gather*}
где $w(x)=-k(x)\cdot u^{\prime}(x)$ -- поток неизвестной функции $u(x)$.

Предположим, что в пределах отрезка $[x_{N-\onehalf},\ell]$ 
неизвестная функция $u(x)$ остается постоянной $u(x)\approx u(\ell)$
(рисунок \ref{fig:ODE_BC_RIGHT_SOL}),
тогда краевое условие на правой границе запишется в виде:
\begin{gather}\label{eq:ODE_BC_RIGHT_SOL}
w(\ell)-w({x_{N-\onehalf}})
+u(\ell)\cdot\int\limits_{x_{N-\onehalf}}^{\ell}q(x)dx
=\int\limits_{x_{N-\onehalf}}^{\ell}f(x)dx
,\end{gather}
где $w({x_{N-\onehalf}})$ -- потоковая величина 
в последнем промежуточном узле $x_{N-\onehalf}$
определяется из соотношения:
\begin{gather*}
u^{\prime}(x)=-\dfrac{w(x)}{k(x)}\quad \Rightarrow \quad
\int\limits_{x_{N-1}}^{\ell}u^{\prime}(x)dx
=-\int\limits_{x_{N-1}}^{\ell}\dfrac{w(x)}{k(x)}dx
\end{gather*}

Предполагая, что тепловой поток $w(x)$ постоянен в пределах области
$[x_{N-1},\ell]$, т.е. $w(x)\approx w(x_{N-\onehalf})$, определим
изменение неизвестной функции $u(x)$ в этой области:
\begin{gather*}
u(\ell)-u(x_{N-1})=-w(x_{N-\onehalf})\cdot\int\limits_{x_{N-1}}^{\ell}\dfrac{dx}{k(x)}
\end{gather*}

Для вычисление определенного интеграла в последнем выражении
воспользуемся формулой трапеций:
\begin{gather*}
\int\limits_{x_{N-1}}^{\ell}\dfrac{dx}{k(x)}
\approx\dfrac{h_N}{2}\cdot\brackets{ \dfrac{1}{k_{N-1}}+\dfrac{1}{k_{N}} }  
\end{gather*}

%
%	ПРАВАЯ ГРАНИЦА
%
\begin{figure}
\begin{center}
\begin{tikzpicture}[background rectangle/.style={fill=olive!10}, show background rectangle]
\def\ul{7.5}
\def\xl{9.5}
\def\xxl{4}
\def\uul{2	}
\def\xxxl{0.5}
\def\uuul{5}
\begin{axis}
[
%	axis on top=true,
	every axis/.style={color=black, solid, thick},
	ytick pos=right,			% ось OY - справа
	ylabel = {\empty},	% подпись оси y
	xmin=0, xmax=10, xtick={\xxxl,(\xxxl+\xxl)/2,\xxl,(\xxl+\xl)/2,\xl}, xticklabels={$x_{N-2}$,$x_{N-\nicefrac{3}{2}}$,$x_{N-1}$,$x_{N-\onehalf}$,$\ell$},
	ymin=0, ymax=10, ytick={\uuul,\uul,\ul}, yticklabels={$u(x_{N-2})$,$u(x_{N-1})$,$u(\ell)$},
	xtick style={thick, black},
	ytick style={thick, black},
	grid=major,		
	major grid style={color=black!20, dashed, thin},
]
\addplot[fill=black!10, draw=none] coordinates {(\xl,0) (\xl,10) ((\xl+\xxl)/2,10) ((\xl+\xxl)/2,0)}\closedcycle;
\addplot[only marks, mark=*, mark size=5pt, 
mark options={fill=lime, draw=black, solid},
const plot mark mid,
very thick,
]
coordinates {(\xxxl,\uuul) (\xxl,\uul) (\xl,\ul)};
\draw[-] (axis cs:{(\xl+\xxl)/2},0) -- (axis cs:\xl,0) node[midway,above] {$\dfrac{h_N}{2}$};
\end{axis}
\end{tikzpicture}
\caption{
Схематическое изображение распределения неизвестной функции $u(x)$
вблизи правой границы $x=\ell$
}
\label{fig:ODE_BC_RIGHT_SOL}
\end{center}
\end{figure}
%*******************************************

Следовательно, потоковая величина в последнем промежуточном узле 
определяется выражением:
\begin{gather}\label{eq:ODE_BC_RIGHT2}
w({x_{N-\onehalf}})=-\overline{k}_N\cdot\dfrac{u(\ell)-u(x_{N-1})}{h_N}
,\end{gather}
где $\overline{k}_N$ -- среднее значение коэффициента $k(x)$ на отрезке
$[x_{N-1},\ell]$:
\begin{gather*}
\overline{k}_N=\dfrac{2k_{N}k_{N-1}}{k_{N}+k_{N-1}}
\end{gather*}

Из общего вида второго краевого условия
\begin{gather*}
-k(\ell)\cdot u^{\prime}(\ell)+\lambda_2\cdot u(\ell)=\mu_2
\end{gather*}
выразим величину потока на границе рассматриваемой области $w(\ell)$:
\begin{gather}\label{eq:ODE_BC_RIGHT3}
w(\ell)=-k(\ell)\cdot u^{\prime}(\ell) = \mu_2-\lambda_2\cdot u(\ell)
\end{gather}

Для вычислений определенных интегралов в
\eqref{eq:ODE_BC_RIGHT_SOL}
воспользуемся формулой прямоугольников:
\begin{gather}\label{eq:ODE_BC_RIGHT4}
\int\limits_{x_{N-\onehalf}}^{\ell}q(x)dx\approx q_N\cdot\dfrac{h_N}{2},\quad
\int\limits_{x_{N-\onehalf}}^{\ell}f(x)dx\approx f_N\cdot\dfrac{h_N}{2}
.\end{gather}

С учетом соотношений \eqref{eq:ODE_BC_RIGHT2}, 
\eqref{eq:ODE_BC_RIGHT3}, \eqref{eq:ODE_BC_RIGHT4}
уравнение баланса вблизи правой границе \eqref{eq:ODE_BC_RIGHT_SOL} 
запишется в разностном виде:
\begin{gather}\label{eq:ODE_BC_RIGHT_MY}
\mu_2-\lambda_2\cdot u(\ell)
+\overline{k}_N\cdot\dfrac{u(\ell)-u(x_{N-1})}{h_N}
+u(\ell)\cdot\brackets{q_N\cdot\dfrac{h_N}{2}}
=f_N\cdot\dfrac{h_N}{2}
.\end{gather}

Объединяя все полученные разностные соотношения 
для дифференциального уравнения
и граничных условий \eqref{eq:ODE_BC_LEFT_MY}, \eqref{eq:ODE_BC_RIGHT_MY}
получаем следующую разностную схему:
\begin{tcolorbox}[colback=blue!5]
\begin{gather*}
\left\{\begin{array}{rcl}
% x=0
u(0)&=&\mu_1\\[1em]
% уравнение
-\overline{k}_{i+1}\cdot\dfrac{u(x_{i+1})-u(x_{i})}{h_{i+1}}
+\overline{k}_i\cdot\dfrac{u(x_{i})-u(x_{i-1})}{h_i}
+\overline{q}_i\cdot h_{i\pm\onehalf}\cdot u(x_i)
&=&\overline{f}_i\cdot h_{i\pm\onehalf}\\[1em]
% x=l
\mu_2-\lambda_2\cdot u(\ell)
+\overline{k}_N\cdot\dfrac{u(\ell)-u(x_{N-1})}{h_N}
+u(\ell)\cdot\brackets{q_N\cdot\dfrac{h_N}{2}}
&=&f_N\cdot\dfrac{h_N}{2}
\end{array}\right.
.\end{gather*}
\end{tcolorbox}
\item
На отрезке введем произвольную сетку $\{x_i\}$ 
и определим шаг сетки $h_i$:
\vspace{1em}

\begin{tabular}{p{1cm} *{10}{p{1cm}}}
\toprule
$i$&0&1&2&3&4&5&6&7&8&9\\
\midmidrule
$x_i$&0&1&2&3&4&4.75&5&5.5&6&7\\
\midrule
$h_i$&&1&1&1&1&0.75&0.25&0.5&0.5&1\\
\bottomrule
\end{tabular}
\vspace{1em}

\item
Вычислим средние значения коэффициентов 
разностной схемы краевой задачи
во всех узлах сетки $\{x_i\}$ $(i=0,1,2,\dots,N)$:
\vspace{1em}

\begin{tabular}{p{1cm} *{10}{p{1cm}}}
\toprule
$i$&0&1&2&3&4&5&6&7&8&9\\
\midmidrule
$\overline{k}_i$&&3.37&3.88&3.48&2.62&2.12&2.02&2.16&2.49&3.12\\
$\overline{q}_i$&1.00&0.54&0.42&0.99&0.65&0.04&0.28&0.71&0.96&0.75\\
$\overline{f}_i$&0.50&0.00&0.50&2.00&4.50&7.03&8.00&10.13&12.50&18.00\\
\bottomrule
\end{tabular}
\vspace{1em}

\item
Определим коэффициенты линейной системы уравнений
$a_i$, $b_i$, $c_i$ и $f_i$\\$(i=0,1,2,\dots,N)$
полученной разностной схемы:



\end{enumerate}


\end{document}


%
% ЗАКЛЮЧЕНИЕ
%
%\begin{Conclusion}
%В работе рассмотрен метод Эйлера численного решения задачи Коши 
%для системы обыкновенных дифференциальных уравнений первого порядка.
%Для заданной системы дифференциальный уравнений первого порядка
%построены рекуррентные соотношения для неизвестных функций $u_1(t)$ и $u_2(t)$,
%которые позволяют последовательно определить их значения в узлах временной сетке $\omega_\tau$.
%
%На основе построенных рекуррентных соотношений найдено 
%численное решение задачи Коши в узлах равномерной сетке 
%$\omega_\tau=\{0,2,4,6,8,10\}$.
%Построены графики функций $u_1(t)$ и $u_2(t)$ на основании 
%вычисленных значений значениях неизвестных функций 
%в различных узлах временной сетки $\omega_\tau$.
%
%Определена предельная абсолютная погрешность приближенного решения задачи Коши
%в пределах всего заданного временного интервала.
%Установлено, что максимальная предельная абсолютная погрешность для $u_1(t)$ 
%составляет $\epsilon_1=5,1$, а для функции $u_2(t)$ -- $\epsilon_2=2,4$.
%
%Приобретен практический навык применения численных методов решения задачи Коши 
%для систем обыкновенных дифференциальных уравнений первого порядка.
%\end{Conclusion}

%
% СПИСОК ИСПОЛЬЗОВАННЫХ ИСТОЧНИКОВ
%
\begin{References}{9}
\bibitem{Samarsky-2009}
Самарский А. А. Введение в численные методы. – Лань, 2009. -- 288 c.
\bibitem{Samarsky-1989}
Самарский А. А., Гулин А. В. Численные методы: учебное пособие для вузов 
// M.: Наука. Гл. ред. физ-мат. лит. 1989. -- 432 с.
\bibitem{Samarsky-2000}
Самарский А. А. и др. Задачи и упражнения по численным методам: Учебное пособие 
// М.: Эдиториал УРСС, 2000.  -- 208 с.
\bibitem{Kalitkin-2011}
Калиткин Н. Н. Численные методы. 2 изд. 
-- СПб.: БХВ-Петербург, 2011. --592 с.
\bibitem{Kalitkin-1978}
Калиткин Н. Н. Численные методы: Учебное пособие. 
-- Наука. Гл. ред. физ.-мат. лит., 1978. -- 511 с.
\bibitem{Demidovich-1967}
Демидович, Б.П. Численные методы анализа: приближение функций, дифференциальные и интегральные уравнения / Б.П. Демидович, И.А. Марон, Э.З. Шувалова ; под ред. Б.П. Демидович. 
-- Изд. 3-е, перераб. -- Москва : Главная редакция физико-математической литературы, 1967. -- 368 с.
\bibitem{LaTeX math symbol's}
\href{https://en.wikipedia.org/wiki/List_of_mathematical_symbols_by_subject}
%{List_of_mathematical_symbols_by_subject}
{Список математических символов \LaTeX}
--URL: \url{https://en.wikipedia.org/wiki/List_of_mathematical_symbols_by_subject}
\bibitem{a1}
Бояршинов, М. Г. Вычислительные методы алгебры и анализа:
учебное пособие / М. Г. Бояршинов. 
-- Саратов : Вузовское образование, 2020. -- 225 c. 
-- ISBN 978-5-4487-0687-5. -- Текст : электронный // Цифровой образовательный ресурс IPR SMART : [сайт]. 
-- URL: https://www.iprbookshop.ru/93065.html (дата обращения: 30.01.2022).
-- Режим до-ступа: для авторизир. пользователей. 
- DOI: https://doi.org/10.23682/93065
\bibitem{a2}
Олейникова, С. А. Численные методы решения оптимизационных задач:
учебное пособие / С. А. Олейникова. 
-- Воронеж : Воронежский государственный технический университет, ЭБС АСВ, 2021. -- 114 c.
-- ISBN 978-5-7731-0960-0. -- Текст : электронный // Цифровой образовательный ресурс IPR SMART : [сайт].
-- URL: https://www.iprbookshop.ru/118626.html (дата обращения: 30.01.2022). 
-- Режим доступа: для авторизир. пользователей
\bibitem{a3}
Гарифуллин, М. Ф. Численные методы интегрирования дифференциальных уравнений / М. Ф. Гарифуллин. 
-- Москва : Техносфера, 2020. -- 192 c.
-- ISBN 978-5-94836-597-8. -- Текст : электронный // Цифровой образовательный ресурс IPR SMART : [сайт].
-- URL: https://www.iprbookshop.ru/99103.html (дата обращения: 30.01.2022).
-- Режим доступа: для авторизир. пользователей
\bibitem{a4}
Ахмадиев, Ф. Г. Математическое моделирование и методы оптимизации:
учебное пособие / Ф. Г. Ахмадиев, Р. М. Гильфанов. 
-- Москва : Ай Пи Ар Медиа, 2022. -- 178 c.
-- ISBN 978-5-4497-1383-4. -- Текст : электронный // Цифровой образовательный ресурс IPR SMART : [сайт].
-- URL: https://www.iprbookshop.ru/116448.html (дата обращения: 30.01.2022).
-- Режим доступа: для авторизир. пользователей
\bibitem{a5}
Рутта, Н. А. Методы и модели принятия оптимальных решений в экономике:
учебное пособие для бакалавров / Н. А. Рутта.
-- Москва : Ай Пи Ар Медиа, 2022. -- 87 c.
-- ISBN 978-5-4497-1534-0. -- Текст : электронный // Цифровой образовательный ресурс IPR SMART : [сайт].
-- URL: https://www.iprbookshop.ru/118015.html (дата обращения: 30.01.2022). 
-- Режим доступа: для авторизир. пользователей
\end{References}

\end{document}