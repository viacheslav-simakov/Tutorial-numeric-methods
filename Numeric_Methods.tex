% !TEX TS-program = xelatex
% !TEX encoding = UTF-8
%%%%%%%%%%%%%%%%%%%%%%%%%%%%%%
%
% Преамбула
%
%%	РАЗМЕР ШРИФТА:
%		\tiny
%		\scriptsize
%		\footnotesize	
%		\small 
%%%	\normalsize	= нормальный размер шрифта
%		\large
%		\Large
%		\huge
%		\Huge
%--------------------------------------------
%	\textrm{текст}	\textsf{текст}	\texttt{текст}
%	\textmd{текст}	\textbf{текст}	\textup{текст}
%	\textit{текст}	\textsl{текст}	\textsc{текст}
%%%%%%%%%%%%%%%%%%%%%%%%%%%%%%%%%%%%%%%%%%%%%
%
%	НАЧАЛО ПРЕАМБУЛЫ
%
\documentclass[14pt,a4paper]{extreport}
%
% языковые пакеты
%
\usepackage[T2A]{fontenc}
\usepackage[utf8]{inputenc}
%\usepackage[cp1251]{inputenc}
%
%	Размеры страницы
%
\usepackage[% https://ctan.org/pkg/geometry
	a4paper,% размер страницы
	mag=1000,
	left=2.5cm,
	right=1.5cm,
	top=2cm,
	bottom=2cm,
	headsep=0.7cm,
	footskip=0.7cm
]{geometry}
% --------------------------------------------
% *** межстрочный интервал ***
% стандартный пропуск строки означает коэффициент 1,2
% (например, высота шрифта 10pt, пропуск базовой строки 12pt).
% Умножьте на \linespread, чтобы вы получили 1,25 * 1,2 = 1,5 
% то есть половину.
% --------------------------------------------
%\linespread{1.5}
%
% МЕЖСТРОЧНЫЙ ИНТЕРВАЛ
%
% межстрочный интервал
\usepackage{setspace}
% полуторный интервал
\onehalfspacing
%\usepackage{indentfirst}
%
% ОТСТУП АБЗАЦА СЛЕВА
%
\setlength{\parindent}{1.25cm}
%
% НУМЕРАЦИЯ СТРАНИЦ 
% http://tug.ctan.org/tex-archive/macros/latex/contrib/fancyhdr/fancyhdr.pdf
%
\usepackage{fancyhdr}% https://www.ctan.org/pkg/fancyhdr
\pagestyle{empty}
\renewcommand{\headrulewidth}{0pt}
\renewcommand{\footrulewidth}{0pt}
%\lhead{\empty}\chead{\empty}\rhead{\empty}
%\lfoot{\empty}\cfoot{\empty}\rfoot{\arabic{page}}
%
% https://latex.org/forum/viewtopic.php?t=8952
%
\makeatletter
\renewcommand{\ps@plain}{%
\renewcommand\@oddhead{}%
\renewcommand\@evenhead{}%
\renewcommand\@oddfoot{\hfil\normalfont\textrm{\thepage}}%
\renewcommand\@evenfoot{\hfil\normalfont\textrm{\thepage}}%
}
\makeatother
%%%%%%%%%%%%%%%%%%%%%%%%%%%%%%%%%%%%%%%%%%%%%
%
% ПАКЕТ МНОГОЯЗЫКОВОЙ ВЁРСТКИ *** XeTeX ***
% https://ru.wikibooks.org/wiki/LaTeX/polyglossia
%
\usepackage{polyglossia}
% устанавливает главный язык документа
\setdefaultlanguage[spelling=modern]{russian}
\setmainlanguage{russian}
% устанавливает второй язык документа
\setotherlanguage{english}
% задаёт свойства шрифтов по умолчанию
\defaultfontfeatures{Ligatures={TeX},Renderer=Basic}
% задаёт основные шрифты документа
\setmainfont{Times New Roman}
\setromanfont{Times New Roman} 
\setsansfont{Arial} 
\setmonofont{Courier New}
% семейство кириллических шрифтов
\newfontfamily{\cyrillicfont}{Times New Roman} 
\newfontfamily{\cyrillicfontrm}{Times New Roman}
\newfontfamily{\cyrillicfonttt}{Courier New}
\newfontfamily{\cyrillicfontsf}{Arial}
%
% *** Новые Цвета ***
%
\usepackage{xcolor}% https://www.ctan.org/pkg/xcolor
\colorlet{darkgreen}{green!40!black}% темно-зеленый
\colorlet{darkred}{red!80!black}% темно-красный
\colorlet{darkorange}{orange!75!black}% темно-оранжевый
\colorlet{darkblue}{blue!70!black}% темно-синий
%
% Unicode-шрифты ДЛЯ ФОРМУЛ
%
\usepackage[intlimits]{amsmath}
\usepackage{amssymb,amsfonts}
% Расширенная матрица
% https://tex.stackexchange.com/questions/2233/whats-the-best-way-make-an-augmented-coefficient-matrix
% \begin{pmatrix}[cc|c]
%   1 & 2 & 3\\
%   4 & 5 & 9
% \end{pmatrix}
\makeatletter
\renewcommand*\env@matrix[1][*\c@MaxMatrixCols c]{%
	\hskip -\arraycolsep
	\let\@ifnextchar\new@ifnextchar
	\array{#1}}
\makeatother
%
% ШРИФТ Times в ФОРМУЛАХ как основной
%
%\usepackage[varg,cmbraces,cmintegrals]{newtxmath}
%
% прямые греческие буквы
%
%\usepackage{upgreek}
%
% Experimental Unicode mathematical typesetting
%
%\usepackage{unicode-math}
%\usepackage[bold-style=TeX]{unicode-math}
\usepackage[% https://ctan.org/pkg/unicode-math
	bold-style=upright%
]{unicode-math}
%
% ССЫЛКИ
%
\usepackage[% https://ctan.org/pkg/hyperref
	unicode=true,%
	bookmarksopen=true,%
	bookmarksnumbered=true,%
	colorlinks,%
	linkcolor=black,%
	urlcolor=darkblue,%
	citecolor=blue%
]{hyperref}
%
% Принудительное размещения рисунка:
% используйте пакет 'float', а затем [H] опцию,
% например, \begin{figure}[H] ... \end{figure}
\usepackage{float}
%%%%%%%%%%%%%%%%%%%%%%%%%%%%%%%%%%%%%%%%%%%%%
%
% ЗАГОЛОВКИ
%
%\usepackage[pagestyles]{titlesec}
\usepackage{titlesec}% https://www.ctan.org/pkg/titlesec
%
% \titleformat{command}
%		[shape]
%		{format}
%		{label}
%		{sep}
%		{before-code}
%		[after-code]
%
%	1) Глава (chapter)
%
\titleformat{\chapter}[hang]% shape
{\bfseries\normalsize\centering}% format
{\hspace{\parindent}\arabic{chapter}}% label
{0.5em}% sep
{}% before-code
[]% after-code
\titleformat*{\section}%
{\bfseries\normalsize}% format
%
% 2) Раздел (section)
%
\titleformat{\section}[hang]% shape
{\bfseries\normalsize}% format
{\hspace{1.25cm}\arabic{section}}% label
{0.5em}% sep
{}% before-code
[]% after-code
%
% 3) Подраздел (subsection)
%
\titleformat{\subsection}[hang]% shape
{\bfseries\normalsize}% format
{\hspace{1.25cm}\arabic{section}.\arabic{subsection}}% label
{0.5em}% sep
{}% before-code
[]% after-code
%
% 4) ПодПодраздел (subsubsection)
%
\titleformat{\subsubsection}[hang]% shape
{\bfseries\normalsize}% format
{\hspace{1.25cm}\arabic{section}.\arabic{subsection}.\arabic{subsubsection}}% label
{0.5em}% sep
{}% before-code
[]% after-code
%
% \titlespacing{command}
%		{before-sep}
%		{after-sep}
%		[right-sep]
%
\titlespacing{\chapter}
{0pt}% before-sep
{0pt}% after-sep
{0pt}% right-sep
\titlespacing{\section}
{0pt}% before-sep
{0pt}% after-sep
{0pt}% right-sep
\titlespacing{\subsection}
{0pt}% before-sep
{0pt}% after-sep
{0pt}% right-sep
\titlespacing{\subsubsection}
{0pt}% before-sep
{0pt}% after-sep
{0pt}% right-sep
%%%%%%%%%%%%%%%%%%%%%%%%%%%%%%%%%%%%%%%%%%%%
%
% СОДЕРЖАНИЕ
%
%%%%%%%%%%%%%%%%%%%%%%%%%%%%%%%%%%%%%%%%%%%%
%\addto\captionsrussian{%
%\renewcommand{\contentsname}{%
%\vspace{-5ex}%
%\begin{center}СОДЕРЖАНИЕ\end{center}%
%\vspace{-5ex}}%
%}
%\addto\captionsrussian{%
%  \renewcommand{\figurename}{Fig.}%
%}
\makeatletter
\renewcommand{\tableofcontents}{%
\newpage%
\noindent{\centering\textbf{СОДЕРЖАНИЕ}\par}%
\vspace{\baselineskip}% пустая строка
\@starttoc{toc}\par}%
\makeatother
% глубина детализации
\setcounter{tocdepth}{3}
\setcounter{secnumdepth}{3}
% переименование формата счетчика
\renewcommand{\thesection}{\arabic{section}}
%\renewcommand{\thesubsection}{\arabic{subsection}}
%	
% \titlecontents{section}
%		[left]
%		{above-code}
%		{numbered-entry-format}
%		{numberless-entry-format}
%		{filler-page-format}
%		[below-code]
%
\usepackage{titletoc}
%	Глава (chapter)
\titlecontents{chapter}
[0pt]% left
{}%above-code
{\thechapter\;}% numbered-entry-format
{}% numberless-entry-format
{\thepage}% filler-page-format
[]% below-code
%
% Раздел (section)
\titlecontents{section}
[0pt]% left
{\normalsize}% above-code
{\thecontentslabel\;}% numbered-entry-format
{}% numberless-entry-format
{(\thepage)}% filler-page-format
[]% below-code
%
% Подраздел (subsection)
\titlecontents{subsection}
[0pt]% left
{}%above-code
{\thecontentslabel\;}% numbered-entry-format
{}% numberless-entry-format
{\thepage}% filler-page-format
[]% below-code
% ПодПодраздел (subsubsection)
\titlecontents{subsubsection}
[0pt]% left [0pt]
{}%above-code
{\thecontentslabel\;}% numbered-entry-format
{}% numberless-entry-format
{\thepage}% filler-page-format
[]% below-code
%
% \dottedcontents{section}
%		[left]
%		{above-code}
%		{label width}
%		{leader width}
\dottedcontents{chapter}%
[1.5em]% left
{}% above-code
{1.5em}% label width
{0.5em}% leader width
\dottedcontents{section}%
[1.0em]% left [1.5em]
{}% above-code
{1.0em}% label width [1.5em]
{0.5em}% leader width
\dottedcontents{subsection}%
[1.5em]% left
{}% above-code
{1.5em}% label width
{0.5em}% leader width
\dottedcontents{subsubsection}%
[2.5em]% left
{}% above-code
{2.5em}% label width
{0.5em}% leader width
%
% Chapter without a pagebreak
%
\makeatletter 
\renewcommand\chapter{\newpage\par%
\thispagestyle{plain}% \global\@topnum\z@
\@afterindentfalse\secdef\@chapter\@schapter}
\makeatother %%%%% <---- Starting chapter without a pagebreak
%
% ВВЕДЕНИЕ
%
\newenvironment{Introduction}{%
%\chapter*{\vspace{-2ex}ВВЕДЕНИЕ}\par%
%\newpage\begin{center}\textbf{ВВЕДЕНИЕ}\end{center}\par
\newpage\phantomsection%
{\centering\textbf{ВВЕДЕНИЕ}\par}%
\addcontentsline{toc}{chapter}{ВВЕДЕНИЕ}%
% пустая линия
%\vspace{\baselineskip}
}{\newpage}
%
% ЗАКЛЮЧЕНИЕ
%
\newenvironment{Conclusion}{%
%\lfoot{\empty}\cfoot{\empty}\rfoot{\arabic{page}}%
\newpage\phantomsection%
%\chapter*{\vspace{-\baselineskip}ЗАКЛЮЧЕНИЕ}\par%
{\centering\textbf{ЗАКЛЮЧЕНИЕ}\par}%
\addcontentsline{toc}{chapter}{ЗАКЛЮЧЕНИЕ}%
% пустая линия
%\vspace{\baselineskip}
}{\newpage}
%
% ПРИЛОЖЕНИЕ
%
\newenvironment{Appendix}{%
%\lfoot{\empty}\cfoot{\empty}\rfoot{\arabic{page}}%
\newpage\phantomsection%
%\chapter*{\vspace{-\baselineskip}ЗАКЛЮЧЕНИЕ}\par%
{\centering\textbf{ПРИЛОЖЕНИЕ}\par}%
\addcontentsline{toc}{chapter}{ПРИЛОЖЕНИЕ}%
% пустая линия
%\vspace{\baselineskip}
}{\newpage}
%
% СПИСОК ИСПОЛЬЗОВАННЫХ ИСТОЧНИКОВ
%
% Пакет поддерживает сжатые, 
% отсортированные списки цитирования
% https://www.ctan.org/pkg/cite
\usepackage{cite}
% формат номера источника
% Заменяем библиографию в квадратных скобках
% http://ftp.tug.org/TUGboat/tb30-1/tb94mori.pdf
% 
\makeatletter%
\renewcommand*{\@biblabel}[1]{\hfill#1\;}
\makeatother
% окружение
\newenvironment{References}[1]{%
\newpage\phantomsection
\addcontentsline{toc}{chapter}{%
СПИСОК ИСПОЛЬЗОВАННЫХ ИСТОЧНИКОВ}
\renewcommand{\bibname}{%
СПИСОК ИСПОЛЬЗОВАННЫХ ИСТОЧНИКОВ}
\begin{thebibliography}{#1}
% пустая линия
\vspace{\baselineskip}
% интервал между библиографическими источниками
\setlength{\itemsep}{0pt}
\setlength{\parskip}{0pt}
}{\end{thebibliography}}
%%%%%%%%%%%%%%%%%%%%%%%%%%%%%%%%%%%%%%%%%%%%%
%
% ТАБЛИЦЫ
%
% Пакет повышает качество таблиц в LaTeX, 
% предоставляя дополнительные команды
% https://www.ctan.org/pkg/booktabs
\usepackage{booktabs}%
% \specialrule{wd}{abovespace}{belowspace}
% горизонтальная линия с отступами сверху и снизу
\renewcommand\midrule{\specialrule{0.5pt}{1ex}{1ex}}
\renewcommand\toprule{\specialrule{1pt}{0ex}{1ex}}
\renewcommand\bottomrule{\specialrule{1pt}{1ex}{0ex}}
%\newcommand\tabsrule{\specialrule{0.5pt}{1ex}{1ex}}
\newcommand\midmidrule{
\specialrule{0.5pt}{1ex}{0.1ex}
\specialrule{0.5pt}{0.1ex}{1ex}
}
% межстрочный интервал в ТАБЛИЦАХ
\renewcommand\arraystretch{1.2}
%%%%%%%%%%%%%%%%%%%%%%%%%%%%%%%%%%%%%%%%%%%%%
%
% СПИСКИ
%
\usepackage[inline]{enumitem}% https://ctan.org/pkg/enumitem
\setlist[enumerate,itemize]{
	left=0pt,
	align=left,
	leftmargin=0pt,
	label = \arabic*),
	leftmargin=*,
	labelsep=1ex,
	itemindent=0pt,
	nosep
}
%%%%%%%%%%%%%%%%%%%%%%%%%%%%%%%%%%%%%%%%%%%%%
%
%	Цветной прямоугольник с текстом
%
\usepackage{tcolorbox}% https://www.ctan.org/pkg/tcolorbox
% Установка опций по умолчанию
\tcbset{
	notitle,
	titlebox=invisible,
	size=title,
	width=\textwidth,
	boxsep=0em,
	left=1ex,
	right=1ex,
	toptitle=0mm,
	top=1ex,
	toprule=0mm,
	bottom=1ex,
	bottomrule=0mm,
	boxrule=0mm,
	arc=0mm,
	colback=orange!10,
}
%%%%%%%%%%%%%%%%%%%%%%%%%%%%%%%%%%%%%%%%%%%%%
%
% TikZ
% https://tex.stackexchange.com/questions/131293/arguments-for-tikz-style
%
\usepackage{tikz}
\usetikzlibrary{%
	pgfplots.groupplots,%
	backgrounds,%
	calc,%
	decorations.pathmorphing,%
	decorations.markings,
	shapes,% геометрические фигуры
	arrows.meta,% стрелки разной формы
	er,% построение диаграмм
	patterns% штриховка областей
}
\tikzset{
	background rectangle/.style={% стиль фона
		fill=olive!10%
	},%
	font=\normalsize%
}
%
% PGF
% http://elib.ict.nsc.ru/jspui/bitstream/ICT/1488/1/pgf-ru-all-method.pdf
%
\usepackage{%
	pgfplots,%
	pgfplotstable%
}
% Последовательность графичеких слоев
% \pgfsetlayers{%
%	background,%
%	pre main,%
%	axis grid,% 
%	axis ticks,% 
%	axis lines,%
%	axis tick labels,%
%	main,%
%	axis descriptions,%
%	axis foreground%
%}
\usepgfplotslibrary{fillbetween}
\usepgflibrary{plotmarks}
% Установка стилей графика
\pgfplotsset{
%	compat=1.9,
	width=8cm,
	every axis/.append style={%
		thick,
		tick style={
			black,
			semithick,
		},
	}
}
%
% НУМЕРАЦИЯ формул, рисунков
%
\renewcommand{\theequation}{\arabic{equation}}
\renewcommand{\thefigure}{\arabic{figure}}
%%%%%%%%%%%%%%%%%%%%%%%%%%%%%%%%%%%%%%%%%%%%%
%
% ПОДПИСИ рисуноков, таблиц
%
\usepackage[% https://www.ctan.org/pkg/caption
	format=plain,% Печатает подписи как обычный абзац
	labelsep=endash,%
	singlelinecheck=false,% отключить центрирование однострочной подписи
	belowskip=0pt,%
	margin={0pt,0pt},%
	indention=0cm,
%	font=onehalfspacing,% полуторный интервал
]{caption}
%	рисунки
\captionsetup[figure]{%
	name=Рисунок,%
	position=below,%
	justification=centering,% центрирование подписи
	font=onehalfspacing,% полуторный интервал
}
%	таблицы
\captionsetup[table]{%
	name=Таблица,%
	position=above,%
%	justification=justified,
	font=onehalfspacing,% полуторный интервал
}
%\captionsetup{belowskip=0pt,margin={0pt,0pt}}


% *** отступы ***
% вертикальный промежуток перед и после объектов,
% местоположение которых соответствует ключу h.
% Имеет естественную длину 12 pt.
\setlength{\intextsep}{\baselineskip}
% вертикальный промежуток между текстом и соответственно 
% одно- и двухколоночными объектами, местоположение 
% которых соответствует ключам t или b.
% Имеет естественную длину 20 pt.
\setlength{\textfloatsep}{\baselineskip}
% вертикальный промежуток между соответственно 
% одно- и двухколоночными объектами, местоположение 
% которых соответствует ключам t или b. 
% Имеет естественную длину 12 pt.
\setlength{\floatsep}{\baselineskip}
% отступ перед названием 
\setlength{\abovecaptionskip}{0.5\baselineskip}
% отступ после названиея
\setlength{\belowcaptionskip}{0ex}
%
% ОТСТУПЫ В ФОРМУЛАХ
%
%\expandafter\def\expandafter\normalsize\expandafter{%
%\normalsize
%\setlength\abovedisplayskip{0.5\baselineskip}
%\setlength\belowdisplayskip{0.5\baselineskip}
%\setlength\abovedisplayshortskip{0.5\baselineskip}
%\setlength\belowdisplayshortskip{0.5\baselineskip}
%}
\AtBeginDocument{%
\abovedisplayskip=0.5\baselineskip
\abovedisplayshortskip=0.5\baselineskip
\belowdisplayskip=0.5\baselineskip
\belowdisplayshortskip=0.5\baselineskip
% отступ перед названием 
%\abovecaptionskip=0.5\baselineskip
% отступ после названиея
%\belowcaptionskip=0\baselineskip
}
%
% Пакет mhchem предоставляет команды для
% набора химических молекулярных формул и уравнений.
\usepackage[version=4,arrows=font]{mhchem}
%\usepackage{expl3,calc}
% длина стрелок
\ExplSyntaxOn
\keys_define:nn { mhchem }
{
arrow-min-length .code:n =
% default is 2em
\cs_set:Npn \__mhchem_arrow_options_minLength:n { {#1} }
}
\ExplSyntaxOff
\mhchemoptions{arrow-min-length=1em}
%
%	Пакет предоставляет макросы для управления строками - 
%	тестирования содержимого строки, извлечения подстрок, 
%	подстановки подстрок и предоставления чисел, 
%	таких как длина строки, позиция или количество 
%	повторов подстроки.
%	https://www.ctan.org/pkg/xstring
%
%	\usepackage{xstring}

%
%	Вставка страниц из PDF-файла
%	Этот пакет упрощает включение внешних многостраничных 
%	PDF-документов в LaTеX документы
% 
\usepackage{pdfpages}
%%%%%%%%%%%%%%%%%%%%%%%%%%%%%%%%%%%%%%%%%%%%%
%
%	МОИ НОВЫЕ КОМАНДЫ
%
% Макрос Существует?
% https://tex.stackexchange.com/questions/164188/ifundefined-actually-defines-macros
\def\ifexists#1{\expandafter\ifx\csname#1\endcsname\relax 0\else 1\fi}
% Pure text from TeX
% https://tex.stackexchange.com/questions/567286/pdfstringdef-turns-accented-characters-into-octal-escape-sequence
\ExplSyntaxOn
\cs_set_eq:NN\textpurify\text_purify:n
\ExplSyntaxOff
% alert
\newcommand{\alert}[2][]{%
#1{\emph{\textcolor{darkred}{#2}}}%
}
% alertx
\newcommand{\alertx}[2][\texttt]{%
#1{\textbf{\textcolor{darkred}{#2}}}%
}
% проверка макроса
\newcommand{\Isdefined}[1]{%
\ifx#1\undefined\relax%
\alertx{\backslash def\detokenize{#1}undefined}%
\else{#1}\fi%
}
% Чистый текст макроса
\newcommand{\MacroTextPurify}[1]{%
\ifx#1\undefined\relax%
\alertx{\backslash def\detokenize{#1}undefined}%
\else\textpurify{#1}%
\fi}
% Пол студента
\newcommand{\detGender}[3]{%
\ifx#1\undefined{для студентов}%
\else%
\ifcase#1\relax{#2}% Gender=0 (женский)
\or{#3}% Gender=1 (мужской)
\else\alertx{Gender=\Gender{ unknow}}%
\fi%
\fi%
}
% пустая строка
\def\emptyline{\par\vspace{\baselineskip}}
% символ '\'
\def\backslash{\char`\\}
% постоянная
\DeclareMathOperator{\const}{const}
% Полужирное начертание для векторов
\newcommand\vect[1]{\mathbfit{#1}}
%\let\vec=\mathbf
% норма
\newcommand{\norma}[1]{\left\lVert#1\right\rVert}
% абсолютное значение
\newcommand{\abs}[1]{\left\lvert#1\right\rvert}
% скалярное произведение векторов
\newcommand{\dotvec}[2]{(\vec{#1},\vec{#2})}
% обыкновенная производная, например $\diff{N_d^{+}}{x}$
\newcommand\diff[2]{ \dfrac{\mathrm{d}#1}{\mathrm{d}#2} }
% обыкновенная производная второго порядка, например $\diff{N_d^{+}}{x}$
\newcommand\diffdiff[2]{ \dfrac{\mathrm{d}^2 #1}{\mathrm{d} #2^2} }
% частная производная, например $\pdiff{N_d^{+}}{x}$
\newcommand\pdiff[2]{ \dfrac{\partial #1}{\partial #2} }
% 1/2
\newcommand\onehalf{ \nicefrac{1}{2} }
% Случайное число \RandInt{min}{max}
\newcommand{\RandInt}[2]{\pgfmathrandominteger{\rndint}{#1}{#2}\rndint}
% Интеграл
\newcommand\intf[4][x]{ \int\limits_{#2}^{#3}{#4}\,\mathrm{d}#1 }
%%%%%%%%%%%%%%%%%%%%%%%%%%%%%%%%%%%%%%%%%%%%%
%
% ТИТУЛЬНЫЙ ЛИСТ
%
% \TitlePage[название документа] или \TitlePage
\newcommand{\TitlePage}[1][\undefined]{
{\centering% начало центрирования
МИНОБРНАУКИ РОССИИ\\
Федеральное государственное бюджетное образовательное учреждение
высшего образования\\
\textbf{<<САРАТОВСКИЙ НАЦИОНАЛЬНЫЙ ИССЛЕДОВАТЕЛЬСКИЙ\\
ГОСУДАРСТВЕННЫЙ УНИВЕРСИТЕТ\\
ИМЕНИ Н.Г. ЧЕРНЫШЕВСКОГО>>}
\emptyline
Кафедра материаловедения,\par
технологии и управления качеством
\emptyline
% название документа (DocumentTitle)
\ifnum1=\ifexists{SubTitle}\MakeUppercase{\SubTitle}\emptyline\fi
%\ifx#1\undefined\relax% не определено
%\else\MakeUppercase{#1}\emptyline%
%\fi
% ЗАГЛАВИЕ
\textbf{\MakeUppercase{\Isdefined{\TITLE}}}
\emptyline
по дисциплине <<\Isdefined{\ModuleTitle}>>\\
% для студентов/студентки/студента
\detGender{\Gender}{студентки}{студента}
\Isdefined{\NoCourse}{ курса }\Isdefined{\NoGroup}{ группы}\\
направления подготовки{ \Isdefined\ProgramCode }
<<\Isdefined{\ProgramTitle}>>
(профиль <<\Isdefined{\ProgramProfile}>>),\\
\Isdefined{\Department}\\% факультет
% Ф.И.О. (родительный падеж)
\ifx\FullNameGenetive\undefined\relax%
%\end{center}%
\else{\FullNameGenetive}%
% пустая строка
\emptyline
% добавляет заполняющее вертикальное пространство
\vfill
% таблица БАРС
\begin{tabular}[l]{b{5cm} b{2.5cm} c}
\toprule
Результат&Баллы&ВСЕГО\\
\midmidrule
Выполнение&&\Isdefined{\Exec}\\
\midrule
Оформление&&\Isdefined{\Polygraphy}\\
\midrule
Устный отчет&&\Isdefined{\OralReport}\\
\bottomrule
\end{tabular}%
% пустая строка
\par\emptyline
% ПОДПИСЬ
\raggedright\noindent
Преподаватель\\профессор, д.т.н., доцент
\hspace{1em}\rule{6.5cm}{0.5pt} 
\hspace{1em} В.В. Симаков\par
\emptyline
\fi%
}\par% окончание центрирования
}
%%%%%%%%%%%%%%%%%%%%%%%%%%%%%%%%%%%%%%%%%%%%%
%
% Колонтитулы
%
% 'Титульный лист'
\fancypagestyle{titlepage}{
\setcounter{page}{1}
% clear all header and footer fields
\fancyhf{}
% линии верхнего и нижнего колонтитулов
\renewcommand{\headrulewidth}{0pt}%
\renewcommand{\footrulewidth}{0pt}%
\cfoot{Саратов~\the\year} % город год
}
% 'Задание'
\fancypagestyle{taskpages}{
% clear all header and footer fields
\fancyhf{}
% нумерация римскими цифрами
%\pagenumbering{roman}
% нумерация арабскими цифрами
\pagenumbering{arabic}
% линии верхнего и нижнего колонтитулов
\renewcommand{\headrulewidth}{0pt}%
\renewcommand{\footrulewidth}{0pt}%
% нижний колонтитул
\lfoot{Задание~
\detGender{\Gender}{получила}{получил}
\hspace{0.25cm}\rule{4cm}{0.5pt}\hspace{0.25cm}%
\Isdefined{\Signature}}% подпись
%\rfoot{\roman{page}}% номера страниц
}
% 'Содержание документа'
\fancypagestyle{bodypages}{
\fancyhf{}
\pagestyle{fancy}
% линии верхнего и нижнего колонтитулов
\renewcommand{\headrulewidth}{0pt}%
\renewcommand{\footrulewidth}{0pt}%
\pagenumbering{arabic}
\rfoot{\arabic{page}}% номера страниц
}
%%%%%%%%%%%%%%%%%%%%%%%%%%%%%%%%%%%%%%%%%%%%%
%
% ЗАДАНИЕ
%
% \TitleTask или \TitleTask['на выполнение чего?']
\newcommand{\TitleTask}[1][лабораторной работы]{%
\newpage
% нумерация страниц с №1
\setcounter{page}{1}
%\newpage\phantomsection
\begin{center}
\textbf{ЗАДАНИЕ}\par
% добавление пункта в СОДЕРЖАНИЕ
%\addcontentsline{toc}{chapter}{Задание на выполнение #1}
%по дисциплине <<\Isdefined{\@ModuleTitle}>>
по дисциплине <<\MacroTextPurify{\ModuleTitle}>>
на выполнение #1 на тему 
<<\textbf{\MacroTextPurify{\TITLE}}>>
\end{center}
\par
}
% home
%	РАЗМЕР ШРИФТА:
%		\tiny
%		\scriptsize
%		\footnotesize	
%		\small 
%%%	\normalsize	= нормальный размер шрифта
%		\large
%		\Large
%		\huge
%		\Huge
%--------------------------------------------
%	\textrm{текст}	\textsf{текст}	\texttt{текст}
%	\textmd{текст}	\textbf{текст}	\textup{текст}
%	\textit{текст}	\textsl{текст}	\textsc{текст}
%%%%%%%%%%%%%%%%%%%%%%%%%%%%%%%%%%%%%%%%%%%%%
%
%	НАЧАЛО ПРЕАМБУЛЫ
%
\documentclass[14pt,a4paper]{extreport}
%
% языковые пакеты
%
\usepackage[T2A]{fontenc}
\usepackage[utf8]{inputenc}
%\usepackage[cp1251]{inputenc}
%
%	Размеры страницы
%
\usepackage[% https://ctan.org/pkg/geometry
	a4paper,% размер страницы
	mag=1000,
	left=2.5cm,
	right=1.5cm,
	top=2cm,
	bottom=2cm,
	headsep=0.7cm,
	footskip=0.7cm
]{geometry}
% --------------------------------------------
% *** межстрочный интервал ***
% стандартный пропуск строки означает коэффициент 1,2
% (например, высота шрифта 10pt, пропуск базовой строки 12pt).
% Умножьте на \linespread, чтобы вы получили 1,25 * 1,2 = 1,5 
% то есть половину.
% --------------------------------------------
%\linespread{1.5}
%
% МЕЖСТРОЧНЫЙ ИНТЕРВАЛ
%
% межстрочный интервал
\usepackage{setspace}
% полуторный интервал
\onehalfspacing
%\usepackage{indentfirst}
%
% ОТСТУП АБЗАЦА СЛЕВА
%
\setlength{\parindent}{1.25cm}
%
% НУМЕРАЦИЯ СТРАНИЦ 
% http://tug.ctan.org/tex-archive/macros/latex/contrib/fancyhdr/fancyhdr.pdf
%
\usepackage{fancyhdr}% https://www.ctan.org/pkg/fancyhdr
\pagestyle{empty}
\renewcommand{\headrulewidth}{0pt}
\renewcommand{\footrulewidth}{0pt}
%\lhead{\empty}\chead{\empty}\rhead{\empty}
%\lfoot{\empty}\cfoot{\empty}\rfoot{\arabic{page}}
%
% https://latex.org/forum/viewtopic.php?t=8952
%
\makeatletter
\renewcommand{\ps@plain}{%
\renewcommand\@oddhead{}%
\renewcommand\@evenhead{}%
\renewcommand\@oddfoot{\hfil\normalfont\textrm{\thepage}}%
\renewcommand\@evenfoot{\hfil\normalfont\textrm{\thepage}}%
}
\makeatother
%%%%%%%%%%%%%%%%%%%%%%%%%%%%%%%%%%%%%%%%%%%%%
%
% ПАКЕТ МНОГОЯЗЫКОВОЙ ВЁРСТКИ *** XeTeX ***
% https://ru.wikibooks.org/wiki/LaTeX/polyglossia
%
\usepackage{polyglossia}
% устанавливает главный язык документа
\setdefaultlanguage[spelling=modern]{russian}
\setmainlanguage{russian}
% устанавливает второй язык документа
\setotherlanguage{english}
% задаёт свойства шрифтов по умолчанию
\defaultfontfeatures{Ligatures={TeX},Renderer=Basic}
% задаёт основные шрифты документа
\setmainfont{Times New Roman}
\setromanfont{Times New Roman} 
\setsansfont{Arial} 
\setmonofont{Courier New}
% семейство кириллических шрифтов
\newfontfamily{\cyrillicfont}{Times New Roman} 
\newfontfamily{\cyrillicfontrm}{Times New Roman}
\newfontfamily{\cyrillicfonttt}{Courier New}
\newfontfamily{\cyrillicfontsf}{Arial}
%
% *** Новые Цвета ***
%
\usepackage{xcolor}% https://www.ctan.org/pkg/xcolor
\colorlet{darkgreen}{green!40!black}% темно-зеленый
\colorlet{darkred}{red!80!black}% темно-красный
\colorlet{darkorange}{orange!75!black}% темно-оранжевый
\colorlet{darkblue}{blue!70!black}% темно-синий
%
% Unicode-шрифты ДЛЯ ФОРМУЛ
%
\usepackage[intlimits]{amsmath}
\usepackage{amssymb,amsfonts}
% Расширенная матрица
% https://tex.stackexchange.com/questions/2233/whats-the-best-way-make-an-augmented-coefficient-matrix
% \begin{pmatrix}[cc|c]
%   1 & 2 & 3\\
%   4 & 5 & 9
% \end{pmatrix}
\makeatletter
\renewcommand*\env@matrix[1][*\c@MaxMatrixCols c]{%
	\hskip -\arraycolsep
	\let\@ifnextchar\new@ifnextchar
	\array{#1}}
\makeatother
%
% ШРИФТ Times в ФОРМУЛАХ как основной
%
%\usepackage[varg,cmbraces,cmintegrals]{newtxmath}
%
% прямые греческие буквы
%
%\usepackage{upgreek}
%
% Experimental Unicode mathematical typesetting
%
%\usepackage{unicode-math}
%\usepackage[bold-style=TeX]{unicode-math}
\usepackage[% https://ctan.org/pkg/unicode-math
	bold-style=upright%
]{unicode-math}
%
% ССЫЛКИ
%
\usepackage[% https://ctan.org/pkg/hyperref
	unicode=true,%
	bookmarksopen=true,%
	bookmarksnumbered=true,%
	colorlinks,%
	linkcolor=black,%
	urlcolor=darkblue,%
	citecolor=blue%
]{hyperref}
%
% Принудительное размещения рисунка:
% используйте пакет 'float', а затем [H] опцию,
% например, \begin{figure}[H] ... \end{figure}
\usepackage{float}
%%%%%%%%%%%%%%%%%%%%%%%%%%%%%%%%%%%%%%%%%%%%%
%
% ЗАГОЛОВКИ
%
%\usepackage[pagestyles]{titlesec}
\usepackage{titlesec}% https://www.ctan.org/pkg/titlesec
%
% \titleformat{command}
%		[shape]
%		{format}
%		{label}
%		{sep}
%		{before-code}
%		[after-code]
%
%	1) Глава (chapter)
%
\titleformat{\chapter}[hang]% shape
{\bfseries\normalsize\centering}% format
{\hspace{\parindent}\arabic{chapter}}% label
{0.5em}% sep
{}% before-code
[]% after-code
\titleformat*{\section}%
{\bfseries\normalsize}% format
%
% 2) Раздел (section)
%
\titleformat{\section}[hang]% shape
{\bfseries\normalsize}% format
{\hspace{1.25cm}\arabic{section}}% label
{0.5em}% sep
{}% before-code
[]% after-code
%
% 3) Подраздел (subsection)
%
\titleformat{\subsection}[hang]% shape
{\bfseries\normalsize}% format
{\hspace{1.25cm}\arabic{section}.\arabic{subsection}}% label
{0.5em}% sep
{}% before-code
[]% after-code
%
% 4) ПодПодраздел (subsubsection)
%
\titleformat{\subsubsection}[hang]% shape
{\bfseries\normalsize}% format
{\hspace{1.25cm}\arabic{section}.\arabic{subsection}.\arabic{subsubsection}}% label
{0.5em}% sep
{}% before-code
[]% after-code
%
% \titlespacing{command}
%		{before-sep}
%		{after-sep}
%		[right-sep]
%
\titlespacing{\chapter}
{0pt}% before-sep
{0pt}% after-sep
{0pt}% right-sep
\titlespacing{\section}
{0pt}% before-sep
{0pt}% after-sep
{0pt}% right-sep
\titlespacing{\subsection}
{0pt}% before-sep
{0pt}% after-sep
{0pt}% right-sep
\titlespacing{\subsubsection}
{0pt}% before-sep
{0pt}% after-sep
{0pt}% right-sep
%%%%%%%%%%%%%%%%%%%%%%%%%%%%%%%%%%%%%%%%%%%%
%
% СОДЕРЖАНИЕ
%
%%%%%%%%%%%%%%%%%%%%%%%%%%%%%%%%%%%%%%%%%%%%
%\addto\captionsrussian{%
%\renewcommand{\contentsname}{%
%\vspace{-5ex}%
%\begin{center}СОДЕРЖАНИЕ\end{center}%
%\vspace{-5ex}}%
%}
%\addto\captionsrussian{%
%  \renewcommand{\figurename}{Fig.}%
%}
\makeatletter
\renewcommand{\tableofcontents}{%
\newpage%
\noindent{\centering\textbf{СОДЕРЖАНИЕ}\par}%
\vspace{\baselineskip}% пустая строка
\@starttoc{toc}\par}%
\makeatother
% глубина детализации
\setcounter{tocdepth}{3}
\setcounter{secnumdepth}{3}
% переименование формата счетчика
\renewcommand{\thesection}{\arabic{section}}
%\renewcommand{\thesubsection}{\arabic{subsection}}
%	
% \titlecontents{section}
%		[left]
%		{above-code}
%		{numbered-entry-format}
%		{numberless-entry-format}
%		{filler-page-format}
%		[below-code]
%
\usepackage{titletoc}
%	Глава (chapter)
\titlecontents{chapter}
[0pt]% left
{}%above-code
{\thechapter\;}% numbered-entry-format
{}% numberless-entry-format
{\thepage}% filler-page-format
[]% below-code
%
% Раздел (section)
\titlecontents{section}
[0pt]% left
{\normalsize}% above-code
{\thecontentslabel\;}% numbered-entry-format
{}% numberless-entry-format
{(\thepage)}% filler-page-format
[]% below-code
%
% Подраздел (subsection)
\titlecontents{subsection}
[0pt]% left
{}%above-code
{\thecontentslabel\;}% numbered-entry-format
{}% numberless-entry-format
{\thepage}% filler-page-format
[]% below-code
% ПодПодраздел (subsubsection)
\titlecontents{subsubsection}
[0pt]% left [0pt]
{}%above-code
{\thecontentslabel\;}% numbered-entry-format
{}% numberless-entry-format
{\thepage}% filler-page-format
[]% below-code
%
% \dottedcontents{section}
%		[left]
%		{above-code}
%		{label width}
%		{leader width}
\dottedcontents{chapter}%
[1.5em]% left
{}% above-code
{1.5em}% label width
{0.5em}% leader width
\dottedcontents{section}%
[1.0em]% left [1.5em]
{}% above-code
{1.0em}% label width [1.5em]
{0.5em}% leader width
\dottedcontents{subsection}%
[1.5em]% left
{}% above-code
{1.5em}% label width
{0.5em}% leader width
\dottedcontents{subsubsection}%
[2.5em]% left
{}% above-code
{2.5em}% label width
{0.5em}% leader width
%
% Chapter without a pagebreak
%
\makeatletter 
\renewcommand\chapter{\newpage\par%
\thispagestyle{plain}% \global\@topnum\z@
\@afterindentfalse\secdef\@chapter\@schapter}
\makeatother %%%%% <---- Starting chapter without a pagebreak
%
% ВВЕДЕНИЕ
%
\newenvironment{Introduction}{%
%\chapter*{\vspace{-2ex}ВВЕДЕНИЕ}\par%
%\newpage\begin{center}\textbf{ВВЕДЕНИЕ}\end{center}\par
\newpage\phantomsection%
{\centering\textbf{ВВЕДЕНИЕ}\par}%
\addcontentsline{toc}{chapter}{ВВЕДЕНИЕ}%
% пустая линия
%\vspace{\baselineskip}
}{\newpage}
%
% ЗАКЛЮЧЕНИЕ
%
\newenvironment{Conclusion}{%
%\lfoot{\empty}\cfoot{\empty}\rfoot{\arabic{page}}%
\newpage\phantomsection%
%\chapter*{\vspace{-\baselineskip}ЗАКЛЮЧЕНИЕ}\par%
{\centering\textbf{ЗАКЛЮЧЕНИЕ}\par}%
\addcontentsline{toc}{chapter}{ЗАКЛЮЧЕНИЕ}%
% пустая линия
%\vspace{\baselineskip}
}{\newpage}
%
% ПРИЛОЖЕНИЕ
%
\newenvironment{Appendix}{%
%\lfoot{\empty}\cfoot{\empty}\rfoot{\arabic{page}}%
\newpage\phantomsection%
%\chapter*{\vspace{-\baselineskip}ЗАКЛЮЧЕНИЕ}\par%
{\centering\textbf{ПРИЛОЖЕНИЕ}\par}%
\addcontentsline{toc}{chapter}{ПРИЛОЖЕНИЕ}%
% пустая линия
%\vspace{\baselineskip}
}{\newpage}
%
% СПИСОК ИСПОЛЬЗОВАННЫХ ИСТОЧНИКОВ
%
% Пакет поддерживает сжатые, 
% отсортированные списки цитирования
% https://www.ctan.org/pkg/cite
\usepackage{cite}
% формат номера источника
% Заменяем библиографию в квадратных скобках
% http://ftp.tug.org/TUGboat/tb30-1/tb94mori.pdf
% 
\makeatletter%
\renewcommand*{\@biblabel}[1]{\hfill#1\;}
\makeatother
% окружение
\newenvironment{References}[1]{%
\newpage\phantomsection
\addcontentsline{toc}{chapter}{%
СПИСОК ИСПОЛЬЗОВАННЫХ ИСТОЧНИКОВ}
\renewcommand{\bibname}{%
СПИСОК ИСПОЛЬЗОВАННЫХ ИСТОЧНИКОВ}
\begin{thebibliography}{#1}
% пустая линия
\vspace{\baselineskip}
% интервал между библиографическими источниками
\setlength{\itemsep}{0pt}
\setlength{\parskip}{0pt}
}{\end{thebibliography}}
%%%%%%%%%%%%%%%%%%%%%%%%%%%%%%%%%%%%%%%%%%%%%
%
% ТАБЛИЦЫ
%
% Пакет повышает качество таблиц в LaTeX, 
% предоставляя дополнительные команды
% https://www.ctan.org/pkg/booktabs
\usepackage{booktabs}%
% \specialrule{wd}{abovespace}{belowspace}
% горизонтальная линия с отступами сверху и снизу
\renewcommand\midrule{\specialrule{0.5pt}{1ex}{1ex}}
\renewcommand\toprule{\specialrule{1pt}{0ex}{1ex}}
\renewcommand\bottomrule{\specialrule{1pt}{1ex}{0ex}}
%\newcommand\tabsrule{\specialrule{0.5pt}{1ex}{1ex}}
\newcommand\midmidrule{
\specialrule{0.5pt}{1ex}{0.1ex}
\specialrule{0.5pt}{0.1ex}{1ex}
}
% межстрочный интервал в ТАБЛИЦАХ
\renewcommand\arraystretch{1.2}
%%%%%%%%%%%%%%%%%%%%%%%%%%%%%%%%%%%%%%%%%%%%%
%
% СПИСКИ
%
\usepackage[inline]{enumitem}% https://ctan.org/pkg/enumitem
\setlist[enumerate,itemize]{
	left=0pt,
	align=left,
	leftmargin=0pt,
	label = \arabic*),
	leftmargin=*,
	labelsep=1ex,
	itemindent=0pt,
	nosep
}
%%%%%%%%%%%%%%%%%%%%%%%%%%%%%%%%%%%%%%%%%%%%%
%
%	Цветной прямоугольник с текстом
%
\usepackage{tcolorbox}% https://www.ctan.org/pkg/tcolorbox
% Установка опций по умолчанию
\tcbset{
	notitle,
	titlebox=invisible,
	size=title,
	width=\textwidth,
	boxsep=0em,
	left=1ex,
	right=1ex,
	toptitle=0mm,
	top=1ex,
	toprule=0mm,
	bottom=1ex,
	bottomrule=0mm,
	boxrule=0mm,
	arc=0mm,
	colback=orange!10,
}
%%%%%%%%%%%%%%%%%%%%%%%%%%%%%%%%%%%%%%%%%%%%%
%
% TikZ
% https://tex.stackexchange.com/questions/131293/arguments-for-tikz-style
%
\usepackage{tikz}
\usetikzlibrary{%
	pgfplots.groupplots,%
	backgrounds,%
	calc,%
	decorations.pathmorphing,%
	decorations.markings,
	shapes,% геометрические фигуры
	arrows.meta,% стрелки разной формы
	er,% построение диаграмм
	patterns% штриховка областей
}
\tikzset{
	background rectangle/.style={% стиль фона
		fill=olive!10%
	},%
	font=\normalsize%
}
%
% PGF
% http://elib.ict.nsc.ru/jspui/bitstream/ICT/1488/1/pgf-ru-all-method.pdf
%
\usepackage{%
	pgfplots,%
	pgfplotstable%
}
% Последовательность графичеких слоев
% \pgfsetlayers{%
%	background,%
%	pre main,%
%	axis grid,% 
%	axis ticks,% 
%	axis lines,%
%	axis tick labels,%
%	main,%
%	axis descriptions,%
%	axis foreground%
%}
\usepgfplotslibrary{fillbetween}
\usepgflibrary{plotmarks}
% Установка стилей графика
\pgfplotsset{
%	compat=1.9,
	width=8cm,
	every axis/.append style={%
		thick,
		tick style={
			black,
			semithick,
		},
	}
}
%
% НУМЕРАЦИЯ формул, рисунков
%
\renewcommand{\theequation}{\arabic{equation}}
\renewcommand{\thefigure}{\arabic{figure}}
%%%%%%%%%%%%%%%%%%%%%%%%%%%%%%%%%%%%%%%%%%%%%
%
% ПОДПИСИ рисуноков, таблиц
%
\usepackage[% https://www.ctan.org/pkg/caption
	format=plain,% Печатает подписи как обычный абзац
	labelsep=endash,%
	singlelinecheck=false,% отключить центрирование однострочной подписи
	belowskip=0pt,%
	margin={0pt,0pt},%
	indention=0cm,
%	font=onehalfspacing,% полуторный интервал
]{caption}
%	рисунки
\captionsetup[figure]{%
	name=Рисунок,%
	position=below,%
	justification=centering,% центрирование подписи
	font=onehalfspacing,% полуторный интервал
}
%	таблицы
\captionsetup[table]{%
	name=Таблица,%
	position=above,%
%	justification=justified,
	font=onehalfspacing,% полуторный интервал
}
%\captionsetup{belowskip=0pt,margin={0pt,0pt}}


% *** отступы ***
% вертикальный промежуток перед и после объектов,
% местоположение которых соответствует ключу h.
% Имеет естественную длину 12 pt.
\setlength{\intextsep}{\baselineskip}
% вертикальный промежуток между текстом и соответственно 
% одно- и двухколоночными объектами, местоположение 
% которых соответствует ключам t или b.
% Имеет естественную длину 20 pt.
\setlength{\textfloatsep}{\baselineskip}
% вертикальный промежуток между соответственно 
% одно- и двухколоночными объектами, местоположение 
% которых соответствует ключам t или b. 
% Имеет естественную длину 12 pt.
\setlength{\floatsep}{\baselineskip}
% отступ перед названием 
\setlength{\abovecaptionskip}{0.5\baselineskip}
% отступ после названиея
\setlength{\belowcaptionskip}{0ex}
%
% ОТСТУПЫ В ФОРМУЛАХ
%
%\expandafter\def\expandafter\normalsize\expandafter{%
%\normalsize
%\setlength\abovedisplayskip{0.5\baselineskip}
%\setlength\belowdisplayskip{0.5\baselineskip}
%\setlength\abovedisplayshortskip{0.5\baselineskip}
%\setlength\belowdisplayshortskip{0.5\baselineskip}
%}
\AtBeginDocument{%
\abovedisplayskip=0.5\baselineskip
\abovedisplayshortskip=0.5\baselineskip
\belowdisplayskip=0.5\baselineskip
\belowdisplayshortskip=0.5\baselineskip
% отступ перед названием 
%\abovecaptionskip=0.5\baselineskip
% отступ после названиея
%\belowcaptionskip=0\baselineskip
}
%
% Пакет mhchem предоставляет команды для
% набора химических молекулярных формул и уравнений.
\usepackage[version=4,arrows=font]{mhchem}
%\usepackage{expl3,calc}
% длина стрелок
\ExplSyntaxOn
\keys_define:nn { mhchem }
{
arrow-min-length .code:n =
% default is 2em
\cs_set:Npn \__mhchem_arrow_options_minLength:n { {#1} }
}
\ExplSyntaxOff
\mhchemoptions{arrow-min-length=1em}
%
%	Пакет предоставляет макросы для управления строками - 
%	тестирования содержимого строки, извлечения подстрок, 
%	подстановки подстрок и предоставления чисел, 
%	таких как длина строки, позиция или количество 
%	повторов подстроки.
%	https://www.ctan.org/pkg/xstring
%
%	\usepackage{xstring}

%
%	Вставка страниц из PDF-файла
%	Этот пакет упрощает включение внешних многостраничных 
%	PDF-документов в LaTеX документы
% 
\usepackage{pdfpages}
%%%%%%%%%%%%%%%%%%%%%%%%%%%%%%%%%%%%%%%%%%%%%
%
%	МОИ НОВЫЕ КОМАНДЫ
%
% Макрос Существует?
% https://tex.stackexchange.com/questions/164188/ifundefined-actually-defines-macros
\def\ifexists#1{\expandafter\ifx\csname#1\endcsname\relax 0\else 1\fi}
% Pure text from TeX
% https://tex.stackexchange.com/questions/567286/pdfstringdef-turns-accented-characters-into-octal-escape-sequence
\ExplSyntaxOn
\cs_set_eq:NN\textpurify\text_purify:n
\ExplSyntaxOff
% alert
\newcommand{\alert}[2][]{%
#1{\emph{\textcolor{darkred}{#2}}}%
}
% alertx
\newcommand{\alertx}[2][\texttt]{%
#1{\textbf{\textcolor{darkred}{#2}}}%
}
% проверка макроса
\newcommand{\Isdefined}[1]{%
\ifx#1\undefined\relax%
\alertx{\backslash def\detokenize{#1}undefined}%
\else{#1}\fi%
}
% Чистый текст макроса
\newcommand{\MacroTextPurify}[1]{%
\ifx#1\undefined\relax%
\alertx{\backslash def\detokenize{#1}undefined}%
\else\textpurify{#1}%
\fi}
% Пол студента
\newcommand{\detGender}[3]{%
\ifx#1\undefined{для студентов}%
\else%
\ifcase#1\relax{#2}% Gender=0 (женский)
\or{#3}% Gender=1 (мужской)
\else\alertx{Gender=\Gender{ unknow}}%
\fi%
\fi%
}
% пустая строка
\def\emptyline{\par\vspace{\baselineskip}}
% символ '\'
\def\backslash{\char`\\}
% постоянная
\DeclareMathOperator{\const}{const}
% Полужирное начертание для векторов
\newcommand\vect[1]{\mathbfit{#1}}
%\let\vec=\mathbf
% норма
\newcommand{\norma}[1]{\left\lVert#1\right\rVert}
% абсолютное значение
\newcommand{\abs}[1]{\left\lvert#1\right\rvert}
% скалярное произведение векторов
\newcommand{\dotvec}[2]{(\vec{#1},\vec{#2})}
% обыкновенная производная, например $\diff{N_d^{+}}{x}$
\newcommand\diff[2]{ \dfrac{\mathrm{d}#1}{\mathrm{d}#2} }
% обыкновенная производная второго порядка, например $\diff{N_d^{+}}{x}$
\newcommand\diffdiff[2]{ \dfrac{\mathrm{d}^2 #1}{\mathrm{d} #2^2} }
% частная производная, например $\pdiff{N_d^{+}}{x}$
\newcommand\pdiff[2]{ \dfrac{\partial #1}{\partial #2} }
% 1/2
\newcommand\onehalf{ \nicefrac{1}{2} }
% Случайное число \RandInt{min}{max}
\newcommand{\RandInt}[2]{\pgfmathrandominteger{\rndint}{#1}{#2}\rndint}
% Интеграл
\newcommand\intf[4][x]{ \int\limits_{#2}^{#3}{#4}\,\mathrm{d}#1 }
%%%%%%%%%%%%%%%%%%%%%%%%%%%%%%%%%%%%%%%%%%%%%
%
% ТИТУЛЬНЫЙ ЛИСТ
%
% \TitlePage[название документа] или \TitlePage
\newcommand{\TitlePage}[1][\undefined]{
{\centering% начало центрирования
МИНОБРНАУКИ РОССИИ\\
Федеральное государственное бюджетное образовательное учреждение
высшего образования\\
\textbf{<<САРАТОВСКИЙ НАЦИОНАЛЬНЫЙ ИССЛЕДОВАТЕЛЬСКИЙ\\
ГОСУДАРСТВЕННЫЙ УНИВЕРСИТЕТ\\
ИМЕНИ Н.Г. ЧЕРНЫШЕВСКОГО>>}
\emptyline
Кафедра материаловедения,\par
технологии и управления качеством
\emptyline
% название документа (DocumentTitle)
\ifnum1=\ifexists{SubTitle}\MakeUppercase{\SubTitle}\emptyline\fi
%\ifx#1\undefined\relax% не определено
%\else\MakeUppercase{#1}\emptyline%
%\fi
% ЗАГЛАВИЕ
\textbf{\MakeUppercase{\Isdefined{\TITLE}}}
\emptyline
по дисциплине <<\Isdefined{\ModuleTitle}>>\\
% для студентов/студентки/студента
\detGender{\Gender}{студентки}{студента}
\Isdefined{\NoCourse}{ курса }\Isdefined{\NoGroup}{ группы}\\
направления подготовки{ \Isdefined\ProgramCode }
<<\Isdefined{\ProgramTitle}>>
(профиль <<\Isdefined{\ProgramProfile}>>),\\
\Isdefined{\Department}\\% факультет
% Ф.И.О. (родительный падеж)
\ifx\FullNameGenetive\undefined\relax%
%\end{center}%
\else{\FullNameGenetive}%
% пустая строка
\emptyline
% добавляет заполняющее вертикальное пространство
\vfill
% таблица БАРС
\begin{tabular}[l]{b{5cm} b{2.5cm} c}
\toprule
Результат&Баллы&ВСЕГО\\
\midmidrule
Выполнение&&\Isdefined{\Exec}\\
\midrule
Оформление&&\Isdefined{\Polygraphy}\\
\midrule
Устный отчет&&\Isdefined{\OralReport}\\
\bottomrule
\end{tabular}%
% пустая строка
\par\emptyline
% ПОДПИСЬ
\raggedright\noindent
Преподаватель\\профессор, д.т.н., доцент
\hspace{1em}\rule{6.5cm}{0.5pt} 
\hspace{1em} В.В. Симаков\par
\emptyline
\fi%
}\par% окончание центрирования
}
%%%%%%%%%%%%%%%%%%%%%%%%%%%%%%%%%%%%%%%%%%%%%
%
% Колонтитулы
%
% 'Титульный лист'
\fancypagestyle{titlepage}{
\setcounter{page}{1}
% clear all header and footer fields
\fancyhf{}
% линии верхнего и нижнего колонтитулов
\renewcommand{\headrulewidth}{0pt}%
\renewcommand{\footrulewidth}{0pt}%
\cfoot{Саратов~\the\year} % город год
}
% 'Задание'
\fancypagestyle{taskpages}{
% clear all header and footer fields
\fancyhf{}
% нумерация римскими цифрами
%\pagenumbering{roman}
% нумерация арабскими цифрами
\pagenumbering{arabic}
% линии верхнего и нижнего колонтитулов
\renewcommand{\headrulewidth}{0pt}%
\renewcommand{\footrulewidth}{0pt}%
% нижний колонтитул
\lfoot{Задание~
\detGender{\Gender}{получила}{получил}
\hspace{0.25cm}\rule{4cm}{0.5pt}\hspace{0.25cm}%
\Isdefined{\Signature}}% подпись
%\rfoot{\roman{page}}% номера страниц
}
% 'Содержание документа'
\fancypagestyle{bodypages}{
\fancyhf{}
\pagestyle{fancy}
% линии верхнего и нижнего колонтитулов
\renewcommand{\headrulewidth}{0pt}%
\renewcommand{\footrulewidth}{0pt}%
\pagenumbering{arabic}
\rfoot{\arabic{page}}% номера страниц
}
%%%%%%%%%%%%%%%%%%%%%%%%%%%%%%%%%%%%%%%%%%%%%
%
% ЗАДАНИЕ
%
% \TitleTask или \TitleTask['на выполнение чего?']
\newcommand{\TitleTask}[1][лабораторной работы]{%
\newpage
% нумерация страниц с №1
\setcounter{page}{1}
%\newpage\phantomsection
\begin{center}
\textbf{ЗАДАНИЕ}\par
% добавление пункта в СОДЕРЖАНИЕ
%\addcontentsline{toc}{chapter}{Задание на выполнение #1}
%по дисциплине <<\Isdefined{\@ModuleTitle}>>
по дисциплине <<\MacroTextPurify{\ModuleTitle}>>
на выполнение #1 на тему 
<<\textbf{\MacroTextPurify{\TITLE}}>>
\end{center}
\par
}
% sgu
%%%%%%%%%%%%%%%%%%%%%%%%%%%%%%
% Стиль графиков
\pgfplotsset{
width=8cm,% размер графиков
xlabel=$x$,ylabel=$f(x)$,
every axis/.style={% оси координат
	font=\small,% размер шрифта
	color=black,solid,thick,
	xtick style={semithick,black},
	ytick style={semithick,black},
	grid=major,% сетка
	major grid style={	thin,dashed,color=gray!25,},
},
every node near coord/.append style={
%font=\footnotesize,%
font=\small,%
color=black,%
%fill=olive!10,%
above,
yshift=+2pt,%
/pgf/number format/fixed,%
/pgf/number format/fixed zerofill,%
/pgf/number format/precision=1,%
},
}
% стиль графиков
\pgfplotsset{
compat=1.15,
PlotDarkRed/.style={
	thick,
	color=darkred,
	mark size=3pt,
	mark=ball,
	mark options={% маркеры
		thin,
		draw=darkred,
		ball color=darkred!50,
	},
},
PlotDarkBlue/.style={
	thick,
	color=darkblue,
	mark size=3pt,
	mark=ball,
	mark options={% маркеры
		thin,
		draw=darkblue,
		ball color=darkblue!50,
	},
},
}
\begin{document}
% колонтитулы
\pagestyle{bodypages}
\newpage
\setcounter{page}{2}
% СОДЕРЖАНИЕ
\tableofcontents
%
% ВВЕДЕНИЕ
%
\begin{Introduction}
Особенно острая потребность в развитии численных методов 
появилась в связи с необходимостью решения новых сложных задач, 
возникающих в ходе развития современной физики и новейших областей техники и технологий. 
Кроме того, использование численных методов допускает применения 
простых и вполне осуществимых вычислений. 


---
Обыкновенными дифференциальными 
уравнениями можно описать задачи движения системы 
взаимодействующих материальных точек, химической кинетики, 
электрических цепей, сопротивления материалов (например, статический 
прогиб упругого стержня) и многие другие. 

Ряд важных задач для уравнений в частных производных также сводится к задачам 
для обыкновенных дифференциальных уравнений. Например, 
если многомерная задача допускает разделение пространственных переменных 
(например, задачи на нахождение собственных колебаний упругих балок 
и мембран простейшей формы, или определение спектра собственных значений энергии частицы 
в сферически-симметричном поле), 
или если ее решение зависит только от некоторой комбинации 
переменных (автомодельные решения), то задача нахождения решения уравнений в частных производных 
сводится к задачам на собственные значения для обыкновенных дифференциальных уравнений. 

Таким образом, решение обыкновенных дифференциальных уравнений 
занимает важное место среди прикладных задач физики, химии и техники.

\textbf{Цель} данной работы состоит в приобретении практических навыков применения 
численных методов решения задачи Коши для систем 
обыкновенных дифференциальных уравнений первого порядка.
\end{Introduction}

% Метод Гаусса
%% Выделение цветом
\pgfkeys{/main/.code=\fcolorbox{gray}{yellow}{$#1$}}
\pgfkeys{/one/.code=\textcolor{darkred}{1}}
\pgfkeys{/zero/.code=\textcolor{gray!50}{0}}
%\pgfkeys{/main=hi!}% пример использования
%
%	Метод Гаусса решения систем линейных уравнений
%
\newpage
\section{Метод Гаусса решения систем линейных уравнений}
К решению систем линейных алгебраических уравнений сводится подавляющее 
большинство задач вычислительной математики и многие численные методы основаны 
на решении систем линейных уравнений вида:
\begin{gather}\label{eq:LS}
\left\{
\begin{matrix}
a_{11}\cdot x_1&+&a_{12}\cdot x_2&+&\dots&+&a_{1m}\cdot x_m&=&b_1\\
a_{21}\cdot x_1&+&a_{22}\cdot x_2&+&\dots&+&a_{2m}\cdot x_m&=&b_2\\
a_{31}\cdot x_1&+&a_{32}\cdot x_2&+&\dots&+&a_{3m}\cdot x_m&=&b_3\\
\hdotsfor{9}\\
a_{m1}\cdot x_1&+&a_{m2}\cdot x_2&+&\dots&+&a_{mm}\cdot x_m&=&b_m
\end{matrix}
\right.,
\end{gather}
где $x_1,x_2,\dots,x_m$ --
неизвестные, которые необходимо определить;
$\{a_{ij}\}$ и \linebreak$b_1,b_2,\dots,b_m$ -- 
известные числовые коэффициенты левой и правой 
частей системы уравнений, соответственно.

Решение системы линейных алгебраических уравнений 
\eqref{eq:LS} представляет совокупность $m$ 
действительных или мнимых чисел $\{s_1,s_2,\dots,s_m\}$, 
таких что их соответствующая подстановка вместо 
$\{x_1,x_2,\dots,x_m\}$ в систему обращает все её 
уравнения в тождества:
\begin{gather*}
\left\{
\begin{matrix}
a_{11}\cdot s_1&+&a_{12}\cdot s_2&+&\dots&+&a_{1m}\cdot s_m&\equiv&b_1\\
a_{21}\cdot s_1&+&a_{22}\cdot s_2&+&\dots&+&a_{2m}\cdot s_m&\equiv&b_2\\
a_{31}\cdot s_1&+&a_{32}\cdot s_2&+&\dots&+&a_{3m}\cdot s_m&\equiv&b_3\\
\hdotsfor{9}\\
a_{m1}\cdot s_1&+&a_{m2}\cdot s_2&+&\dots&+&a_{mm}\cdot s_m&\equiv&b_m
\end{matrix}
\right.,
\end{gather*}

Система линейных алгебраических уравнений \eqref{eq:LS}
может быть представлена в более компактной 
матричной форме:
\begin{gather}\label{eq:LSM}
\mathbf{A}\cdot\vect{x}=\vect{b},
\end{gather}
где $\mathbf{A}$ -- квадратная матрица $m\times m$,
$\vect{x}$ и $\vect{b}$ -- искомый вектор неизвестных и
заданный вектор размерности $1\times m$, правых частей
системы уравнений:
\begin{gather*}
\mathbf{A}=
\begin{pmatrix}
a_{11}&a_{12}&\cdots&a_{1m}\\
a_{21}&a_{22}&\cdots&a_{2m}\\
\vdots&\vdots&\ddots&\vdots\\
a_{m1}&a_{m2}&\cdots&a_{mm}\\
\end{pmatrix},
\quad
\vect{x}=\begin{pmatrix}x_1\\x_2\\\vdots\\x_m\end{pmatrix},
\quad
\vect{b}=\begin{pmatrix}b_1\\b_2\\\vdots\\b_m\end{pmatrix}.
\end{gather*}

Теорема Кронекера--Капелли устанавливает необходимое и 
достаточное условие совместности системы 
линейных алгебраических уравнений посредством 
свойств матричных представлений: 
система совместна тогда и только тогда, когда ранг её матрицы 
совпадает с рангом расширенной матрицы,
полученной путем добавления столбца правых частей $\vect{b}$
матрице системы $\mathbf{A}$: 
\begin{gather}\label{eq:LSAM}
(\mathbf{A}\,\vert\,\vect{b})=
\begin{pmatrix}[cccc|c]
a_{11}&a_{12}&\cdots&a_{1m}&b_1\\
a_{21}&a_{22}&\cdots&a_{2m}&b_2\\
\vdots&\vdots&\ddots&\vdots&\vdots\\
a_{m1}&a_{m2}&\cdots&a_{mm}&b_m\\
\end{pmatrix}
\end{gather}

Преимущество расширенной матрицы заключается в 
возможности выполнения тех же операций с вектором 
правых частей системы уравнений, что и со строками
матрицы.

Предполагается, что определитель матрицы $\mathbf{A}$ 
отличен от нуля $\det\mathbf{A}\ne0$,
так что решение $\vect{x}=(x_1,x_2,\dots,x_m)^\mathrm{T}$
существует и единственно.
Систему линейных уравнений можно решить 
по крайней мере двумя способами:
либо воспользовавшись \emph{формулами Крамера}, 
либо методом последовательного исключения неизвестных 
(\emph{методом Гаусса}).
При больших порядка матрицы $m$ способ Крамера, 
основанный на вычислении определителей,
требует порядка $m!$ арифметических действий, 
в то время как метод Гаусса -- $O(m^3)$ действий.

Для большинства вычислительных задач характерным 
является большой порядок матрицы 
$\mathbf{A}$ ($m\approx10^2\dots10^5$),
поэтому метод Гаусса в различных вариантах широко 
используется при решении задач линейной алгебры на ЭВМ.

%
%	Прямой ход метода Гаусса
%
\emptyline
\subsection{Прямой ход метода Гаусса}
Метод Гаусса состоит в последовательном исключении 
неизвестных $x_i$ из системы линейных уравнений
\eqref{eq:LSAM}.
Например, предположим, что $a_{11}\ne0$, 
тогда разделим первое уравнение системы на $a_{11}$, 
и в результате получим:
\begin{gather*}
\begin{pmatrix}[cccc|c]
1&c_{12}&\cdots&c_{1m}&y_1\\
a_{21}&a_{22}&\cdots&a_{2m}&b_2\\
\vdots&\vdots&\ddots&\vdots&\vdots\\
a_{m1}&a_{m2}&\cdots&a_{mm}&b_m\\
\end{pmatrix},
\end{gather*}
где $c_{1j}$ и $y_1$ -- нормированные коэффициенты 
1-ой строки и правой части 1-го уравнения, соответственно:
\begin{gather*}
c_{1j}=\dfrac{a_{1j}}{a_{11}}\quad (j=2,3,\ldots,m),\quad
y_1=\dfrac{b_1}{a_{11}}.
\end{gather*}

Последовательно умножим первое уравнение системы на $a_{i1}$ и 
вычтем полученное уравнение из каждого $i$-го уравнения системы 
$i=2,3,\dots,m$. В результате получим следующую матрицу:
\begin{gather*}
\begin{pmatrix}[crcr|r]
1&c_{12}&\cdots&c_{1m}&y_1\\
0&a_{22}-a_{21}\cdot c_{12}&\cdots&a_{2m}-a_{21}\cdot c_{1m}&b_2-a_{21}\cdot y_1\\
\vdots&\vdots&\ddots&\vdots&\vdots\\
0&a_{m2}-a_{m1}\cdot c_{12}&\cdots&a_{mm}-a_{m1}\cdot c_{1m}&b_m-a_{m1}\cdot y_1\\
\end{pmatrix}.
\end{gather*}

Запишем полученную матрицу в более компактном виде:
\begin{gather*}
\begin{pmatrix}[cccc|r]
1&c_{12}&\cdots&c_{1m}&y_1\\
0&a_{22}^{(1)}&\cdots&a_{2m}^{(1)}&b_2^{(1)}\\
\vdots&\vdots&\ddots&\vdots&\vdots\\
0&a_{m2}^{(1)}&\cdots&a_{mm}^{(1)}&b_m^{(1)}\\
\end{pmatrix},
\end{gather*}
где $a_{ij}^{(1)}$ и $b_i^{(1)}$
-- модифицированные коэффициенты при неизвестных и правой части,
соответственно.
\begin{gather*}
a_{ij}^{(1)}=(a_{ij}-a_{i1}\cdot c_{1j}), \quad 
b_i^{(1)}=(b_i-a_{i1}\cdot y_{1}), \quad
(i,j=2,3,\dots,m)
\end{gather*}

Если $a_{22}^{(1)}\ne0$, то в модифицированной системе 
аналогично можно исключить неизвестное $x_2$. 
Для этого можно разделить второе уравнение системы 
на коэффициент при второй неизвестной $a_{22}^{(1)}$,
и в результате получим:
\begin{gather*}
\begin{pmatrix}[cccc|r]
1&c_{12}&\cdots&c_{1m}&y_1\\
0&1&\cdots&c_{2m}&y_2\\
\vdots&\vdots&\ddots&\vdots&\vdots\\
0&a_{m2}^{(1)}&\cdots&a_{mm}^{(1)}&b_m^{(1)}\\
\end{pmatrix},
\end{gather*}
где $c_{2j}$ и $y_2$ -- нормированные коэффициенты 
2-ой строки и правой части 2-го уравнения, соответственно.
\begin{gather*}
c_{2j}=\dfrac{a_{2j}^{(1)}}{a_{22}^{(1)}}
\quad(j=3,4,\ldots,m),\quad
y_2=\dfrac{b_1^{(1)}}{a_{22}^{(1)}}.
\end{gather*}

Последовательно умножим второе уравнение системы 
на $a_{i2}^{(1)}$ и вычтем полученное уравнение 
из каждого $i$-го уравнения системы $i=3,4,\ldots,m$.
В результате расширенная матрица примет вид:
\begin{gather*}
\begin{pmatrix}[crcr|r]
1&c_{12}&\cdots&c_{1m}&y_1\\
0&1&\cdots&c_{2m}&y_2\\
\vdots&\vdots&\ddots&\vdots&\vdots\\
0&0&\cdots&a_{mm}^{(1)}-a_{m2}^{(1)}\cdot c_{2m}&b_m^{(1)}-a_{m2}^{(1)}\cdot y_2\\
\end{pmatrix}
\end{gather*}

или в более компактном виде:
\begin{gather*}
\begin{pmatrix}[cccc|r]
1&c_{12}&\cdots&c_{1m}&y_1\\
0&1&\cdots&c_{2m}&y_2\\
\vdots&\vdots&\ddots&\vdots&\vdots\\
0&0&\cdots&a_{mm}^{(2)}&b_m^{(2)}\\
\end{pmatrix}
\end{gather*}
где $a_{ij}^{(2)}$ и $b_i^{(2)}$
-- повторно модифицированные коэффициенты при неизвестных 
и правой части, соответственно:
\begin{gather*}
a_{ij}^{(2)}=(a_{ij}^{(1)}-a_{i2}^{(1)}\cdot c_{2j}),\quad
b_i^{(2)}=(b_i^{(1)}-a_{i2}^{(1)}\cdot y_{2}),\quad
(i,j=3,4,\dots,m)
\end{gather*}

Исключая таким же образом неизвестные $x_3,x_4,\dots,x_m$, 
исходная система линейных уравнений приводится 
к эквивалентному виду:
\begin{gather}\label{eq:UMatrix}
\begin{pmatrix}[cccc|r]
1&c_{12}&\cdots&c_{1m}&y_1\\
0&1&\cdots&c_{2m}&y_2\\
\vdots&\vdots&\ddots&\vdots&\vdots\\
0&0&\cdots&1&y_m\\
\end{pmatrix}
\end{gather}

%
%	Обратный ход метода Гаусса
%
\emptyline
\subsection{Обратный ход метода Гаусса}
Обратный ход заключается в нахождении неизвестных 
$x_1,x_2,\dots,x_m$ полученной эквивалентной системы 
в прямом ходе метода Гаусса.
Поскольку расширенная матрица системы \eqref{eq:UMatrix}
имеет треугольный вид, 
то можно последовательно найти все неизвестные 
$x_m,x_{m-1}, \dots,x_1$:
\begin{gather*}
\left\{\begin{array}{lclclcl}
x_m&=&y_m\\
x_{m-1}&=&y_{m-1}&-&c_{m-1,m}\cdot x_m\\
x_{m-2}&=&y_{m-2}&-&c_{m-2,m-1}\cdot x_{m-1}&-&c_{m-2,m}\cdot x_m\\
\hdotsfor{1}&=&\hdotsfor{5}\\
x_1&=&y_1&-&\sum\limits_{j=2}^{m} c_{1j}\cdot x_j\\
\end{array}\right.
\end{gather*}

Общие формулы обратного хода имеют вид:
\begin{gather*}
x_m=y_m,\quad 
x_i=y_i - \sum\limits_{j=i+1}^{m} c_{ij}\cdot x_j,
\quad i=(m-1,m-2,\dots,1)
\end{gather*}

%
%	Пример решения методом Гаусса
%
\emptyline
\subsection{Численное решение системы линейных 
алгебраических уравнений методом Гаусса}
Представим систему линейных алгебраических уравнений
в матричном виде и запишем расширенную матрицу этой системы,
полученную путем добавления к матрице системы
столбца правой части уравнений:
\begin{gather*}
\left\{\begin{matrix}
2x_1&+&3x_2&+&x_3&=&10\\
4x_1&+&5x_2&+&6x_3&=&31\\
3x_1&+&x_2&+&5x_3&=&22\\
\end{matrix}\right.
\quad\iff\quad
\begin{pmatrix}[ccc|c]
\pgfkeys{/main=2}&3&1&10\\
4&5&6&31\\
3&1&5&22\\
\end{pmatrix}
\end{gather*}

% Прямой ход метода Гаусса
\emph{Прямой ход метода Гаусса.}
\begin{enumerate}
\item
Разделим каждую строку матрицы на значение её
элемента в первом столбце, т.е. первую строку делим на $2$,
вторую на $4$, третью на 3:
\begin{gather*}
\begin{pmatrix}[ccc|c]
\pgfkeys{/main=2}&3&1&10\\
4&5&6&31\\
3&1&5&22\\
\end{pmatrix}\quad\to\quad
% результат
\begin{pmatrix}[ccc|c]
\pgfkeys{/main=1}&\frac{3}{2}&\frac{1}{2}&\frac{10}{2}\\
\pgfkeys{/one}&\frac{5}{4}&\frac{6}{4}&\frac{31}{4}\\
\pgfkeys{/one}&\frac{1}{3}&\frac{5}{3}&\frac{22}{3}\\
\end{pmatrix}
\end{gather*}

Вычитаем из второй и третьей строк матрицы её первую строку:
\begin{gather*}
\begin{pmatrix}[ccc|c]
\pgfkeys{/main=1}&\frac{3}{2}&\frac{1}{2}&\frac{10}{2}\\
1&\frac{5}{4}&\frac{6}{4}&\frac{31}{4}\\
1&\frac{1}{3}&\frac{5}{3}&\frac{22}{3}\\
\end{pmatrix}\quad\to\quad
% результат
\begin{pmatrix}[ccc|c]
\pgfkeys{/main=1}&\frac{3}{2}&\frac{1}{2}&\frac{10}{2}\\
\pgfkeys{zero}&-\frac{1}{4}&1&\frac{11}{4}\\
\pgfkeys{zero}&-\frac{7}{6}&\frac{7}{6}&\frac{7}{3}\\
\end{pmatrix}
\end{gather*}

\item
Разделим вторую строку матрицы на $-\frac{1}{4}$, 
третью строку на $-\frac{7}{6}$:
\begin{gather*}
\begin{pmatrix}[ccc|c]
\pgfkeys{/main=1}&\frac{3}{2}&\frac{1}{2}&\frac{10}{2}\\
\pgfkeys{zero}&\pgfkeys{/main={-\frac{1}{4}}}&1&\frac{11}{4}\\
\pgfkeys{zero}&-\frac{7}{6}&\frac{7}{6}&\frac{7}{3}\\
\end{pmatrix}\quad\to\quad
% результат
\begin{pmatrix}[ccc|c]
\pgfkeys{/main=1}&\frac{3}{2}&\frac{1}{2}&\frac{10}{2}\\
\pgfkeys{zero}&\pgfkeys{/main=1}&-4&-11\\
\pgfkeys{zero}&\pgfkeys{/one}&-1&-2\\
\end{pmatrix}
\end{gather*}

Вычитаем из третьей строки матрицы её вторую строку:
\begin{gather*}
\begin{pmatrix}[ccc|c]
\pgfkeys{/main=1}&\frac{3}{2}&\frac{1}{2}&\frac{10}{2}\\
\pgfkeys{zero}&\pgfkeys{/main=1}&-4&-11\\
\pgfkeys{zero}&1&-1&-2\\
\end{pmatrix}\quad\to\quad
% результат
\begin{pmatrix}[ccc|c]
\pgfkeys{/main=1}&\frac{3}{2}&\frac{1}{2}&\frac{10}{2}\\
\pgfkeys{zero}&\pgfkeys{/main=1}&-4&-11\\
\pgfkeys{zero}&\pgfkeys{zero}&3&9\\
\end{pmatrix}
\end{gather*}

\item
Разделим третью строку матрицы на $3$:
\begin{gather*}
\begin{pmatrix}[ccc|c]
1&\frac{3}{2}&\frac{1}{2}&\frac{10}{2}\\
\pgfkeys{zero}&1&-4&-11\\
\pgfkeys{zero}&\pgfkeys{zero}&3&9\\
\end{pmatrix}\quad\to\quad
% результат
\begin{pmatrix}[ccc|c]
\pgfkeys{/main=1}&\frac{3}{2}&\frac{1}{2}&\frac{10}{2}\\
\pgfkeys{zero}&\pgfkeys{/main=1}&-4&-11\\
\pgfkeys{zero}&\pgfkeys{zero}&\pgfkeys{/main=1}&3\\
\end{pmatrix}
\end{gather*}
\end{enumerate}

\emph{Обратный ход метода Гаусса}.\par
Последовательно определяем неизвестные
в обратном порядке их следования
$x_3\to x_2\to x_1$:
\begin{enumerate}
\item
Из третьего уравнения системы (третья строка матрицы)
определяем неизвестное $x_3$:
\begin{gather*}
x_3=3
\end{gather*}
\item
Из второго уравнения системы (вторая строка матрицы)
определяем неизвестное $x_2$:
\begin{gather*}
x_2-4\,x_3=-11\quad\to\quad
x_2=4\,x_3-11\\
x_2=4\cdot3-11=1
\end{gather*}
\item
Из первого уравнения системы (первая строка матрицы)
определяем неизвестное $x_1$:
\begin{gather*}
x_1+\frac{3}{2}\,x_2+\frac{1}{2}\,x_3=\frac{10}{2}\quad\to\quad
x_1=-\frac{3}{2}\,x_2-\frac{1}{2}\,x_3+5\\[1ex]
x_1=-\frac{3}{2}\cdot{1}-\frac{1}{2}\cdot{3}+5=2
\end{gather*}

Таким образом, найдено решение системы:
\begin{gather*}
\vect{s}=\begin{pmatrix}2\\1\\3\end{pmatrix}
\qquad\iff\qquad
\vect{s}=(2,1,3)^\mathrm{T}
\end{gather*}
\end{enumerate}

Проведём \emph{проверку решения} системы уравнений
методом прямой подстановки найденного 
вектора неизвестных $\vect{s}=(2,1,3)^\mathrm{T}$ 
в исходную систему уравнений:
\begin{gather*}
\mathbf{A}\cdot\vect{s}=\vect{b}
\quad\iff\quad
\begin{pmatrix}[ccc|c]
2&3&1&10\\
4&5&6&31\\
3&1&5&22\\
\end{pmatrix}
\cdot\begin{pmatrix}2\\1\\3\\\end{pmatrix}
\end{gather*}

После умножения расширенной матрицы
системы уравнений на вектор найденного решения
получим тождество:
\begin{gather*}
\begin{pmatrix}[c|c]
2\cdot2+3\cdot1+1\cdot3&10\\
4\cdot2+5\cdot1+6\cdot3&31\\
3\cdot2+1\cdot1+5\cdot3&22\\
\end{pmatrix}
=
\begin{pmatrix}[c|c]
10&10\\
31&31\\
22&22\\
\end{pmatrix}
\end{gather*}

%
%	Метод Гаусса с выбором главного элемента
%
\emptyline
\subsection{Метод Гаусса с выбором главного элемента}
На практике, часто может оказаться, что система \eqref{eq:LSM}
имеет единственное решение, хотя какой-либо 
из угловых миноров матрицы $\mathbf{A}$ равен нулю. 
Кроме того, заранее обычно неизвестно, 
все ли угловые миноры матрицы $\mathbf{A}$ отличны от нуля.
В этих случаях обычный метод Гаусса может оказаться 
\emph{непригодным}.
Избежать указанных трудностей позволяет метод Гаусса 
с выбором главного элемента.

\emph{Основная идея метода} состоит в том, чтобы 
на очередном шаге исключать не следующее по номеру неизвестное,
а то неизвестное, коэффициент при котором 
является \emph{наибольшим по модулю}.
Таким образом, в качестве ведущего элемента здесь выбирается 
\alert{главный}, т.е. наибольший по модулю элемент.
Поэтому, если $\det\mathbf{A}\ne0$, то в процессе вычислений 
не будет происходить деление на нуль.

На практике чаще всего применяется и метод Гаусса 
с выбором главного элемента по всей матрице, 
когда в качестве ведущего выбирается максимальный 
по модулю элемент \emph{среди всех элементов} матрицы системы.

\emptyline
\subsection{Численное решение системы линейных 
алгебраических уравнений методом Гаусса
с выбором главного элемента по всей матрице}
Рассмотрим на примере решение системы линейных уравнений
методом Гаусса с выбором главного элемента по всей
матрице системы:
\begin{gather*}
\left\{\begin{matrix}
2x_1&+&3x_2&+&x_3&=&10\\
4x_1&+&5x_2&+&6x_3&=&31\\
3x_1&+&x_2&+&5x_3&=&22\\
\end{matrix}\right.
\quad\iff\quad
\begin{pmatrix}[ccc|c]
2&3&1&10\\
4&5&6&31\\
3&1&5&22\\
\end{pmatrix}
\end{gather*}

\emph{Прямой ход метода Гаусса с выбором главного элемента}.
\begin{enumerate}
\item
Выбираем максимальный по модулю элемент в матрице 
(главный элемент) во второй строке, третьем столбце
(выделен цветом):
\begin{gather*}
\begin{pmatrix}[ccc|c]
2&3&1&10\\
4&5&\pgfkeys{/main=6}&31\\
3&1&5&22\\
\end{pmatrix}
\end{gather*}
Разделим каждую строку матрицы на значение элемента
матрицы в столбце главного элемента, т.е.
первую строку делим на $1$, вторую строку на $6$,
а третью строку на $5$:
\begin{gather*}
\begin{pmatrix}[ccc|c]
2&3&1&10\\
4&5&\pgfkeys{/main=6}&31\\
3&1&5&22\\
\end{pmatrix}\quad\to\quad
% результат
\begin{pmatrix}[ccc|c]
2&3&\pgfkeys{/one}&10\\
\frac{4}{6}&\frac{5}{6}&\pgfkeys{/main=1}&\frac{31}{6}\\
\frac{3}{5}&\frac{1}{5}&\pgfkeys{/one}&\frac{22}{5}\\
\end{pmatrix}
\end{gather*}

Вычитаем из первой и третьей строки вторую строку:
\begin{gather*}
\begin{pmatrix}[ccc|c]
2&3&1&10\\
\frac{4}{6}&\frac{5}{6}&\pgfkeys{/main=1}&\frac{31}{6}\\
\frac{3}{5}&\frac{1}{5}&1&\frac{22}{5}\\
\end{pmatrix}\quad\to\quad
% результат
\begin{pmatrix}[ccc|c]
\frac{4}{3}&\frac{13}{6}&\pgfkeys{/zero}&\frac{29}{6}\\
\frac{4}{6}&\frac{5}{6}&\pgfkeys{/main=1}&\frac{31}{6}\\
-\frac{1}{15}&-\frac{19}{30}&\pgfkeys{/zero}&-\frac{23}{30}\\
\end{pmatrix}
\end{gather*}

\item
Исключаем из рассмотрения строку с текущим главным 
элементом (вторую строку) и выбираем новый 
главный элемент матрицы (первая строка, второй столбец): 
\begin{gather*}
\begin{pmatrix}[ccc|c]
\frac{4}{3}&\pgfkeys{/main=\frac{13}{6}}&\pgfkeys{/zero}&\frac{29}{6}\\
\frac{4}{6}&\frac{5}{6}&\pgfkeys{/main=1}&\frac{31}{6}\\
-\frac{1}{15}&-\frac{19}{30}&\pgfkeys{/zero}&-\frac{23}{30}\\
\end{pmatrix}
\end{gather*}

Делим каждую строку матрицы на значение элемента
матрицы в столбце главного элемента, т.е.
первую строку делим на $\frac{13}{6}$, а третью строку на $-\frac{19}{30}$:
\begin{gather*}
\begin{pmatrix}[ccc|c]
\frac{4}{3}&\pgfkeys{/main=\frac{13}{6}}&\pgfkeys{/zero}&\frac{29}{6}\\
\frac{4}{6}&\frac{5}{6}&\pgfkeys{/main=1}&\frac{31}{6}\\
-\frac{1}{15}&-\frac{19}{30}&\pgfkeys{/zero}&-\frac{23}{30}\\
\end{pmatrix}\quad\to\quad
% результат
\begin{pmatrix}[ccc|c]
\frac{8}{13}&\pgfkeys{/main=1}&\pgfkeys{/zero}&\frac{29}{13}\\
\frac{4}{6}&\frac{5}{6}&\pgfkeys{/main=1}&\frac{31}{6}\\
\frac{2}{19}&\pgfkeys{/one}&\pgfkeys{/zero}&\frac{23}{19}\\
\end{pmatrix}
\end{gather*}

Вычитаем из третьей строки матрицы её первую строку:
\begin{gather*}
\begin{pmatrix}[ccc|c]
\frac{8}{13}&\pgfkeys{/main=1}&\pgfkeys{/zero}&\frac{29}{13}\\
\frac{4}{6}&\frac{5}{6}&\pgfkeys{/main=1}&\frac{31}{6}\\
\frac{2}{19}&1&\pgfkeys{/zero}&\frac{23}{19}\\
\end{pmatrix}\quad\to\quad
% результат
\begin{pmatrix}[ccc|c]
\frac{8}{13}&\pgfkeys{/main=1}&\pgfkeys{/zero}&\frac{29}{13}\\
\frac{4}{6}&\frac{5}{6}&\pgfkeys{/main=1}&\frac{31}{6}\\
-\frac{126}{247}&\pgfkeys{/zero}&\pgfkeys{/zero}&-\frac{252}{247}\\
\end{pmatrix}
\end{gather*}

\item
Исключаем из рассмотрения строку с текущим главным 
элементом (первую) и выбираем новый главный 
элемент матрицы (третья строка, первый столбец): 
\begin{gather*}
\begin{pmatrix}[ccc|c]
\frac{8}{13}&\pgfkeys{/main=1}&\pgfkeys{/zero}&\frac{29}{13}\\
\frac{4}{6}&\frac{5}{6}&\pgfkeys{/main=1}&\frac{31}{6}\\
\pgfkeys{/main=-\frac{126}{247}}&\pgfkeys{/zero}&\pgfkeys{/zero}&-\frac{252}{247}\\
\end{pmatrix}
\end{gather*}

Делим каждую строку матрицы на значение элемента
матрицы в столбце главного элемента, т.е.
третью строку делим на $-\frac{126}{247}$:
\begin{gather*}
\begin{pmatrix}[ccc|c]
\frac{8}{13}&\pgfkeys{/main=1}&\pgfkeys{/zero}&\frac{29}{13}\\
\frac{4}{6}&\frac{5}{6}&\pgfkeys{/main=1}&\frac{31}{6}\\
\pgfkeys{/main=-\frac{126}{247}}&\pgfkeys{/zero}&\pgfkeys{/zero}&-\frac{252}{247}\\
\end{pmatrix}\quad\to\quad
% результат
\begin{pmatrix}[ccc|c]
\frac{8}{13}&\pgfkeys{/main=1}&\pgfkeys{/zero}&\frac{29}{13}\\
\frac{4}{6}&\frac{5}{6}&\pgfkeys{/main=1}&\frac{31}{6}\\
\pgfkeys{/main=1}&\pgfkeys{/zero}&\pgfkeys{/zero}&2\\
\end{pmatrix}
\end{gather*}
\end{enumerate}

\emph{Обратный ход метода Гаусса 
с выбором главного элемента}.\par
Определим неизвестные из уравнений системы 
в обратном порядке следования номеров столбцов главных элементов,
т.е. $x_1\to x_2\to x_3$:
\begin{enumerate}
\item
Из третьего уравнения системы определим неизвестное $x_1$:
\begin{gather*}
x_1=2
\end{gather*}
\item
Из первого уравнения системы определим неизвестное $x_2$:
\begin{gather*}
\frac{8}{13}\,x_1+x_2=\frac{29}{13}\quad\to\quad
x_2=-\frac{8}{13}\,x_1+\frac{29}{13}\\[1ex]
x_2=-\frac{8}{13}\cdot2+\frac{29}{13}=1
\end{gather*}
\item
Из второго уравнения системы определим неизвестное $x_3$:
\begin{gather*}
\frac{4}{6}\,x_1+\frac{5}{6}\,x_2+x_3=\frac{31}{6}\quad\to\quad
x_3=-\frac{4}{6}\,x_1-\frac{5}{6}\,x_2+\frac{31}{6}\\[1ex]
x_3=-\frac{4}{6}\cdot2-\frac{5}{6}\cdot1+\frac{31}{6}=3
\end{gather*}
Таким образом, найдено решение системы линейных уравнений:
\begin{gather*}
\vect{s}=\begin{pmatrix}2\\1\\3\end{pmatrix}
\qquad\iff\qquad
\vect{s}=(2,1,3)^\mathrm{T}
\end{gather*}
\end{enumerate}

%\end{document}

% Интерполяция полиномом Лагранжа
\newpage
\section{Интерполирование функций}
Задача интерполирования состоит в том, чтобы по 
известным значениям функции $f(x)$ в отдельных
точках отрезка восстановить её значения 
в остальных точках этого отрезка. Такая постановка 
задачи допускает множество решений.

Например, задача интерполирования возникает, 
в том случае, когда известны результаты измерения 
$y_i=f(x_i)$ некоторой физической величины $f$ 
в ограниченном количестве точек $x_i$ ($i=0,1,\dots,n$), 
а требуется оценить значения этой величины в других точках.

Интерполирование используется также, когда вычисление 
значений $f(x)$ трудоемко, например, значение искомой функции
может быть определено как решение сложной задачи, 
в которой $x$ играет роль параметра. При этом 
можно вычислить небольшую таблицу значений функции, 
но прямое нахождение функции при большом числе значений 
аргумента практически затруднительно или нецелесообразно. 

\emptyline
\subsection{Линейная интерполяция функции}
При \emph{линейной интерполяции} функция $f(x)$
на отрезке $x\in[a, b]$ заменяется обобщенным 
интерполяционным полиномом 
$p_n(x)$, который построен в виде линейной комбинации 
$(n+1)$ аналитических функций $\{\phi_i(x)\}$
\begin{gather}\label{eq:Interpolation_Polynom}
p_n(x)=c_0\cdot\phi_0(x)+c_1\cdot\phi_1(x)+\ldots+c_n\cdot\phi_n(x)
=\sum\limits_{i=0}^n c_i\cdot\phi_i(x),
\end{gather}
таким образом, чтобы значения полинома $p_n(x)$ 
в определённых точках отрезка $\{x_0,x_1,\dots,x_n\}$ 
(узлах сетки) совпадают со значениями функции 
в этих точках $\{y_0,y_1,\dots,y_n\}$ (условия сопряжения):
\begin{gather}\label{eq:Interpolation Conjugation}
\left\{\begin{matrix}
p_n(x_0)&=&y_0\\
p_n(x_1)&=&y_1\\
\hdotsfor{3}\\
p_n(x_n)&=&y_n\\
\end{matrix}\right.,\quad\iff\quad
% подробно
\left\{\begin{matrix}
\sum\limits_{i=0}^n c_i\cdot\phi_i(x_0)&=&y_0\\[1em]
\sum\limits_{i=0}^n c_i\cdot\phi_i(x_1)&=&y_1\\
\hdotsfor{3}\\
\sum\limits_{i=0}^n c_i\cdot\phi_i(x_n)&=&y_n\\
\end{matrix}\right..
\end{gather}

Из условий \eqref{eq:Interpolation Conjugation}, 
накладываемых на интерполяционный полином,
формулируется система линейных уравнений относительно 
неизвестных коэффициентов полинома ${c_0,c_1,\dots,c_n}$:
\begin{gather}\label{eq:Interpolation_LS}
\mathbf{A}\cdot\vect{c}=\vect{y},
\end{gather}
где $\mathbf{A}$ -- квадратная матрица $(n+1)\times(n+1)$,
$\vect{c}$ и $\vect{y}$ -- 
вектор неизвестных коэффициентов полинома $p_n(x)$
и вектор значений функции $f(x)$ в заданных точках $\{x_i\}$:
\begin{gather*}
\mathbf{A}=
\begin{pmatrix}
\phi_0(x_0)&\phi_1(x_0)&\cdots&\phi_n(x_0)\\
\phi_0(x_1)&\phi_1(x_1)&\cdots&\phi_n(x_1)\\
\vdots&\vdots&\ddots&\vdots\\
\phi_0(x_n)&\phi_1(x_n)&\cdots&\phi_n(x_n)\\
\end{pmatrix},
\quad
\vect{c}=\begin{pmatrix}c_0\\c_1\\\vdots\\c_n\end{pmatrix},
\quad
\vect{y}=\begin{pmatrix}y_0\\y_1\\\vdots\\y_n\end{pmatrix}.
\end{gather*}

Если среди узлов интерполяции $\{x_i\}$ нет совпадающих
($x_i\ne x_j$ для всех $i,j=0,1,\dots,n$) и 
определитель системы отличен от нуля $\det\mathrm{A}\ne0$,
то задача интерполяции имеет единственное решение, а
система функций $\{\phi_i(x)\}$ называется чебышевской. 
Поэтому при линейной интерполяции необходимо 
строить обобщенный полином $p_n(x)$ на основе 
\emph{чебышевской системы функций}.

Таким образом, для определения коэффициентов 
интерполяционного полинома 
\eqref{eq:Interpolation_Polynom} необходимо 
найти решение системы линейных уравнений
\eqref{eq:Interpolation_LS}, 
любыми аналитическими, приближенными
или численными методами, например, методом Гаусса.

Интерполирование не всегда дает удовлетворительное решение 
задачи о приближении функции \emph{с заданной точностью}
на данном промежутке, так как совпадение функции $f(x)$
с полиномом $p(x)$ в точках $x_i$ и $x_{i+1}$
не гарантирует малость величины
$\abs{f(x)-p(x)}$ на отрезке $[x_i,x_{i+1}]$.

%
% Интерполирование алгебраическими полиномами
%
\emptyline
\subsection{Интерполяция алгебраическими полиномами}
Задача интерполяции алгебраическими полиномами 
сводится к построению полинома степени $n$ по
чебышевской системе алгебраических функций
$\{1,x,x^2,\dots,x^n\}$:
\begin{equation}\label{eq:Interpolation_AP}
p_{n}(x)
=c_{0}+c_{1}\cdot{x}+c_{2}\cdot{x^2}+\ldots+c_{n}\cdot{x^n}
=\sum\limits_{i=0}^{n}c_i\cdot x^i.
\end{equation}

Определитель системы \eqref{eq:Interpolation_LS}
представляет собой определителем Вандермонда, 
который  отличен от нуля $\det\mathbf{A}\ne0$ , 
если среди точек $\{x_i\}$ нет совпадающих,
т.е. $x_i\ne x_j$ для всех $i,j=0,1,\dots,n$:
\begin{gather*}
\det\mathbf{A}=
\begin{vmatrix}
1&x_0&\cdots&x_0^n\\
1&x_1&\cdots&x_1^n\\
\vdots&\vdots&\ddots&\vdots\\
1&x_n&\cdots&x_n^n\\
\end{vmatrix}.
\end{gather*}

Выражение для коэффициентов алгебраического полинома
и вид самого полинома \eqref{eq:Interpolation_AP} 
можно записать различными способами.
Наиболее распространенная запись интерполяционного 
многочлена в форме Лагранжа и в форме Ньютона. 

%
%	Интерполяция функций полиномами Лагранжа
%
\emptyline
\subsection{Интерполяционный полином в форме Лагранжа}
Интерполяционная формула Лагранжа позволяет 
представить многочлен $L_{n}(x)$ в виде 
линейной комбинации значений функции $y(x)$ 
в узлах интерполирования $\{x_i\}$:
\begin{equation}\label{eq:Lagrange}
L_{n}(x)
=\lambda_{0}(x)\cdot{y_0}+\lambda_{1}(x)\cdot{y_1}+\dots+\lambda_{n}(x)\cdot{y_n}
=\sum\limits_{i=0}^n \lambda_i(x)\cdot y_i
\end{equation}
где $\lambda_0(x),\lambda_1(x),\dots,\lambda_n(x)$ -- произвольные неизвестные функции.

Для определения неизвестных функций $\lambda_i(x)$ 
из условий интерполирования следует:
\begin{equation*}\label{inter2}
\left\{\begin{matrix}
\lambda_0(x_0)\cdot{y_0}+\lambda_1(x_0)\cdot{y_1}+\dots+\lambda_n(x_0)\cdot{y_n}&=&y_0\\
\lambda_0(x_1)\cdot{y_0}+\lambda_1(x_1)\cdot{y_1}+\dots+\lambda_n(x_1)\cdot{y_n}&=&y_1\\
\hdotsfor{3}\\
\lambda_0(x_n)\cdot{y_0}+\lambda_1(x_n)\cdot{y_1}+\dots+\lambda_n(x_n)\cdot{y_n}&=&y_n\\
\end{matrix}\right.
\end{equation*}

Эта система уравнений имеет решение если выполняются условия:
\begin{equation*}
\label{uslovia_c}
\lambda_{i}(x_j)=\left\{\begin{matrix}
1, &x_j=x_{i}\\
0, &x_j\ne{x_{i}}
\end{matrix}\right.
\end{equation*}

Коэффициенты $\lambda_{i}(x)$ можно искать в виде 
многочленов степени $n$:
\begin{equation*}
\label{eq_c}
\left\{\begin{matrix}
\lambda_0(x)&=&\alpha_0\cdot(x-x_1)\cdot(x-x_2)
\cdot(x-x_2)\cdot&\ldots&\cdot(x-x_n)\\
\lambda_1(x)&=&\alpha_1\cdot(x-x_0)\cdot(x-x_2)
\cdot(x-x_3)\cdot&\ldots&\cdot(x-x_n)\\
\hdotsfor{5}\\
\lambda_n(x)&=&\alpha_n\cdot(x-x_0)\cdot(x-x_1)
\cdot(x-x_2)\cdot&\ldots&\cdot(x-x_{n-1})
\end{matrix}\right.
\end{equation*}

Определим неизвестные 
$\alpha_0, \alpha_1, \ldots, \alpha_n$ 
из условия для коэффициентов $\lambda_i(x)$:
\begin{equation*}
\left\{
\begin{matrix}
1&=&\alpha_0\cdot(x_0-x_1)\cdot(x_0-x_2)\cdot(x_0-x_2)\cdot&
\dots&\cdot(x_0-x_n)\\
1&=&\alpha_1\cdot(x_1-x_0)\cdot(x_1-x_2)\cdot(x_1-x_3)\cdot&
\dots&\cdot(x_1-x_n)\\
\hdotsfor{5}\\
1&=&\alpha_n\cdot(x_n-x_0)\cdot(x_n-x_1)\cdot(x_n-x_2)\cdot&
\dots&\cdot(x_n-x_{n-1})
\end{matrix}
\right.
\end{equation*}

Таким образом, коэффициенты $\lambda_{i}(x)$ 
интерполяционного многочлена\linebreak
Лагранжа находятся из соотношений:
\begin{equation*}
\left\{\begin{matrix}
\lambda_0(x)&=&\dfrac
{(x-x_1)\cdot(x-x_2)\cdot\dots\cdot(x-x_n)}
{(x_0-x_1)\cdot(x_0-x_2)\cdot\dots\cdot(x_0-x_n)}\\[1em]
\lambda_1(x)&=&\dfrac
{(x-x_0)\cdot(x-x_2)\cdot\dots\cdot(x-x_n)}
{(x_1-x_0)\cdot(x_1-x_2)\cdot\dots\cdot(x_1-x_n)}\\[1em]
\hdotsfor{3}\\[1em]
\lambda_n(x)&=&\dfrac
{(x-x_0)\cdot(x-x_1)\cdot\dots\cdot(x-x_{n-1})}
{(x_n-x_0)\cdot(x_n-x_1)\cdot\dots\cdot(x_n-x_{n-1})}
\end{matrix}\right.,
\end{equation*}
или в более компактной форме:
\begin{equation*}
\lambda_i(x)=\dfrac
{\prod\limits_{j \ne i}^n (x - x_j)}
{\prod\limits_{j \ne i}^n (x_i - x_j)}
\end{equation*}

Итак, интерполяционный многочлен Лагранжа 
\eqref{eq:Lagrange} имеет вид:
\begin{equation}\label{eq:Lagrange_Polynom}
L_{n}(x)=\sum\limits_{i=0}^n\dfrac
{\prod\limits_{j \ne i}^n(x-x_j) }
{\prod\limits_{j \ne i}^n(x_i-x_j)}\cdot{y_i}
\end{equation}

%
% Интерполирование таблично заданной функции
%
\emptyline
\subsection{Интерполяция функции заданной таблично}
Известно множество данных (узлов интерполяции)
$\{x_i\}$, в которых определены 
значения функции $y_i=f(x_i)$:
\begin{table}[h]
\vspace{-0.5\baselineskip}
\caption{Таблично заданная функциональная зависимость}
\label{tab:Interpolation_Data}
\begin{tabular*}{\textwidth}{%
l@{\extracolsep{\fill}}*{4}{r}p{0.25cm}}
\toprule
$i$&$0$&$1$&$2$&$3$\\
\midmidrule
$x_i$&$-0.76$&$-0.09$&$0.22$&$0.55$\\
\addlinespace% дополнительный пробел
$y_i$&$0.08$&$1.84$&$0.40$&$0.96$\\
\bottomrule
\end{tabular*}
\end{table}

Построим обобщенный интерполяционный полином $p_3(x)$
для таблично заданной функции исходя из чебышевской 
системы функций $\{1,x,e^{-x},e^x\}$:
\begin{gather*}
p_4(x) = c_0+c_1\cdot{x}+c_2\cdot e^{-x}+c_3\cdot e^x
\end{gather*}

\begin{enumerate}
\item
Составим расширенную матрица системы уравнений 
\eqref{eq:Interpolation_LS} для определения 
коэффициентов полинома $(c_0,c_1,c_2,c_3)^\mathrm{T}$:
\begin{gather*}
\begin{pmatrix}[cccc|c]
\phi_0(x_0)&\phi_1(x_0)&\phi_2(x_0)&\phi_3(x_0)&y_0\\
\phi_0(x_1)&\phi_1(x_1)&\phi_2(x_1)&\phi_3(x_1)&y_1\\
\phi_0(x_2)&\phi_1(x_2)&\phi_2(x_2)&\phi_3(x_2)&y_2\\
\phi_0(x_3)&\phi_1(x_3)&\phi_2(x_3)&\phi_3(x_3)&y_3\\
\end{pmatrix},\quad\text{здесь}\quad
\left\{\begin{matrix}
\phi_0(x)&=&1\\
\phi_1(x)&=&x\\
\phi_2(x)&=&e^{-x}\\
\phi_3(x)&=&e^x\\
\end{matrix}\right..
\end{gather*}
\item
Воспользуемся данными таблицы \ref{tab:Interpolation_Data}
и заполним числовыми значения элементы расширенной матрицы:
\begin{gather*}
\begin{pmatrix}[cccc|c]
1&-0.76&e^{0.76}&e^{-0.76}&0.08\\
1&-0.09&e^{0.09}&e^{-0.09}&1.84\\
1&0.22&e^{-0.22}&e^{0.22}&0.40\\
1&0.55&e^{-0.55}&e^{0.55}&0.96\\
\end{pmatrix}
\;\iff\;
\begin{pmatrix}[cccc|c]
1&-0.76&\fcolorbox{gray!50}{yellow}{2.138}&0.468&0.08\\
1&-0.09&1.094&0.914&1.84\\
1&0.22&0.803&1.246&0.40\\
1&0.55&0.577&1.733&0.96\\
\end{pmatrix}
\end{gather*}

\item
Решение системы линейных уравнений
найдем методом Гаусса с выбором главного элемента
в расширенной матрице (выделен цветом):
\begin{gather*}
\vect{c}=(-0.393,-81.472,-37.288,39.053)^\mathrm{T}
\end{gather*}

Следовательно, обобщенный интерполяционный полином
для функции заданной таблично можно записать в виде:
\begin{gather*}
\textcolor{darkblue}{
p_3(x) = -0.393-81.472\cdot{x}-37.288\cdot e^{-x}+39.053\cdot e^x
}
\end{gather*}

\item
В таблице \ref{tab:Interpolation_GPD} представлены 
данные расчета коэффициентов обобщенного интерполяционного 
полинома $c_i$, значений этого полинома в узлах сетки $p_3(x_i)$
и абсолютная погрешность интерполяции 
$\varepsilon_i=y_i-p_3(x_i)$.
%
% Таблица результатов
%
\vspace{-0.5\baselineskip}
\begin{table}[H]
\caption{Коэффициенты обобщенного интерполяционного 
полинома $c_i$, значения этого полинома в узлах сетки $p_3(x_i)$
и абсолютная погрешность интерполяции $\varepsilon_i$}
\label{tab:Interpolation_GPD}
\begin{tabular*}{\textwidth}{%
@{\extracolsep{\fill}}*{5}{r}p{2cm}}
\toprule
$i$&$0$&$1$&$2$&$3$\\
\midmidrule% @x,@y
$x_i$&$-0.76$&$-0.09$&$0.22$&$0.55$\\
$y_i$&$0.08$&$1.84$&$0.4$&$0.96$\\
\midrule% коэффициенты полинома
$c_i$&$-0.393$&$-81.472$&$-37.288$&$39.053$\\
\midrule% значение полинома и погрешность
$p_3(x_i)$&$0.057$&$1.832$&$0.422$&$0.973$\\
$\varepsilon_i$&$0.023$&$0.008$&$-0.022$&$-0.013$\\
\bottomrule
\end{tabular*}
\end{table}

\item
На рисунке \ref{fig:Interpolation_GP} представлена 
диаграмма рассеяния (разброса) данных 
функции заданной таблично $y_i=f(x_i)$ (маркеры) и 
результаты вычислений обобщенного интерполяционного 
полинома $p_3(x)$ (сплошная линия).
\begin{figure}[H]\centering
\begin{tikzpicture}
\begin{axis}[ylabel=$p_3(x)$,
xmin=-0.9,xmax=0.7,xtick={-0.8,-0.4,0,0.4},
ymin=-1,ymax=3]
% табличные данные
\addplot[ball darkblue,only marks] coordinates {(-0.76,0.08)(-0.09,1.84)(0.22,0.4)(0.55,0.96)};
% интерполяционный полином
\addplot[darkblue,domain=-0.8:0.6,samples=50]
{-0.393*1-81.472*x-37.288*exp(-x)+39.053*exp(x)}
node[pos=0.7,right] {$p_3(x)$};
\end{axis}
\end{tikzpicture}
\caption{График таблично заданной функции $y_i=f(x_i)$ (маркеры) 
и обобщенного интерполяционного полинома $p_3(x)$
(сплошная линия)}
\label{fig:Interpolation_GP}
\end{figure}
\end{enumerate}

%
% Таблица графика
%
% xmin = -0.8
% xmax = 0.6
% dx   = 0.028
%
\begin{table}[H]
\caption{Рассчётные значения обобщенного интерполяционного 
полинома в узлах сетки $p_3(x_i)$}
\label{tab:Interpolation Plot}
\small
\begin{tabular*}{\textwidth}{%
p{1cm}@{\extracolsep{\fill}}*{9}{r}}
\toprule
$i$&$0$&$1$&$2$&$\dots$&$\dots$&$\dots$&$47$&$48$&$49$\\
\midmidrule
$x_i$&$-0.800$&$-0.771$&$-0.743$&$\dots$&$\dots$&$\dots$&$0.543$&$0.571$&$0.600$\\
$y_i$&$-0.654$&$-0.128$&$0.330$&$\dots$&$\dots$&$\dots$&$0.920$&$1.145$&$1.419$\\
\bottomrule
\end{tabular*}
\end{table}


Построим интерполяционный полином 
в форме Лагранжа $L_3(x)$ на основе данных 
об узлах интерполяции $\{x_i\}$ 
и значений функции в этих узлах $\{y_i\}$:
\begin{gather*}
L_{3}(x)=\sum \limits_{i=0}^3\dfrac
{\prod\limits_{j \ne i}^3(x-x_j)}
{\prod\limits_{j \ne i}^3(x_i-x_j)}\cdot{y_i}
\end{gather*}

\begin{enumerate}
\item
Представим полином Лагранжа в развернутом виде:
\begin{gather*}
\begin{split}
L_{3}(x)=
&\;\dfrac{(x-x_1)\cdot(x-x_2)\cdot(x-x_3)}{(x_0-x_1)\cdot(x_0-x_2)\cdot(x_0-x_3)}\cdot y_0+\\[1ex]
&\;\dfrac{(x-x_0)\cdot(x-x_2)\cdot(x-x_3)}{(x_1-x_0)\cdot(x_1-x_2)\cdot(x_1-x_3)}\cdot y_1+\\[1ex]
&\;\dfrac{(x-x_0)\cdot(x-x_1)\cdot(x-x_3)}{(x_2-x_0)\cdot(x_2-x_1)\cdot(x_2-x_3)}\cdot y_2+\\[1ex]
&\;\dfrac{(x-x_0)\cdot(x-x_1)\cdot(x-x_2)}{(x_3-x_0)\cdot(x_3-x_1)\cdot(x_3-x_2)}\cdot y_3
\end{split}
\end{gather*}

\item
Воспользуемся численными данными об узлах интерполяции 
$\{x_i\}$ и известными значениями интерпретируемой функции 
в этих узлах $\{y_i\}$:
\begin{gather*}
\begin{split}
L_{3}(x)=
&\;\dfrac{(x-(-0.09))\cdot(x-0.22)\cdot(x-0.55)}{(-0.76-(-0.09))\cdot(-0.76-0.22)\cdot(-0.76-0.55)}\cdot0.08+\\[1ex]
&\;\dfrac{(x-(-0.76))\cdot(x-0.22)\cdot(x-0.55)}{(-0.09-(-0.76))\cdot(-0.09-0.22)\cdot(-0.09-0.55)}\cdot1.84+\\[1ex]
&\;\dfrac{(x-(-0.76))\cdot(x-(-0.09))\cdot(x-0.55)}{(0.22-(-0.76))\cdot(0.22-(-0.09))\cdot(0.22-0.55)}\cdot0.40+\\[1ex]
&\;\dfrac{(x-(-0.76))\cdot(x-(-0.09))\cdot(x-0.22)}{(0.55-(-0.76))\cdot(0.55-(-0.09))\cdot(0.55-0.22)}\cdot0.96
\end{split}
\end{gather*}

\item
Проведем необходимые арифметические действия:
\begin{gather*}
\begin{split}
L_{3}(x)=
&\;\dfrac{(x+0.09)\cdot(x-0.22)\cdot(x-0.55)}{(-0.67)\cdot(-0.98)\cdot(-1.31)}\cdot0.08+\\[1ex]
&\;\dfrac{(x+0.76)\cdot(x-0.22)\cdot(x-0.55)}{(0.67)\cdot(-0.31)\cdot(-0.64)}\cdot1.84+\\[1ex]
&\;\dfrac{(x+0.76)\cdot(x+0.09)\cdot(x-0.55)}{(0.98)\cdot(0.31)\cdot(-0.33)}\cdot0.40+\\[1ex]
&\;\dfrac{(x+0.76)\cdot(x+0.09)\cdot(x-0.22)}{(1.31)\cdot(0.64)\cdot(0.33)}\cdot0.96
\end{split}
\end{gather*}

или
\begin{gather*}
\begin{matrix}
L_{3}(x)=
&\;\dfrac{(x+0.09)\cdot(x-0.22)\cdot(x-0.55)}{-0.86}\cdot0.08+\\[1ex]
&\;\dfrac{(x+0.76)\cdot(x-0.22)\cdot(x-0.55)}{0.13}\cdot1.84+\\[1ex]
&\;\dfrac{(x+0.76)\cdot(x+0.09)\cdot(x-0.55)}{-0.10}\cdot0.40+\\[1ex]
&\;\dfrac{(x+0.76)\cdot(x+0.09)\cdot(x-0.22)}{0.28}\cdot0.96&
\end{matrix}
\end{gather*}

Продолжая делать упрощения окончательно получим:
\begin{gather*}
\begin{split}
L_{3}(x)=
&\;(x+0.09)\cdot(x-0.22)\cdot(x-0.55)\cdot(-0.09)+\\
&\;(x+0.76)\cdot(x-0.22)\cdot(x-0.55)\cdot13.84+\\
&\;(x+0.76)\cdot(x+0.09)\cdot(x-0.55)\cdot(-3.99)+\\
&\;(x+0.76)\cdot(x+0.09)\cdot(x-0.22)\cdot3.47
\end{split}
\end{gather*}

\item
Запишем выражение для интерполяционный полином Лагранжа
в каноническом виде:
\begin{gather*}\textcolor{darkred}{
L_{3}(x)=1.37 - 5.248\cdot x + 0.912\cdot x^2 + 13.23\cdot x^3
}
\end{gather*}

\item
В таблице \ref{tab:Interpolation_LPD} представлены 
данные расчета коэффициентов интерполяционного 
полинома Лагранжа $c_i$, значений этого полинома 
в узлах сетки $L_3(x_i)$ и абсолютная погрешность интерполяции 
$\varepsilon_i=y_i-L_3(x_i)$.
%
% Таблица
%
\begin{table}[H]
\caption{Коэффициенты интерполяционного 
полинома Лагранжа $c_i$, значения этого полинома 
в узлах сетки $L_3(x_i)$ и абсолютная погрешность 
интерполяции $\varepsilon_i$}
\label{tab:Interpolation_LPD}
\begin{tabular*}{\textwidth}{%
@{\extracolsep{\fill}}*{6}{r}}
\toprule
$i$&$x_i$&$y_i$&$c_i$&$L_3(x_i)$&$\varepsilon_i$\\
\midmidrule
0&$-0.76$&$0.08$&$1.37$&$0.078$&$0.002$\\
1&$-0.09$&$1.84$&$-5.248$&$1.840$&$0.000$\\
2&$0.22$&$0.40$&$0.912$&$0.400$&$0.000$\\
3&$0.55$&$0.96$&$13.23$&$0.961$&$-0.001$\\
\bottomrule
\end{tabular*}
\end{table}

\item
На рисунке \eqref{fig:Interpolation_LP} представлена 
диаграмму рассеяния (разброса) данных функции 
заданной таблично $y_i=f(x_i)$ (маркеры) и 
результаты вычислений интерполяционного 
полинома Лагранжа $L_3(x)$ (сплошная линия).
% *******************************
%	График функций
%
\begin{figure}[H]\centering
\begin{tikzpicture}
\begin{axis}[% оси координат
ylabel=$L_3(x)$,
xmin=-0.9,xmax=0.7,xtick={-0.8,-0.4,0,0.4},
ymin=-1, ymax=3,
]
% табличные данные
\addplot[ball darkred,only marks]
coordinates {(-0.76,0.08) (-0.09,1.84) (0.22,0.40) (0.55,0.96)};
% полином Лагранжа
\addplot[darkred,mark=none,domain=-0.8:0.6,samples=100] 
{1.37 - 5.248*x + 0.912*x^2 + 13.23*x^3}
node[pos=0.7,right] {$L_3(x)$};
\end{axis}
\end{tikzpicture}
\caption{График таблично заданной функции $y_i=f(x_i)$ (маркеры) 
и интерполяционного полинома Лагранжа $L_3(x)$
(сплошная линия)}
\label{fig:Interpolation_LP}
\end{figure}
% *******************************
\end{enumerate}

% Аппроксимация
\newpage
%
%	Аппроксимация функция
%
\section{Аппроксимация функция}
Задача о приближении функции ставится следующим образом:
данную функцию $f(x)$ необходимо заменить 
обобщенным полиномом $p_m(x)$ заданного порядка $m$ 
так, чтобы отклонение (в известном смысле) функции $f(x)$ 
от обобщенного полинома $p_m(x)$ на указанном множестве 
$\vec{x}=\{x\}$ было наименьшим. 
При этом полином $p_m(x)$ в общем случае 
называется аппроксимирующим.

Если множество $\vec{x}$ состоит из отдельных точек 
$x\in\{x_0, x_1, x_2, \dots x_n\}$ (узлов),
то приближение называется \textit{точечным}.
Если $\vec{x}$ есть отрезок $x_a<x<x_b$, 
то приближение называется \textit{интегральным}. 
Для практики важным является приближение функций 
алгебраическими и тригонометрическими полиномами.

\subsection{Точечное квадратичное аппроксимирование функций}
На практике часто бывает, что заданный порядок $m$ 
приближающего полинома $p_m(x)$ меньше числа 
узлов аппроксимации ${m<n}$, в которых 
известно значение функции $y_i=f(x_i)$ ($i=0,1,2, \cdots, n$).
В этом случае обычно используют точечный 
метод наименьших квадратов и
рассматривается полином степени $m$ вида:
\begin{gather*}
p_m(x)=c_0+c_1\cdot{x}+c_2\cdot{x^2}+\dots+c_m\cdot{x^m}=
\sum\limits_{j=0}^{m}c_j\cdot{x^j}.
\end{gather*}

В качестве меры отклонения $\norma{r}$ полинома $p_m(x)$ 
от известной функции $y(x)$ на множестве точек 
$\{x_0, x_1, x_2,\cdots,x_n\}$, как правило, принимается 
сумма квадратов отклонений полинома от этой функции 
на заданной системе точек:
\begin{gather*}
\norma{r}=\sum_{i=0}^{n}\left(p_m(x_i)-y_i\right)^2
\end{gather*}

Следует отметить, что мера отклонения полинома 
от известной функции есть функция многих переменных
$\norma{r}=g(c_0, c_1, \dots, c_m)$, т.е. коэффициентов полинома
$c_i$ ($i=0,1,\dots,m$), которые необходимо подобрать так, 
чтобы величина меры отклонения была наименьшей 
$\norma{r}\to{\min}$.
Полученный полином называется аппроксимирующим 
для данной функции, а процесс построения этого полинома -- 
точечной квадратичной аппроксимацией или 
точечным квадратичным аппроксимированием функции. 

Для решения задачи точечного квадратичного аппроксимирования,
т.е. определения числовых значений всех коэффициентов 
полинома $p_m(x)$, необходимо найти \emph{положения минимума 
функции} многих переменных $\norma{r}$.

Определим частные производные от величины суммы квадратов отклонений и 
воспользовавшись условием экстремума функции многих переменных, 
составим систему уравнений вида:
\begin{gather*}
\pdiff{\norma{r}}{c_0}=
\pdiff{\norma{r}}{c_1}=
\pdiff{\norma{r}}{c_2}=\cdots=
\pdiff{\norma{r}}{c_m}=0
\end{gather*}

Для определения неизвестных коэффициентов полинома
$c_0, c_1, c_2,\dots, c_m$ необходимо решить систему 
$m+1$ уравнений с $m+1$ неизвестными: 
\begin{gather*}
\renewcommand*{\arraystretch}{1.5}
\left\{\begin{array}{lclcl}
\pdiff{\norma{r}}{a_0}&=&2\cdot\sum\limits_{i=0}^{n}\left(c_0+c_1\cdot{x_i}+c_2\cdot{x_i^2}+\ldots+c_m\cdot{x_i^m} - y_i\right)\cdot1&=&0\\
\pdiff{\norma{r}}{a_1}&=&2\cdot\sum\limits_{i=0}^{n}\left(c_0+c_1\cdot{x_i}+c_2\cdot{x_i^2}+\ldots+c_m\cdot{x_i^m} - y_i\right)\cdot{x_i}&=&0\\
\pdiff{\norma{r}}{a_2}&=&2\cdot\sum\limits_{i=0}^{n}\left(c_0+c_1\cdot{x_i}+c_2\cdot{x_i^2}+\ldots+c_m\cdot{x_i^m} - y_i\right)\cdot{x_i^2}&=&0\\
\hdotsfor{1}&=&\hdotsfor{1}&=&0\\
\pdiff{\norma{r}}{a_m}&=&2\cdot\sum\limits_{i=0}^{n}\left(c_0+c_1\cdot{x_i}+c_2\cdot{x_i^2}+\ldots+c_m\cdot{x_i^m} - y_i\right)\cdot{x_i^m}&=&0\\
\end{array}\right.
\end{gather*}

Таким образом, задача точечной квадратичной аппроксимации 
функции сводится к решению системы линейных уравнений 
относительно неизвестных -- коэффициентов полинома 
$\{c_0, c_1, c_2,\dots, c_m\}$:
\begin{gather*}
\begin{matrix}
\mathbf{A}\cdot\vec{c}=\vec{b}
&\text{или}&
\begin{pmatrix}
a_{00}&a_{01}&\cdots&a_{0m}\\
a_{10}&a_{11}&\cdots&a_{1m}\\
\vdots&\vdots&\ddots&\vdots\\
a_{m0}&a_{m1}&\cdots&a_{mm}\\
\end{pmatrix}
\cdot
\begin{pmatrix}c_0\\c_1\\\vdots\\c_m\end{pmatrix}
=\begin{pmatrix}b_0\\b_1\\\vdots\\b_m\end{pmatrix}
\end{matrix},\end{gather*}
где $\mathbf{A}=\{a_{k\ell}\}$ и $\vec{b}=\{b_k\}$ 
-- квадратная матрица и вектор правых частей 
системы линейных уравнений, соответственно:
\begin{gather*}
a_{k\ell}=\sum\limits_{i=0}^n x_i^k\cdot x_i^\ell,
\quad
b_{k}=\sum\limits_{i=0}^n x_i^k\cdot y_i,
\quad k,\ell=0,1,2,\dots,m
\end{gather*}

Если среди узлов сетки $\{x_i\}$ 
нет совпадающих, а также степень полинома 
меньше чем число узлов аппроксимации $m<n$, 
то определитель системы не равен нулю $\det\mathbf{A}\ne0$.
Следовательно, эта система имеет единственное решение 
$\{\mathring{c}_0, \mathring{c}_1, \mathring{c}_2,\dots, \mathring{c}_m\}$,
а полином $p_m(x)$ с такими коэффициентами $\mathring{c}_i$ 
будет обладать минимальным квадратичным отклонением 
$\norma{r}_{\min}$. 

%
%	Аппроксимирования функций полиномом второй степени $p_2(x)$
%
\subsection{Аппроксимирования функций полиномом
второй степени $p_2(x)$}
Известна таблица данных некоторой функциональной зависимости 
$y(x)$:
\begin{table}[H]
\vspace{-0.5\baselineskip}
\caption{Таблично заданная функциональная зависимость
$y_i=f(x_i)$}
\begin{tabular*}{\textwidth}{%
l@{\extracolsep{\fill}}*{5}{r}p{0.25cm}}
\toprule
$i$&$0$&$1$&$2$&$3$&$4$\\
\midmidrule
$x_i$&$-0.76$&$-0.48$&$-0.09$&$0.22$&$0.55$\\
\addlinespace% дополнительный пробел
$y_i$&$5.15$&$4.39$&$4.10$&$5.71$&$5.30$\\
\bottomrule
\end{tabular*}
\end{table}

Необходимо аппроксимировать функцию $\{y_i\}$,
заданную таблично, алгебраическим полиномом 
второй степени $p_2(x)$:
\begin{gather*}
p_2(x)=c_0 + c_1\cdot x + c_2\cdot x^2
\end{gather*}

\begin{enumerate}
\item
Построим меру отклонения полинома $p_2(x)$ 
от таблично заданной функции $y_i=f(x_i)$
на множестве точек $\{x_0, x_1, x_2, x_3, x_4\}$:
\begin{gather*}
\norma{r}=\sum_{i=0}^{4}\left(c_0+c_1\cdot{x_i}+c_2\cdot{x_i^2}-y_i\right)^2,
\end{gather*}
где $y_i=f(x_i)$ -- значение функции в точке $x_i$.

\item
Запишем меру отклонения $\norma{r}$ в явном виде 
на основе данных из условия задачи:
\begin{gather*}
\begin{split}
\norma{r}=
&\left(c_0 + c_1\cdot(-0.76) + c_2\cdot(-0.76)^2 - 5.15 \right)^2+\\
+&\left(c_0 + c_1\cdot(-0.48) + c_2\cdot(-0.48)^2 - 4.39 \right)^2+\\
+&\left(c_0 + c_1\cdot(-0.09) + c_2\cdot(-0.09)^2 - 4.10 \right)^2+\\
+&\left(c_0 + c_1\cdot(0.22) + c_2\cdot(0.22)^2 - 5.71 \right)^2+\\
+&\left(c_0 + c_1\cdot(0.55) + c_2\cdot(0.55)^2 - 5.30 \right)^2
\end{split}
\end{gather*}

\item
Определим частную производную от меры отклонений $\norma{r}$ 
по аргументу $c_0$ и приравняем её нулю:
\begin{gather*}
\begin{split}
\pdiff{\norma{r}}{c_0}=
&2\cdot\left(a_0 + a_1\cdot(-0.76) + a_2\cdot(-0.76)^2 - 5.15 \right)\cdot 1+\\
&2\cdot\left(a_0 + a_1\cdot(-0.48) + a_2\cdot(-0.48)^2 - 4.39 \right)\cdot 1+\\
&2\cdot\left(a_0 + a_1\cdot(-0.09) + a_2\cdot(-0.09)^2 - 4.10 \right)\cdot 1+\\
&2\cdot\left(a_0 + a_1\cdot(0.22) + a_2\cdot(0.22)^2 - 5.71 \right)\cdot 1+\\
&2\cdot\left(a_0 + a_1\cdot(0.55) + a_2\cdot(0.55)^2 - 5.30 \right)\cdot 1=0
\end{split}
\end{gather*}

Коэффициенты первой строки матрицы $\mathbf{A}$
и первый элемент вектора $\vec{b}$:
\begin{gather*}
\begin{array}{lcl}
a_{00}&=&1+1+1+1+1=5\\
a_{01}&=&(-0.76) + (-0.48) + (-0.09) + (0.22) + (0.55) = -0.56\\
a_{02}&=&(-0.76)^2 + (-0.48)^2 + (-0.09)^2 + (0.22)^2 + (0.55)^2=1.18\\
%
b_0&=&5.15 + 4.39 + 4.10 + 5.71 + 5.30=24.65
\end{array}
\end{gather*}

\item
Определим частную производную от меры отклонений 
$\norma{r}$ по аргументу $c_1$ и приравняем её нулю:
\begin{gather*}
\begin{split}
\pdiff{S}{c_1}=
&2\cdot\left(c_0 + c_1\cdot(-0.76) + c_2\cdot(-0.76)^2 - 5.15 \right)\cdot(-0.76)+\\
&2\cdot\left(c_0 + c_1\cdot(-0.48) + c_2\cdot(-0.48)^2 - 4.39 \right)\cdot(-0.48)+\\
&2\cdot\left(c_0 + c_1\cdot(-0.09) + c_2\cdot(-0.09)^2 - 4.10 \right)\cdot(-0.09)+\\
&2\cdot\left(c_0 + c_1\cdot(0.22) + c_2\cdot(0.22)^2 - 5.71 \right)\cdot(0.22)+\\
&2\cdot\left(c_0 + c_1\cdot(0.55) + c_2\cdot(0.55)^2 - 5.30 \right)\cdot(0.55)=0
\end{split}
\end{gather*}

Коэффициенты второй строки матрицы $\mathbf{A}$
и второй элемент вектора $\vec{b}$:
\begin{gather*}
\begin{array}{lcl}
c_{10}&=&(-0.76) + (-0.48) + (-0.09) + (0.22) + (0.55) = -0.56\\
c_{11}&=&(-0.76)^2 + (-0.48)^2 + (-0.09)^2 + (0.22)^2 + (0.55)^2=1.18\\
c_{12}&=&(-0.76)^3 + (-0.48)^3 + (-0.09)^3 + (0.22)^3 + (0.55)^3=-0.38\\
%
b_1&=&5.15\cdot(-0.76)+4.39\cdot(-0.48)+4.10\cdot(-0.09)+\\
&&5.71\cdot(0.22)+5.30\cdot(0.55)=-2.24
\end{array}
\end{gather*}

\item
Определим частную производную от меры отклонений 
$\norma{r}$ по аргументу $c_2$ и приравняем её нулю:
\begin{gather*}
\begin{split}
\pdiff{\norma{r}}{c_2}=
&2\cdot\left(c_0 + c_1\cdot(-0.76) + c_2\cdot(-0.76)^2 - 5.15 \right)\cdot(-0.76)^2+\\
+&2\cdot\left(c_0 + c_1\cdot(-0.48) + c_2\cdot(-0.48)^2 - 4.39 \right)\cdot(-0.48)^2+\\
+&2\cdot\left(c_0 + c_1\cdot(-0.09) + c_2\cdot(-0.09)^2 - 4.10 \right)\cdot(-0.09)^2+\\
+&2\cdot\left(c_0 + c_1\cdot(0.22) + c_2\cdot(0.22)^2 - 5.71 \right)\cdot(0.22)^2+\\
+&2\cdot\left(c_0 + c_1\cdot(0.55) + c_2\cdot(0.55)^2 - 5.30 \right)\cdot(0.55)^2=0
\end{split}
\end{gather*}

Коэффициенты третьей строки матрицы $\mathbf{A}$
и третий элемент вектора $\vec{b}$:
\begin{gather*}
\begin{array}{lcl}
c_{20}&=&(-0.76)^2 + (-0.48)^2 + (-0.09)^2 + (0.22)^2 + (0.55)^2=1.18\\
c_{21}&=&(-0.76)^3 + (-0.48)^3 + (-0.09)^3 + (0.22)^3 + (0.55)^3=-0.38\\
c_{22}&=&(-0.76)^4 + (-0.48)^4 + (-0.09)^4 + (0.22)^4 + (0.55)^4=0.49\\
%
b_2&=&5.15\cdot(-0.76)^2 +4.39\cdot(-0.48)^2 +4.10\cdot(-0.09)^2+\\
&&5.71\cdot(0.22)^2 +5.30\cdot(0.55)^2=5.94
\end{array}
\end{gather*}

\item
Таким образом, для определения неизвестных коэффициентов $c_0,c_1,c_2$
аппроксимирующего полинома $p_2(x)$ необходимо решить 
систему линейных алгебраических уравнений:
\begin{gather*}
\left\{\begin{matrix}
&5\cdot c_0&-&0.56\cdot c_1&+&1.18\cdot c_2&=&24.65\\
-&0.56\cdot c_0&+&1.18\cdot c_1&-&0.38\cdot c_2&=&-2.24\\
&1.18\cdot c_0&-&0.38\cdot c_1&+&0.49\cdot c_2&=&5.94\\
\end{matrix}\right.
\end{gather*}

\item
Решение этой системы линейных уравнений можно найти методом Гаусса:
\begin{gather*}
\left\{\begin{array}{lcl}
c_0&=&4.66\\
c_1&=&0.80\\
c_2&=&1.52
\end{array}\right.
\end{gather*}

Таким образом, аппроксимирующий полином имеет вид:
\begin{gather*}
p_2(x)=4.66 + 0.80\cdot x + 1.52\cdot x^2
\end{gather*}

\item
На одном графике представим диаграмму рассеяния 
(разброса) данных функции заданной таблично $y_i=f(x_i)$
(маркеры) и результаты вычислений
аппроксимирующего алгебраического полинома 
второго порядка $p_2(x)$ (сплошная линия).
% *******************************
%	График функций
%
\begin{figure}[H]\centering
\begin{tikzpicture}
\begin{axis}[% оси координат
ylabel={$p_2(x)$},
xmin=-1, xmax=0.7, xtick={-0.8,-0.4,0,0.4},
ymin=3.8, ymax=6,
]
\addplot[PlotDarkBlue,only marks]
coordinates {(-0.76,5.15) (-0.48,4.39) (-0.09,4.10) (0.22,5.71) (0.55,5.30)};
\addplot[PlotDarkBlue,mark=none,domain=-0.9:0.6, samples=50] 
{4.66 + 0.80*x + 1.52*x^2} node[pos=0.5,below right] {$p_2(x)$};
\end{axis}
\end{tikzpicture}
\caption{График таблично заданной функции $y_i=f(x_i)$ (маркеры) 
и аппроксимирующего алгебраического полинома $p_2(x)$
(сплошная линия)}
\end{figure}

\end{enumerate}

\end{document}
Получилась система n+1 уравнений с таким же количеством неизвестных аj, причем линейная относительно этих переменных. Эта система называется системой нормальных уравнений. Из ее решения находятся параметры аj аппроксимирующей функции, обеспечивающие minR, т.е. наилучшее возможное квадратичное приближение. Зная коэффициенты, можно (если нужно) вычислить и величину R (например, для сравнения различных аппроксимирующих функций). Следует помнить, что при изменении даже одного значения исходных данных (или пары значений хi, уi, или одного из них) все коэффициенты изменят в общем случае свои значения, так как они полностью определяются исходными данными. Поэтому при повторении аппроксимации с несколько изменившимися данными (например, вследствие погрешностей измерения, помех, влияния неучтенных факторов и т.п.) получится другая аппроксимирующая функция, отличающаяся коэффициентами. Обратим внимание на то, что коэффициенты аj полинома находятся из решения системы уравнений, т.е. они связаны между собой. Это приводит к тому, что если какой-то коэффициент вследствие его малости захочется отбросить, придется пересчитывать заново оставшиеся. Можно рассчитать количественные оценки тесноты связи коэффициентов. Существует специальная теория планирования экспериментов, которая

позволяет обосновать и рассчитать значения хi, используемые для аппроксимации, чтобы получить заданные свойства коэффициентов (несвязанность, минимальная дисперсия коэффициентов и т.д.) или аппроксимирующей функции (равная точность описания реальной зависимости в различных направлениях, минимальная дисперсия предсказания значения функции и т.д.).

В случае постановки другой задачи — найти аппроксимирующую функцию, обеспечивающую погрешность не хуже заданной, — необходимо подбирать и структуру этой функции. Эта задача значительно сложнее предыдущей (найти параметры аппроксимирующей функции заданной структуры, обеспечивающей наилучшую возможную погрешность) и решается в основном путем перебора различных функций и сравнения получающихся мер близости. Для примера на рис. 3.7 приведены для визуального сравнения исходная и аппроксимирующие функции с различной степенью полинома, т.е. функции с различной структурой. Не следует забывать, что с повышением точности аппроксимации растет и сложность функции (при полиномиальных аппроксимирующих функциях), что делает ее менее удобной при использовании.

Пример 3.1. В ходе проведения эксперимента были получены данные, представленные в таблице 3.1. Необходимо способом наименьших квадратов подобрать для заданных значений x и y квадратичную функцию . Построить на одной координатной плоскости экспериментальные данные и аппроксимирующую функцию.

Исходными данными для решения задачи является таблица наблюдений – набор значений независимых переменных и соответствующие им значения функции отклика. Число строк (узлов) таблично заданной функции m называют объемом выборки.

Форма уравнения выбирается исследователем в соответствии с поведением аппроксимируемой функции в области изменения независимых переменных. Результатом же решения задачи аппроксимации являются оценки коэффициентов этого уравнения. Очевидно, что коэффициенты уравнения следует подбирать так, чтобы рассчитываемые по уравнению значения функции отклика максимально близко совпадали с заданными в исходной таблице наблюдений.


http://ru.bmstu.wiki/Аппроксимация_функций,_моделирующих_сигналы
Математические модели сигналов, детально и точно описывающие определенные физические объекты и процессы, могут быть очень сложными и мало пригодными для практического использования, как при математическом анализе физических данных, так и в прикладных задачах, основанных на математическом моделировании КПС. Кроме того, практическая регистрация сигналов выполняется, как правило, с определенной погрешностью или с определенным уровнем шумов, которые по своим значениям могут быть выше теоретической погрешности прогнозирования сигналов при расчетах по сложным, хотя и очень точным формулам. Не имеет большого смысла и проектирование систем обработки и анализа сигналов по высокоточным формулам, если повышение точности расчетов не дает ощутимого эффекта в повышении точности обработки данных. Во всех этих условиях возникает задача аппроксимации – представления произвольных сложных функций 
f
(
x
)
 простыми и удобными для практического использования функциями 
φ
(
x
)
 таким образом, чтобы отклонение 
φ
(
x
)
 от 
f
(
x
)
 в области ее задания было наименьшим по определенному критерию приближения. Функции 
φ
(
x
)
 получили название функций аппроксимации.

Математика очень часто оперирует со специальными математическими функциями решения дифференциальных уравнений и интегралов, которые не имеют аналитических выражений и представляются табличными числовыми значениями 
y
i
 для дискретных значений независимых переменных 
x
i
. Аналогичными таблицами 
{
y
i
,
x
i
}
 могут представляться и экспериментальные данные. Точки, в которых определены дискретные значения функций или данных, называются узловыми. Однако на практике могут понадобиться значения данных величин совсем в других точках, отличных от узловых, или с другим шагом дискретизации аргументов. Возникающая при этом задача вычисления значений функции в промежутках между узами называется задачей интерполяции, за пределами семейства узловых точек вперед или назад по переменным – задачей экстраполяции или прогнозирования. Решение этих задач также обычно выполняется с использованием аппроксимирующих функций.

Сглаживание статистических данных или аппроксимация данных с учетом их статистических параметров относится к задачам регрессии, и рассматриваются в следующей теме. Как правило, при регрессионном анализе усреднение данных производится методом наименьших квадратов (МНК).

Все вышеперечисленные задачи относятся к задачам приближения сигналов и функций и имеют многовековую историю, в процессе которой сформировались классические математические методы аппроксимации, интерполяции, экстраполяции и регрессии функций. В рамках настоящего курса мы не будем углубляться в строгую математическую теорию этих операций. Все современные математические системы (Mathcad, MATLAB, Maple и пр.) имеют в своем составе универсальный аппарат выполнения таких операций, дающий пользователю возможность реализации любых практических задач по обработке данных без отвлечения на теоретические подробности их исполнения. В качестве основной математической системы для примеров использована система Mathcad.
% Численное дифференцирование
\newpage
%
%	Численное дифференцирование
%
\section{Численное дифференцирование}
Задача численного дифференцирования состоит 
в приближенном вычислении производных функции $y(x)$
по заданным в конечном числе точек $\{x_i\}$ 
значениям этой функции.

Численное дифференцирование применяется, 
если функцию $y(x)$ трудно или невозможно 
продифференцировать аналитически, например, если 
функция является таблично заданной, а также 
при решении дифференциальных уравнений 
разностными методами.

Многие формулы численного дифференцирования можно получить, 
используя интерполяционные формулы.
Для этого достаточно заменить функцию $y(x)$ 
интерполяционным полиномом Лагранжа $L_n(x)$ 
и вычислить производные этого многочлена, 
используя его явное представление.

Рассмотрим произвольную сетку $\{x_i\}$ и 
проведем интерполирование функции $y(x)$ в узлах сетки 
$x_{i-1} < x_{i} < x_{i+1}$ полиномом Лагранжа второго
порядка, приближенно полагая $y(x)\approx L_{2}(x)$
для $x\in[x_{i-1},x_{i+1}]$:
% *******************************
%	График функций
%
\begin{figure}[H]\centering
\begin{tikzpicture}
\begin{axis}[
%axis lines = middle,
every axis/.style={color=black, solid, thick},
xlabel = {\empty},		% подпись оси x
ylabel = {\empty},	% подпись оси y
xmin=-0.75, xmax=2,
ymin=-0.5, ymax=3,
%xtick style={thick, black}, 
xtick={-0.5,0,1.75}, xticklabels={$x_{i-1}$,$x_i$,$x_{i+1}$},
%ytick style={thick, black}, 
ytick={2.25,1,0.5625}, yticklabels={$y_{i-1}$,$y_i$,$y_{i+1}$},
grid=major,		
major grid style={color=black!20, dashed, thin},
]
\addplot [only marks,mark=ball,ball color=darkred!75,
mark size=4pt,mark options={draw=darkred,thin}]
coordinates {(-0.5,2.25) (0,1) (1.75,0.5625)};
\addplot [color=darkred, thick, domain=-0.5:1.75] 
{(x-1)^2} node[pos=0.75,above] {$L_2(x)$};
\end{axis}
\end{tikzpicture}
\end{figure}
% *******************************
\begin{equation*}
\label{approx2}
\begin{matrix}
L_{2}(x)
&=&\dfrac
{ (x-x_{i})\cdot(x-x_{i+1}) }
{ (x_{i-1}-x_{i})\cdot(x_{i-1}-x_{i+1}) } \cdot y_{i-1}& + \\
\\
&+&\dfrac
{ (x-x_{i-1})\cdot(x-x_{i+1}) }
{ (x_{i}-x_{i-1})\cdot(x_{i}-x_{i+1}) } \cdot y_{i}& + \\
\\
&+&\dfrac
{ (x-x_{i-1})\cdot(x-x_{i}) }
{ (x_{i+1}-x_{i-1})\cdot(x_{i+1}-x_{i}) } \cdot y_{i+1}
\end{matrix}
\end{equation*}
где $y_{i-1} = y(x_{i-1})$, $y_{i} = y(x_{i})$, $y_{i+1} = y(x_{i+1})$
-- значение функции $y(x)$ в узлах сетки.

%------------------------------------------------------------------
Первая производная многочлена Лагранжа $L_2(x)$:
\begin{gather*}
\begin{matrix}
L^{\prime}_{2}(x)
&=&\dfrac
{ 2x-x_{i}-x_{i+1} }
{ (x_{i-1}-x_{i})\cdot(x_{i-1}-x_{i+1}) } \cdot y_{i-1}& + \\
\\
&+&\dfrac
{ 2x-x_{i-1}-x_{i+1} }
{ (x_{i}-x_{i-1})\cdot(x_{i}-x_{i+1}) } \cdot y_{i}& + \\
\\
&+&\dfrac
{ 2x-x_{i-1}-x_{i} }
{ (x_{i+1}-x_{i-1})\cdot(x_{i+1}-x_{i}) } \cdot y_{i+1}
\end{matrix}
\end{gather*}

Это выражение можно принять за приближенное значение
первой производной  $y^{\prime}(x)$ 
в любой точке отрезка $[x_{i-1},x_{i+1}]$.
Например, в точке $x=x_{i}$ первая производная от функции 
$y(x)$ приближенно равна:
\begin{gather*}
\begin{matrix}
y^{\prime}(x_i)\approx L^{\prime}_{2}(x_i)
&=&\dfrac
{ x_{i}-x_{i+1} }
{ (x_{i-1}-x_{i})\cdot(x_{i-1}-x_{i+1}) } \cdot y_{i-1}& + \\
\\
&+&\dfrac
{ (x_i-x_{i-1}) + (x_{i}-x_{i+1}) }
{ (x_{i}-x_{i-1})\cdot(x_{i}-x_{i+1}) } \cdot y_{i}& + \\
\\
&+&\dfrac
{ x_{i}-x_{i-1} }
{ (x_{i+1}-x_{i-1})\cdot(x_{i+1}-x_{i}) } \cdot y_{i+1}
\end{matrix}
\end{gather*}

%------------------------------------------------------------------
Вторую производную полинома Лагранжа
можно принять за приближенное значение 
второй производной от функции $y(x)$ 
в любой точке отрезка $[x_{i-1}, x_{i+1}]$:
\begin{gather*}
\begin{matrix}
y^{\prime\prime}(x)\approx L^{\prime\prime}_{2}(x)
&=&\dfrac
{ 2 }
{ (x_{i-1}-x_{i})\cdot(x_{i-1}-x_{i+1}) } \cdot y_{i-1}& + \\
\\
&+&\dfrac
{ 2 }
{ (x_{i}-x_{i-1})\cdot(x_{i}-x_{i+1}) } \cdot y_{i}& + \\
\\
&+&\dfrac
{ 2 }
{ (x_{i+1}-x_{i-1})\cdot(x_{i+1}-x_{i}) } \cdot y_{i+1}
\end{matrix}
\end{gather*}

На \emph{равномерной сетке} $\{x_i\}$, расстояние между 
соседними узлами которой одинаково, выражения 
для первой и второй производной в точке $x=x_i$ упрощаются:
\begin{gather*}
y^{\prime}(x_i)\approx
\dfrac{y_{i+1}-y_{i-1}}{2h},
\qquad
y^{\prime\prime}(x_i)\approx
\dfrac{y_{i-1}-2y_{i}+y_{i+1}}{h^2},
\end{gather*}
где $h=(x_i-x_{i-1})=(x_{i+1}-x_i)$ -- шаг сетки.

Для приближенного вычисления производных более высоких порядков
$y^{(n)}(x)$ уже недостаточно полинома Лагранжа второго 
порядка $L_2(x)$. Поэтому необходимо использовать 
полиномы более высокого порядка, что приводит 
к увеличению числа узлов аппроксимации.

Следует отметить, что порядок погрешности аппроксимации 
производных от функции $y(x)$ зависит  как от порядка 
интерполяционного полинома, так и от расположения 
узлов сетки $\{x_i\}$.

%
%	Численного дифференцирование таблично заданной функции
%
\subsection{Численного дифференцирование функции заданной таблично}
% масиив абсцисс [x]
\def\Xarray{{-0.98,-0.76,-0.48,-0.09,0.22,0.55}}
% масиив ординат [y]
\def\Yarray{{4.11,4.83,5.13,5.01,5.13,6.11}}
% элемент массива
\newcommand\x[1]{\pgfmathparse{\Xarray[#1]} \pgfmathresult}
\newcommand\y[1]{\pgfmathparse{\Yarray[#1]} \pgfmathresult}
%
%	График функций
%
\newcommand\FigNumDiff[3]{
\begin{figure}[H]\centering
\begin{tikzpicture}
\begin{axis}[
xlabel = {$x$},		% подпись оси x
ylabel = {$y(x)$},	% подпись оси y
xmin=-1.5, xmax=1,
ymin=3.5, ymax=6.5,
xtick style={thick, black},
ytick style={thick, black},
grid=major,		
major grid style={color=black!20, dashed, thin},
]
% точки графика
\addplot[only marks,mark=*,mark size=3pt,
mark options={fill=gray!25,draw=darkred}] 
coordinates {#1};
% полином Лагранжа
#2;
% отрезок интерполирования
\addplot[only marks,mark=ball,mark size=3pt,
mark options={ball color=darkred!50,draw=darkred}] 
coordinates {#3};
\end{axis}
\end{tikzpicture}
\end{figure}}

Известно множество данных (узлов сетки) $\{x_i\}$
в которых определены значения функции $\{f(x_i)\}$:
\begin{table}[H]
\vspace{-0.5\baselineskip}
\caption{Таблично заданная функциональная зависимость
$y_i=f(x_i)$}
\label{tab:Num_Diff}
\begin{tabular*}{\textwidth}{%
ll@{\extracolsep{\fill}}*{9}{r}}
\toprule
$i$&&$0$&$1$&$2$&$3$&$4$&$5$&$6$&$7$&$8$\\
\midmidrule
$x_i$&&$-1.2$&$-0.98$&$-0.76$&$-0.48$&$-0.09$&$0.22$&$0.32$&$0.55$&$0.76$\\
\addlinespace% дополнительный пробел
$y_i$&&$3.78$&$4.11$&$4.83$&$5.13$&$5.01$&$5.13$&$5.73$&$6.11$&$5.92$\\
\bottomrule
\end{tabular*}
\end{table}

\begin{enumerate}
\item
Построим график функции $y(x)$, используя данные таблицы \ref{tab:Num_Diff}.
% *******************************
%	График функций
%
\begin{figure}[H]\centering
\begin{tikzpicture}
\begin{axis}[
xlabel = {$x$},		% подпись оси x
ylabel = {$y(x)$},	% подпись оси y
xmin=-1.5, xmax=1,
ymin=3.5, ymax=6.5,
xtick style={thick, black},
ytick style={thick, black},
grid=major,		
major grid style={color=black!20, dashed, thin},
]
\addplot[smooth,color=darkred,mark=ball,mark size=4pt,
mark options={draw=darkred,thin,ball color=darkred!75}]
coordinates 
{(-1.2,3.78) (-0.98,4.11) (-0.76,4.83) (-0.48,5.13) (-0.09,5.01) (0.22,5.13) (0.32,5.73) (0.55,6.11) (0.76,5.92)};
\end{axis}
\end{tikzpicture}
\end{figure}
% *******************************
% x1
\item
Аппроксимируем функцию $y(x)$ в узлах $\{x_{0},x_{1},x_{2}\}$
полиномом Лагранжа второго порядка $L_2(x)$, 
используя данные таблицы \ref{tab:Num_Diff}:
\begin{gather*}
\begin{matrix}
L_2(x)&=&\dfrac{(x-(-0.98))(x-(-0.76))}{(-1.20-(-0.98))(-1.20-(-0.76))}\cdot3.78&+\\[1em]
&+&\dfrac{(x-(-1.20))(x-(-0.76))}{(-0.98-(-1.20))(-0.98-(-0.76))}\cdot4.11&+\\[1em]
&+&\dfrac{(x-(-1.20))(x-(-0.98))}{(-0.76-(-1.20))(-0.76-(-0.98))}\cdot4.83
\end{matrix}
\end{gather*}
Проводя элементарные алгебраические преобразования полином Лагранжа 
в пределах отрезка $[x_0,x_2]$ имеет вид:
\begin{gather*}
L_2(x)=4.028925620\cdot x^2+10.28305785\cdot x+10.31801653
\end{gather*}
% *******************************
%	График функций
%
\FigNumDiff
{(-1.2,3.78) (-0.98,4.11) (-0.76,4.83) (-0.48,5.13) (-0.09,5.01) (0.22,5.13) (0.32,5.73) (0.55,6.11) (0.76,5.92)}
{% L2(x)
\addplot[color=darkred,very thick,samples=50,domain=-1.2:-0.76] 
{4.028925620*x^2+10.28305785*x+10.31801653}
node [pos=0.5,right] {$L_2(x)$};
}
{(-1.2,3.78) (-0.98,4.11) (-0.76,4.83)}

Определим первую и вторую производную функции $y(x)$ 
в точке $x_1=-0.98$:
\begin{gather*}
y^{\prime}(-0.98)\approx L_2^{\prime}(-0.98)=2.386363635\\
y^{\prime\prime}(-0.98)\approx L_2^{\prime\prime}(-0.98)=8.057851240
\end{gather*}
%
% x2
\item
Аппроксимация функции $y(x)$ в узлах $\{x_{1},x_{2},x_{3}\}$
полиномом Лагранжа второго порядка $L_2(x)$, 
используя данные таблицы \ref{tab:Num_Diff}:
\begin{gather*}
\begin{matrix}
L_2(x)&=&\dfrac{(x-(-0.76))(x-(-0.48))}{(-0.98-(-0.76))(-0.98-(-0.48))}\cdot4.11&+\\[1em]
&+&\dfrac{(x-(-0.98))(x-(-0.48))}{(-0.76-(-0.98))(-0.76-(-0.48))}\cdot4.83&+\\[1em]
&+&\dfrac{(x-(-0.98))(x-(-0.76))}{(-0.48-(-0.98))(-0.48-(-0.76))}\cdot5.13
\end{matrix}
\end{gather*}
Проводя элементарные алгебраические преобразования полином Лагранжа 
в пределах отрезка $[x_1,x_3]$ имеет вид:
\begin{gather*}
L_2(x)=-4.402597390\cdot x^2-4.387792189\cdot x+4.038218187
\end{gather*}
% График функций
\FigNumDiff
{(-1.2,3.78) (-0.09,5.01) (0.22,5.13) (0.32,5.73) (0.55,6.11) (0.76,5.92)}
{% L2(x)
\addplot[color=darkred, very thick, samples=50, domain=-0.98:-0.48] 
{-4.402597390*x^2-4.387792189*x+4.038218187}
node [pos=0.9,above left] {$L_2(x)$};
}
{(-0.98,4.11) (-0.76,4.83) (-0.48,5.13)}

Определим первую и вторую производную функции $y(x)$ 
в точке $x_2=-0.76$:
\begin{gather*}
y^{\prime}(-0.76)\approx L_2^{\prime}(-0.76)=2.304155844\\
y^{\prime\prime}(-0.76)\approx L_2^{\prime\prime}(-0.76)=-8.805194780
\end{gather*}
%
% x3
\item
Аппроксимация функции $y(x)$ в узлах $\{x_{2},x_{3},x_{4}\}$
полиномом Лагранжа второго порядка $L_2(x)$, 
используя данные таблицы \ref{tab:Num_Diff}:
\begin{gather*}
\begin{matrix}
L_2(x)&=&\dfrac{(x-(-0.48))(x-(-0.09))}{(-0.76-(-0.48))(-0.76-(-0.09))}\cdot4.83&+\\[1em]
&+&\dfrac{(x-(-0.76))(x-(-0.09))}{(-0.48-(-0.76))(-0.48-(-0.09))}\cdot5.13&+\\[1em]
&+&\dfrac{(x-(-0.76))(x-(-0.48))}{(-0.09-(-0.76))(-0.09-(-0.48))}\cdot5.01
\end{matrix}
\end{gather*}
Проводя элементарные алгебраические преобразования полином Лагранжа 
в пределах отрезка $[x_2,x_4]$ имеет вид:
\begin{gather*}
L_2(x)=-2.058389370\cdot x^2-1.480974249\cdot x+4.893385272
\end{gather*}
% График функций
\FigNumDiff
{(-1.2,3.78) (-0.98,4.11) (-0.76,4.83) (-0.48,5.13) (-0.09,5.01) (0.22,5.13) (0.32,5.73) (0.55,6.11) (0.76,5.92)}
{% L2(x)
\addplot[color=darkred,very thick, samples=50, domain=-0.76:-0.09] 
{-2.058389370*x^2-1.480974249*x+4.893385272}
node [pos=0.8,above] {$L_2(x)$};
}
{(-0.76,4.83) (-0.48,5.13) (-0.09,5.01)}

Определим первую и вторую производную функции $y(x)$
в точке $x_3=-0.48$:
\begin{gather*}
y^{\prime}(-0.48)\approx L_2^{\prime}(-0.48)=0.495079546\\
y^{\prime\prime}(-0.48)\approx L_2^{\prime\prime}(-0.48)=-4.116778740
\end{gather*}
%
%x4
\item
Апроксимацию функции $y(x)$ в узлах $\{x_{3},x_{4},x_{5}\}$
полиномом Лагранжа второго порядка $L_2(x)$, 
используя данные таблицы \ref{tab:Num_Diff}:
\begin{gather*}
\begin{matrix}
L_2(x)&=&
\dfrac{(x-(-0.09))(x-0.22)}{(-0.48-(-0.09))(-0.48-0.22)}\cdot5.13&+\\[1em]
&+&\dfrac{(x-(-0.48))(x-0.22)}{(-0.09-(-0.48))(-0.09-0.22)}\cdot5.01&+\\[1em]
&+&\dfrac{(x-(-0.48))(x-(-0.09))}{(0.22-(-0.48))(0.22-(-0.09))}\cdot5.13
\end{matrix}
\end{gather*}
Проводя элементарные алгебраические преобразования полином Лагранжа 
в пределах отрезка $[x_3,x_5]$ имеет вид:
\begin{gather*}
L_2(x)=0.9925558300\cdot x^2+0.2580645177\cdot x+5.025186105
\end{gather*}
% График функций
\FigNumDiff
{(-1.2,3.78) (-0.98,4.11) (-0.76,4.83) (0.32,5.73) (0.55,6.11) (0.76,5.92)}
{% L2(x)
\addplot[color=darkred,very thick,samples=50,domain=-0.48:0.22] 
{0.9925558300*x^2+0.2580645177*x+5.025186105}
node [pos=0.55,below] {$L_2(x)$};
}
{(-0.48,5.13) (-0.09,5.01) (0.22,5.13)}

Определим первую и вторую производную функции $y(x)$ 
в точке $x_4=-0.09$:
\begin{gather*}
y^{\prime}(-0.09)\approx L_2^{\prime}(-0.09)=0.0794044683\\
y^{\prime\prime}(-0.09)\approx L_2^{\prime\prime}(-0.09)=1.985111660
\end{gather*}
%
%x5
\item
Аппроксимация функции $y(x)$ в узлах $\{x_{4},x_{5},x_{6}\}$
полиномом Лагранжа второго порядка $L_2(x)$, 
используя данные таблицы \ref{tab:Num_Diff}:
\begin{gather*}
\begin{matrix}
L_2(x)&=&
\dfrac{(x-0.22)(x-0.32)}{(-0.09-0.22)(-0.09-0.32)}\cdot5.01&+\\[1em]
&+&\dfrac{(x-(-0.09))(x-0.32)}{(0.22-(-0.09))(0.22-0.32)}\cdot5.13&+\\[1em]
&+&\dfrac{(x-(-0.09))(x-0.22)}{(0.32-(-0.09))(0.32-0.22)}\cdot5.73
\end{matrix}
\end{gather*}
Проводя элементарные алгебраические преобразования полином Лагранжа 
в пределах отрезка $[x_4,x_6]$ имеет вид:
\begin{gather*}
L_2(x)=13.69000778\cdot x^2-1.392604236\cdot x+4.773776556
\end{gather*}
% *******************************
% График функций
\FigNumDiff
{(-1.2,3.78) (-0.98,4.11) (-0.76,4.83) (-0.48,5.13) (0.55,6.11) (0.76,5.92)}
{% L2(x)
\addplot[color=darkred,very thick,samples=50,domain=-0.09:0.32] 
{13.69000778*x^2-1.392604236*x+4.773776556}
node[pos=0.2,below right] {$L_2(x)$};
}
{(-0.09,5.01) (0.22,5.13) (0.32,5.73)}

Определим первую и вторую производную функции $y(x)$ 
в точке $x_5=0.22$:
\begin{gather*}
y^{\prime}(0.22)\approx L_2^{\prime}(0.22)=4.630999187\\
y^{\prime\prime}(0.22)\approx L_2^{\prime\prime}(0.22)=27.38001556
\end{gather*}
%x6
\item
Аппроксимация функции $y(x)$ в узлах $\{x_{5},x_{6},x_{7}\}$
полиномом Лагранжа второго порядка $L_2(x)$, 
используя данные таблицы \ref{tab:Num_Diff}:
\begin{gather*}
\begin{matrix}
L_2(x)&=&
\dfrac{(x-0.32)(x-0.55)}{(0.22-0.32)(0.22-0.55)}\cdot5.13&+\\[1em]
&+&\dfrac{(x-0.22)(x-0.55)}{(0.32-0.22)(0.32-0.55)}\cdot5.73&+\\[1em]
&+&\dfrac{(x-0.22)(x-0.32)}{(0.55-0.22)(0.55-0.32)}\cdot6.11
\end{matrix}
\end{gather*}
Проводя элементарные алгебраические преобразования полином Лагранжа 
в пределах отрезка $[x_5,x_7]$ имеет вид:
\begin{gather*}
L_2(x)=-13.17523062\cdot x^2+13.11462456\cdot x+2.882463758
\end{gather*}
% *******************************
%	График функций
%
\begin{figure}[H]\centering
\begin{tikzpicture}
\begin{axis}[
xlabel = {$x$},		% подпись оси x
ylabel = {$f(x)$},	% подпись оси y
xmin=-1.5, xmax=1,
ymin=3.5, ymax=6.5,
xtick style={thick, black},
ytick style={thick, black},
grid=major,		
major grid style={color=black!20, dashed, thin},
]
\addplot[only marks, mark=*, mark size=4pt, mark options={fill=orange, draw=black, solid}] coordinates 
{(-1.2,3.78) (-0.98,4.11) (-0.76,4.83) (-0.48,5.13) (-0.09,5.01) (0.22,5.13) (0.32,5.73) (0.55,6.11) (0.76,5.92)};
\addplot[color=orange, very thick, samples=50, domain=0.22:0.55] {
-13.17523062*x^2+13.11462456*x+2.882463758
};
\draw[color=orange] (axis cs: 0.55,5.5) node {$L_2(x)$};
\end{axis}
\end{tikzpicture}
\end{figure}
% *******************************
Определим первую и вторую производную функции $y(x)$ в точке $x_6=0.32$:
\begin{gather*}
y^{\prime}(0.32)\approx L_2^{\prime}(0.32)=4.682476963\\
y^{\prime\prime}(0.32)\approx L_2^{\prime\prime}(0.32)=-26.35046124
\end{gather*}
% x7
\item
Апроксимацию функции $y(x)$ в узлах $\{x_{6},x_{7},x_{8}\}$
полиномом Лагранжа второго порядка $L_2(x)$, 
используя данные таблицы \ref{tab:Num_Diff}:
\begin{gather*}
\begin{matrix}
L_2(x)&=&
\dfrac{(x-0.55)(x-0.76)}{(0.32-0.55)(0.32-0.76)}\cdot5.73&+\\[1em]
&+&\dfrac{(x-0.32)(x-0.76)}{(0.55-0.32)(0.55-0.76)}\cdot6.11&+\\[1em]
&+&\dfrac{(x-0.32)(x-0.55)}{(0.76-0.32)(0.76-0.55)}\cdot5.92
\end{matrix}
\end{gather*}
Проводя элементарные алгебраические преобразования полином Лагранжа 
в пределах отрезка $[x_6,x_8]$ имеет вид:
\begin{gather*}
L_2(x)=-5.811217790\cdot x^2+6.707933391\cdot x+4.178530017
\end{gather*}
% *******************************
%	График функций
%
\begin{figure}[H]\centering
\begin{tikzpicture}
\begin{axis}[
xlabel = {$x$},		% подпись оси x
ylabel = {$f(x)$},	% подпись оси y
xmin=-1.5, xmax=1,
ymin=3.5, ymax=6.5,
xtick style={thick, black},
ytick style={thick, black},
grid=major,		
major grid style={color=black!20, dashed, thin},
]
\addplot[only marks, mark=*, mark size=4pt, mark options={fill=orange, draw=black, solid}] coordinates 
{(-1.2,3.78) (-0.98,4.11) (-0.76,4.83) (-0.48,5.13) (-0.09,5.01) (0.22,5.13) (0.32,5.73) (0.55,6.11) (0.76,5.92)};
\addplot[color=orange, very thick, samples=50, domain=0.32:0.76] {
-5.811217790*x^2+6.707933391*x+4.178530017
};
\draw[color=orange] (axis cs: 0.65,5.65) node {$L_2(x)$};
\end{axis}
\end{tikzpicture}
\end{figure}
% *******************************
Определим первую и вторую производную функции $y(x)$ в точке $x_7=0.55$:
\begin{gather*}
y^{\prime}(0.55)\approx L_2^{\prime}(0.55)=0.315593822\\
y^{\prime\prime}(0.55)\approx L_2^{\prime\prime}(0.55)=-11.62243558
\end{gather*}
\item
Таким образом, определены значения первой $y^{\prime}(x_i)$ и
второй $y^{\prime\prime}(x_i)$ производной функции $y(x)$ 
в каждом внутреннем узле сетки $\{x_i\}$:
\begin{center}
\begin{tabular}{ l *{7}{l}}
\toprule
$x$&-0,98&-0,76&-0,48&-0,09&0,22&0,32&0,55\\
\midrule
$f^{\prime}(x)$&2,39&2,30&0,50&0,08&4,63&4,68&0,32\\
\midrule
$f^{\prime\prime}(x)$&8,06&-8,81&-4,12&1,99&27,38&-26,35&-11,62\\
\bottomrule
\end{tabular} 
\end{center}
% *******************************
%	График функций
%
\begin{center}
\begin{tikzpicture}
\begin{axis}[
xlabel = {$x$},		% подпись оси x
ylabel = {$\textcolor{blue}{y^{\prime}(x)}$},	% подпись оси y
xmin=-1.5, xmax=1,
ymin=-0.5, ymax=5,
xtick style={thick, black},
ytick style={thick, black},
grid=major,		
major grid style={color=black!20, dashed, thin},
]
\addplot[blue,mark=*, mark size=4pt, mark options={fill=blue!50, draw=black, solid}] coordinates 
{(-0.98,2.39) (-0.76,2.30) (-0.48,0.50) (-0.09,0.08) (0.22,4.63) (0.32,4.68) (0.55,0.32)};
\end{axis}
\end{tikzpicture}
\begin{tikzpicture}
\begin{axis}[
xlabel = {$x$},		% подпись оси x
ylabel = {$\textcolor{red}{y^{\prime\prime}(x)}$},	% подпись оси y
xmin=-1.5, xmax=1,
ymin=-30, ymax=30,
xtick style={thick, black},
ytick style={thick, black},
grid=major,		
major grid style={color=black!20, dashed, thin},
]
\addplot[red,mark=*, mark size=4pt, mark options={fill=red!75, draw=black, solid}] coordinates 
{(-0.98,8.06) (-0.76,-8.81) (-0.48,-4.12) (-0.09,1.99) (0.22,27.38) (0.32,-26.35) (0.55,-11.62)};
\end{axis}
\end{tikzpicture}
\end{center}

\end{enumerate}

%\end{document}
%	Численное интегрирование
\newpage
%
%	Численное интегрирование
%
\section{Численное интегрирование}
Если функция $f(x)$ непрерывна на отрезке $x\in[a,b]$ и 
известна ее первообразная $F(x)$, то определенный интеграл 
от этой функции в пределах от $a$ до $b$ может быть вычислен 
по формуле Ньютона -- Лейбница:
\begin{gather*}
\int\limits_{a}^{b}f(x)dx=F(b)-F(a),
\end{gather*}
где $F^{\prime}(x)=f(x)$ -- 
первообразная подынтегральной функции $f(x)$.

Численное значение интеграла -- это площадь криволинейной 
трапеции, ограниченной линиями графика функции 
и осью абсцисс $Ox$ (выделенная область на рисунке \ref{fig:INT}).

\begin{figure}[H]\centering
\begin{tikzpicture}
\def\xa{-1.6}
\def\xb{2}
\begin{axis}[% оси координат
xlabel=$x$,ylabel=$f(x)$,
xtick={\xa,\xb},xticklabels={$a$,$b$},
ytick={0},yticklabels={0}]
% f(x)
\addplot[name path=A,ball darkblue,mark=none,
samples=50,domain=\xa:\xb]{0.5*(\x-2)*(\x-1)*(\x+1) + 1};
\path[name path=B] (axis cs: \xa,0) -- (axis cs: \xb,0);
\addplot[blue!15] fill between [of=A and B, soft clip={domain=\xa:\xb},];
\end{axis}
\end{tikzpicture}
\caption{Геометрический смысл определенного интеграла}
\label{fig:INT}
\end{figure}

Однако во многих случаях первообразная функция $F(x)$ 
не может быть найдена с помощью элементарных средств 
или является слишком сложной, поэтому 
вычисление определенного интеграла 
может быть затруднительным или даже практически невозможным. 

Кроме того, на практике подынтегральная функция $f(x)$ часто 
задается таблично и тогда само понятие первообразной теряет смысл. 
Аналогичные вопросы возникают при вычислении кратных  
интегралов. Поэтому важное значение имеют приближенные 
и в первую очередь численные методы вычисления 
определенных интегралов. 

\emph{Задача численного интегрирования} функции заключается 
в вычислении значения определенного интеграла на основании ряда  
значений подынтегральной функции $f(x)$.

Обычный прием численного вычисления интеграла состоит 
в том, что данную функцию $f(x)$ на рассматриваемом отрезке 
$x\in[a, b]$ заменяют интерполирующей или аппроксимирующей  
функцией $\varphi(x)$ простого вида (например, полиномом), 
а затем приближенно полагают:
\begin{gather*}
\int\limits_{a}^{b}f(x)dx\approx \int\limits_{a}^{b}\varphi(x)dx
\end{gather*}

Далее рассматриваются способы приближенного вычисления 
определенных интегралов вида:
\begin{gather*}
I=\int\limits_{a}^{b}\varphi(x)dx,
\end{gather*}
основанные на замене интеграла конечной суммой:
\begin{gather*}
I\approx\sum\limits_{i=0}^{n}c_{i}\cdot\varphi(x_i),
\end{gather*}
где $c_{i}$ -- числовые коэффициенты квадратурной формулы; 
$x_i$ -- узлы квадратурной формулы, т.е. точки отрезка 
$[a, b], (i= 0,1,\cdots,n)$.

На основании свойств определенных интегралов, 
$I$ можно представить в виде суммы интегралов
по частичным отрезкам:
\begin{gather*}
\int\limits_{a}^{b}f(x)dx=
\sum\limits_{i=1}^{n}\int\limits_{x_{i-1}}^{x_i}f(x)dx
\end{gather*}

Поэтому, для построения формулы численного интегрирования 
на всем отрезке $[a,b]$ достаточно построить квадратурную формулу 
на частичном отрезке $[x_{i-1},x_i]$ для интеграла:
\begin{gather*}
S_i=\int\limits_{x_{i-1}}^{x_i}f(x)dx
\end{gather*}

%
% Формула прямоугольников
%
\subsection{Формула прямоугольников}
В методе прямоугольников на частичном отрезке 
подынтегральная функция заменяется полиномом нулевой степени,
то есть константу:
\begin{gather*}
f(x)\approx L_0(x)=\const
\end{gather*}

С геометрической точки зрения, в методе прямоугольников
площадь криволинейной трапеции (интеграл от функции) на
частичном отрезке заменяется площадью прямоугольника,
ширина которого будет определяться расстоянием между 
соответствующими соседними узлами интегрирования,
а высота -- значением подынтегральной функции в этих узлах.

В зависимости от выбора узла сетки $\{x_i\}$ для аппроксимации 
подынтегральной функции $f(x)$ на частичном отрезке 
$[x_{i-1},x_{i}]$ различают левую и правую формулы прямоугольников:
если в качестве значения аппроксимирующего полинома
выбирается значение подынтегральной функции 
на левом конце отрезка $L_0\approx f(x_{i-1})=y_{i-1}$
(рисунок \ref{fig:IntRect}), то справедлива 
левая формула прямоугольников:
\begin{gather*}
S^{-}_i\approx\int\limits_{x_{i-1}}^{x_i}L_0(x)dx=
y_{i-1}\cdot(x_{i}-x_{i-1}),
\end{gather*}
а если значение аппроксимирующего полинома
соответствует значению подынтегральной функции 
на правом конце частичного отрезка $L_0\approx f(x_{i})=y_{i}$
(рисунок \ref{fig:IntRect}), то справедлива 
правая формула прямоугольников :
\begin{gather*}
S^{+}_i\approx\int\limits_{x_{i-1}}^{x_i}L_0(x)dx=
y_{i}\cdot(x_{i}-x_{i-1}),
\end{gather*}
%
% Функция построения графика
%
\newcommand\FigInt[3]{
\begin{tikzpicture}[baseline]
\begin{axis}[% оси координат
xlabel=$x$,
name=GRAPH,
ylabel=$f(x)$,
%xmin=-1,xmax=1.5,
ymin=0,%ymax=2.25,
xtick={#1,#2},xticklabels={$x_{i-1}$,$x_i$},
ytick={0,cubic(#1),cubic(#2)},yticklabels={0,$y_{i-1}$,$y_i$},
width=7.5cm,% ширина графика
% определение функции
declare function={
cubic(\x)=0.5*(\x-2)*(\x-1)*(\x+1) + 1;
% полином Лагранжа L1
lagrange(\x) = (\x-#2)/(#1-#2)*cubic(#1) + (\x-#1)/(#2-#1)*cubic(#2);
% промежуточная точка
xc = #1 + (1-0.4)*(#2-#1)/2;
% полином Лагранжа L2
Lagrange(\x) = 
(\x-xc)*(\x-#2)/(xc-#1)/(#2-#1)*cubic(#1) +
(\x-#1)*(#2-\x)/(xc-#1)/(#2-xc)*cubic(xc) +
(\x-#1)*(\x-xc)/(#2-#1)/(#2-xc)*cubic(#2);
},
]
% f(x)
\addplot[name path=F,ball darkblue,fill=blue!10,
mark indices={1,50},samples=50,domain=#1:#2]{cubic(x)}
\closedcycle;
% Ln(x)
\addplot[draw=none,thin,opacity=0.25,fill=red!30,
domain=#1:#2,samples=50] #3 \closedcycle;
\end{axis}
% дополнительно
%\draw[thick,red] ($(GRAPH.south)-(0,0em)$) node {a};
\draw[thick,red] ($(current bounding box.south)-(0,1em)$) node {a)};
\end{tikzpicture}
}
%
% График прямоугольников
%
\begin{figure}[H]\centering
\FigInt{-0.7}{2}{ {cubic(-0.7)}
node[pos=0.7,color=darkred,opacity=1,above] {$L_0(x)$}
}
%
\hskip 10pt
%
\FigInt{-0.7}{2}{ {cubic(2)}
node[pos=0.8,color=darkred,opacity=1,above] {$L_0(x)$}
}\\
%\linebreak
a)\hspace{6cm}b)
\caption{График подынтегральной функции $f(x)$
и аппроксимирующего полинома $L_0(x)$ на частичном отрезке
для формулы прямоугольников}
\label{fig:IntRect}
\end{figure}

%
% Формула трапеций
%
\subsection{Формула трапеций}
Квадратурная \emph{формула трапеций} является следствием замены 
на частичном отрезке подынтегральной функции
интерполяционным полиномом первой степени $f(x)\approx L_1(x)$,
построенным по множеству узлов сетки $\{x_{i-1}, x_i\}$:
\begin{gather*}
L_1(x)=\dfrac{x-x_i}{x_{i-1}-x_i}\cdot y_{i-1} + \dfrac{x-x_{i-1}}{x_i-x_{i-1}}\cdot y_i.
\end{gather*}

Интегрирование интерполяционного полинома Лагранжа 
на частичном отрезке определяет формулу трапеций:
\begin{gather*}
S_i\approx\int\limits_{x_{i-1}}^{x_i}L_1(x)dx=
\dfrac{y_{i}+y_{i-1}}{2}\cdot(x_{i}-x_{i-1})
\end{gather*}
% график
\begin{figure}[H]\centering
\FigInt{-0.7}{2}{ {lagrange(x)} 
node[pos=0.8,color=darkred,opacity=1,above] {$L_1(x)$}
}
\caption{График подынтегральной функции $f(x)$
и аппроксимирующего полинома $L_1(x)$ на частичном отрезке
для формулы трапеций}
\label{fig:IntTrapez}
\end{figure}
%
% Формула Симпсона
%
\subsection{Формула Симпсона}
На частичном отрезке $[x_{i-1},x_{i}]$ квадратурная 
\emph{формула Симпсона} является следствием 
аппроксимации подынтегральной функции 
$f(x)$ интерполяционным полиномом Лагранжа 
второй степени $f(x)\approx L_2(x)$, который построен
по узлам сетки $\{x_{i-1}, x_{i-1/2}, x_{i}\}$:
\begin{gather*}
\begin{matrix}
L_{2}(x)&=&\dfrac
{ (x-x_{i-1/2})\cdot(x-x_{i}) }
{ (x_{i-1/2}-x_{i-1})\cdot(x_{i}-x_{i-1}) } \cdot y_{i-1}& + \\
\\
&+&\dfrac
{ (x-x_{i-1})\cdot(x_{i}-x) }
{ (x_{i-1/2}-x_{i-1})\cdot(x_{i}-x_{i-1/2}) } \cdot y_{i-1/2}& + \\
\\
&+&\dfrac
{ (x-x_{i-1})\cdot(x-x_{i-1/2}) }
{ (x_{i}-x_{i-1})\cdot(x_{i}-x_{i-1/2}) } \cdot y_{i}
\end{matrix},
\end{gather*}
где $x_{i-1/2}$ -- узел вспомогательной сетки,
расположенный между узлами основной сетки
$x_{i-1}<x_{i-1/2}<x_{i}$

Выражение для полинома Лагранжа в каноническом виде:
\begin{gather*}
L_{2}(x)=c_0 + c_1\cdot x + c_2\cdot x^2,
\end{gather*}
где $c_0$, $c_1$, $c_2$ -- коэффициенты при 
соответствующих степенях $x$ интерполяционного полинома 
Лагранжа $L_2(x)$ в пределах частичного отрезка 
$[x_{i-1}, x_{i+1}]$.

Интегрирование интерполяционного полинома Лагранжа $L_2(x)$ 
на частичном отрезке $x\in[x_{i-1}, x_{i+1}]$ определяет формулу Симпсона:
\begin{gather*}
S_i\approx\int\limits_{x_{i-1}}^{x_{i+1}}L_2(x)dx=
c_0\cdot(x_{i+1}-x_{i-1})+
c_1\cdot\dfrac{x_{i+1}^2-x_{i-1}^2}{2}+
c_2\cdot\dfrac{x_{i+1}^3-x_{i-1}^3}{3}.
\end{gather*}

% график
\begin{figure}[H]\centering
\FigInt{-0.7}{2}{ {Lagrange(x)} 
node[pos=0.7,color=darkred,opacity=1,above right] {$L_2(x)$}
}
\caption{График подынтегральной функции $f(x)$ и
аппроксимирующего полинома $L_2(x)$ на частичном отрезке
для формулы Симпсона}
\label{fig:IntSimpson}
\end{figure}

%***********************************
%
%	Численное интегрирования функции заданной таблично
%
%***********************************
\subsection{Численное интегрирования функции заданной таблично}
На множестве узлов сетки $\{x_i\}$ определены 
значения некоторой функции $\{y_i\}=f(x_i)$:
\begin{table}[H]
\vspace{-0.5\baselineskip}
\caption{Таблично заданная функциональная зависимость}
\begin{tabular*}{\textwidth}{%
l@{\extracolsep{\fill}}*{5}{r}p{0.25cm}}
\toprule
$i$&$0$&$1$&$2$&$3$&$4$\\
\midmidrule
$x_i$&$-3.31$&$0.31$&$1.32$&$2.47$&$3.50$\\
\addlinespace% дополнительный пробел
$y_i$&$2.45$&$4.03$&$-3.61$&$4.50$&$3.10$\\
\bottomrule
\end{tabular*}
\end{table}

\begin{enumerate}
% Стиль графиков
\pgfplotsset{%width=8cm,
xmin=-4,xmax=4,xtick={-4,-2,0,2,4},
ymin=-6,ymax=6,ytick={-6,-3,0,3,6},
}
\item
Построим график функции $f(x)$ заданной таблично.
% *******************************
%	График функций
%
\begin{figure}[H]\centering
\begin{tikzpicture}
\begin{axis}
\addplot[name path=A,ball darkblue,smooth] coordinates 
{(-3.31,2.45) (0.31,4.03) (1.32,-3.61) (2.47,4.50) (3.50, 3.1)};
\end{axis}
\end{tikzpicture}
\end{figure}
% *******************************
\item
Воспользуемся левой и правой формулами прямоугольников
для нахождения  численного значения интеграла 
от функции $f(x)$, заданной таблично на отрезке $x\in[x_0,x_4]$.
Для этого разобьем весь отрезок интегрирования 
на частичные отрезки: 
\begin{gather*}
[x_0,x_4]=[x_0,x_1] \cup [x_1,x_2] \cup [x_2,x_3] \cup [x_3,x_4]
\end{gather*}
%	График функций
\begin{figure}[H]\centering
% левые прямоугольники
\begin{tikzpicture}[baseline]
\begin{axis}
\addplot[name path=A,const plot mark left,ball darkblue] coordinates 
{(-3.31,2.45) (0.31,4.03) (1.32,-3.61) (2.47,4.50) (3.50, 3.1)};
%\path [name path=B] (\pgfkeysvalueof{/pgfplots/xmin},0) -- (\pgfkeysvalueof{/pgfplots/xmax},0);
\path[name path=B] (axis cs: -3.31,0) -- (axis cs: 3.5,0);
\addplot[blue!15] fill between [of=A and B, soft clip={domain=-3.31:3.5}];
\end{axis}
% правые прямоугольники
\end{tikzpicture}
\begin{tikzpicture}[baseline]
\begin{axis}
\addplot[name path=A,const plot mark right,ball darkblue] coordinates 
{(-3.31,2.45) (0.31,4.03) (1.32,-3.61) (2.47,4.50) (3.50, 3.1)};
\path[name path=B] (axis cs: -3.31,0) -- (axis cs: 3.5,0);
\addplot[blue!15] fill between [of=A and B, soft clip={domain=-3.31:3.5}];
\end{axis}
\end{tikzpicture}
\caption{Использование квадратурных формул 
левых и правых прямоугольников}
\end{figure}

На каждом частичном отрезке квадратурная формула 
является следствием замены подынтегральной функции 
$f(x)$ интерполяционным полиномом 
нулевой степени $f(x)\approx L_0(x)=\const$, 
построенным но узлам $\{x_{i-1}, x_{i}\}$.

По методу прямоугольников, определим значение 
интеграла на каждом частичном отрезке
(левые прямоугольники):
\begin{gather*}
\begin{array}{lclllll}
S^{-}_1&=&y_0\cdot(x_1-x_0)&=&2.45\cdot(0.31-(-3.31))&\approx&8.87\\
S^{-}_2&=&y_1\cdot(x_2-x_1)&=&4.03\cdot(1.32-0.31)&\approx&4.07\\
S^{-}_3&=&y_2\cdot(x_3-x_2)&=&-3.61\cdot(2.47-1.32)&\approx&-4.15\\
S^{-}_4&=&y_3\cdot(x_4-x_3)&=&4.5\cdot(3.50-2.47)&\approx&4.64\\
\end{array}
\end{gather*}

(правые прямоугольники):
\begin{gather*}
\begin{array}{*7l}
S^{+}_1&=&y_1\cdot(x_1-x_0)&=&4.03\cdot(0.31-(-3.31))&\approx&14.59\\
S^{+}_2&=&y_2\cdot(x_2-x_1)&=&-3.61\cdot(1.32-0.31)&\approx&-3.65\\
S^{+}_3&=&y_3\cdot(x_3-x_2)&=&4.50\cdot(2.47-1.32)&\approx&5.18\\
S^{+}_4&=&y_4\cdot(x_4-x_3)&=&3.10\cdot(3.50-2.47)&\approx&3.19\\
\end{array}
\end{gather*}

Значение интегралов $I^{-}$ и $I^{+}$ на всем отрезке 
интегрирования $[x_0,x_4]$: 
\begin{gather*}
\begin{array}{*7l}
I^{-}&=&S^{-}_1+S^{-}_2+S^{-}_3+S^{-}_4&=&
8.87+4.07-4.15+4.64&=&13.43\\
I^{+}&=&S^{+}_1+S^{+}_2+S^{+}_3+S^{+}_4&=&
14.59-3.65+5.18+3.19&=&19.31
\end{array}
\end{gather*}
 
\item
Рассмотрим \emph{метод трапеций} для нахождения 
численного значения интеграла от функции $f(x)$,
заданной таблично на отрезке $x\in[x_0,x_4]$.
Разобьем весь отрезок интегрирования на частичные отрезки: 
\begin{gather*}
[x_0,x_4]=[x_0,x_1] \cup [x_1,x_2] \cup [x_2,x_3] \cup [x_3,x_4]
\end{gather*}

На каждом частичном отрезке квадратурная формула 
является следствием замены подынтегральной функции 
$f(x)$ интерполяционным полиномом Лагранжа 
первой степени $f(x)\approx L_1(x)$, 
построенным но узлам $\{x_{i-1}, x_{i}\}$, 
т.е. прямой соединяющей два соседних узла.

% *******************************
%	График функций
%
\begin{figure}[H]\centering
\begin{tikzpicture}
\begin{axis}
\addplot[name path=A,
thick,draw=darkblue,mark=ball,mark size=3pt,
mark options={ball color=darkblue!50,thin,draw=darkblue}
] coordinates 
{(-3.31,2.45) (0.31,4.03) (1.32,-3.61) (2.47,4.50) (3.50, 3.1)};
%\path [name path=B] (\pgfkeysvalueof{/pgfplots/xmin},0) -- (\pgfkeysvalueof{/pgfplots/xmax},0);
\path[name path=B] (axis cs: -3.31,0) -- (axis cs: 3.5,0);
\addplot[blue!15] fill between [of=A and B, soft clip={domain=-3.31:3.5},];
\end{axis}
\end{tikzpicture}
\caption{Использование квадратурных формул трапеций}
\end{figure}
% *******************************

По методу трапеций, определим значение интеграла на каждом частичном отрезке:
\begin{gather*}
%\renewcommand*{\arraystretch}{2}
\begin{array}{*7l}
S_1&=&\dfrac{y_1+y_0}{2}\cdot(x_1-x_0)&=&
\dfrac{4.03+2.45}{2}\cdot(0.31-(-3.31))&\approx&11.73\\[1em]
S_2&=&\dfrac{y_2+y_1}{2}\cdot(x_2-x_1)&=&
\dfrac{-3.61+4.03}{2}\cdot(1.32-0.31)&\approx&0.21\\[1em]
S_3&=&\dfrac{y_3+y_2}{2}\cdot(x_3-x_2)&=&
\dfrac{4.50-3.61}{2}\cdot(2.47-1.32)&\approx&0.51\\[1em]
S_4&=&\dfrac{y_4+y_3}{2}\cdot(x_4-x_3)&=&
\dfrac{3.10+4.50}{2}\cdot(3.50-2.47)&\approx&3.91
\end{array}
\end{gather*}

Определим интеграл $I$ на всем отрезке интегрирования 
$[x_0,x_4]$, воспользовавшись свойством аддитивности 
интеграла:
\begin{gather*}
I=S_1+S_2+S_3+S_4=11.73+0.21+0.51+3.91=16.37.
\end{gather*}

\item
Рассмотрим \emph{метод Симпсона} для нахождения 
численное значение интеграла от функции $f(x)$,
заданной таблично на отрезке $x\in[x_0,x_4]$.

Разделим всё множество узлов сетки $\{x_i\}$, в которых
известны значения функции $\{y_i\}$, на основные 
и вспомогательные узлы:
\begin{table}[H]
\vspace{-0.5\baselineskip}
\caption{Таблично заданная функциональная зависимость}
\begin{tabular*}{\textwidth}{%
l@{\extracolsep{\fill}}*{5}{r}p{0.25cm}}
\toprule
$i$&$0$&$1-1/2$&$1$&$1+1/2$&$2$\\
\midmidrule
$x_i$&$-3.31$&$0.31$&$1.32$&$2.47$&$3.50$\\
\addlinespace% дополнительный пробел
$y_i$&$2.45$&$4.03$&$-3.61$&$4.50$&$3.10$\\
\bottomrule
\end{tabular*}
\end{table}
Разобьем весь отрезок интегрирования на частичные отрезки:
\begin{gather*}
[x_0,x_2]=[x_0,x_1] \cup [x_1,x_2].
\end{gather*}

В пределах первого частичного отрезка $[x_0,x_1]$
построим интерполяционный полином Лагранжа $L_2(x)$
по узлам сетки $x_0=-3.31$, $x_{1-1/2}=0.31$, $x_1=1.32$:
\begin{gather*}
\begin{matrix}
L_2(x)&=&\dfrac{(x-0.31)(x-1.32)}{((-3.31-0.31)(-3.31-1.32)}\cdot2.45&+\\[1em]
&+&\dfrac{(x-(-3.31))(x-1.32)}{(0.31-(-3.31))(0.31-1.32)}\cdot4.03&+\\[1em]
&+&\dfrac{(x-(-3.31))(x-0.31)}{(1.32-(-3.31))(1.32-0.31)}\cdot(-3.61)&\\
\end{matrix}
\end{gather*}

После алгебраических преобразований запишем 
интерполяционный полином в каноническом виде:
\begin{gather*}
L_2(x)=5.66-4.74\cdot x-1.73\cdot x^2
\end{gather*}

Определим интеграл от интерполяционного полинома 
$L_2(x)$ на первом частичном отрезке:
\begin{gather*}
I_1=\int\limits_{x_0}^{x_1} L_2(x)dx=
\int\limits_{-3.31}^{1.32}
\left(5.66-4.74\cdot x-1.73\cdot x^2\right)dx=25.88
\end{gather*}

% *******************************
%	График функций
\begin{center}
\begin{tikzpicture}
\begin{axis}[ymax=10]
\addplot[only marks,ball darkblue] coordinates 
{(-3.31,2.45) (0.31,4.03) (1.32,-3.61) (2.47,4.50) (3.50, 3.1)};
%\path [name path=B] (\pgfkeysvalueof{/pgfplots/xmin},0) -- (\pgfkeysvalueof{/pgfplots/xmax},0);
\addplot[name path=A,ball darkblue,mark=none,domain=-3.31:1.32]
{-1.73*x^2-4.74*x+5.66} node[pos=0.4,right] {$L_2(x)$};
\path[name path=B] (axis cs: -3.31,0) -- (axis cs: 1.32,0);
\addplot[darkblue!15] fill between [of=A and B, soft clip={domain=-3.31:1.32}];
\end{axis}
\end{tikzpicture}
\end{center}
% *******************************

В пределах второго частичного отрезка $[x_1,x_2]$
построим интерполяционный полином Лагранжа $L_2(x)$ 
по узлам сетки $x_1=1.32$, $x_{1+1/2}=2.47$, $x_2=3.50$:
\begin{gather*}
\begin{matrix}
L_2(x)&=&\dfrac{(x-2.47)(x-3.50)}{(1.32-2.47)(1.32-3.50)}\cdot(-3.61)&+\\[1em]
&+&\dfrac{(x-1.32)(x-3.50)}{(2.47-1.32)(2.47-3.50)}\cdot4.50&+\\[1em]
&+&\dfrac{(x-1.32)(x-2.47)}{(3.50-1.32)(3.50-2.47)}\cdot3.10\\
\end{matrix}
\end{gather*}

После тривиальных алгебраических преобразований:
\begin{gather*}
L_2(x)=-25.56+21.76\cdot x-3.87\cdot x^2
\end{gather*}

Определим интеграл от интерполяционного полинома 
$L_2(x)$ на втором частичном отрезке:
\begin{gather*}
I_2=\int\limits_{x_1}^{x_2} L_2(x)dx=
\int\limits_{1.32}^{3.50} \left(-25.56+21.76\cdot x-3.87\cdot x^2\right)dx=6.13
\end{gather*}

% *******************************
%	График функций
\begin{center}
\begin{tikzpicture}
\begin{axis}[ymax=10]
% данные
\addplot[only marks,ball darkblue] coordinates 
{(-3.31,2.45) (0.31,4.03) (1.32,-3.61) (2.47,4.50) (3.50, 3.1)};
% Ox
\path [name path=Ox] (axis cs: -3.31,0) -- (axis cs: 3.50,0);
% первый отрезок
\addplot[name path=A,thick,color=darkblue,domain=-3.31:1.32]
{-1.73*x^2-4.74*x+5.66} node[pos=0.4,right] {$L_2(x)$};
\addplot[darkblue!15] fill between [of=A and Ox, 
soft clip={domain=-3.31:1.32}];
% второй отрезок
\addplot[name path=C,thick,color=darkblue,domain=1.32:3.50]
{-3.87*x^2+21.76*x-25.56} node[pos=0.8,above] {$L_2(x)$};
\addplot[darkblue!15] fill between [of=C and Ox, 
soft clip={domain=1.32:3.50}];
\end{axis}
\end{tikzpicture}
\end{center}
% *******************************

Определим интеграл всем отрезке $[x_0,x_2]$ 
воспользовавшись свойством аддитивности интеграла:
\begin{gather*}
I=I_1+I_2=25.88+6.13=32.01
\end{gather*}

\item
Сравнивая численные значения определенного интеграла 
рассчитанные по методам прямоугольников, трапеций и Симпсона,
можно сделать вывод о том, что рассчитанные значения
различаются.
\begin{table}[H]
\vspace{-0.5\baselineskip}
\caption{Численные значения интегралов}
\begin{tabular*}{\textwidth}{%
l@{\extracolsep{\fill}}lp{3cm}}
\toprule
Метод интегрирования&Значение интеграла\\
\midmidrule
Левых прямоугольников&$13.43$\\
Правых прямоугольников&$19.31$\\
Трапеций&$16.37$\\
Симпсона&$32.01$\\
%\addlinespace% дополнительный пробел
\bottomrule
\end{tabular*}
\end{table}
Значение определенного интеграла от функции заданной таблично,
рассчитанное по методу Симпсона является наибольшим, а 
значение рассчитанное по методу левых прямоугольников 
-- наименьшее.
\end{enumerate}

% Задача Коши для систем обыкновенных дифференциальных уравнений
%\input{chapter/ODE_IC}
% Краевые задачи для обыкновенных дифференциальных уравнений
%\input{chapter/ODE_BC}
% Нелинейные уравнения
%\input{chapter/Nonlinear_Equations}

%
% ЗАКЛЮЧЕНИЕ
%
\begin{Conclusion}
В работе рассмотрен метод Эйлера численного решения задачи Коши 
для системы обыкновенных дифференциальных уравнений первого порядка.
Для заданной системы дифференциальный уравнений первого порядка
построены рекуррентные соотношения для неизвестных функций $u_1(t)$ и $u_2(t)$,
которые позволяют последовательно определить их значения в узлах временной сетке $\omega_\tau$.

На основе построенных рекуррентных соотношений найдено 
численное решение задачи Коши в узлах равномерной сетке 
$\omega_\tau=\{0,2,4,6,8,10\}$.
Построены графики функций $u_1(t)$ и $u_2(t)$ на основании 
вычисленных значений значениях неизвестных функций 
в различных узлах временной сетки $\omega_\tau$.

Определена предельная абсолютная погрешность приближенного решения задачи Коши
в пределах всего заданного временного интервала.
Установлено, что максимальная предельная абсолютная погрешность для $u_1(t)$ 
составляет $\epsilon_1=5,1$, а для функции $u_2(t)$ -- $\epsilon_2=2,4$.

Приобретен практический навык применения численных методов решения задачи Коши 
для систем обыкновенных дифференциальных уравнений первого порядка.
\end{Conclusion}

%
% СПИСОК ИСПОЛЬЗОВАННЫХ ИСТОЧНИКОВ
%
\begin{References}{9}
\bibitem{Samarsky-2009}
Самарский А. А. Введение в численные методы. – Лань, 2009. --288 c.
\bibitem{Samarsky-1989}
Самарский А. А., Гулин А. В. Численные методы: учебное пособие для вузов // M.: Наука. Гл. ред. физ-мат. лит. 1989. --432 с.
\bibitem{Samarsky-2000}
Самарский А. А. и др. Задачи и упражнения по численным методам: Учебное пособие // М.: Эдиториал УРСС, 2000.  --208 с.
\bibitem{Kalitkin-2011}
Калиткин Н. Н. Численные методы. 2 изд. --СПб.: БХВ-Петербург, 2011. --592 с.
\bibitem{Kalitkin-1978}
Калиткин Н. Н. Численные методы: Учебное пособие. – Наука. Гл. ред. физ.-мат. лит., 1978. --511 с.
\bibitem{Demidovich-1967}
Демидович, Б.П. Численные методы анализа: приближение функций, дифференциальные и интегральные уравнения / Б.П. Демидович, И.А. Марон, Э.З. Шувалова ; под ред. Б.П. Демидович. – Изд. 3-е, перераб. – Москва : Главная редакция физико-математической литературы, 1967. --368 с.
\end{References}

\end{document}
